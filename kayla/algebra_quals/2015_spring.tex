\documentclass[12pt]{AlgebraQual}
\usepackage{preamble}

\name{Kayla Orlinsky}
\course{Algebra Exam}
\term{Spring 2015}
\hwnum{Spring 2015}

\begin{document}

\begin{problem} $\,$
Use Sylow's theorems and other results to describe, up to isomorphism, the possible structures of a group of order $1005$.
\end{problem}


\begin{solution}$\,$
Let $G$ be a group of order $1005=3\cdot5\cdot 67.$ By Sylow, $n_{67}|15$ and $n_{67}\equiv 1\mod67$ so clearly $n_{67}=1$.

Now, we examine the cases.

\boxed{\text{Abelian}} Then $G\cong\mathbb{Z}_{1005}.$

Let $P_{67},P_5,P_3$ be Sylow $67,5,3$-subgroups respectively. Now, by the recognizing semi-direct products theorem. Since $P_{67}$ is normal, $P_3P_{67}$ and $P_5P_{67}$ are subgroups of $G$, and since $$|P_3P_5P_{67}|=3\cdot5\cdot67/|P_3\cap(P_5P_{67})|=1005=|G|$$ we have that $G$ is a semi-direct product of its Sylow subgroups.

Since $P_5P_{67}$ is a subgroup and has index $3$ which is the smallest prime dividing the order of the group (see \textbf{Spring 2010: Problem 2 Claim 1}).

Therefore, $P_5P_{67}$ is normal in $G$. Now, since $P_5$ is also a Sylow $p$-subgroup of $P_5P_{67}$ and $n_5=1$ in $P_5P_{67}$ by Sylow. Therefore, by \textbf{Fall 2011: Problem 5 Claim 3}, $P_5$ is also normal in $G$.

Finally, we have that $P_3P_5$ is a subgroup of $G$ and so to determine possible structures of $G$ as a semi-direct product, we need only look at three homomorphisms.

\boxed{\varphi:P_3P_5\to\aut(P_{67})} Since $P_3P_5$ is of order $pq$ where $p\nmid(q-1)$, we have that $P_3P_5\cong\mathbb{Z}_{15}.$

Furthermore, $\aut(P_{67})\cong\mathbb{Z}_{66}$.

Thus, if $P_3\cong\langle a\rangle$, $P_5\cong\langle b\rangle$, and $P_{67}\cong\langle c\rangle$, we have that $\varphi(b)=\id$ since $5$ does not divide the order of $\mathbb{Z}_{66}$ and $\varphi(a)=\alpha$ where $\alpha$ has order $3.$

Since $\mathbb{Z}_{66}$ is abelian and $66=2\cdot3\cdot 11$, there are exactly two non-trivial options for $\alpha$. Note that one will be the square of the other. Namely, if $\varphi_1(a)=\alpha$ and $\varphi_2(a)=\alpha^2$, then $\varphi_1(a^2)=\varphi_2(a)$ and since $a\mapsto a^2$ is an automorphism of $\mathbb{Z}_3$, these will generate isomorphic semi-direct products.

Thus, we need only find one element of order $3$ in $\mathbb{Z}_{66}.$

This element is given by $\alpha:\mathbb{Z}_{67}\to\mathbb{Z}_{67}$ defined by $\alpha(c)=c^{29}$.

Once can check that $$\alpha^3(c)=\alpha^2(c^{29})=\alpha(c^{37})=c.$$

Therefore, we obtain a possible multiplication for $G$ given by $bcb^{-1}=\varphi(b)(c)=c$ and $aca^{-1}=\varphi(a)(c)=c^{29}.$

Thus, $$G\cong\langle a,b,c\,|\, a^3=b^5=c^{67}=1,ab=ba,bc=cb,ac=c^{29}a\rangle.$$

\boxed{\varphi:P_3P_{67}\to \aut(P_5)} Since $P_5$ is normal, we can check $\varphi:P_3P_{67}\to P_5$, however $\aut(P_5)\cong\mathbb{Z}_4$ and $P_3P_{67}$ have no elements of order $2$ or $4$, so only the trivial homomorphism is possible.

\boxed{\varphi:P_3\to \aut(P_5P_{67})} since $5$ and $67$ are coprime, $\aut(P_5P_{67})\cong\mathbb{Z}_4\times\mathbb{Z}_{66}$. However, since there are no elements in $\mathbb{Z}_4$ of order $3$, the only possible non-trivial homomorphisms will generate the same multiplication as the first case.

Therefore, there are only two groups of order $1005$.
\begin{center}
\begin{framed}
$$\mathbb{Z}_{1005}$$

$$\langle a,b,c\,|\, a^3=b^5=c^{67}=1,ab=ba,bc=cb,ac=c^{29}a\rangle\cong\mathbb{Z}_3\rtimes_\varphi\mathbb{Z}_{67}\times\mathbb{Z}_5$$
\end{framed}
\end{center}
\end{solution}
\newpage


\begin{problem} $\,$
Let $R$ be a commutative ring with $1$. Let $M,N$ and $V$ be $R$-modules.
\begin{enumerate}[label=(\alph*)]
    \item Show that if $M$ and $N$ are projective, then so is $M\otimes_RN$.
    \item Let $\tr(V)=\{\sum_i\varphi_i(v_i)\,|\,\varphi\in\Hom_R(V,R),v_i\in V\}\subset R.$ If $1\in\tr(V)$, show that up to isomorphism, $R$ is a direct summand of $V^k$ for some $k.$
\end{enumerate}
\end{problem}


\begin{solution}$\,$
\begin{enumerate}[label=(\alph*)]
    \item Since $M$ and $N$ are projective, there exists $A$, $B$, $R$-modules such that $$M\oplus A\cong R^m\qquad N\oplus B\cong R^n$$ where $R^m\cong\bigoplus_{i=1}^mR_i$ and $R^n$ are free modules of dimension $m$ and $n$ respectively.

    Thus,  \begin{align*}
        (M\otimes_RN)\oplus[(A\otimes_R N)\oplus B^m]&=[(M\oplus A)\otimes_RN]\oplus B^m\\
        &=[R^m\otimes_RN]\oplus B^m\\
        &=[(R\oplus R\oplus\cdots\oplus R)\otimes_R N]\oplus B^m\\
        &=[N\oplus N\oplus \cdots\oplus N]\oplus B^m\tag{1}\\
        &=(N\oplus B)\oplus (N\oplus B)\oplus \cdots \oplus (N\oplus B)\\
        &=R^n\oplus R^n\oplus\cdots\oplus R^n\\
        &=R^{nm}
    \end{align*}

    with (1) because $R\otimes_R N=N.$ Therefore, $M\otimes_R N$ is the summand of a free module so it is projective.

    \item Let $$\tr(V)=\{\sum_i\varphi_i(v_i)\,|\,\varphi\in\Hom_R(V,R),v_i\in V\}\subset R.$$

    Now, we note that $\tr(V)=\sum\varphi(V)$ where the sum is taken over all $\varphi\in\hom_R(V,R)$. Furthermore, because $\varphi$ is homomorphism, it is easily verified that $\varphi(V)$ is an ideal of $R$ for all $\varphi$.

    Now, $\tr(V)$ is an ideal of $R$ since it is clearly closed under addition and for any $r\in R$, $$r\sum_i\varphi_i(v_i)=\sum_i\varphi_i(rv_i)\in\tr(V)$$ since the $\varphi$ are homomorphisms and $rv_i\in V$ since $V$ is an $R$-modules. This gives that $\tr(V)$ is a left ideal and since $R$ is commutative it will be a right ideal as well.

    Therefore, if $1\in\tr(V)$ then $\tr(V)=R$. Thus, there exists finitely many $\varphi_i\in\Hom_R(V,R)$ and $v_i\in V$ such that $$1=\varphi_1(v_1)+\cdots+\varphi_k(v_k)\qquad k\text{ minimal}.$$

    Namely, for every $r\in R$, there exists $w_j\in V$ such that $$r=\varphi_1(w_1)+\cdots+\varphi_k(w_k).$$

   Now, because $k$ is minimal, if $$r\in \varphi_i(V)\cap\bigoplus_{j\not=i}\varphi_j(V)$$ then $$r=\varphi_i(w_i)=\sum_{j\not=i}\varphi_j(w_j)$$

   Thus, we can define \begin{align*}
       f:V^k&\to R\\
       (w_1,...,w_k)&\mapsto\sum_{i=1}^k\varphi_i(w_i)
   \end{align*}
   which we have already found to be surjective.

   Therefore, we have a short exact sequence
   \begin{center}
   \begin{tikzcd}
   0 \arrow[r] & \ker(f) \arrow[r] & V^k\arrow[r] & R\arrow[r] &0
   \end{tikzcd}
   \end{center}

   However, $R$ is a free module over itself, so $R$ is projective. Therefore, the above short exact sequence is split and so by the splitting lemma, $$V^k\cong R\oplus\ker(f).$$

   Therefore, $R$ is a direct summand of $V^k.$
\end{enumerate}
\end{solution}
\newpage



\begin{problem} $\,$
Let $F$ be a field and $M$ a maximal ideal of $F[x_1,...,x_n]$. Let $K$ be an algebraic closure of $F$. Show that $M$ is contained in at least $1$ and in only finitely many maximal ideals of $K[x_1,...,x_n].$
\end{problem}


\begin{solution}$\,$
First, by generalized Nullstellensatz, $V(M)\not=\varnothing$ as a subset of $K^n$, since $M$ is maximal in $F[x_1,...,x_n]$ so $1\notin M.$

Namely, there exists $(a_1,...,a_n)\in K^n$ such that $(a_1,...,a_n)\in V(M).$

Therefore, again by Nullstellensatz, for every $f\in M$, there exists $m$ such that $f^m\in(x_1-a_1,...,x_n-a_n)$ which is a maximal ideal of $K[x_1,...,x_n].$

However, maximal ideals are prime, and so inductively, we get that $f\in(x_1-a_1,...,x_n-a_n)$. Therefore, $M\subset (x_1-a_1,...,x_n-a_n)$ so $M$ is contained in at least one maximal ideal.

%Now, for each distinct point $(b_1,...,b_n)\in V(M)$ we could make the same argument, namely that $M\subset(x_1-b_1,...,x_n-b_n).$

Next, we prove a claim about $L=F[x_1,...,x_n]/M$.

\begin{claim} If $L=F[x_1,...,x_n]/M$ is a field, then it is a finite field extension of $F$.
\begin{proof} We proceed by induction on $n.$

Basecase: let $L=F[a_1]$ be a field. Then for $f(a_1)\in L$ there exists $g(a_1)\in L$ such that $f(a_1)g(a_1)=1\in L$ and so $a_1$ satisfies $h(x)=f(x)g(x)-1$. Namely, $a_1$ is algebraic over $F$ and so $L$ is a finite field extension of $F.$

Assume $L=F[a_1,...,a_k]$ is a finite field extension of $F$ for all $k\le n$.

Then let $L=F[a_1,...,a_n][a_{n+1}]$. Since $L$ is a field, by the same reasoning as the basecase, $L$ is algebraic over $F[a_1,...,a_n]$. However, by the inductive hypothesis, $F[a_1,...,a_n]$ is a finite field extension of $F$ and so $$[L:F]=[L:F[a_1,...,a_n]][F[a_1,...,a_n]:F]<\infty.$$
\end{proof}
\end{claim}

Now, if $N$ is a maximal ideal of $K[x_1,...,x_n]$ such that $M\subset N$, then we will clearly have an embedding $$L\hookrightarrow K[x_1,...,x_n]/N\cong K$$ induced by the embedding $M\hookrightarrow N$. Note that since $K[x_1,...,x_n]/N$ is a finite field extension of $K$ which is algebraically closed, it must be isomorphic to $K.$

Namely, each embedding of $L$ is associated to exactly one maximal ideal $N$ of $K[x_1,...,x_n]$ such that $M\subset N$.

\begin{claim} If $L$ is a finite field extension of $F$, then there exists only finitely many embeddings of $L$ into $K$ the algebraic closure of $F.$
\begin{proof} We proceed by induction.

Basecase: let $L=F(a_1)$. Because $a_1$ is algebraic over $F$, it has minimal (irreducible) polynomial $$f(x)=x^n+\alpha_{n-1}x^{n-1}+\cdots+\alpha_1x+\alpha_0\in F[x].$$

Now, if $\varphi:L\hookrightarrow K$, because $\varphi(1)=1$, $\varphi$ is $F$-linear and so $$\varphi(f(a_1))=\varphi(a_1)^n+\alpha_{n-1}\varphi(a_1)^{n-1}+\cdots+\alpha_1\varphi(a_1)+\alpha_0=0$$ so $\varphi$ permutes the roots of $f(x).$ Note that $K$ is the algebraic closure of $F$ and so contains all such roots.

Thus, there are only finitely many possible choices of $\varphi$ since there are only finitely many roots of $f(x).$

Now, assume there are only finitely many injections of $L=F(a_1,...,a_k)$ to $K$ for $k\le n$.

Then we examine $L=F(a_1,...,a_n,a_{n+1})=F(a_1,...,a_n)(a_{n+1}).$ Then there are only finitely many $F(a_1,...,a_n)$-linear injections from $L\hookrightarrow K$ by the same reasoning as the basecase, and by the induction hypothesis, only finitely many $F$-linear injections from $F(a_1,...,a_n)\hookrightarrow K$.

Since any injection $L\hookrightarrow K$ will be defined by where it sends the $a_i,$ and since there are only finitely many choices for where to send $a_1,...,a_n$ and only finitely many choices for where to send $a_{n+1}$, we have only finitely many possible injections of $L$ into $K.$
\end{proof}
\end{claim}

Finally, since there are only finitely many possible embeddings of $F[x_1,...,x_n]/M$ to $K[x_1,...,x_n]/N$ there can be only finitely many maximal ideals $M\subset N$.
\end{solution}
\newpage


\begin{problem} $\,$
Let $F$ be a finite field.
\begin{enumerate}[label=(\alph*)]
    \item Show that there are irreducible polynomials over $F$ of every positive degree.
    \item Show that $x^4+1$ is irreducible over $\mathbb{Q}[x]$ but is reducible over $\mathbb{F}_p[x]$ for every prime $p$ (hint: show there is a root in $\mathbb{F}_{p^2}[x]).$
\end{enumerate}
\end{problem}


\begin{solution}$\,$
\begin{enumerate}[label=(\alph*)]
    \item Let $F$ be a finite field of $q=p^k$ elements. Fix a positive integer $n$.

    Then let $K$ be the field of $q^n=p^{nk}$ elements. Then $K^\times$ is a cyclic multiplicative group. Now, because finite fields of the same order are isomoprhic, $K$ is isomorphic to a field extension of $F.$

    Therefore, $$[K:F]=\frac{[K:F_p]}{[F:F_p]}=\frac{nk}{k}=n$$ where $F_p$ is the field of $p$ elements. Thus, there exists an element $\alpha\in K$ such that $\alpha$ has minimal polynomial of degree $n$ over $F.$

    By definition, the minimal polynomial is irreducible and has degree $n$ over $F.$

    \item First, $x^4+1$ has no roots in $\mathbb{Q}$ so if it reduces it has no linear terms. Namely, it can only reduce into a product of two quadratic polynomials. However, $x^4+1=(x^2-i)(x^2+i)$ over $\mathbb{C}[x]$ and since $i\notin\mathbb{Q}$, we have that $x^4+1$ is irreducible.

    Now, we examine $x^4+1$ as a polynomial over $\mathbb{F}_p[x]$.

    If $p=2$, then $$x^4+1=(x^2)^2+1^2=(x^2+1)^2$$ and so it is reducible.

    If $p$ is odd, then $p=2k+1$ and $k\ge1$. $$p^2=(2k+1)^2=4k^2+4k+1=4(k^2+k)+1=8r+1$$ since if $k$ is even, $k^2+k$ is also even, and if $k$ is odd, then $k^2+k$ is a sum of two odds and so it is also even.

    Namely, $p^2\equiv 1\mod 8$ for any odd $p.$ Therefore, $$(x^8-1)|(x^{p^2-1}-1).$$ However, then if $\alpha$ is a root of $x^4+1$, then $\alpha$ is a root of $(x^4+1)(x^4-1)=x^8-1$ and so it is a root of $x^{p^2-1}-1$. Finally, we have that $$\alpha^{p^2-1}=1\implies \alpha^{p^2}=\alpha$$ and so $\alpha\in\mathbb{F}_{p^2}.$

     Now, if $x^4+1$ is irreducible over $\mathbb{F}_p[x]$ and $\alpha$ is a root of $x^4+1$, then $[\mathbb{F}_p(\alpha):\mathbb{F}_p]=4$.

    However, $\alpha\in \mathbb{F}_{p^2}$ and so $$2=[\mathbb{F}_{p^2}:\mathbb{F}_p]=[\mathbb{F}_{p^2}:\mathbb{F}_p(\alpha)][\mathbb{F}_p(\alpha):\mathbb{F}_p]=[\mathbb{F}_{p^2}:\mathbb{F}_p(\alpha)]4$$ which is clearly a contradiction.

    Thus, $x^4+1$ is reducible over $\mathbb{F}_p$.
\end{enumerate}
\end{solution}
\newpage



\begin{problem} $\,$
Let $F$ be a field and $M$ a finitely generated $F[x]$-module. Show that $M$ is artinian if and only if $\dim_FM$ is finite.
\end{problem}


\begin{solution}$\,$

\boxed{\implies} Assume $M$ is artinian. Because $M$ is finitely generated, $$M=F[x]m_1+\cdots+F[x]m_n$$ for some $m_i\in M.$

We proceed by induction on $n.$

Assume $M=F[x]m_1$ for some $m_1\in M$.

Then let \begin{align*}
    \varphi:F[x]&\to M\\
    f(x)&\mapsto f(x)m_1
\end{align*}

The $\ker(\varphi)=\ann(m_1)$ by definition. Therefore, $$F[x]/\ann(m_1)\cong M$$ which is Artinian. Namely, $F[x]/\ann(m_1)$ must be a field extension of $F$ since the only artinian domains are fields.

\begin{claim} An artinian integral domain $F$ is a field.
\begin{proof} Let $a\in F$ be nonzero. Then we have a decreasing chain of ideals $$(a)\supset (a^2)\supset (a^3)\supset\cdots$$ which must terminate after a finite number of steps. Thus, $(a^l)=(a^k)$ for all $l\ge k$ for some $k.$

Namely, $a^{k+1}b=a^k$ for some $b\in F$.

However, then $a^k(ab-1)=0$ and since $F$ is a domain, $a\not=0$ implies that $a^k\not=0$ and so $ab=1$. Thus, $a$ has a right inverse.

Similarly, $a$ has a left inverse so $a$ is invertible. Therefore, $F$ is a field.
\end{proof}
\end{claim}

Now, since $F[x]/\ann(m_1)$ is a field extension of $F$, and since $F[x]$ is a PID, $\ann(m_1)$ must be generated by an irreducble polynomial. Therefore, $[F[x]/\ann(m_1):F]<\infty$ since it is an algebraic extension of $F$, and so $M\cong F[x]/\ann(m_1)$ is a finite dimensional $F$-vector space.

Now, assume $$M=F[x]m_1+\cdots+F[x]m_k$$ is a finite dimensional $F$-vector space for all $k\le n$.

Then assume $$M=F[x]m_1+\cdots+F[x]m_n+F[x]m_{n+1}.$$

Then $N=F[x]m_1+\cdots+F[x]m_n$ is a submodule of $M$ which a finite dimensional $F$-vector space by the inductive hypothesis.

Thus, $M/N\cong F[x]m_{n+1}$ is an artinian $F[x]$-module and so it is finite dimensional $F$-vector space by the same reasoning as the basecase. Thus, $M/N$ and $N$ are both finite dimensional over $F$ and so $M$ must be finite dimensional over $F$.

\boxed{\impliedby} Because $M$ is finitely generated as an $F[x]$-module $$M=F[x]m_1+\cdots+F[x]m_n$$ for some $m_i\in M.$ However, because $M$ is a finite dimensional vector space over $F$, $M=x_1F+\cdots+x_mF$ for $x_1,...,x_m\in M$ linearly independent. Thus, $f(x)m_i$ can be written as a unique linear combination of the $x_i$, and so any $F[x]$-submodule of $M$ will be an $F$-subspace of $M.$

Therefore, any decreasing chain of submodules of $M$ is a decreasing chain of finite dimensional subspaces which must terminate after a finite number of steps. Thus, $M$ is artiniain as an $F[x]$-module.

\end{solution}
\newpage



\begin{problem} $\,$
Let $R$ be a right Artinian ring with a faithful irreducible right $R$-module. If $x,y\in R$, set $[x,y]:=xy-yx$. Show that if $[[x,y],z]=0$ for all $x,y,z\in R$, then $R$ has no nilpotent elements.
\end{problem}


\begin{solution}$\,$
A faithful right $R$-module is a right $R$-module where $\ann(M)=0.$

An irreducible $R$-module is equivalent to a simple $R$-module.

Since $J(R)$ is also defined as the intersection of the annihilators of all simple right $R$-modules, $J(R)=0$ since $R$ has a simple right-module with trivial annihilator.

Therefore, by Artin-Wedderburn, $R$ is semi-simple and so $$R\cong M_{n_1}(D_1)\oplus\cdots\oplus M_{n_k}(D_k)$$ as a right $R$-module where $D_k$ are division rings over $R.$

Let $n_i>1$ for some $i$. Then we define the following matrices:

Let $$x=\begin{bmatrix}
    0 & 0 & \cdots & 0 & 1\\
    0 & 0 & \cdots & 0 & 0\\
    & \vdots & \ddots & \vdots &\\
    0 & 0 & \cdots & 0 & 0\\
    0 & 0 & \cdots & 0 & 0
    \end{bmatrix} \qquad y=\begin{bmatrix}
    0 & 0 & \cdots & 0 & 1\\
    0 & 0 & \cdots & 1 & 0\\
    & \vdots & \ddots & \vdots &\\
    0 & 1 & \cdots & 0 & 0\\
    1 & 0 & \cdots & 0 & 0
    \end{bmatrix}\qquad z=x $$

    Then $$x^2=0$$ and $$xyx=\begin{bmatrix}
    1 & 0 & \cdots & 0 & 0\\
    0 & 0 & \cdots & 0 & 0\\
    & \vdots & \ddots & \vdots &\\
    0 & 0 & \cdots & 0 & 0\\
    0 & 0 & \cdots & 0 & 0
    \end{bmatrix}\begin{bmatrix}
    0 & 0 & \cdots & 0 & 1\\
    0 & 0 & \cdots & 0 & 0\\
    & \vdots & \ddots & \vdots &\\
    0 & 0 & \cdots & 0 & 0\\
    0 & 0 & \cdots & 0 & 0
    \end{bmatrix}=\begin{bmatrix}
    0 & 0 & \cdots & 0 & 1\\
    0 & 0 & \cdots & 0 & 0\\
    & \vdots & \ddots & \vdots &\\
    0 & 0 & \cdots & 0 & 0\\
    0 & 0 & \cdots & 0 & 0
    \end{bmatrix}=x$$

Therefore, \begin{align*}
    [[x,y],x]&=[(xy-yx),x]\\
    &=(xy-yx)x-x(xy-yx)\\
    &=xyx-yx^2-x^2y+xyx\\
    &=2xyx\\
    &=2x\\
    &=0
\end{align*} however, $2x\not=0$ and this contradicts the assumption that $[[x,y],z]=0$ for all $x,y,z\in R$ and so $n_i=1$ for all $i.$

Namely, $R$ is a direct sum of division rings and so has no nilpotent elements.
\end{solution}
\newpage


\end{document}
