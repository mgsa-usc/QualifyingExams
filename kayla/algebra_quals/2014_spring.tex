\documentclass[12pt]{AlgebraQual}
\usepackage{preamble}

\name{Kayla Orlinsky}
\course{Algebra Exam}
\term{Spring 2014}
\hwnum{Spring 2014}

\begin{document}

\begin{problem} $\,$
Let $L$ be a Galois extension of a field $F$ with $\gal(L/F)\cong D_{10}$, the dihedral group of order $10$. How many subfields $F\subset M\subset L$ are there, what are their dimensions over $F,$ and how many are Galois over $F?$
\end{problem}


\begin{solution}$\,$
$|D_{10}|=10=2\cdot 5$. Thus, by Sylow, $n_5\equiv 1\mod 5$ and $n_5|2$ so $n_5=1$. Thus, $D_{10}$ has one Sylow $5$-subgroup which is normal. Since $D_{10}$ is not abelian, $n_2\not=1$. Thus, $n_2\equiv 1\mod 2$ and $n_2|5$ so $n_2=5$.

There is the trivial subgroup $\{e\}$ which corresponds to the basefield $F$ which is trivially galois over itself.

There are $5$ subgroups $P_i$ $i=1,...,5$ of order $2$, which are not normal in $G$. Thus, there are $5$ intermediate fields $F\subset M_i\subset L$ $i=1,...,5$, such that $|P_i|=[L:M_i]=2$ so $[M_i:F]=5$ and $M_i/F$ is not a galois extension for $i=1,...,5$.

There is $1$ normal subgroup of order $5$ $Q$. Thus, there is one intermediate field $F\subset K\subset L$ with $|Q|=5=[L:K]$ and $[K:F]=2$ and $K/F$ is a galois extension.

Finally, there is the top field $L$ which corresponds to $D_{10}=\gal(L/F)$ which is galois over $F$ and $[L:F]=10$.
\end{solution}
\newpage


\begin{problem} $\,$
Up to isomorphism, using direct and semi-direct products, describe the possible structures of a group of order $5\cdot 11\cdot 61$.
\end{problem}


\begin{solution}$\,$
Let $G$ be a group of order $5\cdot 11\cdot 61.$ Then by Sylow, $n_{61}\equiv 1\mod 61$ and $n_{61}|5\cdot 11$. Thus, $n_{61}=1$. Also, $n_{11}\equiv 1\mod 11$ and $n_{11}|5\cdot 61$. Since $61\not\equiv 1\mod 11$ and $305\equiv 8\mod 11$, we have that $n_{11}=1$ as well.

Therefore, $G$ has a normal Sylow $11$-subgroup $P_{11}$ and a normal Sylow $61$-subgroup $P_{61}$.

\boxed{\text{Abelian}} If $G$ also has a normal Sylow $5$-subgroup $P_5$, then $G$ is abelian and $$G\cong\mathbb{Z}_{3355}.$$

Else, we have by the recognizing of semi-direct products theorem that $G$ is a semi-direct product of its Sylow subgroups.

There are $3$ possible homomorphisms to check.

\boxed{\varphi:P_5P_{11}\to\aut(P_{61})} Since $P_{11}$ is normal, $P_5P_{11}$ is a subgroup of $G$. Let $\varphi:P_5P_{11}\to\aut(P_{61})\cong\mathbb{Z}_{60}$ be a homomorphism.

Let $P_5\cong\langle a\rangle$, $P_{11}\cong\langle b\rangle$, $P_{61}\cong\langle c\rangle.$

Then because $\mathbb{Z}_{60}$ has no elements of order $11$, $\varphi$ is determined by where it sends $P_5$. Since $\mathbb{Z}_{60}$ is abelian, it has one normal Sylow $5$-subgroup so $\varphi(a)$ will be some generator of the Sylow $5$-subgroup of $\mathbb{Z}_{60}$. Namely, $\varphi_1(a)$ will be some power of $\varphi_2(a)$ for any two homomorphisms $\varphi_1$ and $\varphi_2$. Therefore, $\varphi_1$ and $\varphi_2$ will generate ismorophic semi-direct products since $a\mapsto a^i$ is an isomorphism of $P_5$ for $i=1,...,4$.

Thus, we need to only find one automorphism of $P_{61}$ of order $5.$

The map $\sigma:c\mapsto c^9$ has order $5$. This defines multiplication on $G$ by $bcb^{-1}=\varphi(b)(c)=c$ and $aca^{-1}=\varphi(a)(c)=c^9$.

Thus, $$G\cong\langle a,b,c\,|\, a^5=b^{11}=c^{61}=1,ab=ba,bc=cb,ac=c^9a\rangle$$

\boxed{\varphi:P_5P_{61}\to\aut(P_{11})} $\varphi:P_5P_{61}\to\aut(P_{11})\cong \mathbb{Z}_{10}$.

Again, $\mathbb{Z}_{10}$ has one Sylow $5$-subgroup and no elements of order $61$ so again, we will obtain only one unique structure defined by $\varphi(a)$ having order $5$.

Since $\tau:c\mapsto c^3$ has order $5$ we have $$G\cong\langle a,b,c\,|\, a^5=b^{11}=c^{61}=1,ac=ca,bc=cb,ab=b^3a\rangle$$

\boxed{\varphi:P_5\to\aut(P_{11}P_{61})} Since $11$ and $61$ are coprime, $\aut(P_{11}P_{61})\cong\aut(P_{11})\times\aut(P_{61})$.

Now, $\varphi(a)=(\sigma,\id),(\id,\tau)$ will define the same structures that we have already found. Namely, the only new structure would be defined by $\varphi(a)=(\sigma,\tau)$.

Thus, the final structure is $$G\cong\langle a,b,c\,|\, a^5=b^{11}=c^{61}=1,ab=b^3a,bc=cb,ac=c^9a\rangle$$

Thus, there are four possible group structures for $G.$

\begin{center}
    \begin{framed}
    $$\mathbb{Z}_{3355}$$

    $$\langle a,b,c\,|\, a^5=b^{11}=c^{61}=1,ab=ba,bc=cb,ac=c^9a\rangle$$

    $$\langle a,b,c\,|\, a^5=b^{11}=c^{61}=1,ac=ca,bc=cb,ab=b^3a\rangle$$

    $$\langle a,b,c\,|\, a^5=b^{11}=c^{61}=1,bc=cb,ab=b^3a,ac=c^9a\rangle$$
    \end{framed}
\end{center}
\end{solution}
\newpage



\begin{problem} $\,$
Let $I$ be a nonzero ideal of $R=\mathbb{C}[x_1,...,x_n]$. Show that $R/I$ is a finite dimensional algebra over $\mathbb{C}$ if and only if $I$ is contained in only finitely many maximal ideals of $R.$
\end{problem}


\begin{solution}$\,$
This is the same question as \textbf{Spring 2012: Problem 1}. We provide the same proof here as we did there.
\boxed{\implies} Assume $R/I$ is a finite dimensional algebra over $\mathbb{C}$. Then $R/I$ is artinian, since proper ideals are sub-algebras of strictly smaller degree.

Thus, if $S=\{M_1M_2\cdots M_k\,|\, M_i$ maximal ideal of $R/I\}$ is the set of finite products of maximal ideals in $R/I$. $S$ is nonempty so $S$ contains a minimal element in $R/I$, $M_1M_2\cdots M_k$. Let $N$ be some other maximal ideal of $R/I$. Then $NM_1\cdots M_k\subset M_1\cdots M_k$ so $$NM_1\cdots M_k= M_1\cdots M_k\subset N.$$ However, $N$ is maximal and so prime, thus $M_i\subset N$ for some $i$. However, by maximality, $M_i=N$.

Thus, these are the only maximal ideals of $R/I$. By the correspondence theorem, there is a $\oto$ correspondence between maximal ideals of $R$ containing $I$ and maximal ideals of $R/I$.

Since $R/I$ has only finitely many maximal ideals, there are only finitely many maximal ideals of $R$ containing $I.$

\boxed{\impliedby} Assume $I$ is contained in only finitely many maximal ideals of $R$. Note that $R$ is Noetherian by the Hilbert Basis theorem, and so all ideals are finitely generated.

Since $I$ is contained in only finitely many maximal ideals, $V(I)$ contains only finitely many points. Namely, by Nullstellensatza, $$\sqrt{I}\bigcap_{a\in\mathbb{C}^n}M_a\qquad \text{is a finite intersection}$$ where $M_a$ is the maximal ideal (by Nullstellensatz) of the form $(x_1-a_1,...,x_n-a_n)$ for $a=(a_1,...,a_n)$.

Thus, $\sqrt{I}=\bigcap_{i=1}^nM_{a_i}$ where $I\subset M_{a_i}$ for all $i$.

Since $\sqrt{I}$ is finitely generated, $\sqrt{I}=(f_1,f_2,...,f_k)$, and for each $f_i$ there exists $m_i$ so $f_i^{m_i}\in I$.

Let $m=\lcm\{m_i\}.$ Then $$I\subset\sqrt{I}=\bigcap_{i=1}^nM_{a_i}$$

and $$I\supset (\sqrt{I})^m=\left(\bigcap_{i=1}^nM_{a_i}\right)^m=\bigcap_{i=1}^nM_{a_i}^m.$$

Thus, the Chinese remainder theorem, since $M_{a_i}$ are pairwise coprime, $M_{a_i}^m$ are all pairwise coprime (since if $M_{a_i}^m+M_{a_j}^m$ is contained in some maximal ideal $M$, then $M$ contains both $M_{a_i}^m$ and $M_{a_j}^m$ and so must contain both $M_{a_i}$ and $M_{a_j}$ which forces $M=R$).

Therefore, $$R/\sqrt{I}^m\cong R/\cap_iM_{a_i}^m\cong R/\prod_iM_{a_i}^m\cong \prod R/M_{a_i}^m.$$

\begin{claim} If $F$ is a field and if $L=F[x_1,...,x_n]/M$ is a field, then $L$ is a finite field extension of $F$.
\begin{proof} We proceed by induction on $n.$

Basecase: let $L=F[a_1]$ be a field. Then for $f(a_1)\in L$ there exists $g(a_1)\in L$ such that $f(a_1)g(a_1)=1\in L$ and so $a_1$ satisfies $h(x)=f(x)g(x)-1$. Namely, $a_1$ is algebraic over $F$ and so $L$ is a finite field extension of $F.$

Assume $L=F[a_1,...,a_k]$ is a finite field extension of $F$ for all $k\le n$.

Then let $L=F[a_1,...,a_n][a_{n+1}]$. Since $L$ is a field, by the same reasoning as the basecase, $L$ is algebraic over $F[a_1,...,a_n]$. However, by the inductive hypothesis, $F[a_1,...,a_n]$ is a finite field extension of $F$ and so $$[L:F]=[L:F[a_1,...,a_n]][F[a_1,...,a_n]:F]<\infty.$$
\end{proof}
\end{claim}

Thus, by the claim, $R/M_{a_i}$ is a finite field extension of $\mathbb{C}$ and so namely, it is finite dimensional over $\mathbb{C}$.

Then, $R/M_{a_i}^m$ is also finite dimensional since $M_{a_i}^m\subset M_{a_i}$ so we can inject $R/M_{a_i}^m\hookrightarrow R/M_{a_i}$ which is finite dimensional, so $R/M_{a_i}^m$ is finite dimensional, and so $R/\sqrt{I}^m$ is finite dimensional since it is a product of finite dimensional algebras.

Finally, $$R/I\cong (R/\sqrt{I}^m)/(I/\sqrt{I}^m)$$ is a quotient of a finite dimensional algebra, and so $R/I$ is a finite dimensional $\mathbb{C}$-algebra.
\end{solution}
\newpage


\begin{problem} $\,$
Let $R$ be a commutative ring with $1$, and $M$ a noetherian $R$-module. For $N$ a noetherian $R$ module show that $M\otimes_R N$ is a noetherian $R$-module. When $N$ is an artinian $R$ module show that $M\otimes_RN$ is an artinian $R$ module.
\end{problem}


\begin{solution}$\,$
Since $M$ is noetherian, $M$ is finitely generated. Namely, $M=m_1R+\cdots+m_nR$ for some $m_1,...,m_n\in M$.

Thus, we can define a module isomorphism \begin{align*}
    f:M&\to R^n\\
    m_i&\mapsto (0,0,...,0,1,0,...,0)\qquad i\thh\text{-position}
\end{align*}

Therefore, we have a short exact sequence
\begin{center}
    \begin{tikzcd}
    0 \arrow[r] & R^n\arrow[r] & M\arrow[r] & 0
    \end{tikzcd}
\end{center} and since tensor products are right-regular,

\begin{center}
    \begin{tikzcd}
     R^n\otimes_R N\arrow[r] & M\otimes_R N\arrow[r] & 0
    \end{tikzcd}
\end{center}

and so $$R^n\otimes_RN\cong N^n\text{ (direct sum) }\cong M\otimes_RN.$$

Since $N$ is noetherian, a direct sum of $n$ copies of $N$ is noetherian and so $M\otimes_RN$ is noetherian.

Similarly, if $N$ is artinian, a direct sum of $n$ copies of $N$ is artinian and so $M\otimes_R N$ is artinian.

\end{solution}
\newpage



\begin{problem} $\,$
For $n\ge 5$ show that the symmetric group $S_n$ cannot have a subgroup $H$ with $3\le [S_n:H]<n$ ($[S_n:H]$ is the index of $H$ in $S_n$).
\end{problem}


\begin{solution}$\,$
Note that $A_n$ is always a subgroup of $S_n$ of index $2.$

Let $H$ be a subgroup of $S_n$ such that $2<[S_n:H]=k<n$. Then let $S_n$ act on $X=S_n/H$ the set of left cosets (not necessarily a group) by left multiplication.

This defines a map \begin{align*}
    \varphi:S_n&\to S_{|X|}=S_k\\
    a&\mapsto \sigma_a
\end{align*}

where $\sigma_a:X\to X$ is defined by $\sigma_a(bH)=abH$.

Now, if $a\in\ker(\varphi)$ then $abH=bH$ for all $b$. Then $abh=bh'$ for $h,h'\in H$ so $a=bh'h^{-1}b^{-1}\in bHb^{-1}.$

Thus, $$a\in \bigcap_{b\in S_n}bHb^{-1}\subset H.$$

Therefore, $\ker(\varphi)\subset H$. However, the only normal subgroups of $S_n$ for $n\ge 5$ are the trivial one, $S_n$ itself, or $S_n.$

Since $|H|<|A_n|$, $|\ker(\varphi)|\not=n!/2,n!$, so the kernel is trivial.

However, then $S_n$ has an isomorphic copy inside $S_k$, which is not possible since $k<n$ so $k!<n!.$

Thus, $H$ cannot exist.
\end{solution}
\newpage




\begin{problem} $\,$
Let $R$ be the group algebra $\mathbb{C}[S_3]$. How many nonisomorphic, irreducible, left modules does $R$ have and why?
\end{problem}


\begin{solution}$\,$
First, by classification theorems for group algebras, $\mathbb{C}[S_3]$ is semi-simple and has $3$ simple components because $S_3$ has $3$ conjugacy classes.

Furthermore, $|S_3|=6=n_1^2+n_2^2+n_3^2$ by Mashke's theorem where $n_i$ correspond to the simple components $M_{n_i}(\mathbb{C})$ comprising $\mathbb{C}[S_3]$.

Therefore, if $n_3\le 2$, and since $S_3$ is not abelian, not all the $n_i$ are $1$. Thus, if $n_3=2$, then $6=n_1^2+n_2^2+4$ so $n_1=n_2=1$.

Therefore, $$\mathbb{C}[S_3]\cong\mathbb{C}^2\oplus M_2(\mathbb{C}).$$

Since the number of non-isomorphic simple left $R$-module is exactly the number of simple components in the decomposition, $R$ has $3$ non-isomorphic simple left $R$-modules.

\begin{mybox}
***Although it was not asked, the simple left $\mathbb{C}[S_3]$-modules are exactly $\mathbb{C}[S_3]/I$ for some maximal left ideal $I.$

Since maximal ideals of $\mathbb{C}[S_3]$ are \begin{align*}
    I_1&=(0)\oplus\mathbb{C}\oplus M_2(\mathbb{C})\\
    I_2&=\mathbb{C}\oplus(0)\oplus M_2(\mathbb{C})\\
    I_3&=\mathbb{C}\oplus\mathbb{C}\oplus\mathbb{C}^2
\end{align*} since the maximal left ideal of $\mathbb{C}$ is $(0)$ and the maximal left ideals of $M_2(\mathbb{C})$ are the column spaces, namely, $\mathbb{C}^2$.

Therefore, the non-isomorphic simple left $\mathbb{C}[S_3]$-modules are $$M_1\cong\mathbb{C}\text{ (first component),}\qquad M_2\cong\mathbb{C}\text{ (second component),}\qquad M_3\cong\mathbb{C}^2.$$
\end{mybox}

\end{solution}
\newpage



\begin{problem} $\,$
Let each $g_1(x),g_2(x),...,g_n(x)\in\mathbb{Q}[x]$ be irreducible of degree four and let $L$ be a splitting field over $\mathbb{Q}$ for $\{g_1(x),...,g_n(x)\}.$ Show there is an extension field $M$ of $L$ that is a radical extension of $\mathbb{Q}.$
\end{problem}


\begin{solution}$\,$
Since the $g_i$ are irreducible over $\mathbb{Q},$ they are separable.

Let $L_i$ be the splitting field of $g_i$ over $\mathbb{Q}$.

Then since $L_i$ is the splitting field of a separable polynomial, it is a Galois extension of $\mathbb{Q}$. Since $G_i=\gal(L_i/\mathbb{Q})$ is a subgroup of $S_4$ (because $|G|=[L_i/\mathbb{Q}]\le 4!$ so $G$ embeds into $S_4$) which is solvable, and since subgroups of solvable groups are solvable, $G_i$ is solvable.

Thus, $g_i(x)$ is solvable by radicals and $L_i$ is a radical extension.

Therefore, we obtain a chain, $$\mathbb{Q}\subset L_1\subset L_1L_2\subset\cdots\subset L_1L_2\dots L_n=M$$ where each product of the $L_i$ is radical over $\mathbb{Q}$ and so $M$ is certainly a radical extension.

Therefore, $L\subset L_1\cdots L_n=M$ is contained in a radical extension of $\mathbb{Q}.$
\end{solution}



\end{document}
