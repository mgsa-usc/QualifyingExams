\documentclass[12pt]{AlgebraQual}
\usepackage{preamble}

\name{Kayla Orlinsky}
\course{Algebra Exam}
\term{Fall 2014}
\hwnum{Fall 2014}

\begin{document}

\begin{problem} $\,$
Let $G$ be a group of order $56$ having at least $7$ elements of order $7$. Let $S$ be a Sylow $2$-subgroup of $G.$
\begin{enumerate}[label=(\alph*)]
    \item Prove that $S$ is normal in $G$ and $S=C_G(S).$
    \item Describe the possible structures of $G$ up to isomorphism. (Hint: How does an element of order $7$ act on the elements of $S.)$
\end{enumerate}
\end{problem}


\begin{solution}$\,$
\begin{enumerate}[label=(\alph*)]
    \item Since by Sylow $n_7|8$ and $n_7\equiv 1\mod 7$, $n_7=1,8$. Because $G$ has at least $7$ elements of order $7$, $n_7\not=1$ so $n_7=8.$

    Thus, because Sylow $7$-subgroups are cyclic in $G$ and they are conjugates, $G$ actually has $6\cdot8=48$ elements of order $7.$

    Since $56-48=8$, $G$ can have only $7$ elements of even order.

    Thus, $G$ has one Sylow $2$-subgroup, $S.$

    Now, by assumption $G$ is non-abelian (its Sylow $7$-subgroup is not normal). Thus, $C_G(S)\not=G.$

    Now, because $S$ is normal, $C_G(S)$ will also be normal in $G.$ if $a\in G$, $x\in C_G(S)$, $s\in S$, then \begin{align*}
        axa^{-1}s(axa^{-1})^{-1}&=axa^{-1}sax^{-1}a^{-1}\\
        &=axs_0x^{-1}a^{-1}\qquad a^{-1}sa=s_0,\\
        &=as_0a^{-1}\\
        &=s\qquad s=as_0a^{-1}
    \end{align*}

    Thus, $axa^{-1}\in C_G(S).$

    Therefore, if $|C_G(S)|=56/2=28$, then $C_G(S)$ will be normal in $G$.

    However, in $C_G(S),$ $n_7\equiv 1\mod7$, $n_7|4$ so $C_G(S)$ has a normal Sylow $7$-subgroup. However, normal Sylow subgroups of normal subgroups are normal in the whole group (see \textbf{Fall 2011: Problem 5 Claim 3}). This contradicts that $G$ has $8$ Sylow $7$-subgroups.

    If $|C_G(S)|=14$, then again $C_G(S)$ has a normal Sylow $7$-subgroup and so again, this would force $G$ to have a normal Sylow $7$-subgroup.

    Finally, $|C_G(S)|\not=7$ because again, $C_G(S)$ is normal in $G.$

    Therefore, $|C_G(S)|$ has even order so $C_G(S)\subset S.$

    Thus, $C_G(S)=Z(S)$ and so it cannot be trivial since $S$ is a $p$-group and so has non-trivial center by the class equation.

    Let $s\in S$ not be in $C_G(S).$ Then there exists $a,b\in S$ with $a\not=b$ and $sas^{-1}=b.$ If $a\in C_G(S)$ then $$b=sas^{-1}=sas^{-1}a^{-1}a=ss^{-1}a=a$$ which is a contradiction. Similarly, $b\not\in C_G(S)$. Note that clealry $s\not=a$ and $s\not=b.$

    However, this implies that there are an odd number of elements not in $C_G(S)$ which is impossible since $C_G(S)\subset S$ and so has even order.

    To see this, note that so far we have found $3$ total elements not in $C_G(S)$ and since $C_G(S)$ has even order, there must exist at least one more $c\in S$ such that $c\notin C_G(S)$ and $c$ is distinct from $a$ and $b.$

    However, $scs^{-1}$ cannot be $a$ or $b$, else we would get that $c$ is one of the $a$ or $b.$ Namely, if $scs^{-1}=b$ then $scs^{-1}=sas^{-1}$ so $c=a.$

    Therefore, there must exist some $d$ such that $scs^{-1}=d$, where $d\not=a,b,c,s$. Furthermore, by the same reasoning as before, $d\notin C_G(S)$. Else $$c=s^{-1}ds=s^{-1}dsd^{-1}d=s^{-1}sd=d$$ which contradicts that $c\notin C_G(S).$

    However, we now have $5$ distinct elements not in $C_G(S).$ Repeating we obtain a contradiction, that $C_G(S)$ is trivial.

    Finally, $C_G(S)=Z(S)=S$ and so $S$ is abelian.

    \item Since $S$ is abelian, $$S\cong\mathbb{Z}_8,\mathbb{Z}_4\times\mathbb{Z}_2,\mathbb{Z}_2^3.$$

    This will give three possible structures for $G$.

    Note that by the recognizing semi-direct products theorem, $G\cong P_7S$ where $P_7$ is a Sylow $7$-subgroup.

    \boxed{\varphi:P_7\to\aut(\mathbb{Z}_8)} Let $$\varphi:P_7\to\aut(\mathbb{Z}_8)\cong\mathbb{Z}_8^\times\cong\mathbb{Z}_{8-2}\cong\mathbb{Z}_6$$

    Since there are no elements of order $7$ in $\aut(\mathbb{Z}_8)$, only the trivial homomorphism is well defined. Since this would define an abelian structure on $G$, this cannot lead to a possible structure for $G.$

    \boxed{\varphi:P_7\to\aut(\mathbb{Z}_4\times\mathbb{Z}_2)} Let $\varphi:P_7\to\aut(\mathbb{Z}_4\times\mathbb{Z}_2)$.

    First, if $\sigma:\mathbb{Z}_4\times\mathbb{Z}_2\to\mathbb{Z}_4\times\mathbb{Z}_2$ is an automorphism, one can check that to ensure the kernel of $\sigma$ is trivial, we only have the following choices for $\sigma:$

    $$\begin{matrix}
    \sigma(1,0)=(1,0) &\text{and}& \sigma(0,1)=(0,1)\\
    \sigma(1,0)=(1,1) &\text{and}& \sigma(0,1)=(0,1)\\
    \sigma(1,0)=(3,0) &\text{and}& \sigma(0,1)=(0,1)\\
    \sigma(1,0)=(3,1) &\text{and}& \sigma(0,1)=(0,1)\\
    \sigma(1,0)=(3,1) &\text{and}& \sigma(0,1)=(1,0)\\
    \sigma(1,0)=(3,0) &\text{and}& \sigma(0,1)=(1,0)\\
    \sigma(1,0)=(1,1) &\text{and}& \sigma(0,1)=(1,0)\\
    \sigma(1,0)=(1,0) &\text{and}& \sigma(0,1)=(1,0)
    \end{matrix}$$

    Namely, $\aut(\mathbb{Z}_4\times\mathbb{Z}_2)$ has order $8$ and so again, there are no elements of order $7$ for $\varphi$ to map.

    \boxed{\varphi:P_7\to\varphi:P_7\to\aut(\mathbb{Z}_2^3)} $\varphi:P_7\to\aut(\mathbb{Z}_2^3)\cong GL_3(\mathbb{F}_2)$. Since $$|GL_3(\mathbb{F}_2)|=(2^3-1)(2^3-2)(2^3-2^2)=7\cdot6\cdot4=2^3\cdot 3\cdot 7$$

    Therefore, there exists a non-trivial homomorphism $\varphi$ under which we can define the semi-direct product structure for $G.$

    Now, any $\varphi$ must map $P_7$ to a Sylow $7$-subgroup of $\aut(\mathbb{Z}_2^3)$. Since Sylow subgroups are conjugates, any two different homomorphisms $\varphi_1,\varphi_2$ will have conjugate images. Namely, they will generate isomorphic semi-direct products.

    Thus, there is only one possible group $G$ with non-normal Sylow $7$-subgroup.

    To actually write down a presentation for $G$, we must find an element of order $7$ in $\aut(\mathbb{Z}_2^3)$.

    Let $S\cong\langle a,b,c\rangle$ and $P_7\cong\langle d\rangle.$

    After some effort, one obtains that the automorphism of $S$ defined by $a\mapsto b$, $b\mapsto bc$, $c\mapsto a$ defined an automorphism of order $7.$

    Therefore, we get the following multiplication for $$G\cong\mathbb{Z}_2^3\times_\varphi\mathbb{Z}_7$$, $$G\cong\langle a,b,c,d\,|\, a^2=b^2=c^2=d^7=1,dad^{-1}=b,dbd^{-1}=bc,dcd^{-1}=a\rangle.$$

    This is the only possible structure for $G.$
\end{enumerate}
\end{solution}
\newpage


\begin{problem} $\,$
Show that a finite ring with no nonzero nilpotent elements is commutative.
\end{problem}


\begin{solution}$\,$
Let $R$ be a finite ring with no nonzero nilpotent elements.

Let $r\in J(R)$. Then $1-r$ is invertible in $R$ because $J(R)$ is quasi-regular.

Now, because $R$ is artinian (it is finite), we have a decreasing chain $$(r)\supset(r^2)\supset\cdots$$ which must terminate after a finite number of steps. Namely, $(r^n)=(r^m)$ for all $n\ge m$.

However, $r^m=ar^{m+1}$ for some $a\in R$. Thus, $$r^m(1-ar)=0.$$ Since $ar\in J(R)$, $1-ar$ has an inverse so $r^m=0$. Namely, $r$ is nilpotent.

Since $r\in R,$ it must be that $r=0.$

Thus, $J(R)=0.$

Therfore, by Artin Wedderburn, $R\cong M_{n_1}(D_1)\oplus\cdots\oplus M_{n_k}(D_k)$ for some division rings $D_i.$ Note that because $R$ is finite, $D_i$ must be finite, and since finite division rings are fields, $D_i\cong\mathbb{F}_{q_i}$ a field of $q_i$ elements.

However, since $R$ has no nonzero nilpotent elements and $$A=\begin{bmatrix}
0 & 0 & \cdots & 0 & 1\\
0 & 0 &\cdots & 0 & 0\\
& \vdots & \ddots & \vdots & \\
0 & 0 & \cdots & 0 & 0\\
0 & 0 & \cdots & 0 & 0
\end{bmatrix}$$ is a nilpotent matrix over any field (any division ring really), $n_i=1$ for all $i.$

Thus, $R\cong \mathbb{F}_{q_1}\oplus\cdots\oplus \mathbb{F}_{q_k}$ and so it is commutative.
\end{solution}
\newpage



\begin{problem} $\,$
If $R=M_n(\mathbb{Z})$, and $A$ is an additive subgroup of $R$, show that as additive subgroups $[R:A]$ is finite if and only if $R\otimes_\mathbb{Z}\mathbb{Q}=A\otimes_\mathbb{Z}\mathbb{Q}.$
\end{problem}


\begin{solution}$\,$

\boxed{\implies} Assume $[R:A]=m<\infty$. Then for all $X\in R/A$, with $X\not=0$ (in other words for $X\notin A$), $mX=0\in R/A$ since $R/A$ is a finite group, (in other words, $mX\in A$).

Therefore for all $X\in R$ with $X\notin A$, and all $q\in\mathbb{Q}$, $X\otimes mq=mX\otimes q\in A\otimes_\mathbb{Z}\mathbb{Q}$ and clearly if $X\in A$, then $X\otimes q\in A\otimes_\mathbb{Z}\mathbb{Q}$ so $R\otimes_\mathbb{Z}\mathbb{Q}=A\otimes_\mathbb{Z}\mathbb{Q}$.

\boxed{\impliedby} Note that $R\cong\mathbb{Z}^{n^2}$ and so $R$ is a finitely generated free module over a PID ($\mathbb{Z}$ is a PID).

Therefore, $A$ is also a free finitely generated $\mathbb{Z}$-module so $A\cong\mathbb{Z}^m$ for some $m$ since submodules of free module over PIDs are also free and additive subgroups are submodules.

Therefore, if $R\otimes_\mathbb{Z}\mathbb{Q}=A\otimes_\mathbb{Z}\mathbb{Q}$ then $\mathbb{Z}^{n^2}\otimes_\mathbb{Z}\mathbb{Q}=\mathbb{Z}^m\otimes_\mathbb{Z}\mathbb{Q}$, so of course $n^2=m$.

Therefore, $[R:A]<\infty$ since if $[R:A]=\infty$ then there exists $i,j$ so there are an infinite number of possible values in the $ij\thh$ entry of every matrix of $R/A$. Namely, $X\in R/A$ can have any infinite number of possible values in its $ij\thh$ entry.

However, then $R/A$ has an isomorphic copy of $\mathbb{Z}$ in it, and so namely, it has rank $\ge1$. However, this is not possible since rank of $R/A$ is rank$(R)-$rank$(A)=n^2-n^2=0$.

Thus, $[R:A]<\infty.$
\end{solution}
\newpage


\begin{problem} $\,$
Let $R$ be a commutative ring with $1$, $n$ a positive integer and $A_1,...,A_k\in M_n(R)$. Show that there is a noetherian subring $S$ of $R$ containing $1$ with all $A_i\in M_n(S).$
\end{problem}


\begin{solution}$\,$
First, we note that since $\varphi:\mathbb{Z}\to R$ defined by $\varphi(1)=1_R$ has kernel which is an ideal of $\mathbb{Z}$, namely an additive subgroup, so either $\mathbb{Z}$ or $\mathbb{Z}_n$ has an isomorphic copy in $R$.

Therefore, we can consider $S\cong \mathbb{Z}[A_1,...,A_k]$, the subring generated by the entries of the $A_i$. Then since $S$ is a finitely generated algebra over a PID, it is a noetherian subgring of $R$ and $M_n(S)$ contains all the $A_i.$

\end{solution}
\newpage



\begin{problem} $\,$
Let $R=\mathbb{C}[x,y]$. Show that there exists a positive integer $m$ such that $((x+y)(x^2+y^4-2))^m$ is in the ideal $(x^3+y^2,y^3+xy)$.
\end{problem}


\begin{solution}$\,$
This question is from \textbf{Fall 2012: Problem 3}, thus we provide the same proof here that we did there.

By Nullstellensatz, if $(x+y)(x^2+y^4-2)$ satisfies every point $(a,b)\in V(x^3+y^2,y^3+xy)$, then $(x+y)(x^2+y^4-2)\in\sqrt{I}$ and there exists an integer $m$ such that $((x+y)(x^2+y^4-2))^m\in(x^3+y^2,y^3+xy)$.

Thus, we compute $V(x^3+y^2,y^3+xy).$

If $x^3+y^2=0$ and $y^3+xy=0$ simultaneously, then $x^3y+y^3-y^3-xy=0$ so $x^3y-xy=0$ so $xy(x^2-1)=0$. Thus, we have $x=0,1,-1$ or $y=0$. This gives the following points $(0,0),(1,i),(1,-i),(-1,1),(-1,-1)\in V(x^3+y^2,y^3+xy)$.

Since $(x+y)(x^2+y^4-2)$ $(0,0),(-1,1)$ immediately satisfy $(x+y)$, we need only check $(x^2+y^4-2)$.

Since $1^2+(i)^4-2=1+1-2=0$, $1^2+(-i)^4-2=0$, $(-1)^2+(-1)^4-2=2-2=0$, we have by Nullstellensatz that $(x+y)(x^2+y^4-2)$ is satisfied by every point $(a,b)\in V(x^3+y^2,y^3+xy)$, so $(x+y)(x^2+y^4-2)\in\sqrt{I}$ and there exists an integer $m$ such that $((x+y)(x^2+y^4-2))^m\in(x^3+y^2,y^3+xy)$.
\end{solution}
\newpage




\begin{problem} $\,$
Let $f(x)\in\mathbb{Q}[x]$ be an irreducible polynomial of degree $n\ge5$. Let $L$ be the splitting field of $f$ and let $\alpha$ be a zero of $f$. Given that $[L:\mathbb{Q}]=n!$, prove that $\mathbb{Q}[\alpha^4]=\mathbb{Q}[\alpha].$
\end{problem}


\begin{solution}$\,$
Since $f$ is irreducible and $\mathbb{Q}$ is characteristic $0$, $f$ is separable.

Thus, $L/\mathbb{Q}$ is Galois.

Now, since $G=\gal(L/\mathbb{Q})$ embeds into $S_n$ (since $f$ has degree $n$), $G\cong S_n$ for $n\ge 5.$

Now, by \textbf{Spring 2014: Problem 5}, for $n\ge 5$, $S_n$ has no subgroups of index $2<[S_n:H]<n$.

Now, we simply note that by the fundamental theorem of Galois Theory, subfields of $L$ over $\mathbb{Q}$ correspond exactly to subgroups of $G=S_n.$

Specifically, subgroups $H$ of $S_n$ correspond to subfields $\mathbb{Q}\subset K\subset L$ satisfying that $|H|=[L:K]$ and $[S_n:H]=[K:\mathbb{Q}].$

Now, $L/\mathbb{Q}(\alpha^4)$ corresponds to a subgroup $H$ of $S_n$ such that $$[S_n:H]=[\mathbb{Q}(\alpha^4):\mathbb{Q}].$$

Thus, $[\mathbb{Q}(\alpha^4):\mathbb{Q}]\ge n$ or $[\mathbb{Q}(\alpha^4):\mathbb{Q}]\le 2$.

Now, because $\alpha$ has minimal polynomial $f(x)$ over $\mathbb{Q}$, thus we have that $$[\mathbb{Q}(\alpha):\mathbb{Q}(\alpha^4)]=\frac{[\mathbb{Q}(\alpha):\mathbb{Q}]}{[\mathbb{Q}(\alpha^4):\mathbb{Q}]}=\frac{n}{[\mathbb{Q}(\alpha^4):\mathbb{Q}]}.$$

Therefore, $[\mathbb{Q}(\alpha^4):\mathbb{Q}]\le n$.

Next, $[\mathbb{Q}(\alpha^4):\mathbb{Q}]\not=1$ since then $\alpha$ would have a minimal polynomial of degree $4$ over $\mathbb{Q}$, but the minimal polynomial of $\alpha$ has degree $5.$

Now, if $[\mathbb{Q}(\alpha^4):\mathbb{Q}]=2$ then $H=A_n$ and $\alpha^4$ has minimal polynomial $x^2+ax+b$ for $a,b\in\mathbb{Q}$. However, then $\alpha^4=\frac{-a\pm\sqrt{a^2-4b}}{2a}$ and so the minimal polynomial of $\alpha$ over $\mathbb{Q}$, which is $f(x)$, is solvable by radicals, which is not possible since $S_n$ is not solvable for $n\ge 5.$

Thus, $[\mathbb{Q}(\alpha^4):\mathbb{Q}]=n$ and so $[\mathbb{Q}(\alpha):\mathbb{Q}(\alpha^4)]=1.$ Namely, $\mathbb{Q}(\alpha)=\mathbb{Q}(\alpha^4).$
\end{solution}



\end{document}
