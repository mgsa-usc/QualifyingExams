\documentclass[12pt]{AlgebraQual}
\usepackage{preamble}

\name{Kayla Orlinsky}
\course{Algebra Exam}
\term{Spring 2012}
\hwnum{Spring 2012}

\begin{document}

\begin{problem} $\,$
Let $I$ be an ideal of $R=\mathbb{C}[x_1,...,x_n]$. Show that $\dim_\mathbb{C}R/I$ is finite if and only if $I$ is contained in only finitely many maximal ideals of $R.$
\end{problem}


\begin{solution}$\,$
\boxed{\implies} Assume $R/I$ is a finite dimensional algebra over $\mathbb{C}$. Then $R/I$ is artinian, since proper ideals are sub-algebras of strictly smaller degree.

Thus, if $S=\{M_1M_2\cdots M_k\,|\, M_i$ maximal ideal of $R/I\}$ is the set of finite products of maximal ideals in $R/I$. $S$ is nonempty so $S$ contains a minimal element in $R/I$, $M_1M_2\cdots M_k$. Let $N$ be some other maximal ideal of $R/I$. Then $NM_1\cdots M_k\subset M_1\cdots M_k$ so $$NM_1\cdots M_k= M_1\cdots M_k\subset N.$$ However, $N$ is maximal and so prime, thus $M_i\subset N$ for some $i$. However, by maximality, $M_i=N$.

Thus, these are the only maximal ideals of $R/I$. By the correspondence theorem, there is a $\oto$ correspondence between maximal ideals of $R$ containing $I$ and maximal ideals of $R/I$.

Since $R/I$ has only finitely many maximal ideals, there are only finitely many maximal ideals of $R$ containing $I.$

\boxed{\impliedby} Assume $I$ is contained in only finitely many maximal ideals of $R$. Note that $R$ is Noetherian by the Hilbert Basis theorem, and so all ideals are finitely generated.

Since $I$ is contained in only finitely many maximal ideals, $V(I)$ contains only finitely many points. Namely, by Nullstellensatza, $$\sqrt{I}\bigcap_{a\in\mathbb{C}^n}M_a\qquad \text{is a finite intersection}$$ where $M_a$ is the maximal ideal (by Nullstellensatz) of the form $(x_1-a_1,...,x_n-a_n)$ for $a=(a_1,...,a_n)$.

Thus, $\sqrt{I}=\bigcap_{i=1}^nM_{a_i}$ where $I\subset M_{a_i}$ for all $i$.

Since $\sqrt{I}$ is finitely generated, $\sqrt{I}=(f_1,f_2,...,f_k)$, and for each $f_i$ there exists $m_i$ so $f_i^{m_i}\in I$.

Let $m=\lcm\{m_i\}.$ Then $$I\subset\sqrt{I}=\bigcap_{i=1}^nM_{a_i}$$

and $$I\supset (\sqrt{I})^m=\left(\bigcap_{i=1}^nM_{a_i}\right)^m=\bigcap_{i=1}^nM_{a_i}^m.$$

Thus, the Chinese remainder theorem, since $M_{a_i}$ are pairwise coprime, $M_{a_i}^m$ are all pairwise coprime (since if $M_{a_i}^m+M_{a_j}^m$ is contained in some maximal ideal $M$, then $M$ contains both $M_{a_i}^m$ and $M_{a_j}^m$ and so must contain both $M_{a_i}$ and $M_{a_j}$ which forces $M=R$).

Therefore, $$R/\sqrt{I}^m\cong R/\cap_iM_{a_i}^m\cong R/\prod_iM_{a_i}^m\cong \prod R/M_{a_i}^m.$$

\begin{claim} If $F$ is a field and if $L=F[x_1,...,x_n]/M$ is a field, then $L$ is a finite field extension of $F$.
\begin{proof} We proceed by induction on $n.$

Basecase: let $L=F[a_1]$ be a field. Then for $f(a_1)\in L$ there exists $g(a_1)\in L$ such that $f(a_1)g(a_1)=1\in L$ and so $a_1$ satisfies $h(x)=f(x)g(x)-1$. Namely, $a_1$ is algebraic over $F$ and so $L$ is a finite field extension of $F.$

Assume $L=F[a_1,...,a_k]$ is a finite field extension of $F$ for all $k\le n$.

Then let $L=F[a_1,...,a_n][a_{n+1}]$. Since $L$ is a field, by the same reasoning as the basecase, $L$ is algebraic over $F[a_1,...,a_n]$. However, by the inductive hypothesis, $F[a_1,...,a_n]$ is a finite field extension of $F$ and so $$[L:F]=[L:F[a_1,...,a_n]][F[a_1,...,a_n]:F]<\infty.$$
\end{proof}
\end{claim}

Thus, by the claim, $R/M_{a_i}$ is a finite field extension of $\mathbb{C}$ and so namely, it is finite dimensional over $\mathbb{C}$.

Then, $R/M_{a_i}^m$ is also finite dimensional since $M_{a_i}^m\subset M_{a_i}$ so we can inject $R/M_{a_i}^m\hookrightarrow R/M_{a_i}$ which is finite dimensional, so $R/M_{a_i}^m$ is finite dimensional, and so $R/\sqrt{I}^m$ is finite dimensional since it is a product of finite dimensional algebras.

Finally, $$R/I\cong (R/\sqrt{I}^m)/(I/\sqrt{I}^m)$$ is a quotient of a finite dimensional algebra, and so $R/I$ is a finite dimensional $\mathbb{C}$-algebra.
\end{solution}
\newpage


\begin{problem} $\,$.
If $G$ is a group with $|G|=7^2\cdot 11^2\cdot 19$, show that $G$ must be abelian and describe the possible structures of $G.$
\end{problem}


\begin{solution}$\,$
By Sylow, $n_7\equiv 1\mod 7$ and $n_7|11^2\cdot 19$. Since $11^2\equiv 2\mod 7$, $11\cdot 19\equiv 6 \mod 7$, $11^2\cdot 19\equiv 3\mod 7$, $n_7=1$.

Thus, $G$ has a normal Sylow $7$-subgroup $P_7$.

Thus, $H=P_7P_{11}$ is a subgroup of $G$ where $P_{11}$ is a Sylow $11$-subgroup of $G.$

Now, let $X=G/H$ the set of let cosets of $H$. Then $|X|=19$.

Let $G$ act on $X$ by left multplication. Then this defines a homomorphism \begin{align*}
    \varphi:G&\to S_{|X|}=S_{19}\\
    a&\mapsto \sigma_a:X\to X\qquad \sigma_a(gH)=agH
\end{align*}

Note that $\varphi$ is not an embedding since $11^2$ does not divide $19!$. Therefore, $11$ divides $|\ker(\varphi)|$ and so there exists an element $x\in\ker(\varphi)$ of order $11$.

Now, if $\varphi(a)=\id$, then $agH=gH$ for all $g\in G$ so $a\in gHg^{-1}$ for all $g\in G.$

Namely, $\displaystyle \ker(\varphi)=\bigcap_{g\in G}gHg^{-1}$. Note also that $P_7$ is normal in $G$ and so because $gP_7g^{-1}=P_7\subset gHg^{-1}$ for all $g.$

Therefore, $$|\ker(\varphi)|=\left|\bigcap_{g\in G}gHg^{-1}\right|\ge 7^2\cdot 11.$$

Namely, $|\varphi(G)|=11\cdot 19,$ or $19$, namely $\varphi(G)$ is abelian by Sylow.

However, $G$ acts transitively on $X$, since for $gH,aH\in X$, $$gH=ga^{-1}aH=ga^{-1}(aH)=g_0(aH)\qquad g_0=ga^{-1}.$$ Therefore, $\varphi(G)$, which is necessarily abelian based on its order, is a transitive subgroup of $S_{19}$, and so it has order $19$. If the order were larger, then there would exist $x=\varphi(a)(1)=\varphi(b)(1)$ and $\varphi(a)(y)\not=\varphi(b)(y)$. Thus, by transitivity, there is $\varphi(c)(x)=y$, then $$\varphi(c)\varphi(a)\varphi(c)\varphi(b)(1)=\varphi(c)\varphi(a)\varphi(c)(x)=\varphi(c)\varphi(a)(y)$$

and

$$\varphi(c)\varphi(b)\varphi(c)\varphi(a)(1)=\varphi(c)\varphi(b)\varphi(c)(x)=\varphi(c)\varphi(b)(y)$$

which cannot be equal to $\varphi(b)(y)\not=\varphi(a)(y)$ which contradicts that $\varphi(G)$ is abelian.

Thus, $|\varphi(G)|=19$ so $|\ker(\varphi)|=|H|$ so $\ker(\varphi)=H$ and so $H$ is normal in $G$. Therefore, because $H$ has a normal Sylow $11$-subgroup (since $n_{11}|49$ and $n_{11}\equiv 1\mod 11$ in $H$, $n_{11}=1$) and since normal Sylow subgroups of normal subgroups are normal in the whole group (see \textbf{Fall 2011: Problem 5 Claim 3}), $G$ has a normal Sylow $11$-subgroup.

Now, by the recognizing of semi-direct products theorem, if $G$ is not abelian then it is a semi-direct product of its Sylow subgroups.

However, $\aut(P_7P_{11})\cong\aut(P_7)\times\aut(P_{11})$ since $11$ and $7$ are coprime. Thus, depending on whether $P_7\cong\mathbb{Z}_7\times\mathbb{Z}_7$ or $\mathbb{Z}_{49}$ we have that $$\aut(P_7)\cong\mathbb{Z}_{49-7}=\mathbb{Z}_{42}\qquad \aut(P_7)\cong GL_2(\mathbb{F}_7).$$ In either case, $|\aut(P_7)|=42$ or $(7^2-1)(7^2-7)=48\cdot 42$ and $19$ does not divide either of these.

Similarly, $\aut(P_{11})$ has order $11^2-11=110$ or $(11^2-1)(11^2-11)=120\cdot 110$, and again there are no elements of order $19$ to choose from.

Therefore, any homomorphism $\varphi:P_{19}\to\aut(P_7P_{11})$ will be trivial and so the only possible structure for $G$ is as an abelian group.

There are $4$ possible abelian structures for $G$.

\begin{center}
    \begin{framed}
    $$\mathbb{Z}_{7^2}\times\mathbb{Z}_{11^2}\times\mathbb{Z}_{19}$$

    $$\mathbb{Z}_7\times\mathbb{Z}_7\times\mathbb{Z}_{11^2}\times\mathbb{Z}_{19}$$

    $$\mathbb{Z}_{7^2}\times\mathbb{Z}_{11}\times \mathbb{Z}_{11}\times\mathbb{Z}_{19}$$

    $$\mathbb{Z}_7\times\mathbb{Z}_7\times\mathbb{Z}_{11}\times \mathbb{Z}_{11}\times\mathbb{Z}_{19}$$
    \end{framed}
\end{center}



\end{solution}
\newpage



\begin{problem} $\,$
Let $F$ be a finite field and $G$ a finite group with GCD$\{char F,|G|\}=1$. The group algebra $F[G]$ is an algebra over $F$ with $G$ as an $F$-basis, elements $\alpha=\sum_Ga_gg$ for $a_g\in G,$ and multiplication that extends $ag\cdot bh=ab\cdot gh$. Show that any $x\in F[G]$ that is not a zero left divisor (i.e. if $xy=0$ for $y\in F[G]$ then $y=0$) must be invertible in $F[G].$
\end{problem}


\begin{solution}$\,$
Let $x\in F[G]$ be not a zero left divisor. Then because $F[G]$ is a finite field and $G$ is a finite group, $F[G]$ is a finite dimensional $F$-algebra and so it is artinian (both left and right artinian) as an $F$-algebra.

Namely, we can construct a decreasing chain of left ideals $$(x)\supset (x^2)\supset\cdots$$ which must terminate after a finite number of steps. Namely, there exists $n$ so $(x^m)=(x^n)$ for all $m\ge n$.

Thus, $(x^{n+1})=(x^n)$ so there exists $y\in F[G]$ such that $x^n=yx^{n+1}$. Namely, $(1-yx)x^n=0$. Since $x$ is not a left-zero divisor, $(1-yx)x^{n-1}=0$, and recursivley we obtain that $(1-yx)=0$ so $yx=1$. Namely, $x$ has a left inverse in $G$.

Now, assume $x$ is a right zero-divisor. Then there exists $a\in F[G]$ so $xa=0$. Thus, $$(yx)a=1a=a\qquad y(xa)=y(0)=0\implies a=0.$$

Therefore, $x$ is not a right-zero divisor, and since $F[G]$ is right artinian we could preform the same reasoning as before on $(x),(x^2),...$ as right ideals to obtain that $x$ has a right inverse $z$.

Now, since $$(yx)z=1z=z\qquad y(xz)=y1=y$$ we have that $y=z$ and so $y$ is a $2$-sided inverse for $x.$
\end{solution}
\newpage



\begin{problem} $\,$
If $p(x)=x^8+2x^6+3x^4+2x^2+1\in\mathbb{Q}[x]$ and if $\mathbb{Q}\subset M\subset\mathbb{C}$ is a splitting field for $p(x)$ over $\mathbb{Q}$, argue that $\gal(M/\mathbb{Q})$ is solvable.
\end{problem}


\begin{solution}$\,$
Let $u=x^2$ and $h(u)=u^4+2u^3+3u^2+2u+1$. Then the zeros of of $p(x)$ are precisely the square roots of the zeros of $h(u)$. Namely, if $L$ is the splitting field of $h(u)$ over $\mathbb{Q}$ then $M/L$ will certainly be a radical extension so we need only check $L/\mathbb{Q}.$

Now, $h'(u)=4u^3+6u^3+6u+2$ and $h'(u)<0$ for all $u\le -1/3$ and $h'(u)>0$ for all $u\ge0$. However, for any $\alpha\in(-1/3,0)$, $$h(\alpha)=\alpha^4+2\alpha^3+3\alpha^2+2\alpha+1>-\frac{2}{9}-\frac{2}{3}+1=-\frac{8}{9}+1>0.$$

Therefore, $h$ has no real roots, and namely no rational roots. Thus, $h$ has a pair of complex conjugate roots, $\alpha,\overline{\alpha},\beta,\overline{\beta}$.

Therefore, $L$ is the splitting field of a separable polynomial over $\mathbb{Q}$ and so $L$ is Galois over $\mathbb{Q}$.

Since $[L:\mathbb{Q}]\le 4!$, and $\gal(L/\mathbb{Q})\hookrightarrow S_4$ which is solvable, we have that $\gal(L/\mathbb{Q})$ is solvable since subgroups of solvable groups are solvable.

Finally, if $G=\gal(M/\mathbb{Q})$, then by the fundamental theorem of Galois theory, $H=\gal(M/L)$ is normal in $G$ and $G/H=\gal(L/\mathbb{Q})$. Namely, since $H$ is normal in $G$ and is solvable (as we already discussed $M/L$ is a radical extensions) and $G/H$ is solvable, we have that $G$ is solvable.
\end{solution}
\newpage




\begin{problem} $\,$
Let $R$ be a commutative ring with $1$ and let $x_1,...,x_n\in R$ so that $x_1y_1+\cdots+x_ny_n=1$ for some $y_i\in R.$ Let $A=\{(r_1,...,r_n)\in R^n\,|\,x_1r_1+\cdots+x_nr_n=0\}.$ Show that $R^n\cong_RA\oplus R$, that $A$ has $n$ generators, and that when $R=F[x]$ for $F$ a field then $A_R$ is free of rank $n-1$.
\end{problem}


\begin{solution}$\,$
Let \begin{align*}
    \varphi:R^n&\to R\\
    (r_1,...,r_n)&\mapsto x_1r_1+\cdots+x_nr_n
\end{align*}

Then $\varphi$ is an $R$-module homomorphism and is surjective since $(y_1,...,y_n)\mapsto 1$.

Clearly $\ker(\varphi)=A$, thus we have a short exact sequence \begin{center}
    \begin{tikzcd}
    0 \arrow[r] & A\arrow[r] & R^n\arrow[r] & R\arrow[r] & 0
    \end{tikzcd}
\end{center}

and since $R$ is a projective $R$-module (both left and right because $R$ is commutative), this implies that $$R^n\cong R\oplus A.$$

Since $R=x_1R+x_2R+\cdots+x_nR$, $R^n=(x_1R+x_2R+\cdots+x_nR)^n$ and since $A$ is a submodule of $R^n$, $A$ has less than or equal to $n$ generators.

However, because $R^n/A\cong R$ is cyclic, $A$ has at least $n$ generators.

Thus, $A$ has exactly $n$ generators.

When $R=F[x]$, then $R$ is a PID and so because $A$ is finitely generated, by the structure theorem, $A$ is a direct sum of its free and torsion parts.

Namely, we have that $A\cong R^a\oplus T(A)$ where $T(A)$ is the torsion part of $A$.

Now, since $$R^n\cong R\oplus R^a\oplus T(A)\cong R^{a+1}\oplus T(A)$$ it must be that $a+1=n$ so $a=n-1$ and $T(A)=0$.

Thus, $A$ is a free $R$-module of rank $n-1.$
\end{solution}
\newpage




\begin{problem} $\,$
For $p$ a prime let $F_p$ be the field of $p$ elements and $K$ an extension field of $F_p$ of dimension $72$.
\begin{enumerate}[label=(\alph*)]
    \item Describe the possible structures of $\gal(K/F_p)$.
    \item If $g(x)\in F_p[x]$ is irreduicble of degree $72$, argue that $K$ is a splitting field of $g(x)$ over $F_p.$
    \item Which integers $d>0$ have irreducibles in $F_p[x]$ of degree $d$ that split in $K?$
\end{enumerate}
\end{problem}


\begin{solution}$\,$
\begin{enumerate}[label=(\alph*)]
    \item Since $K$ has $q=p^{72}$ elements, $K$ is the splitting field of $x^q-x$,which is separable over $F_p$. Thus, $K/F_p$ is Galois.

    Since Galois extensions over finite fields are always cyclic extensions, $\gal(K/F_p)\cong\mathbb{Z}_{72}$.

    \item If $g(x)\in F_p[x]$ is irreduicble of degree $72$, and $\alpha$ is a root of $g(x)$, then $[F_p(\alpha):F_p]=72=[K:F_p]$. Therefore, since finite fields of the same order are isomorphic, $K=F_p(\alpha)$ and so $\alpha\in K.$

    \item If $d|72$ then by the same reasoning, any polynomial of degree $d$ will split completely in $K$.
\end{enumerate}
\end{solution}
\newpage


\end{document}
