\documentclass[12pt]{Homework}

% Changed from \usepackage{prelude}
\usepackage{preamble}
\usepackage{amssymb}
\usepackage{enumitem}
\usepackage{mathrsfs}
\def\upint{\mathchoice%
    {\mkern13mu\overline{\vphantom{\intop}\mkern7mu}\mkern-20mu}%
    {\mkern7mu\overline{\vphantom{\intop}\mkern7mu}\mkern-14mu}%
    {\mkern7mu\overline{\vphantom{\intop}\mkern7mu}\mkern-14mu}%
    {\mkern7mu\overline{\vphantom{\intop}\mkern7mu}\mkern-14mu}%
  \int}
\def\lowint{\mkern3mu\underline{\vphantom{\intop}\mkern7mu}\mkern-10mu\int}
\usepackage[mathscr]{euscript}
\usepackage{comment}
\usepackage{bbding}
\renewcommand\qedsymbol{\Peace}
\newcommand\placeqed{\nobreak\enspace\Peace}
\usepackage{MnSymbol}
\usepackage{tikz}
\usepackage{halloweenmath}
\newcommand{\contradiction}{\null\hfill\large{$\mathghost$}\normalsize}

\name{Kayla Orlinsky}
\course{Real Analysis Exam}
\term{Fall 2015}
\hwnum{Fall 2015}

\begin{document}

\begin{problem} $\,$
Prove that for almost all $x\in[0,1]$, there are at most finitely many rational numbers with reduced form $p/q$ such that $q\ge 2$ and $|x-p/q|<1/(q\log q)^2$. (Hint: Consider intervals of length $2/(q\log q)^2$ centered at rational points $p/q$).
\end{problem}


\begin{solution}$\,$
Let $Q(x)=\{p/q\in\mathbb{Q}\,|\,(p,q)=1,q\ge 2, |x-p/q|<1/(q\log q)^2\}$.

Then $Q(x)$ counts the number of such rationals stated in the problem. 
Note that if $\int_\mathbb{R}Q(x)dx<\infty$, then $Q(x)$ must be finite a.e.

Now, since \begin{align*}
    \int_\mathbb{R}Q(x)dx&=\sum_{q\ge 2\in\mathbb{N}}\int_{B(1/(q\log q)^2,p/q)}Q(x)dx\\
    &=\sum_{q=2}^\infty(q-1)m(B(1/(q\log q)^2,p/q)) \tag{1}\\
    &=\sum_{q=2}^\infty\frac{2(q-1)}{q^2\log^2q}\\
    &=2\sum_{q=2}^\infty\frac{1}{q\log^2q}-2\sum_{q=2}^\infty\frac{1}{q^2\log^2q}
\end{align*}

\boxed{(1)} Because the Lebesgue measure is translation invariant, we may take $p<q$. Since the number of rationals in an interval is also invariant under translation, if $p>q$, then $p$ is an integer shift of some $p'<q$. 

The $(q-1)$ comes from the number of integers $1<p<q$.

Now, we check the convergence of both sums. 

\boxed{\sum_{q=2}^\infty\frac{1}{q^2\log^2q}} converges by limit comparison test. Specifically, $$\lim_{q\to\infty}\frac{1}{q^2\log^2q}/\frac{1}{q^2}=\lim_{q\to\infty}\frac{1}{\log^2q}=0\implies\text{ since }\sum_{q=2}^\infty\frac{1}{q^2}<\infty\text{ then }\sum_{q=2}^\infty\frac{1}{q^2\log^2q}<\infty.$$

\boxed{\sum_{q=2}^\infty\frac{1}{q\log^2q}} converges by integral test. 

$$\int_2^\infty\frac{1}{q\log^2q}dq=\int_{\log2}^\infty\frac{1}{u^2}du<\infty\qquad\text{ since }\log2>0$$
\[
\begin{matrix}
    u=\log q & \quad  q:[2,\infty]\\
    du=\frac{1}{q}dq & \quad u:[\log2,\infty]
\end{matrix}
\]

Thus, $\int Q(x)dx<\infty$ and so $Q(x)$ must be finite a.e.
\end{solution}
\newpage

\begin{problem} $\,$
Suppose that the real-valued function $f(x)$ is nondecreasing on the interval $[0,1]$. Prove that there exists a sequence of continuous functions $f_n(x)$ such that $f_n\to f$ pointwise on this interval.
\end{problem}


\begin{solution}$\,$
First, since $f$ is increasing, it has at most countably many discontinuities and so is measurable. Specifically, $f^{-1}((-\infty,a))=\{x\,|\,f(x)<a\}=[0,\inf f^{-1}(a))\backslash N$ for $N$ some null set. 

Thus, because $f$ is measurable, there exists a sequence of simple functions $\phi_1\le\phi_2\le\cdots\le f$ with $\phi_n\to f$ pointwise. 

Thus, it suffices to show that there exists a sequence of continuous functions converging to $\phi$ a simple function. Furthermore, since the $\phi_n$ are an increasing sequence converging to an increasing function, we may take our $\phi$ to also be increasing.

Then, let $\phi=\sum_{i=1}^ma_i\chi_{E_i}$ be the standard representation of $\phi$. Since $\phi$ is increasing, and by definition can have only a finite set as its range, $\phi$ can only have a finite number of discontinuities. Furthermore, $a_i\le a_{i+1}$ for all $1\le i\le m-1$. 

Now, to construct a continuous approximation to $\phi$, we simply use a trapezoid approximation. Let $x_1,x_2,...,x_{m-1}$ be the points where the jump discontinuities of $\phi$ occur. Now,
\[
    \begin{matrix}
    \text{If } \phi(x_i)=a_i & \qquad & \text{If } \phi(x_i)=a_{i+1}\\
    \text{Let } y_i=n(a_{i+1}-a_i)(x-(x_i+\frac{1}{n})) & \qquad & y_i=n(a_{i+1}-a_i)(x-(x_i-\frac{1}{n}))\\
    \chi_i(x)=\chi_{[x_i,x_i+\frac{1}{n}]}(x) & \qquad & \chi_i(x)=\chi_{[x_i-\frac{1}{n},_i)}(x)
    \end{matrix}
\]

\begin{center}
\begin{tikzpicture}
\draw[thick] (-3,0) -- (-2,0);
\draw[thick] (-2,1) -- (-1,1);
\fill[black] (-2,0) circle (0.05cm);
\filldraw[white,draw=black] (-2,1) circle (0.05cm);
\draw[thick,dotted] (-2,0) -- (-1.8,1);
\draw[thick] (3,0) -- (4,0);
\draw[thick] (4,1) -- (5,1);
\fill[black] (4,1) circle (0.05cm);
\filldraw[white,draw=black] (4,0) circle (0.05cm);
\draw[thick,dotted] (4,1) -- (3.8,0);
\end{tikzpicture}
\end{center}

Then, the $y_i$ are the line segments connecting the jumps of $\phi$ and always connected to $\phi(x_i)$.

Let $$g_n(x)=\phi(x)+\sum_{i=1}^my_i\chi_i(x).$$

Then, we note that $g_n(x)=\phi(x)$ for all $x$ except within $\frac{1}{n}$ of $x_i$. (Note that $g_n(x_i)=\phi(x_i)$ for all $x_i$.

Based on our construction, it is immediate that the $g_n$ are continuous.

Thus, if $\phi(x_i)=a_i$ and we can show that $g_n(x_i+\delta)\to \phi(x_i+\delta)$ for each $x_i$, and similarly for $\phi(x_i)=a_{i+1}$ and $g_n(x_i-\delta)\to\phi(x_i-\delta)$ we will be done.

Let $\varepsilon>0$. Then, for all $\delta>0$, there exists an $N\in\mathbb{N}$ such that $\frac{1}{N}<\delta$. Then, for all $n\ge N$, $$|g_n(x_i+\delta)-\phi(x_i+\delta)|=|a_{i+1}-a_{i+1}|=0<\varepsilon.$$

Similarly for $x_i-\delta$.

Finally, since there are only finitely many discontinuites of $\phi$, for whatever the minimum distance between any two $x_i$ is, there exists an $N\in\mathbb{N}$ such that $\frac{1}{n}$ is less than that distance for all $n>N$. Thus, aside from possibly discarding the first finite $N$, $g_n\to \phi$ pointwise.
\end{solution}
\newpage

\begin{problem} $\,$
Let $(X,\mu)$ be a finite measure space. Assume that a sequence of integrable functions $f_n$ satisfies $f_n\to f$ in measure, where $f$ is measurable. Assume that $f_n$ satisfies the following property: For every $\varepsilon>0$, there exists $\delta>0$ such that $$\mu(E)\le \delta\implies\int_E|f_n|d\mu\le\varepsilon.$$

Prove that $f$ is integrable and that $$\lim_n\int_X|f_n-f|d\mu=0.$$
\end{problem}


\begin{solution}$\,$
Since $f_n\to f$ in measure, there exists a subsequence $\{f_{n_k}\}$ such that $f_{n_k}\to f$ a.e.

Since $f_n\in L^1(\mu)$ or all $n$, $E=\{x\,|\,f_n(x)=\infty\}$ is $\mu$-null. Thus, each $f_{n_k}$, for any $\varepsilon>0$ and associated $\delta$ from the problem, there exists some finite $M>0$ such that $\mu(\{x\,|\,f_n(x)>M\}<\delta$.

Thus, \begin{align*}
    \int|f|d\mu&=\int_E|f|d\mu+\int_{E^c}|f|d\mu\\
    &=\int_E\liminf_{n_k}|f_{n_k}d\mu+\int_{E^c}\liminf_{n_k}|f_{n_k}d\mu\\
    &\le \liminf_{n_k}\int_E|f_{n_k}d\mu+\liminf_{n_k}\int_{E^c}|f_{n_k}d\mu\qquad\text{ Fatou's Lemma }\\
    &\le \liminf_{n_k}\varepsilon+\liminf_{n_k}M\mu(X)<\infty \tag{1}
\end{align*}
\boxed{(1)} Since $\delta$ is from the problem, and $\mu(E)<\delta$, $\int_E|f_{n_k}d\mu\le \varepsilon$ and $\mu(X)<\infty$. 

Therefore, $f\in L^1$. 

\begin{claim} The above property for $f_n$ holds for $f$.
\begin{proof} Since $f\in L^1$, for the subsequence $\{f_{n_k}\}$ converging to $f$ a.e., by Fatou's we have that, for all $\varepsilon>0$ and $\delta$ stated in the problem, if $\mu(E)<\delta$ then
$$\int_E|f|d\mu=\int_E\liminf_{n_k}|f_{n_k}|d\mu\le \liminf_{n_k}\int_E|f_{n_k}|d\mu\le\liminf_{n_k}\varepsilon=\varepsilon.$$
\end{proof}
\end{claim}

Now, let $\varepsilon>0$ be given and $\delta$ be as from the problem. Let $F=\{x\,|\,|f_n(x)-f(x)|\ge\varepsilon\}$. Then since $f_n\to f$ in measure, there exists some $N$ such that $\mu(F)<\delta$ for all $n\ge N$.

Then $$\int|f_n-f|d\mu=\int_F|f_n-f|d\mu+\int_{F^c}|f_n-f|d\mu\le 2\varepsilon+\varepsilon\mu(X).$$

Since $\varepsilon$ is arbitrary, we have that $\int|f_n-f|d\mu\to 0$ as $n\to\infty.$
\end{solution}
\newpage

\begin{problem}[Folland, 2.3.25, p.59] $\,$
Consider the following statements about a functioin $f:[0,1]\to\mathbb{R}$.
\begin{enumerate}[label=(\roman*)]
    \item $f$ is continuous almost everywhere
    \item $f$ is equal to a continuous function $g$ almost everywhere.
\end{enumerate}

Does (i) imply (ii)? Prove or give a counterexample. Does (ii) imply (i)? Prove or give a counter example.
\end{problem}


\begin{solution}$\,$
\boxed{(i)\not\implies (ii)} Let \[
f(x)=\begin{cases}
0 & \text{ if }0\le x\le\frac{1}{2}\\
1 & \text{ if }\frac{1}{2}<x\le 1
\end{cases}
\]
 then $f(x)$ is continuous a.e. since it is only discontinuous at $\frac{1}{2}$.
 
 Now, assume there is some continuous function $g(x)=f(x)$ a.e.
 
 Let $\frac{1}{2}>\varepsilon>0$ be given. Then, by continuity of $g$, there exists a $\delta$ such that for all $y\in(\frac{1}{2}-\delta,\frac{1}{2}+\delta)$, $|g(\frac{1}{2})-g(y)|<\varepsilon$.
 
 However, because $f=g$ a.e., there exists a $x_0\in (\frac{1}{2}-\delta,\frac{1}{2})$ such that $f(x_0)=g(x_0)=0$ and there exists $y_0\in(\frac{1}{2},\frac{1}{2}+\delta)$ such that $f(y_0)=g(y_0)=1$. 
 
 however, then $|x_0-y_0|<\delta$ and $|g(y_0)-g(x_0)|=1>\varepsilon$ which is a contradiction of the continuity of $g$. \contradiction
 
 Therefore, $f$ is not equal to a continuous function a.e.
 
\boxed{(ii)\not\implies (i)} Let $f(x)=\chi_\mathbb{Q}$. Let $g(x)=0$. Then $f(x)=g(x)$ a.e. (since $f(x)\not=0$ only when $x\in\mathbb{Q}$ which is a Lebesgue-null set). 

However, $f(x)$ is discontinuous at every point and so $f(x)$ is not continuous a.e.
\end{solution}
\vspace{0.5cm}

\end{document}
 