\documentclass[12pt]{Homework}

% Changed from \usepackage{prelude}
\usepackage{preamble}
\usepackage{amssymb}
\usepackage{enumitem}
\usepackage{mathrsfs}
\def\upint{\mathchoice%
    {\mkern13mu\overline{\vphantom{\intop}\mkern7mu}\mkern-20mu}%
    {\mkern7mu\overline{\vphantom{\intop}\mkern7mu}\mkern-14mu}%
    {\mkern7mu\overline{\vphantom{\intop}\mkern7mu}\mkern-14mu}%
    {\mkern7mu\overline{\vphantom{\intop}\mkern7mu}\mkern-14mu}%
  \int}
\def\lowint{\mkern3mu\underline{\vphantom{\intop}\mkern7mu}\mkern-10mu\int}
\usepackage[mathscr]{euscript}
\usepackage{comment}
\usepackage{bbding}
\renewcommand\qedsymbol{\Peace}
\newcommand\placeqed{\nobreak\enspace\Peace}
\usepackage{MnSymbol}
\usepackage{tikz}
\usepackage{halloweenmath}
\newcommand{\contradiction}{\null\hfill\large{$\mathghost$}\normalsize}
\renewenvironment{framed}[1][\hsize]
   {\MakeFramed{\hsize#1\advance\hsize-\width \FrameRestore}}%
   {\endMakeFramed}

\name{Kayla Orlinsky}
\course{Real Analysis Exam}
\term{Spring 2010}
\hwnum{Spring 2010}

\begin{document}

\begin{problem} $\,$
A function $f:\mathbb{R}\to\mathbb{R}$ is said to be \textit{upper semicontinuous} (or \textit{u.s.c}) if for all $x\in\mathbb{R}$ and all $\varepsilon>0$ there exists $\delta>0$ such that $f(y)<f(x)+\varepsilon$ whenever $|y-x|<\delta$.
\begin{enumerate}[label=(\alph*)]
    \item Show that every u.s.c. function is Borel measurable. HINT: Consider $\{x\,|\,f(x)<a\}$.
    \item Suppose $\mu$ is a finite measure on $\mathbb{R}$ and $A$ is a closed subset of $\mathbb{R}$. Using (a) or otherwise, show that the function $x\mapsto\mu(x+A)$ is measurable. Here $x+A=\{x+y\,|\,y\in A\}$.
 \end{enumerate}
\end{problem}


\begin{solution}$\,$
\begin{enumerate}[label=(\alph*)]
    \item Since $$f^{-1}((-\infty,a))=\{x\,|\,f(x)<a\}=A,$$ we check that $f^{-1}((-\infty,a))$ is open. Let $x\in A.$ Then since $f$ is usc, for all $\varepsilon>0$ there exists a $\delta$ such that $$f(y)-\varepsilon<f(x)<a\qquad\text{ whenever }|y-x|<\delta.$$
    
    Now, for $\varepsilon=a-f(x)$, there is some $\delta$ where, for all $|y-x|<\delta$ we have that $$f(y)-f(x)<\varepsilon=a-f(x)\implies f(y)<a.$$ Thus, $B(\delta,x)\subset A$.
    
    Thus, $A$ is open and since all open sets are Borel, we have that $A$ is Borel.
    \item Since $A$ is closed, $A^c$ is open and so $A$ is a Borel set. Thus, there exists some $E$, which is a union of finitely many open intervals such that $\mu(A\Delta E)<\varepsilon.$
    
    Thus, it suffices to check the statement holds for $x\mapsto\mu(x+E).$
    
    Let \begin{align*}
        f(x):&\mathbb{R}\to\mathbb{R}\\
        &x\mapsto\mu(x+E)
    \end{align*}
    
    We would like to show that $f$ is usc. Let $\varepsilon>0$ be given and $x$ be fixed. WLOG, let $$E=\bigcup_{i=1}^N(a_i,b_i)\qquad x+E=\bigcup_{i=1}^N(a_i+x,b_i+x).$$
    
    Now, fix $b\in\mathbb{R}$. Let $B_n=(b,b+\frac{1}{n})$. Then $B_1\supset B_2\supset \cdots$ and since $\mu(X)<\infty$, by continuity $$0=\mu(\bigcap^\infty B_n)=\lim_{n\to\infty}\mu(b,b+\frac{1}{n}).$$ Since, $b$ was arbitrary, this implies that for all $\varepsilon>0$ there exists some $\delta$ such that for all $x<y<x+\delta$, $\mu(b+x,b+y)<\varepsilon$.
    
    Let $\delta_i$ be such that $\mu(a_i+x,b_i+x+\delta_i)<\varepsilon.$
    
    Now, let $$\delta=\frac{1}{N}\max_i\{\mu(b_i+x,b_i+x+\delta_i)\}.$$
    
    Then, for all $x<y<x+\delta$, we have that \begin{align*}
        f(y)=\mu(y+E)&\le m(x+E)+\mu(\cup_{i=1}^N\mu(b_i+x,b_i+y)\\
        &\le\mu(x+E)+\sum_{i=1}^N\mu(b_i+x,b_i+y)\\
        &<\mu(x+E)+\sum_{i=1}^N\delta\\
        &<\mu(x+E)+\varepsilon=f(x)+\varepsilon
    \end{align*}
    
    Thus, $f$ is usc and so it is measurable by (a).
 \end{enumerate}
\end{solution}
\newpage

\begin{problem} $\,$
Suppose $\{f_n\}$ and $f$ are measurable functions on $(X,\mathscr{M},\mu)$ and $f_n\to f$ in measure. Is it necessarily true that $f_n^2\to f^2$ in measure if 
\begin{enumerate}[label=(\alph*)]
    \item $\mu(X)<\infty$
    \item $\mu(X)=\infty$
\end{enumerate}
\end{problem}


\begin{solution}$\,$
\begin{enumerate}[label=(\alph*)]
    \item Let $$E_n=\{x\,|\,|f_n^2(x)-f^2(x)|\ge\varepsilon\}$$
    
    Then \begin{align*}
        \mu(E_n)&=\mu(\{x\,|\,|f_n(x)-f(x)||f_n(x)+f(x)|\ge\varepsilon\})\\
        &=\mu(\{x\,|\,|f_n(x)-f(x)||f_n(x)+f(x)|\ge\varepsilon\text{ and }|f_n(x)+f(x)|\ge k\})\\
        &+\mu(\{x\,|\,|f_n(x)-f(x)||f_n(x)+f(x)|\ge\varepsilon\text{ and }|f_n(x)+f(x)|<k\})\\
        &\le\mu(\{x\,|\,|f_n(x)+f(x)|\ge k\})+\mu(\{x\,|\,|f_n(x)-f(x)|\ge\frac{\varepsilon}{k})
    \end{align*}
    
    Now, we assume that $f(x)<\infty$ a.e. which is safe it is not specified that $f$ is defined over the extended reals.
    
    Thus, $$\mu(E_n)\le\lim_{k\to\infty}\lim_{n\to\infty}\mu(\{x\,|\,|f_n(x)+f(x)|\ge k\})+\mu(\{x\,|\,|f_n(x)-f(x)|\ge\frac{\varepsilon}{k})=0.$$
    \item Let $f_n(x)=x+\frac{1}{n}$ then $|f_n-x|=|\frac{1}{n}|$ and since for all $\varepsilon>0$, there exists an $N$ such that $$\frac{1}{n}\le\varepsilon$$ for all $n\ge N$, we have that $$\mu(\{x\,|\,|f_n(x)-x|\ge\varepsilon\})\to0\quad\text{ as }n\to\infty.$$
    
    However, assuming $\mu=m$, and $X=[0,\infty)$, we have that $f_n^2=x^2+\frac{2x}{n}+\frac{1}{n^2}$ and $$|f_n^2-f^2|\ge\varepsilon\implies \left|\frac{2x}{n}+\frac{1}{n^2}\right|\ge\varepsilon\implies x\ge\frac{n\varepsilon}{2}-\frac{1}{2}.$$
    
    Thus,
     $$m(\{x\,|\,|f_n-f|\ge\varepsilon\})=m\left(\{x\,|\,x\ge\frac{n\varepsilon}{2}-\frac{1}{2}\right)=m\left(\frac{n\varepsilon}{2}-\frac{1}{2},\infty\right)=\infty\quad\text{ for all }n.$$
     
     Thus, $f_n^2\not\to f^2$ in measure.
\end{enumerate}
\end{solution}
\newpage

\begin{problem} $\,$
Suppose $f:[0,1]\to\mathbb{R}$ is as strictly increasing absolutely continuous function. Let $m$ denote the Lebesgue measure. If $m(E)=0$ show that $m(f(E))=0$.
\end{problem}


\begin{solution}$\,$
Let $E\subset[0,1]$ with $m(E)=0.$

Now, since $f$ is absolutely continuous and strictly increasing we have that $f$ is one-to-one on $[0,1]$. Thus, if $y\in f(E)$, then $y=f(x)$ for exactly one $x\in E$ and similarly, if $x\in E$ then there is one $y=f(x)\in f(E).$ Thus, $$\chi_E(x)=\chi_{f(E)}(y).$$

Furthermore, $f'$ exists a.e. by the Fundamental Theorem of Lebesgue Integrals.

Thus, $$m(f(E))=\int\chi_{f(E)}(y)dy=\int\chi_E(u)f'du=0$$ $$u=y=f(x)\qquad du=f'(x)dx$$
\end{solution}
\newpage

\begin{problem} $\,$
For $n\ge1$ define $h_n$ on $[0,1]$ by $$h_n=\sum_{j=1}^n(-1)^j\chi_{(\frac{j-1}{n},\frac{j}{n}]}.$$ Here $\chi_E$ denotes the characteristic function of $E$. If $f$ is Lebesgue integrable on $[0,1]$, show that $$\lim_{n\to\infty}\int_{[0,1]}fh_ndm=0.$$

HINT: First consider $f$ in a suitably smaller function space.
\end{problem}


\begin{solution}$\,$
Let $f(x)=\chi_E(x)$ with $E\subset[0,1]$. Then for fixed $n$, $h_n\in L^1([0,1])$ since $|h_n|=(0,1]$ so we can apply Fubini to $fh_n$ on $m\times \nu$ with $\nu$ the counting measure on $\mathbb{N}$. Thus,

\begin{align*}
    \int_{[0,1]}fh_ndm&=\int_{[0,1]}\sum_{j=1}^n(-1)^j\chi_{(\frac{j-1}{n},\frac{j}{n}]\cap E)}dm\\
    &=\sum_{j=1}^n(-1)^j\int_{[0,1]}\chi_{(\frac{j-1}{n},\frac{j}{n}]\cap E)}dm\\
    &=\sum_{\text{even j}}^n\int_{[0,1]}\chi_{(\frac{j-1}{n},\frac{j}{n}]\cap E)}dm-\sum_{\text{odd j}}^n\int_{[0,1]}\chi_{(\frac{j-1}{n},\frac{j}{n}]\cap E)}dm\\
    &=\sum_{\text{even j}}^nm\left(\left(\frac{j-1}{n},\frac{j}{n}\right]\cap E\right)-\sum_{\text{odd j}}^nm\left(\left(\frac{j-1}{n},\frac{j}{n}\right]\cap E\right)\\
    &=m\left(\bigcup_{\text{ even j}}^n\left(\frac{j-1}{n},\frac{j}{n}\right]\cap E\right)-m\left(\bigcup_{\text{ odd j}}^n\left(\frac{j-1}{n},\frac{j}{n}\right]\cap E\right)
\end{align*}

Let $$A_n=\bigcup_{\text{ even }j}^n\left(\frac{j-1}{n},\frac{j}{n}\right]\quad\text{ so }\quad A_n^c=\bigcup_{\text{ odd }j}^n\left(\frac{j-1}{n},\frac{j}{n}\right].$$

Let $E=(a,b)\subset[0,1]$. Then for all $\varepsilon>0$ there exists $N$ such that for some $j,k$ $|j/N-a|<\varepsilon$ and $|k/n-b|<\varepsilon$. Then $E$ will be almost perfectly partitioned. Specifically, $$|m(A_n\cap E)-m(A^c\cap E)|\le\frac{1}{N}+2\varepsilon.$$ 

Thus, $$m(A_n\cap E)-m(A^c\cap E)\to0\qquad\text{ as }n\to\infty.$$

Therefore, the same is true for finite unions of open intervals. 

Now, for all $E$, and for all $\varepsilon$ there exists $F$, a finite union of open intervals such that $$m(E\Delta F)<\varepsilon.$$

Thus, \begin{align*}
    |m(A_n\cap E)-m(A_n^c\cap E)|&=|m(A_n\cap E)-m(A_n^c\cap E)+m(A_n\cap F)\\
    &\qquad\qquad-m(A_n^c\cap F)+m(A_n^c\cap F)-m(A_n\cap F)|\\
    &\le|(m(A_n\cap E)-m(A_n\cap F))-(m(A_n^c\cap E)-m(A_n^c\cap F))|\\
    &\qquad\qquad+|m(A_n\cap F)-m(A_n^c\cap F)|\\
    &=|m(A_n\cap (E\backslash F))-m(A_n^c\cap (E\backslash F))|+|m(A_n\cap F)-m(A_n^c\cap F)|\\
    &\le2\varepsilon+|m(A_n\cap F)-m(A_n^c\cap F)|
\end{align*}

And since we have already seen that $$|m(A_n\cap F)-m(A_n^c\cap F)|\to0$$ for $F$, we have that the same holds for $E$ and so $$\lim_{n\to\infty}\int_{[0,1]}\chi_Eh_ndm=0.$$

Thus, the above holds for all simple functions $f$ by linearity of the integral. 

Finally, since for all $\varepsilon>0$ there exists a simple function $\phi$ such that $\int|f-\phi|dm<\varepsilon$, we have that $$\left|\int_{[0,1]}fh_ndm-\int_{[0,1]}\phi h_ndm\right|=\left|\int_{[0,1]}(f-\phi)h_ndm\right|\le\int_{[0,1]}|f-\phi||h_n|dm=\int_{[0,1]}|f-\phi|dm<\varepsilon$$ so, tending $\varepsilon$ to $0$ we have our result.
\end{solution}
\vspace{0.5cm}

\end{document}
 