\documentclass[12pt]{Homework}

% Changed from \usepackage{prelude}
\usepackage{preamble}
\usepackage{amsmath}
\usepackage{amssymb}
\usepackage{enumitem}
\usepackage{mathrsfs}
\usepackage[mathscr]{euscript}
\usepackage{comment}
%\usepackage{MnSymbol}
\usepackage{tikz,float}
\usepackage{tikz-cd}
\usepackage{graphicx}
\usepackage{caption, threeparttable}
%\captionsetup{labelfont = sc, textfont = it}
\usepackage{halloweenmath}
\newcommand{\contradiction}{\null\hfill\large{$\mathghost$}\normalsize}
\usepackage[skins]{tcolorbox}
\newtcolorbox{mybox}{enhanced,sharp corners=all,colback=white,colframe=gray,toprule=0pt,bottomrule=0pt,leftrule=1pt,rightrule=1pt,overlay={
    \draw[gray,line width=1pt] (frame.north west) -- ++(2cm,0pt);
    \draw[gray,line width=1pt] (frame.south east) -- ++(-2cm,0pt);
}}
\usepackage{bbding}
\renewcommand\qedsymbol{\Peace}
\newcommand\placeqed{\nobreak\enspace\Peace}
\newcommand{\im}{\mathscr{I}\text{m}}
\newcommand{\re}{\mathscr{R}\text{e}}
\newcommand{\res}{\text{Res}}

\name{Kayla Orlinsky}
\course{Real Analysis Exam}
\term{Spring 2017}
\hwnum{Spring 2017}

\begin{document}

\begin{problem} $\,$
Assume that $f$ is a positive absolutely continuous function on $[0,1]$. Prove that $1/f$ is also absolutely continuous on $[0,1]$.
\end{problem}


\begin{solution}$\,$
Since $f$ is absolutely continuous and strictly positive, it has a minimum $f(x_0)=\delta>0$ for some $x_0\in[0,1].$ Namely, for all $x\in[0,1]$, $$f(x)\ge\alpha\implies \frac{1}{f(x)}\le \frac{1}{\alpha}$$

Now, let $\varepsilon>0$ and $\delta>0$ such that for all $\{(a_i,b_i)\}_{i=1}^n$ finite collections of disjoint subintervals of $[0,1]$, $$\sum_{i=1}^n(b_i-a_i)<\delta\implies \sum_{i=1}^n|f(b_i)-f(a_i)|<\alpha^2\varepsilon.$$

Then, \begin{align*}
    \sum_{i=1}^n\left|\frac{1}{f(b_i)}-\frac{1}{f(a_i)}\right|&=\sum_{i=1}^n\left|\frac{f(a_i)-f(b_i)}{f(b_i)f(a_i)}\right|\\
    &\le\sum_{i=1}^n\frac{|f(b_i)-f(a_i)|}{\alpha^2}\\
    &<\frac{1}{\alpha^2}\alpha^2\varepsilon=\varepsilon
\end{align*}

so $\frac{1}{f}$ is absolutely continuous on $[0,1].$
\end{solution}
\newpage




\begin{problem} $\,$
Assume that $E$ is Lebesgue measurable. 
\begin{enumerate}[label=(\alph*)]
    \item Suppose $m(E)<\infty$, where $m$ is the Lebesgue measure. Show that $$f(x)=\int\chi_E(y)\chi_E(y-x)dm(y)$$ is continuous. (Here, $\chi_A$ denotes the characteristic function of a set $A\subset\mathbb{R})$.
    \item Suppose $0<m(E)\le\infty$. Show that $S=E-E=\{x-y:x,y\in E\}$ contains an open interval $(-\varepsilon,\varepsilon)$ for some $\varepsilon>0$.
\end{enumerate}
\end{problem}


\begin{solution}$\,$
 \begin{enumerate}[label=(\alph*)]
    \item Suppose $m(E)<\infty$. First, we note that $y-x\in E$ if and only if $y\in E+x$. Therefore, \begin{align*}
        f(x)&=\int\chi_E(y)\chi_E(y-x)dm(y)\\
        &=\int\chi_E(y)\chi_{E+x}(y)dm(y)\\
        &=m(E\cap (E+x)).
    \end{align*}
    
    Now, since $m(E)<\infty,$ then for all $\varepsilon>0$, there exists $$A=\bigcup_{i=1}^n(a_i,b_i)$$ a finite union of disjoint open intervals such that $m(E\Delta A)<\varepsilon.$
    
    Thus, if $g(x)=m(A\cap(A+x))$ is continuous, then because \begin{align*}
        |g(x)-f(x)|&=|m(A\cap(A+x))-m(E\cap (E+x))|\\
        &=|m(A\cap(A+x)\cap E)+m(A\cap(A+x)\cap E^c)\\
        &\qquad\qquad-m(E\cap(E+x)\cap A)-m(E\cap(E+x)\cap A^c)|\\
        &<|m(A\cap(A+x)\cap E)-m(E\cap(E+x)\cap A)|+\varepsilon\\
        &=|m(A\cap(A+x)\cap E\cap(E+x))+m(A\cap(A+x)\cap E\cap(E+x)^c)\\
        &\qquad\qquad-m(E\cap(E+x)\cap A\cap(A+x))-m(E\cap(E+x)\cap A\cap(A+t)^c)|+\varepsilon\\
        &<4\varepsilon
    \end{align*} so $f(x)$ must also be continuous.
    
    Now, $g$ is clearly continuous since $A$ is a finite union of intervals so $$\lim_{x\to y}g(x)=g(y)$$ for all $x,y\in\mathbb{R}$ since we can make the difference $|g(x)-g(y)|$ as small as we like by making $|x-y|$ small.
    
    Namely, $f$ is continuous.
    \item Suppose $0<m(E)\le\infty$. 
    
    If $m(E)<\infty$ then $g(x)=m(E\cap(E+x))$ is continuous by (a).
    
    Now, $g(0)=m(E)>0$ by assumption. Therefore, by continuity, there exists $\varepsilon>0$ small enough that $g(z)>0$ for all $z\in(-\varepsilon,\varepsilon)$.
    
    Let $z\in(-\varepsilon,\varepsilon)$ and $z\notin E-E$. Then there does not exist $x,y\in E$ such that $z=x-y.$
    
    Namely, for all $y\in E$, $x=z+y\notin E$. Therefore, $E\cap(E+z)=\varnothing$ so $m(E\cap (E+z))=0.$
    
    This is a contradiction of $g(z)>0$ so no such $z$ can exist. Namely, $(-\varepsilon,\varepsilon)\in E-E$.
    
    Now, if $m(E)=\infty$, because $\mathbb{R}$ is $\sigma$-finite, $E\subset\mathbb{R}$ is also $\sigma$-finite. Namely, there exists $\{E_k\}_{k=1}^\infty$ where $E_k$ are disjoint and $m(E_k)<\infty$ for all $k$ and $$E=\bigcup_{k=1}^\infty E_k.$$
    
    Therefore, $$(-\varepsilon,\varepsilon)\in E_k-E_k\subset E-E.$$
\end{enumerate}
\end{solution}
\newpage




\begin{problem} $\,$
Assume that $f$ is a continuous function on $[0,1]$. Prove that $$\lim_{n\to\infty}\int_0^1nx^{n-1}f(x)dx=f(1).$$
\end{problem}


\begin{solution}$\,$
Since $f$ is continuous on a closed interval, it is bounded and so the Lebesgue integral and Riemann integrals are the same.

Therefore, letting $u=x^n$, we get that $du=nx^{n-1}dx$ and so $$\lim_{n\to\infty}\int_0^1nx^{n-1}f(x)dx=\lim_{n\to\infty}\int_0^1f(u^{1/n})du.$$

Now, we apply DCT.
\begin{enumerate}
    \item $f(u^{1/n})$ is measurable.
    \item $$\lim_{n\to\infty}f(u^{1/n})=f\left(\lim_{n\to\infty}u^{1/n}\right)=f(1)$$ for a.e. $u\in[0,1]$ since $f$ is continuous. 
    \item $f(u^{1/n})\le M\in L^1([0,1])$ where $M=\max|f(x)|$ for a.e. $u\in[0,1].$
\end{enumerate}

Therefore, $$\lim_{n\to\infty}\int_0^1nx^{n-1}f(x)dx=\lim_{n\to\infty}\int_0^1f(u^{1/n})du=\int_0^1\lim_{n\to\infty}f(u^{1/n})du=\int_0^1f(1)du=f(1).$$
\end{solution}
\newpage



\begin{problem} $\,$
Let $(\Omega,\mathscr{F},\mu)$ be a $\sigma$-finite measure space. Let $f,g$ be measurable real valued functions. Show that $$\int|f-g|d\mu=\int_{-\infty}^\infty\int\left|\chi_{(t,\infty)}(f(x))-\chi_{(t,\infty)}(g(x))\right|d\mu(x)dt.$$
\end{problem}


\begin{solution}$\,$
Note that for fixed $t$, $$f(x)>t,g(x)\le t\implies \chi_{(t,\infty)}(f(x))-\chi_{(t,\infty)}(g(x))=1$$ and $$f(x)\le t,g(x)>t\implies \chi_{(t,\infty)}(f(x))-\chi_{(t,\infty)}(g(x))=-1$$ all other cases make the integrand $0$.

Let $$E=\{x\,|\, f(x)>t\ge g(x)\}.$$

Therefore, by linearity of the integral we get that
\begin{align*}
    \int_{-\infty}^\infty\int&\left|\chi_{(t,\infty)}(f(x))-\chi_{(t,\infty)}(g(x))\right|d\mu(x)dt\\
    &=\int_{-\infty}^\infty\int_E\chi_{(t,\infty)}(f(x))-\chi_{(t,\infty)}(g(x))d\mu(x)dt\\
    &\qquad\qquad\qquad-\int_{-\infty}^\infty\int_{E^c}\chi_{(t,\infty)}(f(x))-\chi_{(t,\infty)}(g(x))d\mu(x)dt\\
    &=\int_{-\infty}^\infty\int_E1d\mu(x)dt+\int_{-\infty}^\infty\int_{E^c}1d\mu(x)dt\\
    &=\int_{-\infty}^\infty\int_E\chi_{(g(x),f(x))}(x)d\mu(x)dt+\int_{-\infty}^\infty\int_{E^c}\chi_{(f(x),g(x))}(x)d\mu(x)dt\\
    &=\int_E\int_{-\infty}^\infty\chi_{(g(x),f(x))}(x)dtd\mu(x)+\int_{E^c}\int_{-\infty}^\infty\chi_{(f(x),g(x))}(x)dtd\mu(x)\\
    &=\int_E\int_{g(x)}^{f(x)}dtd\mu(x)+\int_{E^c}\int_{f(x)}^{g(x)}dtd\mu(x)\\
    &=\int_Ef(x)-g(x)d\mu(x)+\int_{E^c}g(x)-f(x)d\mu(x)\\
    &=\int_X|f(x)-g(x)|\mu(x)
\end{align*}

with (1) since $\Omega$ is $\sigma$-finite and the integrand is clearly in $L^+(\mu\times m)$ so the integrals can be swapped. 
\end{solution}
\vspace{0.5cm}

\end{document}
 