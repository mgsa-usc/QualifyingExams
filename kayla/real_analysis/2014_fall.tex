\documentclass[12pt]{Homework}

% Changed from \usepackage{prelude}
\usepackage{preamble}
\usepackage{amssymb}
\usepackage{enumitem}
\usepackage{mathrsfs}
\def\upint{\mathchoice%
    {\mkern13mu\overline{\vphantom{\intop}\mkern7mu}\mkern-20mu}%
    {\mkern7mu\overline{\vphantom{\intop}\mkern7mu}\mkern-14mu}%
    {\mkern7mu\overline{\vphantom{\intop}\mkern7mu}\mkern-14mu}%
    {\mkern7mu\overline{\vphantom{\intop}\mkern7mu}\mkern-14mu}%
  \int}
\def\lowint{\mkern3mu\underline{\vphantom{\intop}\mkern7mu}\mkern-10mu\int}
\usepackage[mathscr]{euscript}
\usepackage{comment}
\usepackage{MnSymbol}
\usepackage{bbding}
\renewcommand\qedsymbol{\Peace}
\newcommand\placeqed{\nobreak\enspace\Peace}
\usepackage{tikz}
\usepackage{halloweenmath}
\newcommand{\contradiction}{\null\hfill\large{$\mathghost$}\normalsize}
\renewenvironment{framed}[1][\hsize]
   {\MakeFramed{\hsize#1\advance\hsize-\width \FrameRestore}}%
   {\endMakeFramed}

\name{Kayla Orlinsky}
\course{Real Analysis Exam}
\term{Fall 2014}
\hwnum{Fall 2014}

\begin{document}

\begin{problem} $\,$
Assume that $f$ is integrable on $(0,1)$. Prove that $$\lim_{a\to\infty}\int_0^1f(x)x\sin(ax^2)dx=0.$$
\end{problem}


\begin{solution}$\,$
First, let $f(x)=\chi_E(x)$ for some measurable set $E\subset[0,1]$. Then \begin{align*}
    \int_0^1\chi_E(x)x\sin(ax^2)dx&=\int_Ex\sin(ax^2)dx\\
    &\le\int_0^1x\sin(ax^2)dx\\
    &=\int_0^a\frac{\sin(u)}{2a}du\qquad \begin{matrix}
    u=ax^2 &  x:[0,1]\\
    du=2axdx & u:[0,a]
\end{matrix}
&=\frac{-\cos(u)}{2a}\bigg|_0^a\\
&=\frac{-\cos(a)}{2a}+\frac{1}{2a}\to0\qquad\text{ as }a\to\infty.
\end{align*}

Therefore, by linearity of the integral, the statement holds for simple functions.

Now, since $f\in L^1$, there exists some $\phi$ simple function such that for all $\varepsilon>0$, $$\int_0^1|f-\phi|dx<\varepsilon.$$

Thus, \begin{align*}
    \left|\int_0^1f(x)x\sin(ax^2)dx-\int_0^1\phi(x)x\sin(ax^2)dx\right|&\le\int_0^1|(f(x)-\phi(x))x\sin(ax^2)|dx\\
    &\le\int_0^1|f(x)-\phi(x)|dx<\varepsilon
\end{align*} since $x\sin(ax^2)\le1$ for all $x\in(0,1)$ and all $a$.

Thus, since we already showed that $$\int_0^1\phi(x)x\sin(ax^2)dx\to0$$ we are done.
\end{solution}
\newpage

\begin{problem} $\,$
Let $(X,\mathscr{M},\mu)$ be a measure space, and let $f$ and $f_1,f_2,f_3,...$ be real valued measurable functions on $X$. If $f_n\to f$ in measure and if $F:\mathbb{R}\to\mathbb{R}$ is uniformly continuous, prove that $F\circ f_n\to F\circ f$ in measure.
\end{problem}


\begin{solution}$\,$
Let $\varepsilon>0$ be given. Then, since $F$ is uniformly continuous, there exists a $\delta>0$ such that whenever $|x-y|<\delta$, $|F(x)-F(y)|<\varepsilon.$

Let $$E=\{x\,|\,|f_n(x)-f(x)|\ge\delta\}$$
$$F=\{x\,|\,|(F\circ f_n)(x)-(F\circ f)(x)|\ge\varepsilon\}.$$

Now, we note that if $x\in E^c$, then $|f_n(x)-f(x)|<\delta$ and so $|F(f_n(x))-F(f(x))|<\varepsilon$ which implies that $x\in F^c$.

Thus, $E^c\subset F^c$ and so $F\subset E$.

Finally, this implies that since $\mu(E)\to0$ as $n\to\infty$ (since $f_n\to f$ in measure, then $\mu(F)\to0$ as $n\to\infty$ as well.
\end{solution}
\newpage

\begin{problem} $\,$
Let $f_n$ be \textbf{nonnegative} measurable functions on a measure space $(X,\mathscr{M},\mu)$ which satisfy $\int f_nd\mu=1$ for all $n=1,2,... .$ Prove that $$\limsup_n(f_n(x))^{1/n}\le1$$ for $\mu$-a.e. $x.$
\end{problem}


\begin{solution}$\,$
First, for all $n$, let $$E_n=\{x\,|\,f_n(x)>n\}.$$ Then \begin{align*}
    1&=\int f_nd\mu\\
    &=\int_{E_n}f_nd\mu+\int_{E_n^c}f_nd\mu\\
    &\ge\int_{E_n}nd\mu+\int_{E_n^c}f_nd\mu\\
    &\ge n\mu(E_n)+0\qquad\text{ since }f_n\ge0\\
    \implies&\mu(E_n)\le\frac{1}{n} 
\end{align*}

Thus, except on a set of shrinking measure, $f_n(x)\le n$, and so (again except on a set of shrinking measure) $f_n(x)^{1/n}\le n^{1/n}$.

Finally, since \begin{align*}
    y&=\lim_{n\to\infty}n^{1/n}\\
   \implies \ln y&=\lim_{n\to\infty}\frac{\ln n}{n}\\
    &=\lim_{n\to\infty}\frac{1}{n}\\
    &=0\\
    \implies y&=e^0=1
\end{align*}

we have that $$\limsup_n(f_n(x))^{1/n}\le\limsup_n(n^{1/n})=\lim_{n\to\infty}n^{1/n}=1.$$

\begin{comment}
Let $$g_n=\sup_{k\ge n}(f_k(x))^{1/k}<\infty$$ a.e. since if $f_k(x)$ grows arbitrarily large, this growth must occur on sets of increasingly small measure (since $f_k$ are integrable).  Then, by definition of $\sup$, for all $\varepsilon>0$, there exists some $f_k(x)^{1/k}$ such that $$f_k(x)^{1/k}\ge g_n(x)-\frac{1}{n}.$$

Let $\varepsilon=\frac{1}{n}$ and call $f_{k_n}(x)^{1/k_n}$ the function satisfying the above. 

Now, $$\inf_{n}f_{k_n}(x)^{1/k_n}\ge\inf_n\left(g_n(x)-\frac{1}{n}\right)$$
\end{comment}

\end{solution}
\newpage

\begin{problem} $\,$
Let $-\infty<a<b<\infty$. Suppose $F:[a,b]\to\mathbb{C}$. 
\begin{enumerate}[label=(\alph*)]
    \item Define what it means for $F$ to be absolutely continuous on $[a,b]$.
    \item Give an example of a function which is uniformly continuous but not absolutely continuous. (Remember to justify your answer.)
    \item Prove that if there exists $M$ such that $|F(x)-F(y)|\le M|x-y|$ for all $x,y\in[a,b]$, then $F$ is absolutely continuous. Is the converse true? (Again, remember to justify your answer).
\end{enumerate}
\end{problem}


\begin{solution}$\,$
\begin{enumerate}[label=(\alph*)]
    \item $\,$
    
    \begin{framed}[.75\textwidth]
    $F(x)$ is absolutely continuous on [a,b] if for all $\varepsilon>0$ there exists a $\delta>0$ such that for any finite collection $\{(a_i,b_i)\}_1^n$ of disjoint subintervals of $[a,b]$ satisfying $\displaystyle\sum_{i=1}^n|b_i-a_i|<\delta$ implies that $\displaystyle\sum_{i=1}^n|F(b_i)-F(a_i)|<\varepsilon$
    \end{framed}
    \item Let $f(x)$ be the Cantor Function on $[0,1]$. Since $f(x)$ is continuous on a closed interval, it is uniformly continuous.
    
    Now, assume that $f(x)$ is absolutely continuous. We already know that $f'(x)=0$ a.e. since it is only non-constant on the Cantor set (which has Lebesgue Measure $0$). 
    
    Thus, if $f(x)$ is absolutely continuous on $[0,1]$, then by the Fundamental Theorem of Lebesgue Integrals, $$1=f(1)-f(0)=\int_0^1f'(x)dx=\int_0^10dx=0.\qquad\qquad$$
    
    Thus, $f(x)$ cannot be absolutely continuous on $[0,1]$.
    \item Clearly if such an $M$ exists, it must be nonnegative. If $M=0$, then $F$ is constant on $[a,b]$ and so it is clearly absolutely continuous. 
    
    If $M>0$, then for all $\varepsilon>0$, we can let $\delta=\frac{\varepsilon}{M}$. Then, for any finite collection of disjoint intervals $\{(a_i,b_i)\}_1^n$ satisfying $$\sum_{i=1}^n(b_i-a_i)<\delta,$$ we have that $$\sum_{i=1}^n|F(b_i)-F(a_i)|\le\sum_{i=1}^nM|b_i-a_i|=M\sum_{i=1}^n|b_i-a_i|<M\delta=M\frac{\varepsilon}{M}=\varepsilon.$$
    
    Thus, $F$ is absolutely continuous.
    
    Now, let $F(x)=\sqrt{x}$ on $[0,1]$. Then $F'(x)=\frac{1}{2\sqrt{x}}$ exists except at $0$ (a null set) and since $$\lim_{b\to0}\int_b^1\frac{1}{2\sqrt{x}}dx=\lim_{b\to0}\sqrt{x}\bigg|_b^1=\lim_{b\to0}(1-\sqrt{b})=1$$ so $F'(x)\in L^1$ and by the same computation $$F(x)-F(0)=\int_0^xF'(x)dx.$$
    
    Thus, again by the Fundamental Theorem of Lebesgue Integrals, $F$ is absolutely continuous on $[0,1]$.
    
    However, for all $y>x$, we have $$|F(y)-F(x)|=\sqrt{y}-\sqrt{x}=\frac{y-x}{\sqrt{y}+\sqrt{x}}.$$
    
    If there exists some $\infty>M>0$ such that $\frac{y-x}{\sqrt{y}+\sqrt{x}}\le M(y-x)$, then $M\ge\frac{1}{\sqrt{y}+\sqrt{x}}.$ However, for $x,y$ very small near $0$, we can force $M$ to grow as large as we like and so no such finite $M$ exists.
\end{enumerate}
\end{solution}
\vspace{0.5cm}

\end{document}
 