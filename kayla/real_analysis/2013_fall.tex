\documentclass[12pt]{Qual}
\usepackage{preamble}

\name{Kayla Orlinsky}
\course{Real Analysis Exam}
\term{Fall 2013}
\hwnum{Fall 2013}

\begin{document}

\begin{problem} $\,$
Let $\mu$ be a finite Borel Measure on $\mathbb{R}$, which is absolutely continuous with respect to the Lebesgue measure $m$. Prove that $x\mapsto\mu(A+x)$ is continuous for every Borel set $A\subset\mathbb{R}$.
\end{problem}


\begin{solution}$\,$
First, we will denote $A+x=A_x$. Now, because $A$ is Borel and $\mu(A)<\infty$, for all $\varepsilon>0$ there exists some set $E$ which is a finite union of open intervals such that $\mu(A\Delta E)<\varepsilon$.

Now, because $\mu<<m$, and \begin{align*}
    m(A_x\Delta E_x)&=m(A_x\cup E_x)-m(A_x\cap E_x)\\
    &=m((A\cup E)_x)-m((A\cap E)_x)\\
    &=m(A\cup E))-m(A\cap E)\\
    &=m(A\Delta E)
\end{align*}
we have that $\mu(A_x\Delta E_x)<\varepsilon$ for all $x$ (using the fact that we can ensure $E$ satisfies that $m(A_x\Delta E_x)<\delta$ and so $\mu(A_x\Delta E_x)<\varepsilon$ by absolute continuity).

Thus, since $|\mu(A_x)-\mu(E_x)|<\varepsilon$ for all $x$, it suffices to show that \begin{align*}
    f:&\mathbb{R}\to\mathbb{R}\\
    &x\mapsto \mu(E_x)
\end{align*} is continuous.

WLOG, let $E=\bigcup_{i=1}^N(a_i,b_i)$.

Let $\varepsilon>0$ be given and $x$ be fixed. Then let $\delta$ be such that $$m(F)<\delta\implies \mu(F)<\varepsilon$$ by absolute continuity.

Then, whenever $|x-y|<\frac{\delta}{N}$, we have that \begin{align*}
    |f(x)-f(y)|&=|\mu(E_x)-\mu(E_y)|\\
    &=|\mu(E_x)-(\mu(E_y\backslash E_x)+\mu(E_y\cap E_x))|\\
    &=|\mu(E_x)-\mu(E_x\cap E_y)-\mu(E_y\backslash E_x))|\\
    &=|\mu(E_x\backslash E_y)-\mu(E_y\backslash E_x)|\\
    &\le \mu(E_x\backslash E_y).
\end{align*}

Now, since for each interval of $E_x\backslash E_y$ we have $(a_i+x,b_i+x)\backslash (a_i+y,b_i+y)=I_{{x\backslash y}_i}$ has length at most $\frac{\delta}{N}$, so $$m(E_x\backslash E_y)\le\sum_{i=1}^Nm(I_{{x\backslash y}_i})<N\frac{\delta}{N}=\delta$$ so $\mu(E_x\backslash E_y)<\varepsilon$.

Thus, $f$ is continuous.
\end{solution}
\newpage

\begin{problem} $\,$
Let $f$ be a Lebesgue integrable function on $\mathbb{R}$, and assume that $$\sum_{n=1}^\infty\frac{1}{|a_n|}<\infty.$$
Prove that $g(x)=\sum_{n=1}^\infty f(a_nx)$ converges almost everywhere and is integrable on $\mathbb{R}$. Also, find an example of a Lebesgue integrable function $f$ on $\mathbb{R}$ such that $g(x)=\sum_{n=1}^\infty f(nx)$ converges almost everywhere but is not integrable.
\end{problem}


\begin{solution}$\,$
Let $(\mathbb{N},\nu)$ be the counting measure. Then since the $\sigma$-algebra of $\nu$ is the powerset of $\mathbb{N}$, $f$ is $m\times\nu$ measurable.

Let $M=\int|f(x)|dx.$

Furthermore, since $m$ and $\nu$ are $\sigma$-finite measure spaces, and $|f(a_nx)|\in L^+(m\times\nu)$, by Tonelli, \begin{align*}
    \int|g(x)|dx&=\int\left|\sum_{n=1}^\infty f(a_nx)\right|dx\\
    &\le\int\sum_{n=1}^\infty|f(a_nx)|dx\\
    &=\sum_{n=1}^\infty\int|f(a_nx)|dx\\
    &=\sum_{n=1}^\infty\int\frac{|f(u)|}{a_n}du\qquad \begin{matrix}
    u=a_nx   \\
    du=a_ndx
\end{matrix}\\
&=\sum_{n=1}^\infty\frac{M}{a_n}<\infty
\end{align*}since $$\sum_{n=1}^\infty \frac{1}{a_n}\le\sum_{n=1}^\infty \frac{1}{|a_n|}<\infty.$$

Now, let $f(x)=\chi_{[0,1]}(x)$. Then $f\in L^1$. Furthermore, $$f(nx)=\chi_{[0,1]}(nx)=\chi_{[0,\frac{1}{n}]}(x)\qquad\text{ since }0\le nx\le 1\implies 0\le x\le\frac{1}{n}.$$

$$\sum_{n=1}^\infty\chi_{[0,\frac{1}{n}]}(x)<\infty$$ for all $x\not=0$ since for all $x\in[0,\frac{1}{n}]$, there exists some $N$ such that $\frac{1}{N}<x$ so $\sum_{n=N}^\infty f(nx)=0$. Thus, $g(x)$ converges a.e..

Now, since $f(nx)\in L^+(m\times\nu)$ by the same reasoning as before, $$\int\sum_{n=1}^\infty \chi_{[0,\frac{1}{n}]}(x)dx=\sum_{n=1}^\infty\int\chi_{[0,\frac{1}{n}]}(x)dx=\sum_{n=1}^\infty m([0,\frac{1}{n}])=\sum_{n=1}^\infty\frac{1}{n}=\infty.$$
\end{solution}
\newpage

\begin{problem} $\,$
Assume $b>0$. Show that the Lebesgue Integral $$\int_1^\infty x^{-b}e^{\sin x}\sin(2x)dx$$ exists if and only if $b>1$.
\end{problem}


\begin{solution}$\,$
\boxed{b>1} $$\int_1^\infty x^{-b}e^{\sin x}\sin(2x)dx\le\int_1^\infty ex^{-b}dx<\infty$$ by $p$-test and since $|\sin a|\le1$ for all $a$.

\boxed{0<b\le 1} $$\int_1^\infty x^{-b}e^{\sin x}\sin(2x)dx\ge\sum_{n=1}^\infty\int_{n\pi/12}^{(4n+1)\pi/12} e^{-1}x^{-b}\frac{1}{2}dx=\infty.$$

In other words, we restrict the domain to where $\sin(2x)\ge\frac{1}{2}$ and this integral is still infinite. This implies that $\int f^+dx=\infty$. Similarly, restricting to where $\sin(2x)\le-\frac{1}{2}$ gives that $\int f^-dx=\infty$.

Thus, the integral does not exist.
\end{solution}
\newpage

\begin{problem} $\,$
Suppose that $F$ is the distribution function of a Borel measure $\mu$ on $\mathbb{R}$ with $\mu(\mathbb{R})=1$. Prove that $$\int_{-\infty}^\infty(F(x+a)-F(x))dx=a$$ for all $a>0$.
\end{problem}


\begin{solution}$\,$
First, since $x+a>x$ because $a>0$
$$\int_\mathbb{R}(F(x+a)-F(x))dx=\int_\mathbb{R}\mu([x,x+a))dx=\int_\mathbb{R}\int_\mathbb{R}\chi_{[x,x+a)}d\mu dm$$

Now, since $\mu$ is finite and $m$ is $\sigma$-finite, and $\chi_{[x,x+a)}\in L^+(m\times\mu)$ because $\mu$ is a Borel measure, the order of integration can be swapped and so we have $$\int_\mathbb{R}\int_\mathbb{R}\chi_{[x,x+a)}d\mu dm=\int_\mathbb{R}\int_\mathbb{R}\chi_{[x,x+a)}dmd\mu=\int_\mathbb{R} ad\mu=a\mu(\mathbb{R})=a.$$
\end{solution}
\vspace{0.5cm}

\end{document}
