\documentclass[12pt]{Qual}
\usepackage{preamble}

\name{Kayla Orlinsky}
\course{Real Analysis Exam}
\term{Spring 2015}
\hwnum{Spring 2015}

\begin{document}

\begin{problem} $\,$
Consider the sequence $$f_n(x)=\left(1+\frac{x}{n}\right)^{-n}\cos\left(\frac{x}{n}\right),\qquad n=1,2,...\,\,.$$

Evaluate $$\lim_n\int_0^\infty f_n(x)dx,$$ being careful to justify your answer.
\end{problem}


\begin{solution}$\,$
We would like to use Dominated Convergence Theorem.
\begin{enumerate}
    \item $\{f_n\}$ is measurable for all $n$.
    \item \begin{align*}
        y&=\lim_{n\to\infty}\left(1+\frac{x}{n}\right)^{-n}\\
        \implies \ln(y)&=\lim_{n\to\infty}-n\ln\left(1+\frac{x}{n}\right)\\
        &=\lim_{n\to\infty}\frac{\ln\left(1+\frac{x}{n}\right)}{\frac{-1}{n}}\\
        &=\lim_{n\to\infty}\frac{\frac{1}{1+\frac{x}{n}}\frac{-x}{n^2}}{\frac{1}{n^2}}\qquad\text{L'Hopital's Rule.}\\
        &=\lim_{n\to\infty}\frac{-x}{1+\frac{x}{n}}\\
        &=-x\\
        \implies y&=e^{-x}
    \end{align*}
    Thus, $$\lim_{n\to\infty}f_n(x)=\left(\lim_{n\to\infty}\left(1+\frac{x}{n}\right)^{-n}\right)\left(\lim_{n\to\infty}\cos\left(\frac{x}{n}\right)\right)=e^{-x}\cos(0)=e^{-x}$$ since both limits exist separately. Furthermore, this limit holds for all $x$.
    \item Now, for all $n>1$, $$\left(1+\frac{x}{n}\right)^{-n}\le\left(1+x\right)^{-n}\le(1+x)^{-2}\in L^1.$$

    Note that since $$\frac{1}{(1+x)^n}=\left(\frac{1}{1+x}\right)^n$$ and $\frac{1}{1+x}\le 1$ for all $x\ge0$, we have that $(1+x)^{-n}\le(1+x)^{-n+1}$ for all $n$.
\end{enumerate}

Thus, by the Dominated Convergence Theorem, $$\lim_{n\to\infty}\int_0^\infty f_n(x)dx=\int_0^\infty\lim_{n\to\infty}f_n(x)dx=-e^{-x}\bigg|_0^\infty=1.$$
\begin{comment}

    Now, using calculus, we let $g(x)=-n\ln\left(1+\frac{x}{n}\right)+x$. Since $x\ge0$, for all $n>1$, we have that $$\frac{d}{dx}g(x)=\frac{d}{dx}\left(-n\ln\left(1+\frac{x}{n}\right)+x\right)=\frac{-n}{n+x}+k=\frac{(k-1)n+x}{n+x}\ge0$$ and since $g(0)=0$, we have that $g(x)\ge0$ for all $x\ge0$ and so $$-n\ln\left(1+\frac{x}{n}\right)\ge -x\implies \left(1+\frac{x}{n}\right)^{-n}\ge e^{-x}.$$

    Finally, This implies that

    $$f_n(x)\le \left(1+\frac{x}{n}\right)^{-n}\le e^{-x}\in L^1([0,\infty)).$$
    \begin{center}
    \includegraphics[scale=0.75]{hw/Qual1.jpg}
\end{center}
\end{comment}
\end{solution}
\newpage

\begin{problem} $\,$
Suppose that $f:[0,\infty)\to\mathbb{R}$ is Lebesgue integrable.
\begin{enumerate}[label=(\alph*)]
    \item Show that there exists a sequence $x_n\to\infty$ such that $f(x_n)\to0$.
    \item Is it true that $f(x)$ must converge to $0$ as $x\to\infty$? Give a proof or a counter example.
    \item Suppose additionally that $f$ is differentiable and $f'(x)\to0$ as $x\to\infty$. Is it true that $f(x)$ must converge to $0$ as $x\to\infty$? Give a proof or counter example.
\end{enumerate}
\end{problem}


\begin{solution}$\,$
\begin{enumerate}[label=(\alph*)]
    \item Let $\{x_n\}$ be such that for all $n$, $x_n>n$ and $f(x_n)<\frac{1}{n}$.

    If no such sequence exists, then for all sequences with $x_n>n$, $f(x_n)\ge\frac{1}{n}$. However, then $$\int_n^\infty f(x)dx\ge\int_n^\infty\frac{1}{n}=\infty$$ which contradicts that $f\in L^1$. Thus, the sequence given exists.
    \item No. Let $f(x)=\chi_\mathbb{Q}$. Then $f\in L^1$ since $m(\mathbb{Q})=0$, however $\lim_{x\to\infty} f(x)$ does not exist.
    \item Assume that $f\not\to 0$ as $x\to\infty$. Then there exists some $\{x_n\}$ tending to infinity with $f(x_n)\ge\varepsilon$ for all $n$. (WLOG we take $f(x_n)\ge0$, however if $f$ is everywhere negative, then $-f(x_n)\ge\varepsilon$ and the rest of the proof is similar).

    Since $|f'(x_n)|\le \frac{\varepsilon}{2}$ for large enough $n$, and since differentiability implies continuity, we may apply the Fundamental Theorem of Calculus. (Note that on any closed interval $[x_n,x_n+1]$, $f$ must be bounded) so for all $x_n\le x\le x_n+1$ $$|f(x)-f(x_n)|=\left|\int_{x}^{x_n+1}f'(t)dt\right|\le\left|\int_{x_n}^{x_n+1}f'(t)dt\right|\le\int_{x_n}^{x_n+1}\frac{\varepsilon}{2} dt=\frac{\varepsilon}{2}.$$

    However, \begin{align*}
        |f(x)-f(x_n)|&\le\frac{\varepsilon}{2}\\
        -\frac{\varepsilon}{2}&\le f(x)-f(x_n)\\
      \varepsilon-\frac{\varepsilon}{2}&\le f(x_n)-\frac{\varepsilon}{2}\le f(x)
    \end{align*}

    However, then $$\int f(t)dt\ge \sum_{n=N}^\infty\int_{x_n}^{x_n+1}f(t)dt\ge\sum_{n=N}^\infty\frac{\varepsilon}{2}=\infty\qquad\qquad\qquad\text{\contradiction}$$

    Again, this contradicts $f\in L^1$ and so no such sequence can exist. Namely, $f\to0$ as $x\to\infty$.
\end{enumerate}
\end{solution}
\newpage

\begin{problem} $\,$
Define $f_n(x)=ae^{-nax}-be^{-nbx}$ where $0<a<b$.
\begin{enumerate}[label=(\alph*)]
    \item Show that $$\sum_{n=1}^\infty\int_0^\infty f_n(x)dx=0$$ and $$\int_0^\infty\sum_{n=1}^\infty f_n(x)dx=\log(b/a).$$
    \item What can you deduce about the value of $$\int_0^\infty\sum_{n=1}^\infty|f_n(x)|dx?$$
\end{enumerate}
\end{problem}


\begin{solution}$\,$
\begin{enumerate}[label=(\alph*)]
    \item \begin{align*}
        \int_0^\infty f_n(x)dx&=\int_0^\infty ae^{-nax}-be^{-nbx}dx\\
        &=\frac{ae^{-nax}}{-na}-\frac{be^{-nbx}}{-nb}\bigg|_0^\infty\\
        &=0-\left(\frac{-1}{n}+\frac{1}{n}\right)=0.
    \end{align*}

    Thus, $$\sum_{n=1}^\infty\int_0^\infty f_n(x)dx=0.$$

    Now, using the convergence of Geometric Series (because $e^{ax}\ge1$ for all $a>0$ and all $x\ge0$), we have that $$\sum_{n=1}^\infty f_n(x)=a\sum_{n=1}^\infty\left(\frac{1}{e^{ax}}\right)^n-b\sum_{n=1}^\infty\left(\frac{1}{e^{bx}}\right)^n=\frac{ae^{-ax}}{1-e^{-ax}}-\frac{be^{-bx}}{1-e^{-bx}}.$$

    Thus, \begin{align*}
        \int_0^\infty\sum_{n=1}^\infty f_n(x)dx&=\int_0^\infty\frac{ae^{-ax}}{1-e^{-ax}}dx-\int_0^\infty\frac{be^{-bx}}{1-e^{-bx}}dx\\
        &=\int_0^1\frac{du}{u}-\int_0^1\frac{dw}{w}\qquad\qquad \begin{matrix}
    u=1-e^{-ax} &   x:[0,\infty]\\
    du=ae^{-ax} & u:[0,1]
\end{matrix}\quad\text{ similarly for }w\\
        &=\ln(u)-\ln(w)\\
        &=\ln|1-e^{ax}|-\ln|1-e^{-bx}|\bigg|_0^\infty\\
        &=\ln(1)-\lim_{x\to0}\ln\left(\frac{1-e^{-ax}}{1-e^{-bx}}\right)\\
        &=\lim_{x\to0}\ln\left(\frac{1-e^{-bx}}{1-e^{-ax}}\right)\qquad\text{ absorbing the negative}\\
        &=\ln\left(\lim_{x\to0}\frac{1-e^{-bx}}{1-e^{-ax}}\right)\qquad\ln \text{ is continuous}\\
        &=\ln\left(\lim_{x\to0}\frac{be^{-bx}}{ae^{-ax}}\right)\qquad\text{L'Hopital's Rule}\\
        &=\ln\left(\frac{b}{a}\right).
    \end{align*}

   Note that it was necessary for $b>a>0$.
    \item $f_n(x)$ is certainly a continuous function for all $x$ and all $n$, thus $f_n$ is measurable. Furthermore, if $(\mathbb{N},\nu)$ is the counting measure space, then $f_n(x)$ will certainly be measurable with respect to $m\times\nu$.

    Since both $([0,\infty),m)$ and $(\mathbb{N},\nu)$ are $\sigma$-finite measure spaces, and $|f_n(x)|\in L^+(m\times\nu)$, by Tonelli, the integral and summation of $|f_n(x)|$ can be swapped.

    However, from (a), we saw that swapping the order for $f_n(x)$ gave different results. It must then be the case that Fubini does not apply to $f_n(x)$ and so $f_n(x)\notin L^1(m\times \nu)$. Thus, $$\int|f_n(x)|d(m\times\nu)=\int_0^\infty\sum_{n=1}^\infty|f_n(x)|dx=\infty$$
\end{enumerate}
\end{solution}
\newpage

\begin{problem} $\,$
Assume that $f$ is integrable on $[0,1]$ with respect to the Lebesgue measure $m$, and let $F(x)=\int_0^xf(t)dt.$ Assume that $\phi:\mathbb{R}\to\mathbb{R}$ is Lipschitz, i.e., there exists a constant $C\ge0$ such that $$|\phi(x_1)-\phi(x_2)|\le C|x_1-x_2|,\qquad x_1,x_2\in\mathbb{R}.$$

Prove that there exists a function $g$ which is integrable on $[0,1]$ such that $\phi(F(x))=\int_0^xg(t)dt$ for $x\in[0,1]$.
\end{problem}


\begin{solution}$\,$
First, since $F:[0,1]\to\mathbb{R}$ and $F(x)-F(0)=F(x)=\int_0^xf(t)dt$ with $f\in L^1([0,1])$, by the Fundamental Theorem of Lebesgue Integrals, $F$ is absolutely continuous.

Furthermore, we may replace $f$ with $F'$ (as the two are equal a.e.).

Now, $\phi$ is certainly absolutely continuous. If $C=0$, then $\phi$ is contant and absolute continuity is immediate. If $C>0$, then for all $\varepsilon>0$, letting $\delta=\frac{\varepsilon}{C}$, for all finite disjoint collections of intervals $\{(a_i,b_i)\}_1^n$ satisfying that $$\sum_{i=1}^n(b_i-a_i)<\delta$$ we have that $$\sum_{i=1}^n|\phi(b_i)-\phi(a_i)|\le\sum_{i=1}^nC|b_i-a_i|=C\sum_{i=1}^n(b_i-a_i)<C\delta=C\frac{\varepsilon}{C}=\varepsilon.$$

Thus, $\phi$ is absolutely continuous. Finally, let $\varepsilon>0$ be given. Let $\delta_F$ and $\delta_\phi$ be the associated constants for the definition of absolute continuity of $F$ and $\phi$ respectively.

Then let $$\delta=\min\left\{\delta_F,\delta_\phi\right\}.$$

Then, for any finite collection of disjoint intervals $\{(a_i,b_i)\}_1^n$ satisfying $$\sum_{i=1}^n(b_i-a_i)<\delta,$$ we have that $$\sum_{i=1}^n|\phi(F(b_i))-\phi(F(a_i))|\le\sum_{i=1}^nC|F(b_i)-F(a_i)|=C\sum_{i=1}^n|F(b_i)-F(a_i)|<C\varepsilon.$$

Thus, $\phi(F(x))$ is absolutely continuous and since $\phi(F(x)):[0,1]\to\mathbb{R}$, by the Fundamental Theorem of Lebesgue Integrals, there must exist a function $g\in L^1([0,1])$ such that $$\phi(F(x))-\phi(F(0))=\phi(F(x))-\phi(0)=\int_0^xg(t)dt.$$

With possibly shifting $g$ by a constant we obtain our result.
\end{solution}
\vspace{0.5cm}

\end{document}
