\documentclass[12pt]{Qual}
\usepackage{preamble}

\name{Kayla Orlinsky}
\course{Complex Analysis Exam}
\term{Fall 2018}
\hwnum{Fall 2018}

\begin{document}

\begin{problem} $\,$
Let $a>0$. Compute $$\int_0^\pi\frac{d\theta}{a^2+\sin^2\theta}$$
\end{problem}


\begin{solution}$\,$
Since $\sin^2\theta=\sin^2(\theta+\pi)$, we have that $$\int_0^{2\pi}\frac{d\theta}{a^2+\sin^2\theta}=2\int_0^\pi\frac{d\theta}{a^2+\sin^2\theta}$$ and so

\begin{align*}
    \int_0^\pi\frac{d\theta}{a^2+\sin^2\theta}&=\frac{1}{2}\int_0^{2\pi}\frac{d\theta}{a^2+\sin^2\theta}\\
    &=\frac{1}{2}\int_0^{2\pi}\frac{d\theta}{a^2+\frac{1-\cos\theta}{2}}\\
    &=\frac{1}{2}\int_0^{2\pi}\frac{2d\theta}{2a^2+1-\frac{e^{i\theta}+e^{-i\theta}}{2}}\\
    &=\frac{1}{2}\int_0^{2\pi}\frac{4d\theta}{4a^2+2-e^{i\theta}-e^{-i\theta}}\\
    &=\frac{1}{2}\int_0^{2\pi}\frac{-4e^{i\theta}d\theta}{e^{2i\theta}-(4a^2+2)e^{i\theta}+1}\\
    &=\frac{1}{2}\int_{\{|z|=1\}}\frac{4idz}{z^2-(4a^2+2)z+1}\qquad z=e^{i\theta}\\
    &=\frac{4i}{2}\int_{\{|z|=1\}}\frac{dz}{(z-\alpha)(z-\beta)}\tag{1}\\
    &=2i\left(2\pi i\res_{z=\beta}\frac{1}{(z-\alpha)(z-\beta)}\right)\tag{2}\\
    &=-4\pi\frac{1}{\beta-\alpha}\\
    &=\frac{-4\pi}{-4a\sqrt{a^2+1}}\\
    &=\frac{\pi}{a\sqrt{a^2+1}}\\
\end{align*}

with (1) since \begin{align*}
    x^2-(4a^2+2)x+1=0\implies x&=\frac{4a^2+2\pm\sqrt{(4a^2+2)^2-4}}{2}\\
    &=\frac{4a^2+2\pm\sqrt{16a^4+16a^2+4-4}}{2}\\
    &=\frac{4a^2+2\pm\sqrt{16a^4+16a^2}}{2}\\
    &=\frac{4a^2+2\pm4a\sqrt{a^2+1}}{2}\\
    &=2a^2+1\pm2a\sqrt{a^2+1}\\
    \alpha&=2a^2+1+2a\sqrt{a^2+1}\\
    \beta&=2a^2+1-2a\sqrt{a^2+1}
\end{align*}

and (2) because $\alpha>1$ for all $a>0$.
\end{solution}
\newpage



\begin{problem} $\,$
Find the number of solutions of the equation $z-2-e^{-z}=0$ in $H=\{z\in\mathbb{C}:\re(z)>0\}.$
\end{problem}

\begin{solution}$\,$
Let $f(z)=z-2-e^{-z}$. Then $$f(iy)=iy-2-e^{-iy}=iy-2-\cos(y)-i\sin(y)=-2-\cos(y)+i(y-\sin(y))$$ and so $\re(f(iy))\le -1<0$ for all $y\in\mathbb{R}$. Thus, $f$ sends the imaginary axis to the left-half plane, away from the origin.

Now, if $z-2-e^{-z}=0$ then $$|e^{-z}|=e^{\re(-z)}=|z-2|\ge|z|-2>1\qquad\text{ for all }|z|>3$$ and so for if $z$ is a root of $f$ and $|z|>3$, then $\re(-z)=-\re(z)>0$ and so $\re(z)<0$ and $z$ lies in the left-half plane. Namely, the total change in argument is at most $\pi.$

Now, let $z=Re^{i\theta}$ for $\theta\in(-\pi/2,\pi/2).$

Then $$\frac{1}{R}f(Re^{i\theta})=e^{i\theta}-\frac{2}{R}-\frac{e^{-Re^{i\theta}}}{R}\to e^{i\theta}\qquad R\to\infty$$ and so by the argument principle, $f$ has a total change in argument of $\pi$.

Namely, $f$ can have at most one zero in the right half plane.

Since $f(0)<0$ and $f(10)=8-\frac{1}{e^{10}}>7>0$ by the intermediate value theorem, $f$ has a zero on the positive real axis.

Namely, $f$ has exactly one zero in the right-half plane.

\end{solution}
\newpage




\begin{problem} $\,$
Let $\Omega\not=\mathbb{C}$ be simply connected and let for any $c\in\Omega$, the mapping $\varphi_c:\Omega\to\mathbb{D}=\{z\in\mathbb{C}:|z|<1\}$ be conformal so that $\varphi_c(c)=0.$ Let $g_c(z)=\log|\varphi_c(z)|,$ $z\in\Omega\backslash\{c\}.$

Show that $g_a(b)=g_b(a)$ for any distinct $a,b\in\Omega.$
\end{problem}


\begin{solution}$\,$
Let $a,b\in\Omega$ be distinct and let \begin{align*}
    T:\mathbb{D}&\to\mathbb{D}\\
    z&\mapsto\frac{z-\varphi_a(b)}{1-\overline{\varphi_a(b)}z}
\end{align*} Then $T$ is a conformal map and isomorphism of the unit disk.

Note that $\varphi_c$ being conformal for all $c$, implies that $\varphi_c$ is locally invertible by the inverse function theorem.

 Now, we examine $f(z)=(T\circ\varphi_a\circ\varphi_b^{-1})(z)$.

Then $f:\mathbb{D}\to\mathbb{D}$ with $$f(0)=T(\varphi_a(\varphi_b^{-1}(0)))=T(\varphi_a(b))=0.$$

Therefore, by Schwarz' Lemma, $$|f(z)|\le|z|\qquad z\in\mathbb{D}$$ and so namely, \begin{align*}
    |f(\varphi_b(a))|&=|T(\varphi_a(\varphi_b^{-1}(\varphi_b(a))))|\\
    &=|T(\varphi_a(a))|\\
    &=|T(0)|\\
    &=|\varphi_a(b)|\\
    &\le|\varphi_b(a)|
\end{align*}

Similarly, we can show using a different isomorphism of the disk that $$|\varphi_b(a)|\le|\varphi_a(b)|$$ and so $$|\varphi_a(b)|=|\varphi_b(a)|\implies g_a(b)=g_b(a).$$
\end{solution}
\newpage





\begin{problem} $\,$
Let $a\in\mathbb{D}=\{z\in\mathbb{C}:|z|<1\},$ and $$f_a(z)=\frac{a-z}{1-\overline{a}z},\qquad z\in\overline{\mathbb{D}}.$$ Show that $f_a$ is a holomorphic bijective mapping of $\mathbb{D}$  onto $\mathbb{D}$ which is its own inverse.
\end{problem}


\begin{solution}$\,$
$f$ is a Mobius Transform, and so namely it is bijective with inverse $$f_a^{-1}(z)=\frac{z-a}{\overline{a}z-1}=\frac{a-z}{1-\overline{a}z}=f_a(z).$$

Note that being able to write the inverse directly implies that $f$ is bijective.

Since $|a|<1$, $\left|\frac{1}{\overline{a}}\right|=\frac{1}{|a|}>1$ and so $f_a$ does not have any poles inside the unit disk. Namely, $f_a$ is holomorphic, bijective, and analytic in the unit disk.
\end{solution}


\end{document}
