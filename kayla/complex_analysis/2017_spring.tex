\documentclass[12pt]{Qual}
\usepackage{preamble}

\name{Kayla Orlinsky}
\course{Complex Analysis Exam}
\term{Spring 2017}
\hwnum{Spring 2017}

\begin{document}

\begin{problem} $\,$
Let $\gamma$ be a circle with radius $1$ and center $0$ with positive direction of integration. Compute $$\int_\gamma e^{1/z}dz$$
\end{problem}


\begin{solution}$\,$
\begin{align*}
    \int_{|z|=1}e^{1/z}dz&=\int_\{|z|=1\}\sum_{n=0}^\infty\frac{1}{z^nn!}dz\\
    &=\sum_{n=0}^\infty\int_{|z|=1}\frac{1}{z^nn!}dz\qquad\text{ since both the integral and sums exist}\\
    &=\sum_{n=1}^\infty\int_{|z|=1}\frac{1}{z^nn!}dz\qquad\text{first integral is of an analytic function so it dies}\\
    &=\sum_{n=1}^\infty\left(\frac{2\pi if^{(n-1)}(0)}{(n!)^2}\right)\qquad f(z)=1\\
    &=2\pi i
\end{align*}
\end{solution}
\newpage



\begin{problem} $\,$
Assume that $f$ is a holomorphic function on the unit disk $D=\{z\in\mathbb{C}:|z|<1\}$ satisfying $$f(z)^3=\overline{f(z)}\qquad z\in D.$$ Prove that $f$ is constant.
\end{problem}

\begin{solution}$\,$
$$f^4(z)=\overline{f(z)}f(z)=|f(z)|^2\implies f^2(z)=|f(z)|\in\mathbb{R}\qquad z\in D.$$

Thus, for each $z\in D$, $f(z)$ must be either purely real or purely imaginary. Namely, $$f(D)\subset\{x:x\in\mathbb{R}\}\bigcup\{iy:y\in\mathbb{R}\}.$$

However, by the open mapping theorem, if $f$ is nonconstant, then $f(D)$ must be an open subset of the plane, since $f$ is holomorphic. Since subsets of the real and imaginary lines are not open in $\mathbb{C}$, $f$ must be constant.

\begin{mybox}
Alternatively, $$|f(z)|^3=|\overline{f(z)}|=|f(z)|\implies |f(z)|^3=|f(z)|\implies |f(z)|=0\text{ or }1\forall z\in\mathbb{D}.$$

Thus, $f$ is either identically $0$ (constant) or $f$ sends the open disk to the unit circle which contradicts the open mapping theorem.
\end{mybox}
\end{solution}
\newpage




\begin{problem} $\,$
Let $f(z)=\sum_{n=1}^\infty z^{n!}$.
\begin{enumerate}[label=(\alph*)]
    \item Show that $f(z)$ is holomorphic in the unit disk $D=\{z\in\mathbb{C}:|z|<1\}$.
    \item Show that $f$ does not have any holomorphic extension, that is, there exists no $g$ holomorphic on some open set $U\supset D$ such that $U\not=D$ and $f=g|_D$. (Hint: Consider $e^{i\theta}$ where $\theta$ is rational.)
\end{enumerate}
\end{problem}


\begin{solution}$\,$
This question is nearly identical to \textbf{Spring 2015: Problem 2}. Thus, the proof here is the same as the one given there.
\begin{enumerate}[label=(\alph*)]
    \item $f$ is holormophic if and only if the sum converges uniformly. For $|z|<r<1$, we have that $|z|^{n!}\le r^{n!}\le r^n$ since $r<1$, and so $f$ converges uniformly by Weierstrass M-test on $\{|z|<r\}.$ Since this holds for all $r<1$, we have that $f$ converges uniformly and is therefore analytic on $D.$
    \item Let $D\subsetneq U$ open. Since $U\not=D$ and $U$ is open, $U$ must contain some $z_0$ with $|z_0|=1.$

    Again, since $U$ is open, any neighborhood of $z_0$ must contain some $z=e^{i\pi/m}$ for $m\in\mathbb{Z}$ since $|z|=1.$

    However, then $$f(z)=\sum_{n=0}^{m-1}e^{in!\pi/m}+\sum_{n=m}^\infty e^{i\pi n!/m}=\sum_{n=0}^{m-1}e^{in!\pi/m}+\sum_{n=m}^\infty 1=\infty$$ since $n!/m\in\mathbb{Z}$ for $n\ge m.$

    Now, $f$ has a non-removable singularity in every neighborhood of $z_0$.

    Namely, $f$ cannot be extended to a neighborhood to $z_0$ since it will not converge in any punctured neighborhood of $z_0.$
\end{enumerate}
\end{solution}
\newpage





\begin{problem} $\,$
Let $f$ be an entire function such that $$f(z+m+ni)=f(z),\qquad z\in\mathbb{C}\qquad m,n\in\mathbb{Z}.$$ Prove that $f$ is a constant function.
\end{problem}


\begin{solution}$\,$
WLOG, let $f(0)=0$, else we can let $g(z)=f(z)-f(0)$.

Let $z\in\mathbb{C}$. Then let $m$ be the integer component of $\re(z)$ and $n$ be the integer part of $\im(z).$ Then for some $\alpha,\beta\in(0,1),$ we can write $$z=(m+\alpha)+(n+\beta)i\implies f(z)=f((\alpha+\beta i)+(m+ni))=f(\alpha+\beta i).$$

Namely, $f$ is uniquely determined by the values in takes in $\mathbb{D}=\{|z|<1\}$. Since $f$ is entire, by the maximum modulus principle, $f$ must attain a maximum on $\partial\overline{\mathbb{D}}=\{|z|=1\}.$

Now, because $f$ is entire, it has no poles, so $$M=\sup_{z\in\partial\overline{\mathbb{D}}}|f(z)|<\infty$$ and $$|f(z)|\le M\qquad\text{ for all }z\in\mathbb{C}$$ so by Louiville's $f$ is entire and bounded so it is constant.
\end{solution}


\end{document}
