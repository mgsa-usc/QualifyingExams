\documentclass[12pt]{AlgebraQual}
\usepackage{preamble}

\name{Kayla Orlinsky}
\course{Algebra Exam}
\term{Fall 2011}
\hwnum{Fall 2011}

\begin{document}

\begin{problem} $\,$
Let $I$ and $J$ be ideals of $R=\mathbb{C}[x_1,x_2,...,x_n]$ that define the same variety of $\mathbb{C}^n$. Show that for any $x\in (I+J)/I$ there is $m=m(x)>0$ with $x^m=0_{R/I}$. Show that there is an integer $M>0$ so that for any $y_1,y_2,...,y_M\in (I+J)/I$, $y_1y_2\cdots y_m=0_{R/I}.$
\end{problem}


\begin{solution}$\,$
To do this, we simply note the following claim.

\begin{claim} for $I,J$ ideals of $k[x_1,...,x_n]$, $V(I+J)=V(I)\cap V(J)$
\begin{proof} \boxed{\supset} Let $\alpha\in V(I)\cap V(J)$. Then $f(\alpha)=0$ for all $f\in I$ and $g(\alpha)=0$ for all $g\in J$.

However, $$I+J=\{\sum_{i=1}^kf_ig_i\,|\,f_i\in I,g_i\in J\}.$$ Therefore, since $f_i$ vanish and $g_i$ vanish at $\alpha$ for all $i$, we get that $any h\in I+J$ will also vanish at $\alpha.$

Namely, $V(I+J)\supset V(I)\cap V(J).$

\boxed{\subset} Since $I\subset I+J$ and $J\subset I+J$, $V(I+J)\subset V(I)$ and $V(I+J)\subset V(J)$. Therefore, $V(I+J)\subset V(I)\cap V(J).$
\end{proof}
\end{claim}

Now, by the \textbf{Claim 1}, $$V(I+J)=V(I)\cap V(J)=V(I)\cap V(I)=V(I)\qquad V(I)=V(J)\text{ by assumption.}$$

Therefore, if $f(x)\in I+J$, then for all $\alpha\in V(I),$ $\alpha\in V(I+J)$ and so $f(\alpha)=0.$

Thus, by Nullstellenzatz Part II, there exists an $m>0$ such that $f^m\in I$.

Namely, if $\overline{f}\in (I+J)/I$, there exists $m$ such that $\overline{f}=0\in R/I$.

Finally, since $R$ is Noetherian by the Hilbert Basis Theorem, all ideals of $R$ are finitely generated.

Therefore, if $J=(f_1(x),...,f_n(x))$ we can let $m_i$, $i=1,...,n$ be the values found earlier such that $f_i^{m_i}(x)\in I$ for all $i$. Let $m=\max\{m_i\}$.

We would like to show that $[(I+J)/I]^{nm}=0$. Note that $$[(I+J)/I]^{nm}\cong[J/(I\cap J)]^{nm}\cong J^{nm}/(I\cap J)$$ by the second isomorphism theorem.

Now, we can let $$J=Rf_1(x)\oplus\cdots\oplus Rf_n(x).$$ Then $$J^{nm}=\bigoplus_{r_1+\cdots+r_n=nm}Rf_1^{r_1}(x)\cdots f_n^{r_n}(x).$$

Since $r_1+\cdots+r_n=nm,$ there must exist some $r_i\ge m$. Otherwise, $r_1+\cdots+r_n<nm.$ However, then $f_i^{r_i}(x)\in I$ by the above and so then $J^{nm}\subset I\cap J$ and so namely, $$J^{nm}/(I\cap J)\cong [(I+J)/I]^{nm}=0\in R/I.$$

Therefore, $M=nm$ is such that any product of $M$ things in $(I+J)/I$ will be trivial.
\end{solution}
\newpage


\begin{problem} $\,$.
If $K\subset L$ are finite fields with $|K|=p^n$ and $[L:K]=m$ then show that for each $1\le t<nm$, any $a\in L-K$ has a $p^t$-th root in $L$. When $m=3,$ show that every $b\in K$ has a cube root in L.
\end{problem}


\begin{solution}$\,$
Since $|L|=p^{nm}$ $L$ is the splitting field of $x^{p^{nm}}-x.$

Therefore, for all $a\in L$ and for all $1\le t<nm$ $$a=a^{p^m}=a^{p^{nm-t}p^t}=\left(a^{p^{nm-t}}\right)^{p^t}.$$

Now, let $m=3$, and let $b\in K$.

Then we can let \begin{align*}
    \varphi:L^\times&\to L^\times\\
    a&\mapsto a^3
\end{align*}

Since $L^\times\cong\langle a\rangle$ is a cyclic group of order $p^{3n}-1$, we have two cases. First, if $3$ is coprime to $p^{3n}-1$, then $\varphi$ is a group isomorphism. Namely, every element of $L$ (and thus $K$) has a cube root in $L$.

If $3$ is not coprime to $p^{3n}-1$, then $3|(p^{3n}-1)$ so $p^{3n}=3t+1$ some $t.$

Now, we want to show that $K^\times\subset\varphi(L^\times)$ since this will show that every $b\in K$ can be written as some $(a^x)^3$ for $a^x\in L.$

Now, if $a^x\in\ker\varphi$, then $(a^x)^3=a^{3x}=1$. However, then $\frac{p^{3n}-1}{3}$ divides $x$ since the order of $a$ is $p^{3n}-1$. The converse is clearly also true. Thus, $x=0,\frac{p^{3n}-1}{3},2\frac{p^{3n}-1}{3}$ and these are the only possibilities. Namely, $$|\varphi(L^\times)|=\frac{|L^\times|}{|\ker\varphi|}=\frac{p^{3n}-1}{3}.$$

Now, since $\varphi(L^\times)=\langle a^x\rangle$ is also cyclic, letting $K^\times=\langle a^y\rangle$ for some $y$, we have that $K^\times\subset\varphi(L^\times)$ if $x|y$.

\begin{claim} $x$ divides $y$ if and only if $(p^n-1)$ divides $\frac{p^{3n}-1}{3}$.
\begin{proof} \boxed{\implies} If $x|y$ then $K^\times\subset\varphi(L^\times)$ and so $|K^\times|=p^n-1$ must divide $|\varphi(L^\times)|=\frac{p^{3n}-1}{3}$.

\boxed{\impliedby} Assume $(p^n-1)$ divides $\frac{p^{3n}-1}{3}$. Then the order of $a^y$ divides the order of $a^x$.

Namely, $$(a^x)^\frac{p^{3n}-1}{3}=(a^x)^{(p^n-1)t}=(a^{xt})^{p^n-1}=1=(a^y)^{p^n-1}$$ for some $t.$

And so the order of $a^{xt}$ is $p^n-1$ as well. Thus, $\langle a^{xt}\rangle=\langle a^y\rangle$ and so $x|y.$
\end{proof}
\end{claim}

From the claim, we need only show that $(p^n-1)$ divides $\frac{p^{3n}-1}{3}$.

However, $\frac{p^{3n}-1}{3}=\frac{(p^n-1)(p^{2n}+p^n+1)}{3}$.

If $3|(p^{2n}+p^n+1)$ then we are done, so assume not.

Then $3|(p^n-1)$. Namely, $p^n\equiv 1\mod 3$. So $p^{2n}+p^n+1\equiv 1+1+1\equiv 0\mod 3$ and so again, $3|(p^{2n}+p^n+1)$.

Therefore, $p^n-1$ divides $(p^n-1)\frac{p^{2n}+p^n+1}{3}$ and so $K^\times\subset\varphi(L^\times)$ is a subgroup.

Finally, this gives that for every $b\in K$, $b$ has a cube root in $L$.
\end{solution}
\newpage



\begin{problem} $\,$
Let $F$ be an algebraically closed field and $A$ an $F$-algebra with $\dim_FA=n$. If every element of $A$ is either nilpotent or invertible, show that the set of nilpotent elements of $A$ is an ideal $M$ of $A,$ that $M$ is the unique maximal ideal of $A$, and that $\dim_FM=n-1.$
\end{problem}


\begin{solution}$\,$
Let $M$ be the set of nilpotent elements of $A$. Namely, $M$ is the nilradical of $A$. $M$ is always an ideal since $A$ is commutative (being an algebra) and so if $x,y\in M$ with $x^s=0$ and $y^t=0$, then $(x-y)^{st}=0$ and so $x-y\in M.$ Similarly, if $a\in A$ then $(ax)^s=a^sx^s=0$ so $ax\in M$.

Thus, $M$ is an ideal.

Now, let $M\subsetneq M'\subset A$ with $M'$ another ideal of $A$.

Then let $x\in M'$ and $x\notin M$. Since $x$ is not nilpotent, it is invertible. Therefore, $x^{-1}x=1\in M'$ and so $M'=A$.

Thus, $M$ is maximal.

Using this same argument, we get that $M$ must be unique, since any other element not in $M$ is invertible and so cannot be contained in any proper ideal.

Finally, since $A/M$ is a field, it is a field extension of $F$. However, since $F$ is algebraically closed, $A/M\cong F$.

Therefore, $$1=\dim_F(A/M)=\dim_FA-\dim_FM=n-\dim_FM\implies \dim_FM=n-1.$$
\end{solution}
\newpage



\begin{problem} $\,$
Let $M$ be a finitely generated $F[x]$ module, for $F$ a field.
\begin{enumerate}[label=(\alph*)]
    \item Show that if $f(x)m=0$ for $f(x)\not=0$ forces $m=0$, then $M$ is a projective $F[x]$ module.
    \item If $H$ is an $F[x]$ submodule of $M$ show that $M=H\oplus K$ for a submodule $K$ of $M$ if and only if: $f(x)m\in H$ for $f(x)\not=0$ implies $m\in H.$
\end{enumerate}
\end{problem}


\begin{solution}$\,$
\begin{enumerate}[label=(\alph*)]
    \item Since $M$ is finitely generated over $F[x]$ which is a PID, we may apply the structure theorem. Note also that because $F[x]$ is a PID, projective is equivalent to free.

    Thus, by the structure theorem, $$M\cong P\oplus T(M)$$ with $P$ the free part of $M$ and $T(M)$ the torsion part.

    Now, let $m\in T(m)$. Then there exists $f(x)\in F[x]$ with $f(x)\not=0$ such that $f(x)m=0$. However, by assumption, this implies that $m=0$.

    Thus, $T(M)=0$ and so $M\cong P$ for some free module $P$. Therefore, $M$ is free and so it is projective.

    \item \boxed{\implies} Assume $H$ is an $F[x]$ submodule of $M$.

    Further, assume that $M=H\oplus K$ for a submodule $K$ of $M$.

    Because $M$ is free, $H$ is free, and so $H\cap K=(0)$ because $H$ is projective.

    Now, let $f(x)m\in H$ for $f(x)\not=0$. If $m\notin H$, then $m\in K$ because $M=H\oplus K$ and $H\cap K=(0)$.

    However, then $f(x)m\in K$ because $K$ is a submodule, which is a contradiction.

    Thus, $m\in H$.

    \boxed{\impliedby} Assume if $f(x)m\in H$ and $f(x)\not=0$, then $m\in H$.

    Then, by (a), $H$ is projective, and so letting $\varphi:M\to H$ be any surjective homomorphism, (which exists since both $M$ and $H$ are free and have bases over $F[x]$) we get a short exact sequence of the form
    \begin{center}
        \begin{tikzcd}
        0 \arrow[r] & K \arrow[r] & M \arrow[r] & H \arrow[r] & 0
        \end{tikzcd}
    \end{center} where $K=\ker\varphi$.

    Since $H$ is projective, the sequence is split and $M\cong H\oplus K$ so we are done.
\end{enumerate}
\end{solution}
\newpage



\begin{problem} $\,$
Up to isomorphism, describe the possible structures of any group of order $987=3\cdot 7\cdot 47.$
\end{problem}


\begin{solution}$\,$
\boxed{\text{Abelian}} There is an abelian group of order $987$ isomorphic to $$G\cong\mathbb{Z}_3\times\mathbb{Z}_7\times\mathbb{Z}_{47}.$$

Now, for $G$ non-abelian, using the Sylow theorems it is immediate that $n_{47}=1$ since $n_{47}|21$ and $n_{47}\equiv 1\mod 47$ so $n_{47}=1$.

Thus, $G$ has a normal Sylow $47$-subgroup. Let $P_{47}$ be the normal Sylow $47$-subgroup, and $P_3$, $P_7$ be Sylow $3$-subgroups and Sylow $7$-subgroups.

\begin{claim} If $N$ is normal in $G$ and $P$ is a normal Sylow $p$-subgroup of $N$, then $P$ is normal in $G$.
\begin{proof} Let $N$ be normal in $G$ and $P$ be a normal Sylow $p$-subgroup of $N$.

Let $g\in G$. Then $gNg^{-1}=N$, therefore, since $P$ is a subgroup of $N$, $gPg^{-1}\subset N$.

Therefore, if $p\in P$, $gpg^{-1}=n\in N$. However, conjugation is an automorphism and preserves order, so $n\in N$ has order dividing $|P|$. Thus, $n$ lies in some Sylow $p$-subgroup of $N.$ However, since $P$ is normal in $N,$ $P$ is the only Sylow $p$-subgroup of $N$ and so $n\in P$.

Thus, $G=N_G(P)$.
\end{proof}
\end{claim}

Therefore, since $P_{47}$ is normal, $P_7P_{47}$ is a subgroup of $G$ and since it has index $3$ which is the smallest prime dividing the order of $G$, it is normal by \textbf{Spring 2010: Problem 2 Claim 1}.

Clearly $P_7$ is normal in $P_7P_{47}$ since $n_7|47$ and $n_7\equiv 1\mod 7$ so $n_7=1$ so by \textbf{Claim 3}, $P_7$ is normal in $G.$

\boxed{P_{47}\rtimes P_3\times P_7}. Let $\varphi:P_3P_7\to\Aut(P_{47})$. Since $P_{47}\cong\mathbb{Z}_{47}$, $$\varphi:\mathbb{Z}_{21}\to\mathbb{Z}_{47}^\times\cong\mathbb{Z}_{46}.$$

Since $\mathbb{Z}_{46}$ has no elements of order $3$ or order $7$, $\varphi$ can only be the trivial homomorphism and this gives no new group structures aside from the abelian one.

Now, since $P_{47}$ is normal, $P_{47}P_3$ and $P_{47}P_7$ are subgroups of $G$.

\boxed{P_3\rtimes P_7\times P_{47}} if $P_3\trianglelefteq G$, then by similar arguments as before, $G\cong P_3\rtimes P_7\times P_{47}$.

Let $$\varphi:\mathbb{Z}_{7\cdot 47}\to\Aut(P_3)\cong\mathbb{Z}_2.$$

Since there are no order $2$ elements in $\mathbb{Z}_{7\cdot 47}$ this gives nothing interesting.

\boxed{P_7\times P_{47}\rtimes P_3} Since $3$ is the smallest prime dividing $|G|,$ and $[G:P_7P_{47}]=3$, and $P_7P_{47}\cong P_7\times P_{47}$ is subgroup of $G$, and it is normal by \textbf{Spring 2010, Problem 2, Claim 1}.

Let $$\varphi:P_3\to \Aut(P_7\times P_{47})\cong \Aut(P_7)\times\Aut(P_{47})\cong\mathbb{Z}_6\times\mathbb{Z}_{46}.$$

There are exactly two elements of order $3$ in $\mathbb{Z}_6\times\mathbb{Z}_{46}$, namely $(2,0)$ and $(4,0)$.

Thus, we have two non-trivial homomorphisms, $\varphi_1(1)=(2,0)$ and $\varphi_2(1)=(4,0)$.

Let $P_3\cong\langle a\rangle$ and $P_7\times P_{47}\cong \langle b\rangle\times\langle c\rangle$.

Then $(2,0)$ and $(4,0)$ correspond to the maps $\psi_1$ and $\psi_2$ respectively, with
\begin{center}
\begin{minipage}{0.4\textwidth}
\begin{align*}
    \psi_1: \langle b\rangle\times\langle c\rangle&\to \langle b\rangle\times\langle c\rangle\\
    (b,c)&\mapsto (b^2,c)
\end{align*}
\end{minipage}
\begin{minipage}{0.4\textwidth}
\begin{align*}
    \psi_2: \langle b\rangle\times\langle c\rangle&\to \langle b\rangle\times\langle c\rangle\\
    (b,c)&\mapsto (b^4,c)
\end{align*}
\end{minipage}
\end{center}

It is crucial to note that $\psi_2=\psi_1^2$.

Thus, if \begin{align*}
    \gamma:P_3&\to P_3\\
    a&\mapsto a^2
\end{align*} then $\gamma$ is an automorphism of $P_3$ since $a^2$ also a generator of $P_3$ and since $\varphi_2=\varphi_1\circ\gamma$, we get that $\varphi_2$ and $\varphi_1$ define isomorphic semi-direct products.

Thus, the multiplication for $\varphi_1$ is $a ba^{-1}=\varphi_1(a)(b)=\psi_1(b)=b^2$ and $aca^{-1}=\varphi_1(a)(c)=\psi_1(c)=c.$

And so we get one group, $$G\cong\mathbb{Z}_7\times\mathbb{Z}_{47}\rtimes_{\varphi_1}\mathbb{Z}_3=\langle a,b,c\,|\,a^3=b^7=c^{47}=1,ac=ca,bc=cb,ab=b^2a\rangle$$

\boxed{P_3\times P_{47}\rtimes P_7} If $P_3$ is normal, then $P_3\times P_{47}$ is a normal subgroup of $G$ and so we can examine $$\varphi:P_7\to\Aut(P_3\times P_{47})\cong\mathbb{Z}_2\times\mathbb{Z}_{46}.$$

However, again, no non-trivial homomorphism exists.

\boxed{P_7\rtimes P_3\times P_{47}} If $P_7$ is normal, then we can look at $$\varphi:\mathbb{Z}_3\times\mathbb{Z}_{47}\to\Aut(P_7)\cong\mathbb{Z}_6$$

However, this will give two non-trivial homomorphisms, $\varphi_1(1)=(2)$ and $\varphi_2(1)=4$.

Since these automorphisms are given by $\psi_1(b)=b^2$ and $\psi_2(b)=b^4$, we quickly see that both of these yield the same multiplicative structure as before.

Namely, for $\varphi_1$ we get $aba^{-1}=b^2$ and $cbc^{-1}=b$ and for $\varphi_2$ we get $aba^{-1}=b^4$ and $cbc^{-1}=b$ and $ac=ca$. These were already described in an earlier case.


\boxed{P_3\times P_7\rtimes P_{47}} If $P_3\times P_7$ is normal, then we can examine $$\varphi:P_{47}\to\Aut(P_3\times P_7)\cong\mathbb{Z}_2\times\mathbb{Z}_6.$$ Clear this forces $\varphi$ to be trivial.

Therefore, there are exactly $2$ possible groups up to isomorphism.

\begin{center}
    \begin{framed}
    \vspace{-0.25cm}
    $$\mathbb{Z}_3\times\mathbb{Z}_7\times\mathbb{Z}_{47}$$
    \vspace{-0.5cm}
    $$\langle a,b,c\,|\,a^3=b^7=c^{47}=1,ac=ca,bc=cb,ab=b^2a\rangle$$
    \end{framed}
\end{center}\qedhere
\end{solution}
\newpage



\begin{problem} $\,$
Let $R=\mathbb{Z}[x_1,x_2,....,x_n,...]$ and let $\{f_i(X)\,|\,i\ge 1\}\subseteq R$ satisfy $$f_1(X)R\subseteq f_2(X)R\subseteq\cdots\subseteq f_t(X)R\subseteq\cdots.$$ Show that $f_s(X)R=f_m(X)R$ for some $m$ and all $s\ge m.$
\end{problem}


\begin{solution}$\,$
Since each $f_i$ is a polynomial, we may take each $f_i$ to be comprised of a finite number of variables.

Namely, $f_1\subset\mathbb{Z}[x_{k_1},...,x_{k_{n_1}}]$ for some $k_j$.

Now, $(f_1(X))\subset(f_2(X))$ and so there exists $g_2(X)$ such that $f_1(X)=f_2(X)g_2(X)$. %If $g_2$ is a unit then we are done since $f_2(X)\in(f_1(X)).$

Now, since $\mathbb{Z}$ is a UFD, $\mathbb{Z}[x_{k_1},...,x_{k_{n_1}}]$ is also a UFD, and so $f_1$ can be uniquely factored into irreducibles (which are primes in a UFD), $f_1=p_1\cdots p_t$.

Then, since $$f_1(X)=p_1(X)\cdots p_t(X)=f_2(X)g_2(X)$$ we get that $f_2g_2\in\mathbb{Z}[x_{k_1},...,x_{k_{n_1}}]$ and so each $p_j$ divides either $f_2$ or $g_2$

Namely, $f_2(X)\in\mathbb{Z}[x_{k_1},...,x_{k_{n_1}}]$.

Therefore, inductively, we get that $f_i(X)\in\mathbb{Z}[x_{k_1},...,x_{k_{n_1}}]$ for all $i$ and so namely, if $R'=\mathbb{Z}[x_{k_1},...,x_{k_{n_1}}]$, then we can write $$f_1(X)R'\supset f_2(X)R'\supset\cdots.$$

Since $\mathbb{Z}$ is Noetherian, by the Hilbert Basis Theorem, $R'=\mathbb{Z}[x_{k_1},...,x_{k_{n_1}}]$ is also Noetherian and so the chain must terminate at some finite $m.$

Since $(f_m(X))=(f_n(X))\subset R'\subset R$ for all $n\ge m,$ we are done.
\end{solution}
\newpage



\begin{problem} $\,$
Let $U$ be the set of all $n$-th roots of unity in $\mathbb{C},$ for all $n\ge3$, and set $F=\mathbb{Q}(U)$. For primes $p_1<\cdots<p_k$ and nonzero $a_1,...,a_k\in\mathbb{Q}$, set $M=F(a_1^{1/p_1},...,a_k^{1/p_k})\subseteq\mathbb{C}$. Show that $M$ is Galois over $F$ with a cyclic Galois group. For any subfield $F\subseteq L\subseteq M,$ show that there is a subset $T$ of $\{a_j^{1/p_j}\}$ so that $L=F(T).$
\end{problem}


\begin{solution}$\,$
$M$ is Galois over $F$ if $M$ is the splitting field of a separable polynomial over $F.$

Since $a_i^{1/p_i}$ has minimal polynomial $f_i(x)=x^{p_i}-a_i,$ which has roots $\xi_i^la_i^{1/p_i}$ for $\xi_i$ a $p_i^{th}$ root of unity and $0\le l\le p_i-1$, $f_i$ splits completely in $M$.

Therefore, $M$ is the splitting field of $\prod_{i=1}^kf_i(x)$ which is a polynomial over $F$. Thus, $M$ is Galois over $F.$

Note that $[M:F]\le \prod_{i=1}^kp_i$. However, $$[M:F]=[M:F(a_i^{1/p_i})][F(a_i^{1/p_i}):F]=[M:F(a_i^{1/p_i})]p_i$$ and so $p_i|[M:F]$ for all $i=1,...,k$. Therefore, $[M:F]=p_1\cdots p_k.$

Now, let $G=\Gal(M/F).$ Using the same logic, we obtain that $$K=F(a_1^{1/p_1},...,a_{i-1}^{1/p_{i-1}},a_{i+1}^{1/p_{i+1}},...,a_k^{1/p_k})\text{ is Galois over }F$$ and since $[M:K]=p_i$, $G$ has a normal subgroup of order $p_i$. Namely, $G$ has a normal Sylow $p_i$-subgroup for all $i.$

This is only possible if $G$ is abelian and so $$G\cong\mathbb{Z}_{p_1}\times\cdots\times\mathbb{Z}_{p_k}\cong\mathbb{Z}_{p_1\cdots p_k}\qquad\text{cyclic}.$$

Finally, let $F\subset L\subset M.$

Then $L$ corresponds to some subgroup of $G$. However, the subgroups of $G$ correspond exactly to products of the $\mathbb{Z}_{p_i}$. Thus, if $L$ corresponds to $\mathbb{Z}_{p_{i_1}}\times\cdots\times\mathbb{Z}_{p_{i_l}}$ with $l\le k$, then $L=F(a_{i_1}^{1/p_{i_1}},...,a_{i_l}^{1/p_{i_l}})$.
\end{solution}
\newpage


\end{document}
