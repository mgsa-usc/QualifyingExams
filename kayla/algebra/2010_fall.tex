\documentclass[12pt]{AlgebraQual}
\usepackage{preamble}

\name{Kayla Orlinsky}
\course{Algebra Exam}
\term{Fall 2010}
\hwnum{Fall 2010}

\begin{document}

\begin{problem} $\,$
Use Sylow's Theorems to show that any group of order $(99^2-4)^3$ is solvable.
\end{problem}


\begin{solution}$\,$
First, we decompose the number. \begin{align*}
    99^2-4&=(100-1)^2-4\\
    &=10,000-200+1-4\\
    &=10,000-200-3\\
    &=9,800-3\\
    &=9,797\\
    &=97\cdot 101
\end{align*}

Since both $97$ and $101$ are prime, $(99^2-4)^3=97^3\cdot 101^3.$

Now, it is merely tedious to check that, by the Sylow theorems, $n_{97}|101^3$ and $n_{97}\equiv 1\mod 97$ implies that $n_{97}=1.$
Since the Sylow-$97$ subgroups $P_{97}$ is a $p$ group, it has non-trivial center by the class equation and so we obtain a subnormal series for $P_{97}.$

Namely, $$1\le Z(P_{97})\le P_{97}$$ since $Z(P_{97})=P_{97}$ so $P_{97}$ is abelian, or $|Z(P_{97})|=97,97^2$ in which case $P_{97}/Z(P_{97})$ is abelian.

In any case, $P_{97}$ is solvable.

Finally, since $G/P_{97}$ is also a $p$-group of order $101^3$, it will be solvable for the same reason.

Thus, $G$ contains a normal solvable subgroup such that $G/N$ is solvable and so $G$ is solvable.
\end{solution}
\newpage

\begin{problem} $\,$
For any finite group $G$ and positive integer $m$, let $n_G(m)$ be the number of elements $g$ of $G$ that satisfy $g^m=e_G$. If $A$ and $B$ are finite abelian groups so that $n_A(m)=n_B(m)$ for all $m,$ show that as groups $A\cong B$.
\end{problem}


\begin{solution}$\,$
By the fundamental theorem of Abelian groups, we can write \begin{align*}
    A&\cong(\mathbb{Z}_{p_1^{\alpha_1}})^{n_1}\oplus\cdots\oplus(\mathbb{Z}_{p_k^{\alpha_k}})^{n_k}\\
    B&\cong(\mathbb{Z}_{z_1^{\beta_1}})^{m_1}\oplus\cdots\oplus(\mathbb{Z}_{q_l^{\beta_l}})^{m_l}
\end{align*}

with $p_i,q_j$ primes, $\alpha_i,\beta_i$ distinct and $n_i,m_j$ not zero. We note that $N_A(m),N_B(m)\ge1$ for all $m$ since $e_A$ and $e_B$ will always be counted.

Now, $N_A(p_i)>1$ since each copy of $\mathbb{Z}_{p_i^{\alpha_i}}$ contains an element of order $p_i$ by Lagrange's theorem.

However, $N_A(p_i)=N_B(p_i)$ and so then $B$ contains a non-trivial element with order dividing $p_i$. Namely, $B$ contains an element of order $p_i$.

Since $p_i$ is prime and the $q_i$ are primes, it must be that $p_i=q_j$ for some $j$.

Since this holds for all $p_i$ and $q_j$, we can conclude that $k=l$ and $p_i=q_i$.

Now, $N_A(p_i^{\alpha_i})=n_i(p_i^{\alpha_i}-1)+1$ since, if $g\in A$ satisfies that $g^{p_i^{\alpha_i}}=e_A$, then $g\in \mathbb{Z}_{p_i^{\alpha_i}}$. Since there are $p_i^{\alpha_i}-1$ non-identity elements in each copy, and $n_i$ copies plus $1$ identity element, we conclude the above value.

In fact, $N_A(p_i^n)=n_i(p_i^{\alpha_i}-1)+1$ for all $n\ge \alpha_i$.

Therefore, $\beta_i=\alpha_i$ for all $i$. Else, if $N_B(p_i^{\beta_i})$ would be larger or smaller than $N_A(p_i^{\alpha_i})$.

Finally,

However, then $$N_A(p_i^{\alpha_i})=n_i(p_i^{\alpha_i}-1)+1=m_i(p_i^{\alpha_i}-1)+1=N_B(p_i^{\alpha_i})$$ and so $m_i=n_i$ for all $i$.

Therefore, $A\cong B.$
\end{solution}
\newpage

\begin{problem} $\,$
If $g(x)=x^5+2\in\mathbb{Q}[x],$ for $\mathbb{Q}$ the field of rational numbers, compute the Galois group of a splitting field $L$ over $\mathbb{Q}$ of $g(x).$ How many subfields of $L$ containing $\mathbb{Q}$ are Galois over $\mathbb{Q}$?
\end{problem}


\begin{solution}$\,$
First, if $g(z)=0$ then $z^5=-2$. Letting $z=Re^{i\theta}$ we get that $R=\sqrt[6]{2}$ and $5\theta=(2k+1)\pi$ so, letting $z=e^{i\frac{\pi}{5}}$, we have that the roots of $g$ are $Rz,-Rz^2,Rz^3,-Rz^4,Rz^5$.

Since $Rz^5=-2=-R\zeta^5$ where $\zeta$ is a primitive $5^{th}$-root of unity, we can let $z=-\zeta$.

Thus, the splitting field for $g$ is $L=\mathbb{Q}(R,\zeta)$.

Now, it is clear that $R\zeta$ has minimal polynomial $g$ and so $$[L:\mathbb{Q}]=[L:\mathbb{Q}(R\zeta)][\mathbb{Q}(R\zeta):\mathbb{Q}]=[L:\mathbb{Q}(R\zeta)]5$$

and similarly, $\zeta$ has minimal polynomial $x4+x^3+x^2+x+1$ and so $$[L:\mathbb{Q}]=[L:\mathbb{Q}(\zeta)][\mathbb{Q}(\zeta):\mathbb{Q}]=4[\mathbb{Q}(\zeta):\mathbb{Q}]$$

Thus, $20|[\mathbb{Q}(\zeta):\mathbb{Q}]$ and since $[\mathbb{Q}(\zeta):\mathbb{Q}]\ge20$ we have that $[\mathbb{Q}(\zeta):\mathbb{Q}]=20.$

Now, $g$ is separable, the extension is Galois and so $|\Gal(g)|=[L:\mathbb{Q}]=20.$

Now, we must work to identify $G=\Gal(g)$.

First, let
\begin{center}
 \begin{minipage}{.3\textwidth}
        \centering
    \begin{align*}
    \sigma:L&\to L\\
    R&\mapsto R\zeta\\
    \zeta&\mapsto \zeta
\end{align*}
\end{minipage}%
\begin{minipage}{.3\textwidth}
        \centering
\begin{align*}
    \tau:L&\to L\\
    R&\mapsto R\\
    \zeta&\mapsto \zeta^3
\end{align*}
\end{minipage}
\end{center}

Then both of these are automorphisms of $L$ and furthermore, they do not commute since \begin{align*}
    \sigma(\tau(R))&=\sigma(R)=R\zeta\\
    \tau(\sigma(R))&=\tau(R\zeta)=R\zeta^3
\end{align*} we have that $G$ is not abelian.

Now, $$\tau^4(\zeta)=\tau^3(\zeta^3)=\tau^2(\zeta^4)=\tau(\zeta^2)=\zeta$$ we have that $\tau$ is an element of order $4$ and so $G$ contains $\langle\tau\rangle\cong\mathbb{Z}_4$ as a subgroup.

Now, by the Sylow Theorems, $n_5\equiv 1\mod 5$ and $n_5|4$ so $n_5=1$. Namely, $G$ has one Sylow $5$-subgroup and it is normal.

Therefore, \begin{center}
\begin{tikzcd}
0 \arrow[r] &  P_5  \arrow[r] & G \arrow[r] & P_4 \arrow[r] & 0
\end{tikzcd}\end{center}
is split because $P_5\cap P_4=\{e\}$ and so $|P_5P_4|=\frac{|P_5||P_4|}{|P_5\cap P_4|}=\frac{5\cdot 4}{1}=20=|G|$ and so $$G\cong P_5\rtimes P_4\cong\mathbb{Z}_5\rtimes\mathbb{Z}_4.$$

Finally, by the Galois Correspondence Theorem, to count the number of Galois extensions, we need to determine number of normal subgroups of $G$.

This requires exactly determining $G$ up to isomorphism.

Let $\varphi:\mathbb{Z}_4\to\Aut(\mathbb{Z}_5)\cong\mathbb{Z}_4$. We have already seen that $\langle \tau\rangle\cong P_4\cong\mathbb{Z}_4$ and it is easy to show that $\langle \sigma\rangle=P_5\cong\mathbb{Z}_5$

Then because $G$ can be characterized as a semi-direct product, $\tau\sigma\tau^{-1}=\varphi(\tau).$

Therefore, since $$\tau(\sigma(\tau^{-1}(R)))=\tau(\sigma(R))=\tau(R\zeta)=R\zeta^3=\sigma^3(R).$$ Thus, $$G\cong\langle\sigma,\tau\,|\,\sigma^5=\tau^4=1,\tau\sigma\tau^{-1}=\sigma^3\rangle.$$

Now, we must count normal subgroups of $G.$

The trivial subgroup as well as $G$ itself are both normal subgroups and so $L$ and $\mathbb{Q}$ are both Galois extensions of $\mathbb{Q}$.

We already have that $P_5$ is a normal subgroup and $P_4$ is not, so that adds one more. Note that $P_4$ is not normal since the above computation for $G$ gave that $$\sigma^{-1}\tau\sigma=\sigma^2\tau\notin P_4.$$ Namely, $$\sigma(\tau(\sigma^{-1}(R)))=\sigma(\tau(R\zeta^4))=\sigma(R\zeta^2)=R\zeta^3\not=\tau^i$$ for any $i$.

Finally, if $G$ has a subgroup of order $10$ it will be normal since it will have index $2$ which is the smallest prime dividing $|G|.$ \textit{(To see a proof of this see \textbf{Spring 2010, Problem 2, Claim 1})}.

Now, if $H$ is a subgroup of $G$ of order $10$, then it necessarily contains a copy of $P_5$ and since $P_5$ is the unique subgroup of $G$ of order $5$, $\sigma\in H$.

Now, it is not difficult to check that this forces $H=\langle \sigma,\tau^2\rangle$ since if $H$ must contain some power of $\tau^i$ with $i\not=1$ (else $H=G$).

Thus, $H$ is the unique normal subgroup of $G$ of order $10.$

Now, $G$ is not a direct product since it is non-abelian and is defined as the semi-direct product of two abelian groups. Therefore, if $G$ has a normal subgroup $K$ of order $2$ it must be contained in $H$, else $|HK|=\frac{|H||K|}{|H\cap K|}=\frac{10\cdot 2}{1}=|G|$ and so $HK\cong H\times K\cong G.$

Now, if $K$ is normal in $G$, then it must be normal in $H$ and since $K\cong Q_2$ the Sylow $2$-subgroup of $H$, it suffices to check if $n_2=1$ with $n_2=$ the number of Sylow $2$-subgroups of $H$.

However, $n_2\not=1$ since $\langle \tau^2\rangle$ and $\langle \sigma^2\tau^2\rangle$ both represent distinct Sylow $2$-subgroups of $H$. This is because $$(\sigma^2\tau^2)^2=\sigma^2\tau^2\sigma^2\tau^2=\sigma^2\tau\sigma\tau^3=\sigma^2\sigma^3\tau^4=1.$$

Thus, $n_2\not=1$ and so $G$ has no normal subgroups of degree $2.$

Finally, the total number of Galois extensions of $\mathbb{Q}$ contained in $L$ is $2+1+1=4$ which are associated to the trivial subgroup, $G$ itself, $P_5$ which is $G$'s Sylow $5$-subgroup, and $H$ the normal subgroup in $G$ of order $10.$
\end{solution}
\newpage

\begin{problem} $\,$
Let $P$ be a minimal prime ideal in the commutative ring $R$ with $1$; that is, if $Q$ is a prime ideal in $R$ and if $Q\subset P,$ then $Q=P$. Show that each $x\in P$ is a zero divisor in $R.$
\end{problem}


\begin{solution}$\,$
Let $S=R\backslash P$ as a set. Since $P$ is a prime ideal, if $a, b\in R\backslash P$ then $ab\in R\backslash P$ (else if $ab\in P$ then $a\in P$ or $b\in P$ which is a contradiction).

Thus, $S$ is closed under multiplication and since $0\notin S$ (because $0\in P$), $R'=S^{-1}R$ is a well defined ring.

Now, we claim that $PR'=\left\{\frac{p}{s}\,|\, p\in P,s\in S\right\}$ is the unique maximal ideal of $R'$.

\begin{claim} $PR'$ is the unique maximal ideal of $R'.$
\begin{proof} Let $Q$ be an ideal of $R'$. If there exists some $\frac{q}{s}\in Q$ such that $\frac{q}{s}\notin PR'$, then $q\notin P$. However, then $q\in S$ and so $\frac{q}{q}=1\in Q$ and namely, $Q=R'$.

Therefore, all proper ideals of $R'$ are contained in $PR'.$
\end{proof}
\end{claim}

\begin{claim} $PR'$ is the unique prime ideal of $R'.$
\begin{proof} Now, assume that there is a $Q$ prime ideal of $R'$. By the previous claim, $Q\subset P'R.$ Thus, if $q\in Q$ then $\frac{p}{s}\in PR'$ so we have that $\frac{p}{s}=q\in Q$ for some $q$.

Thus, $p=qs\in QS$ and so $qs\in P$. Therefore, $q\in P$ or $s\in P.$

If $s\in P$ then $\frac{s}{s}=1\in PR'$ which is a contradiction since $P\not=R$. Thus, $q\in P$ and so namely, $QS\in P$. Since $Q$ was assumed to be prime, $QS$ will also be a prime ideal of $R$ and so $P=QS$. Therefore, $Q=PR'.$
\end{proof}
\end{claim}

Finally, we use the fact that the nilradical of $R'$, which is the intersection of all prime ideals of $R'$, which is exactly the set of nilpotent elements of $R'$, is $PR'$ (the only prime ideal of $R'$).

Therefore, every element of $PR'$ is nilpotent, and $$\left(\frac{p}{s}\right)^n=\frac{p^n}{s^n}=0\implies p^n=0$$ because $S$ is closed under multiplication and does not contain $0$ so namely, $s^n\not=0$ for all $s\in S$ and all $n.$

Therefore, every element of $P$ is nilpotent.
\end{solution}
\newpage

\begin{problem} $\,$
Let $R=\mathbb{C}[x_1,...,x_n]$ with $n\ge3$ and $\mathbb{C}$ the field of complex numbers. Consider the ideal $I$ of $R$ defined by $$I=(x_1\cdots x_{n-1}-x_n,x_1\cdots x_{n-2}x_n-x_{n-1},...,x_2\cdots x_n-x_1)$$ so the generators of $I$ are obtained by subtracting each $x_j$ from the product of the others. Show that ther are fixed positive integers $s$ and $t$ so that for each $0\le i\le n$, $(x_i^s-x_i)^t\in I$. (Hint: Consider the product of the generators of $I$.)
\end{problem}


\begin{solution}$\,$
We examine $V(I).$

First, if $x_i=0$ for any $i$, then $x_k=0$ for all $k$. This is immediate since $x_i=x_1x_2\cdots x_{i-1}x_{i+1}\cdots x_{n-1}x_n$ for all $i.$

Now, taking $x_i\not=0$ for all $i$, we have that \begin{align*}
    x_i&=x_1\cdots x_{i-1}x_{i+1}\cdots x_n\\
    x_{i+1}&=x_1\cdots x_ix_{i+2}\cdots x_n\\
    \frac{x_{i+1}}{x_1\cdots x_{i-1}x_{i+2}\cdots x_n}&=x_i\\
    &=x_1\cdots x_{i-1}x_{i+1}\cdots x_n\\
    1&=x_1^2\cdots x_{i-1}^2x_{i+2}^2\cdots x_n^2\\
    &=\frac{x_i^2}{x_{i+1}^2}\\
    x_i^2&=x_{i+1}^2\qquad\text{ for all }i.
\end{align*}

Therefore, as long as $x_i\not=0$ for all $i,$ $$1=x_i^{2(n-1)}$$ so namely, the $x_i$ are equal to $2n-2$-roots of unity.

Namely, $$x_i=x_i^{2n-1}$$ for all $i$. That is to say that $x_i^{2n-1}-x_i\in\sqrt{I}$ for all $i$ and so namely, for each $i$, there exists a $t$ such that $(x_i^{2n-1}-x_i)^t\in I$.
\end{solution}
\newpage

\begin{problem} $\,$
Let $R$ be a right artinian algebra over an algebraically closed field $F$. Show that $R$ is algebraic over $F$ of bounded degree. That is, show there is a fixed positive integer $m$ so that for any $r\in R$ there is a non $g_r(x)\in F[x]$ with $g_r(r)=0$ and with $\deg g\le m$.
\end{problem}


\begin{solution}$\,$
First, we note that $J(R/J(R))=0$ trivially.

Now, there is a correspondence between maximal ideals of $R/J(R)$ and max ideals of $R$ containing $J(R).$ However, $J(R)\subset M$ for all $M$ maximal ideals of $R$ by definition and so there is a $1-1$ correspondence between max ideals of $R$ and max ideals of $J(R).$

Now, we claim that $R/J(R)$ has only finitely many maximal ideals.

Let $$M_1\supset M_1M_2\supset\cdots $$ be a descending chain of maximal ideals of $R/J(R)$. Because $R$ is artinian, $R/J(R)$ is also artinian since quotients of artinian rings are artinian and so the chain terminates.

However, if the chain terminates at $M_1\cdots M_n$, then these must be the only maximal ideals of $R/J(R).$

\begin{claim} $M_1,...,M_n$ are the only ideals of $R/J(R).$
\begin{proof}
Assume not, then if $x\in M_1\cdots M_n$ and there is some maximal ideal of $R/J(R)$ such that $x\notin M$, we have that $MM_1\cdots M_n\subsetneq M_1\cdots M_n$ and therefore extends the chain which is a contradiction.
\end{proof}
\end{claim}

Now, let \begin{align*}
    \varphi: R/J(R)&\to\bigoplus_{i=1}^n\frac{R/J(R)}{M_i}\\
    r&\mapsto(r+M_1,...,r+M_n)
\end{align*}

Then $\varphi$ is injective since clearly $$\ker\varphi\subset \bigcap M_i=J(R/J(R))=0.$$ Furthermore, $\varphi$ is clealry surjective so $R/J(R)$ is semi-simple since $(R/J(R))/M_i$ is a field for all $i.$

Therefore, by Artin-Wedderburn, $$R/J(R)\cong M_{n_1}(D_1)\oplus\cdots\oplus M_{n_k}(D_k)$$ for some integers $n_i$ and some division rings over $F$, $D_i$.

Namely, $R/J(R)$ is finite dimensional over $F$.

Now, because the center of $D_i$, $Z(D_i)$ is a field, by Schur's Lemma, $\psi:F\to Z(D_i)$ is either trivial or an isomoprhism.

However, $F$ being commutative (by definition of field) and $R/J(R)$ being an algebra over $F$, we have that $$F\in Z(R/(J(R)))\cong  Z(M_{n_1}(D_1))\oplus\cdots\oplus Z(M_{n_k}(D_k)0\cong Z(D_1)\oplus Z(D_n)$$ and so namely, we can define a projection map to send $F\to Z(D_i)$ for all $i.$ This map must be non-trivial for all $i$ since $F\in Z(R/(J(R)))$ and so $F\cong Z(D_i)$ for all $i.$

Now, let $\alpha\in D_i$. Since $[F(\alpha):F]<\infty$ (because $[D_i:F]<\infty$ by semi-simpleness of $R/J(R)$), we have that $\alpha$ is algebraic over $F$ and thus satisfies a monic irreducible polynomial with coefficients in $F$. However, $F$ is algebraically closed and so the only monic irreducible polynomials over $F$ are linear. Namely, $\alpha\in F.$

Thus, $D_i=F$ for all $i$.

Now, $R/J(R)$ is a finite dimensional $F$-algebra and so $R/J(R)$ is algebraic over $F$. That is, $a+J(R)$ is algebaric over $F$ for all $a\in R.$

Finally, $J(R)$ is algebraic over $F$ because $R$ is artinian and so $J(R)$ is nilpotent. Namely, $x$ satisfies $x^n=0$ for all $x\in J(R)$.

Since the sum of two algebraic elements is algebraic, this implies that $t=a+x$ and $x$ is algebraic so $t-x=a$ is algebraic for all $a\in R$, and for all $x\in J(R).$
\end{solution}
\newpage


\end{document}
