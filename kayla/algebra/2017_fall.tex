\documentclass[12pt]{AlgebraQual}
\usepackage{preamble}

\name{Kayla Orlinsky}
\course{Algebra Exam}
\term{Fall 2017}
\hwnum{Fall 2017}

\begin{document}

\begin{problem} $\,$
Assume $S$ is a commutative integral domain, and $R\subset S$ is a subring. Assume $S$ is finitely generated as an $R$-module, i.e., there exists elements $s_1,...,s_n\in S$ such that $S=s_1R+s_2R+\cdots s_nR$. Show that $R$ is a field if and only if $S$ is a field. Is the statement true if the assumption that $S$ is an integral domain is dropped?
\end{problem}


\begin{solution}$\,$
\begin{mybox}
***Note that here, finitely generated as an $R$-module is far stronger than finitely generated as an $R$-algebra.

If $S$ were a finitely generated $R$-algebra, then $S=R[s_1,...,s_n]$, namely, $S$ would consist of all polynomials in the $s_i$ with coefficients in $R$.

To say that $S$ is finitely generated as an $R$-module, is to say that every element of $S$ is a finite sum of the $s_i$ with coefficients in $R$.
\end{mybox}

\boxed{\implies} Assume $R$ is a field. Since $S$ is commutative and has no zero divisors, to show that $S$ is a field, we need only show that every $s\in S^\times$ is a unit.

Fix $0\not=s\in S$ \begin{align*}
    \varphi:S&\to S\\
    t&\mapsto st
\end{align*}

$\varphi$ is clearly an $R$-module homomorphism since it is linear.

Furthermore, since $S$ is a domain, $\varphi$ is injective since if $\varphi(t)=0$ then $st=0$ so either $s=0$ or $t=0$, but $s\not=0$ so $t=0$.

However, since $S$ is a finitely generated module over a field, $S$ is an $R$-vector space. Therefore, since $S$ is a finitely generated vector space, it has a finite basis and is finite dimensional.

Finally, this forces $\varphi$ to also be surjective by rank-nullity theorem.

Thus, $1\in R\subset S$ and so, there exists $t\in S$ so $\varphi(t)=st=1.$ Namely, $s$ has an inverse in $S.$

Since $s\in S^\times$ was arbitrary, we have that $S$ is a field.

\boxed{\impliedby} Assume $S$ is a field. Since $s_i^k\in S$ for all $k$, ($S$ is a ring), we have that $R[s_i]\subset S$. However, since $S$ is finitely generated as an $R$-module, then $R[s_i]$ is also finitely generated as an $R$-module, and so namely, $s_i$ is transcendental over $R$.

To see this, note that if $R[s_i]$ is spanned by $\{1,f_1(s_i),...,f_l(s_i)\}$ where $f_j\in R[s_1,...,s_n]$, then if $m$ is the maximal degree of the $f_j$, $$s_i^{m+1}=r_0+\sum_{j=1}^nr_jf_j(s_i)\qquad r_j\in R$$ and so $s_i$ satisfies a monic polynomial with coefficeints in $R$.

Therefore, $S$ is an algebraic extension of $R$.

However, now we are done. Let $0\not=r\in R$, then $r^{-1}\in S$ since $S$ is a field.

However, $r^{-1}$ is algebraic over $R$, meaning that there exists $a_i\in R$ not all $0$ so \begin{align*}
    (r^{-1})^m+a_{m-1}(r^{-1})^{m-1}+\cdots+a_1r^{-1}+a_0&=0\\
     r^{m-1}((r^{-1})^m+a_{m-1}(r^{-1})^{m-1}+\cdots+a_1r^{-1}+a_0)&=0\\
     r^{-1}+a_{m-1}+a_{m-2}r+\cdots+a_1r^{m-2}+a_0r^{m-1}&=0\\
     r^{-1}&=-a_0r^{m-1}-a_1r^{m-2}-\cdots-a_{m-2}r-a_{m-1}\in R
\end{align*}

Therefore, $R$ is a field.

\begin{mybox}
The statement is false if the assumption that $S$ is an integral domain is dropped.

\boxed{\centernot\implies} Let $R=\mathbb{Z}_3$, $S=R[\sqrt{3}]$. Then $S$ is finitely generated as a $\mathbb{Z}_3$ module since $S=\mathbb{Z}_3+\sqrt{3}\mathbb{Z}_3$. Furthermore, $S$ is not an integral domain since $\sqrt{3}\sqrt{3}=3=0\in S$ but $\sqrt{3}\not=0$. Finally, $R$ is a field and $S$ is not a field since $\sqrt{3}(a+b\sqrt{3})=a\sqrt{3}+3b=a\sqrt{3}\not=1$ for $a=0,1,2$.

\boxed{\impliedby} The other direction is true, since if we assume that $S$ is a field, then it must be a commutative integral domain, and so the proof holds.
\end{mybox}

\end{solution}
\newpage


\begin{problem} $\,$
Suppose $R$ is a commutative unital ring, $\mathfrak{p}\subset R$ is a prime ideal and $M$ is a finitely generated $R$-module. Recall that the annihilator ideal $\Ann_R(M)$ consists of elements $r\in R$ such that $rm=0$ for all $m\in M$. Show the localized module $M_\mathfrak{p}$ is \textit{nonzero} if and only if $\Ann_R(M)\subset\mathfrak{p}$.
\end{problem}


\begin{solution}$\,$
Since $M$ is finitely generated, there exists $m_1,...,m_n$ such that $$M=m_1R+m_2R\cdots+m_nR.$$

\boxed{\implies} Assume $M_P$ is nonzero. Recall that $M_P=S^{-1}M$ where $S=R\backslash P$.

Now, recall that $\frac{m}{s}=0\in M_P$ if and only if there exists $t\in S$ so $tm=0\in M$.

Assume there exists an $x\in\Ann_R(M)$ with $x\notin P$. Then $x\in S$ and since $xm_i=0\in M$ for all $i$, we have that $\frac{m_i}{1}=0\in M_P$ for all $i$ and all $s\in S$. Namely, $\frac{m}{s}=0\in M_P$ for all $m\in M$ and all $s\in S$ and so $M_P=0$.

This is a contradiction and so no such $x$ can exist. Namely, $\Ann_R(M)\subset P.$

\boxed{\impliedby} Assume $\ann_R(M)\subset P$. Now, assume $M_P=0$. Then, as stated ealier, for all $m\in M$, there exists $s\in S$ so $sm=0$.

Namely, for $m_i$, there exists $s_i$ so $s_im_i=0$ for all $i=1,...,n$.

Let $s=s_1\cdots s_n$. Then $sm=0$ for all $m\in M$.

This is clear, since $m\in M$ is of the form $a_1m_1+\cdots+a_nm_n$ with $a_i\in R$, since $M$ is a finitely generated $R$-module.

Thus, \begin{align*}
    sm&=s\sum_{i=1}^na_im_i\\
    &=\sum_{i=1}^n(s_1\cdots s_na_im_i)
    &=\sum_{i=1}^n(s_1\cdots s_{i-1}s_{i+1}\cdots s_na_is_im_i)\qquad R\text{ commutative }
    &=\sum_{i=1}^n0\\
    &=0
\end{align*}

However, then $s\in \ann_R(M)$ by definition and since we assumed that $\ann_R(M)\subset P$, this is a contradiction because $S=R\backslash P$.

Therefore, $M_P\not=0$.

\end{solution}
\newpage



\begin{problem} $\,$
Let $f(x)=x^5+1$. Describe the splitting field $K$ of $f(x)$ over $\mathbb{Q}$ and compute the Galois group $\gal(K/\mathbb{Q}).$
\end{problem}


\begin{solution}$\,$
The roots $z$ of $f(x)$ all must satisfy that $z^5=-1$. Thus, if $z=e^{i\theta}$, then $5\theta=\pi,3\pi,5\pi,7\pi,9\pi$.

Clearly $\xi=e^{i\pi/5}$ is a primitive root, since it generates the others, and so $K=\mathbb{Q}(\xi)$.

Now, we note that $-1$ is a root of $f(x)$ and dividing out, we see that \[
     \polylongdiv{x^5+1}{x+1}
   \]

   and so $$x^5+1=(x+1)(x^4-x^3+x^2-x+1).$$

   \begin{claim} If A polynomial $f(x)$ is irreducible over $\mathbb{Z}_p$ for any $p$ which does not divide the leading coefficient of $f$, then $f(x)$ is irreducible over $\mathbb{Q}$.
   \begin{proof} First, since $f$ is irreducible over $\mathbb{Q}$ if and only if it is irreducible over $\mathbb{Z}$, it suffices to consider $f(x)$ a polynomial over $\mathbb{Z}$.

   Now, if $f$ is reducible in $\mathbb{Z}$, then $f(x)=g(x)h(x)$ in $\mathbb{Z}$. However, both $g$ and $h$ have the same degree over $\mathbb{Z}_p$ as they do over $\mathbb{Z}$ since $p$ does not divide the leading coefficient of $f$, so it cannot divide the leading coefficient of $g$ or $h$.

   Namely, $f(x)=g'(x)h'(x)$ in $\mathbb{Z}_p$ where neither $g'$ nor $h'$ are constant, and so $f$ is reducible over $\mathbb{Z}_p.$
   \end{proof}
   \end{claim}


From the claim, over $\mathbb{Z}_2$, $x^4-x^3+x^2-x+1$ becomes $x^4+x^3+x^2+x+1$. Now, if this factors into two quadratics, then we would have $(x^2+ax+b)(x^2+cx+d)$, with $a,b,c,d=0,1$.

Then $$1=a+c=b+d+ac=ad+cb=bd.$$ So $b=d=1$ and either $a=0$ or $c=0$. However, then $1=1+1+0=0$ which is a contradiction.

Therefore, the polynomial cannot factor into two quadratics, and since all the roots are complex, it cannot factor into linear terms, so the polynomial is irreducible over $\mathbb{Z}_2$ and hence over $\mathbb{Q}$.

Finally, we have that $$[K:\mathbb{Q}]=4.$$

Since $K$ is the splitting field of a separable polynomial (all roots are distinct) $K/\mathbb{Q}$ is Galois, and there are only two groups of order $4$, so $G=\gal(K/\mathbb{Q})$ is either $\mathbb{Z}_4$ or $\mathbb{Z}_2\times\mathbb{Z}_2$.

Now, we note that the roots are exactly, $\xi,\xi^3,\xi^5,\xi^7,\xi^9.$ Since $\xi^{10}=1$, we can rewrite this as $\xi,\xi^3,-1,-\xi^2,-\xi^4.$

Now, $\sigma:K\to K$ defined by $\sigma(\xi)=\xi^3$, defines a map in $G$.

Furthermore, $$\sigma^4(\xi)=\sigma^3(\xi^3)=\sigma^2(\xi^9)=\sigma^2(-\xi^4)=\sigma(-\xi^{12})=\sigma(-\xi^2)=-\xi^6=\xi$$

and so $\sigma$ has order $4$ and therefore, $G\cong\mathbb{Z}_4.$

\end{solution}
\newpage




\begin{problem} $\,$
Let $\alpha$ be the real positive $16\thh$ root of $3$ and consider the field $F=\mathbb{Q}(\alpha)$ generated by $\alpha$ over the field of rational numbers. Observe that there is a chain of indeterminate fields $$\mathbb{Q}\subset\mathbb{Q}(\alpha^8)\subset\mathbb{Q}(\alpha^4)\subset\mathbb{Q}(\alpha^2)\subset\mathbb{Q}(\alpha)=F.$$

Compute the degrees of these intermediate field extensions and conclude they are all distinct. Show that every intermediate field $K$ between $\mathbb{Q}$ and $F$ is one of the above (hint: consider the constant term of the minimal polynomial of $\alpha$ over $K).$
\end{problem}


\begin{solution}$\,$
The chain is clear. Now, $$[F:\mathbb{Q}(\alpha^2)]=2$$ since $\alpha$ clearly satisfies $f(x)=x^2-\alpha^2\in\mathbb{Q}(\alpha^2)[x]$. Note that since $\alpha$ is real, it is not possible that $\alpha=a+b\alpha^2$ for any $a,b\in\mathbb{Q}$. Otherwise, $\alpha$ would be a root of $g(x)=bx^2-x+a$ which is not possible, since $\alpha$ has minimal polynomial $x^{16}-3$ over $\mathbb{Q}$. Namely, $f(x)$ is the minimal polynomial $\alpha$ satisfies over $\mathbb{Q}(\alpha^2).$

Similarly, $$[\mathbb{Q}(\alpha^2):\mathbb{Q}(\alpha^4)]=[\mathbb{Q}(\alpha^4):\mathbb{Q}(\alpha^8)]=2$$ and since $\alpha^{16}=3$, $\alpha^8$ satisfies $f(x)=x^2-3$ so $$[\mathbb{Q}(\alpha^8):\mathbb{Q}]=2$$ as well.

Therefore, each field in the chain as a proper subfield of the next.

Now, let $\mathbb{Q}\subsetneq K\subsetneq F$. If $K$ contains no powers of $\alpha$, then $K=\mathbb{Q}$.

Let $\alpha^{2k+1}\in K$ for some $0<k<8$. Then $$(\alpha^{2k+1})^8=\alpha^{16k+8}=3^k\alpha^8\in K$$ so $\alpha^8\in K$. Therefore, $$(\alpha^{2k+1})^{2k+1}\alpha^8=\alpha^{4k^2+4k+8}\alpha=\alpha$$ since $4k^2+4k+8=4(k^2+k+2)=16l$ because $k^2+k+2$ is an even integer strictly greater than $2$ for all non-zero positive integers $k$.

This is a contradiction, and so $K$ can contain no odd powers of $\alpha$.

However, now we are basically done. Since $K\not=\mathbb{Q}$, $K$ must contain some even power of $\alpha$. Let $\alpha^{2k}\in K$ where $0<k<8$ is minimal. Then $k=1,2,4$.  If $k=3$, then $$(\alpha^6)^3=\alpha^2=\alpha^{2\cdot 1}$$ so the minimality of $k$ is contradicted. Similarly, if $k=5$, then $(\alpha^{10})^2=\alpha^4=\alpha^{2\cdot 2}$, and if $k=6$, then $(\alpha^{12})^2=\alpha^8=\alpha^{2\cdot 4}$, and if $k=7$, then $(\alpha^{14})^2=\alpha^{12}=\alpha^{2\cdot 6}$ all of which contradict our choice of $k.$

Therefore, $K$ can only contain powers of $\alpha$ of the form $\alpha^2,\alpha^4,\alpha^8$ and so any intermediate $K$ must be one of the three fields $\mathbb{Q}(\alpha^2),\mathbb{Q}(\alpha^4),\mathbb{Q}(\alpha^8)$.

\end{solution}
\newpage



\begin{problem} $\,$
A finite group is said to be \textit{perfect} if it has nontrivial abelian homomorphic image. Show that a perfect group has no nontrival solvable homomorphac image. Next, suppose that $H\subset G$ is a normal subgroup with $G/H$ perfect. If $\theta:G\to S$ is a homomorphism from $G$ to a solvable group $S$ and if $N=\ker\theta$, show that $G=NH$ and deduce that $\theta(H)=\theta(G)$.
\end{problem}


\begin{solution}$\,$
Assume $G$ is perfect. Let $\varphi:G\to S$ be some group homomorphism such that $\varphi(G)\subset S$ is solvable.

Let $K$ be the kernel of $\varphi$. Then $G/K\cong\varphi(G)$ and so $G/K$ is solvable.

Namely, Since $\varphi(G)$ is not abelian, there exists a normal subgroup $N/K\subset G/K$ such that $$(G/K)/(N/K)\cong G/N\qquad\text{ is abelian}.$$

However, then the quotient map $\pi:G\to G/N$ is certainly a surjective homomorphism into an abelian group, which contradicts that $G$ is perfect.

Thus, $G$ can have no solvable homomorphic image.

Now, suppose that $G$ has a normal subgroup $H$ and that $G/H$ is perfect.

Let $\theta:G\to S$ be a homomorphism with $S$ solvable and $N=\ker\theta$. If $\theta$ is trivial, then we are done since $N=NH=G$. Assume $\theta$ is non-trivial.

Then $G/N\cong\theta(G)$ which is solvable since subgroups of solvable groups are also solvable.

Now, let $f:G/H\to\theta(G)$ defined by $f(gH)=\theta(g)$. Then $f$ is well defined since if $gH=g'H$, then $g=g'h$ for some $h\in H$ so $f(gH)=f(g'hH)=f(g'H)$.

Now, since $G/H$ is perfect, $f$ must be the zero map. Namely, $\theta(g)=0$ for all $gH\in G/H$.

Thus, $\theta(g)=0$ for all $g\notin H$. Therefore, if $g\notin H$, then $g\in N$.

Since $N$ is normal, $NH$ is a subgroup of $G$ and since any $g\notin H$ implies $g\in N$, and $G$ is finite, $G=NH$.
\end{solution}
\newpage



\begin{problem} $\,$
Let $A$ be a finite dimensional $\mathbb{C}$-algebra. Given $a\in A$, write $L_a$ for the left multiplication operatire, i.e., $L_a(b)=ab.$ Define a map $(-,-):A\times A\to\mathbb{C}$ by means of the formula $(a,b):=\tr(L_aL_b)$.
\begin{enumerate}[label=(\alph*)]
    \item Show that $(-,-)$ is a symmetric bilinear form on $A.$
    \item If one defines the radical Rad$(-,-)$ as $\{a\in A\,|\,(a,b)=0 \forall b\in A\}$, then show that Rad$(-,-)$ is a two-sided ideal in $A$.
    \item Show that Rad$(-,-)$ coincides with the Jacobson radical of $A$.
\end{enumerate}
\end{problem}


\begin{solution}$\,$
\begin{enumerate}[label=(\alph*)]
    \item First, we note that $$L_{ab}(x)=abx=aL_b(x)$$ for all $a,b,x\in A$ and $$L_{a+b}(x)=(a+b)x=ax+bx=L_a(x)+L_b(x)$$ Therefore, since the trace is a linear operation, for $a\in\mathbb{C}$ and $x,y,z\in A$, we have that \begin{align*}
        (ax+ay,z)&=\tr(L_{ax+ay}L_z)\\
        &=\tr((L_{ax}+L_{ay})L_z)\\
        &=\tr((aL_x+aL_y)L_z)\\
        &=a\tr(L_xL_z)+a\tr(L_yL_z)\\
        &=a(x,z)+a(y,z)
    \end{align*} and similarly $$(z,ax+ay)=a(z,x)+a(z,y).$$

    Therefore, $(-,-)$ is bilinear. It is symmetric, since $\tr(AB)=\tr(BA)$ so $$(x,y)=\tr(L_xL_y)=\tr(L_yL_x)=(y,x).$$
    \item Let $x,y\in $ Rad$(-,-)$. Then $$(x-y,b)=(x,b)-(y,b)=0-0=0$$ for all $b\in A$ so $x-y\in $ Rad$(-,-)$.

    Similarly, if $r\in A$ then $(rx,b)=(x,rb)=0$ for all $b\in A$ so $rx\in $ Rad$(-,-)$ and $(xr,b)=(x,rb)=0$ for all $r\in A$.

    Therefore, Rad$(-,-)$ defines an ideal in $A$.

    \item Since $A$ is finite dimensional, it is Artinian, so $J(A)$ is nilpotent.

    let $x\in J(A)$. Then for all $b\in A$, $xb\in J(A)$ since $J(A)$ is a $2$-sided ideal. Now, there exists an $n$ so $(xb)^n=0$ since $J(A)$ is nilpotent so $$L_{(xb)^n}=(L_{xb})^n=0.$$ So $L_{xb}$ is nilpotent. Since nilpotent matrices always have zero-trace, $$(x,b)=(xb,1)=\tr(L_{xb})=0.$$ And since $b\in A$ was arbitrary, then $x\in $ Rad$(-,-)$.

    \begin{mybox}
    Recall: If a matrix $M$ is nilpotent, then $M^n=0$ for some $n$. Let $\lambda$ be an eigenvalue of $M$ and $v$ a non-zero eigenvector. Then $M^nv=\lambda^nv=0$ so $\lambda=0$.

    Thus, $M$ has only zero eigenvalues, and since $\tr(M)$ is the sum of the eigenvalues, $\tr(M)=0.$
    \end{mybox}

    Let $a\in$ Rad$(-,-)$. Then, we note that $(a^n,1)=\tr(L_{a^n})=\tr(L_a^n)=0$ for all $n$, so $\sum_{i=1}^n\lambda_i^n=0$ for all $n,$ where $\lambda_i$ are the (not necessarily distinct) eigenvalues of $L_a$.

    Now, we note that if characteristic polynomial of $L_a$ is $p(x)$, then $p(x)=\prod_{i=1}^n(x-\lambda_i)$ and by Cayley Hamilton, $L_a$ satisfies $p(x)$.

    Since $p(x)$ is a polynomial with coefficeints that are symmetric in $\lambda_i$ and since $\sum_{i=1}^n\lambda_i^n=0$ for all $n$ implies that all the symmetric polynomials in the $\lambda_i$ are $0$, we have that $p(x)=x^n$.

    Namely, $L_a$ has only $0$ as an eigenvalue and so it is nilpotent.

    Thus, there exists an $n$ such that $L_{a^n}=L_a^n=0$. Therefore, $a^n1=a^n=0$ so $a$ is nilpotent.

    Since all nilpotent elements are quasi-regular and since $J(R)$ is the largest quasi-regular $2$-sided ideal, it must be that Rad$(-,-)\subset J(R)$.
\end{enumerate}
\end{solution}
\newpage



\begin{problem} $\,$
Suppose $F$ is an algebraically closed field, $V$ is a finite-dimensional $F$-vector space, and $A\in\End_F(V)$. Show that there exists polynomial $f,g\in F[x]$ such that \begin{enumerate}[label=(\roman*)]
    \item $A=f(A)+g(A)$
    \item $f(A)$ is diagonalizable and $g(A)$ is nilpotent
    \item $f$ and $g$ both vanish at $0.$
\end{enumerate}
\end{problem}


\begin{solution}$\,$
Let $$p_A(x)=\prod_{i=1}^m(x-\lambda_i)^{k_i}$$ be the minimal polynomial of $A$. If $x$ divides $p(x)$, then let $p(x)=p_A(x)$ else we let $p(x)=xp_A(x)$ and WLOG, let $\lambda_0=0$.

Let $$q_i(x)=\frac{p(x)}{(x-\lambda_i)^{k_i}}\qquad i=0,...,m$$ Note that $q_i(A)\not=0$ since $q_i$ has degree strictly smaller than $p$. Then for all $i\not=j$, $q_i$ and $(x-\lambda_i)^{k_i}$ are coprime, and so there exists polynomials $a_i(x)$ so $$1=\sum_{i=0}^mf_i(x)\qquad f_i(x)=a_i(x)q_i(x).$$

Now, let $$f(x)=\sum_{i=0}^m\lambda_if_i(x).$$

Then, $$\lambda_jI-f(A)=\lambda_j\sum_{i=0}^mf_i(A)-\sum_{i=0}^m\lambda_if_i(A)=\sum_{i=0}^m(\lambda_j-\lambda_i)f_i(A).$$

Next, since $p_A(x)$ divides $q_i(x)q_j(x)$ for all $i\not=j$, we have that $f_i(A)f_j(A)=0$ for all $i\not=j$. Namely, $$f_j(A)=f_j(A)\sum_{i=0}^mf_i(x)=f_j(A)^2$$ for all $j$.

Therefore, $$f^2(A)=\left(\sum_{i=0}^m\lambda_if_i(A)\right)^2=\sum_{i=0}^m\lambda_i^2f_i^2(A)$$

Thus, \begin{align*}
    (\lambda_jI-f(A))(\lambda_kI-f(A))&=\lambda_j\lambda_kI-(\lambda_j+\lambda_j)f(A)+f^2(A)\\
    &=\lambda_j\lambda_k\sum_{i=0}^mf_i(A)-(\lambda_j+\lambda_k)\sum_{i=0}^m\lambda_if_i(A)+\sum_{i=0}^m\lambda_i^2f_i^2(A)\\
    &=\sum_{i=0}^m(\lambda_j\lambda_k-(\lambda_j+\lambda_k)\lambda_i+\lambda_i^2)f_i(A)\\
    &=\sum_{i=0}^m(\lambda_j-\lambda_i)(\lambda_k-\lambda_i)f_i(A)
\end{align*}

Finally, $$\prod_{i=0}^m(f(A)-\lambda_jI)=\sum_{i=1}^m\prod_{i=0}^m(\lambda_j-\lambda_i)f_i(A)=0$$ since there is a $\lambda_j-\lambda_j=0$ term in every product.

Thus, $f(A)$ has minimal polynomial dividing $\prod_{i=0}^m(x-\lambda_j)$, and since if $v$ is an eigenvector of $A$ associated to eigenvalue $\lambda_i$, then $f(A)v=\lambda_iv$ by construction. Therefore, $f(A)$ has the same eigenvalues as $A$ and is diagonalizable.

Finally, let $g(x)=x-f(x)$. Then $$g(A)=A-f(A)=\sum_{i=1}^m(A-\lambda_iI)f_i(A).$$

Then, let $$k=\max_{i=0,...,m}k_i.$$ Then $$g^k(A)=(A-f(A))^k=\sum_{i=1}^m(A-\lambda_iI)^kf_i(A)=0$$ since $p_A(x)$ divides $(A-\lambda_iI)^kf_i(A)$.

At last, we have that $$A=f(A)+g(A)$$ where $f(A)$ is diagonalizable and $g(A)$ is nilpotent, and by construction, $f$ and $g$ both vanish at $0.$
\end{solution}
\newpage



\end{document}
