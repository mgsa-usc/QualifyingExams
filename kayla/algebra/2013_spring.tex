\documentclass[12pt]{Qual}
\usepackage{preamble}

\name{Kayla Orlinsky}
\course{Algebra Exam}
\term{Spring 2013}
\hwnum{Spring 2013}

\begin{document}

\begin{problem} $\,$
Let $p>2$ be a prime. Describe, up to isomorphism, all groups of order $2p^2$.
\end{problem}


\begin{solution}$\,$
Let $G$ be a group of order $2p^2$. Then by Sylow, $n_p\equiv 1\mod p$ and $n_p|2$ so because $p>2$, $n_p=1$. Thus, $G$ has a normal Sylow $p$-subgroup.

\boxed{\text{Abelian}} If $G$ also has a normal Sylow $2$-subgroup, then $G$ is abelian and $$G\cong\mathbb{Z}_2\times\mathbb{Z}_{p^2}$$ or $$G\cong\mathbb{Z}_2\times\mathbb{Z}_p\times\mathbb{Z}_p$$ depending on whether or not $P_p$ the Sylow $p$-subgroup of $G$ is isomorphic to $\mathbb{Z}_{p^2}$ or $\mathbb{Z}_p\times\mathbb{Z}_p$.

Now, if $P_2$ is a non-normal Sylow $2$-subgroup of $G$, then by the recognizing of semi-direct products theorem, $G$ is a semi-direct product of its Sylow $2$ and Sylow $p$-subgroups.

\boxed{P_p\cong\mathbb{Z}_{p^2}} If $P_p$ is cyclic, then we can let $\varphi:P_2\to\aut(\mathbb{Z}_{p^2})\cong\mathbb{Z}_{p^2}^\times\cong\mathbb{Z}_{p^2-p}$ be a homomorphism.

Let $P_2=\langle a\rangle$ and $P_p=\langle b\rangle.$

Then because $\mathbb{Z}_{p^2-p}$ is of even order, there is a nontrivial homomorphism $\varphi$ which will give a semi-direct product structure to $G.$

Since $\mathbb{Z}_{p^2-p}$ is cyclic, its Sylow $2$-subgroup is also cyclic and so can only have one element of order $2$. This is because if the Sylow $p$-subgroup is $\langle x\rangle$ where $x$ has order $2^n$, then if $i<2^{n-1}$, $2i<2^n$ so $x^i$ does not have order $2$, and if $i>2^{n-1}$, then $i=2^{n-1}+r$ for $0<r<2^{n-1}$ so $(x^i)^2=x^{2i}=x^{2^n+2r}=x^{2r}$ and $2r<2^n$ so again, $x^i$ does not have order $2$.

Thus, the only element of order $2$ is $x^{2^{n-1}}$.

Thus, we have one possible homorphism $\varphi(a)=\sigma$ where $\sigma:P_p\to P_p$ is defined by $\sigma(b)=b^{-1}$, this defined multipliation on $G$ by $bab^{-1}=\varphi(a)(b)=b^{-1}$.

This gives a structure for $G$ as $$G\cong\langle a,b\,|\,a^2=b^{p^2}=1,ab=b^{-1}a\rangle\cong D_{2p^2}$$ the dihedral group of order $2p^2$.

\boxed{P_p\cong\mathbb{Z}_p\times\mathbb{Z}_p}. Then if $\varphi:P_2\to\aut(\mathbb{Z}_p\times\mathbb{Z}_p)\cong GL_2(\mathbb{F}_p)$ we have that $|GL_2(\mathbb{F}_p)|=(p^2-1)(p^2-p)$ and since $p^2-p$ is even, again there exists a nontrivial homomorphism $\varphi.$

Let $P_p\cong\langle b\rangle\times\langle c\rangle.$

Since $\varphi(a)$ will have order $2$ and $P_2$ can either act trivially on one or neither of the copies of $\mathbb{Z}_p$ inside $P_p$, we have two possible homomorphisms which generate different semi-direct products,

$\varphi_1(a)(b)=b^{-1}$ and $\varphi_1(a)(c)=c$, $\varphi_2(a)(b)=b^{-1}$ and $\varphi_2(a)(c)=c^{-1}$.

\begin{mybox}
***Note that the swap function $P_p\to P_p$ where $(b,c)\mapsto (c,b)$ is an automorphism, and so $\varphi_3(a)(b)=b$ and $\varphi_3(a)(c)=c^{-1}$ is conjugate to $\varphi_1(a)$, namely, $\varphi_1$ and $\varphi_3$ generate isomorphic semi-direct products.
\end{mybox}

This gives two possible structures for $G$. $$G\cong\langle a,b,c\,|\,a^2=b^p=c^p=1,bc=cb,ab=b^{-1}a,ac=ca\rangle\cong D_{2p}\times\mathbb{Z}_p$$

$$G\cong\langle a,b,c\,|\,a^2=b^p=c^p=1,bc=cb,ab=b^{-1}a,ac=c^{-1}a\rangle\cong \mathbb{Z}_2\rtimes_{\varphi_2}\mathbb{Z}_p^2$$

Thus, there are $5$ possible structures for $G.$

\begin{center}
    \begin{framed}
    $$\mathbb{Z}_2\times\mathbb{Z}_{p^2}$$

    $$\mathbb{Z}_2\times\mathbb{Z}_p\times\mathbb{Z}_p$$

    $$D_{2p^2}$$

    $$D_{2p}\times\mathbb{Z}_p$$

    $$\langle a,b,c\,|\,a^2=b^p=c^p=1,bc=cb,ab=b^{-1}a,ac=c^{-1}a\rangle\cong \mathbb{Z}_2\rtimes_{\varphi_2}\mathbb{Z}_p^2$$
    \end{framed}
\end{center}
\end{solution}
\newpage


\begin{problem} $\,$
Let $R$ be a commutative Noetherian ring with $1$. Show that every proper ideal of $R$ is the product of finitely many (not necessarily distinct) prime ideals of $R$. (Hint: Consider the set of ideals that are not products of finitely many prime ideals. Also, note that if $R$ is not a prime ring then $IJ=(0)$ for some non-zero ideals $I$ and $J$ of $R$).
\end{problem}


\begin{solution}$\,$
Let $S$ be the set of proper ideals of $R$ which are not products of finitely many prime ideals.

Assume $S$ is nonempty. Because $R$ is noetherian, $S$ contains a maximal element $I$.

Since $I$ is not prime, there exists a product of elements $ab\in I$ such that $a\notin I$ and $b\notin I$ (if no such $ab$ existed then $I$ would be prime).

Then $(a)(b)\in I$ since sums of products of $ab\in I$ but $(a)\not\subset I$ and $(b)\not\subset I$.

Now, if $I+(a)=R$, then $1=x+ra$ where $x\in I$ and $r\in R$. However, then $1-ar=x\in I$ so $b-rab=xb\in I$ because $I$ is an ideal, and $rab\in I$ since $ab\in I$, so $b\in I$, which is a contradiction because $I\in S$.

However, $I\subset I+(a)\not=R$ and so $I+(a)$ cannot be in $S$ by maximality of $I.$

Thus, there exists a finite set of prime ideals $P_1,...,P_n$ such that $I+(a)=P_1P_2\cdots P_n$.

Similarly, there exists $Q_1,...,Q_m$ so $I+(b)=Q_1Q_2\cdots Q_m$.

However, then $$(Q_1Q_2\cdots Q_m)(P_1P_2\cdots P_n)=(I+(a))(I+(b))=I.$$

This contradicts that $I$ is in $S$, so $S$ must in fact be empty.
\end{solution}
\newpage



\begin{problem} $\,$
In the polynomial ring $R=\mathbb{C}[x,y,z]$ show that there is a positive integer $m,$ and polynomials $f,g,h\in R$ such that $$(x^{16}y^{25}z^{81}-x^7z^{15}-yz^9+x^5)^m=(x-y)^3f+(y-z)^5g+(x+y+z-3)^7h.$$
\end{problem}


\begin{solution}$\,$
By Nullsetellensatz, if $I=((x-y)^3,(y-z)^5,(x+y+z-3)^7)$, and $g(x,y,z)\in R$ is such that $g(a,b,c)=0$ for all $(a,b,c)\in V(I)$, then $g\in\sqrt{I}$ and so there exists an integer $m$ such that $g^m\in I.$

Thus, we need only check that $g(x,y,z)=x^{16}y^{25}z^{81}-x^7z^{15}-yz^9+x^5$ is satisfied by every point in $V(I).$

The points in $V(I)$ correspond exactly to the zeros of the generators of $I$. Namely, we have that $(x-y)^3=0,(y-z)^5=0,(x+y+z-3)^7=0$ simultaneously.

Thus, $x=y,y=z,x+y+z=3$, so $x+x+x=3$ so $x=1$, so the only point in $V(I)$ is $(1,1,1)$ which is clearly satisfied by $g$ since $1\cdot 1\cdot 1-1\cdot 1-1+1=0.$

Thus, there exists an $m$ so $g^m\in I.$
\end{solution}
\newpage


\begin{problem} $\,$
Let $R\not=(0)$ be a finite ring such that for any $x\in R$ there is $y\in R$ with $xyx=x$. Show that $R$ contains an identity element such that, for $a,b\in R$, if $ab=1$ then $ba=1$.
\end{problem}


\begin{solution}$\,$
\begin{mybox}
***As written, this problem is not quite correct. Let $R=\{0,a,b\}$ where addition is defined by $a+a=0$, $b+b=0$, $a+b=b+a=0$, and $0$ behaves as usual. And multiplication is given by $a^2=a$, $b^2=b$, $ab=b$, $ba=a$, and $0$ behaves as usual.

\begin{itemize}
    \item $R$ is nonempty and has a $0$ element.
    \item Addition in $R$ is associative and commutative.
    \item $R$ has additive inverses.
    \item Distributivity is immediate since the sum of any two elements in $R$ is zero so multiplication trivially distributes.
    \item For multiplicative associativity, we check each case: $$\begin{matrix}
    (ab)a=ba=a & a(ba)=aa=a\\
    (ba)b=ab=b & b(ab)=bb=b\\
    (ab)b=bb=b & a(bb)=ab=b\\
    (ba)a=aa=a & b(aa)=ba=a\\
    (aa)b=ab=b & a(ab)=ab=b\\
    (bb)a=ba=a & b(ba)=ba=a\\
    aaa=aa=a & bbb=bb=b
    \end{matrix}$$
\end{itemize}

Finally, $R$ is a finite nonzero ring, $aba=a$ and $bab=b$ so for $a$ and $b$, there is an element in $R$ satisfying the hypothesis of the problem. However, $R$ does not contain a multiplicative identity element $1$, since $ab\not=ba$.

The issue here, is that the $y$ satisfying $aya=a$ is not unique. $aba=a$ and $aaa=a$, where $a\not=b$ by assumption. Thus $R$ need not contain an identity at all in this case.
\end{mybox}

Assume that $R$ is a nonzero ring such that for each $x\in R$, there exists a \textit{unique} $y\in R$ so $xyx=x$.

Let $x\in R$ be nonzero. Assume that there exists some $a\in R$ with $xa=0.$ Then $$x(y+a)x=xyx+xax=xyx=x.$$

Now, because $y$ is unique, we have that $y+a=y$ and so $a=0$.

Thus, $xa=0$ implies $a=0.$ Similarly $ax=0$ also implies $a=0$.

This shows that $R$ contains no zero divisors.

Now, define \begin{align*}
        \varphi_x: R&\to R\\
        y&\mapsto xy
    \end{align*}

If $\varphi_x$ is injective, then it is surjective (because $R$ is finite) and so namely, every $y\in R$ can be written as $xy$. Namely, $x$ is a left identity for $R.$

If $\varphi_x$ is not injective, $\ker\varphi_x$ is not trivial. However, then $xa=0$ for some $0\not=a\in R$ which is a contradiction by the above.

Therefore, $\varphi_x$ is injective and so it is an isomorphism. Namely, $x$ is a left identity of $R$ via the isomorphic association $y\sim_\varphi xy$.

Similarly, we can show that $x$ is also a right identity and namely, we may call $x=1\in R.$

Now, assume that $ab=1\in R.$

We have already seen that $R$ has no zero divisors, namely, $$bab=b\implies bab-b=0\implies (ba-1)b=0$$ and so $ba=1$ since $b$ is not a zero divisor.

\begin{mybox}
***Note that since $R$ has no zero divisors, $xyx=x$ actually implies that $x(yx-1)=0$ and so $yx=1$. Similarly, $(xy-1)x=0$ so $xy=1$. Namely, every element of $R$ is invertible and so $R$ is a finite field.
\end{mybox}

\end{solution}
\newpage



\begin{problem} $\,$
Let $f(x)=x^{15}-2$, and let $L$ be the splitting field of $f(x)$ over $\mathbb{Q}$.
\begin{enumerate}[label=(\alph*)]
    \item What is $[L:\mathbb{Q}]$?
    \item Show there exists a subfield $F$ of degree $8$ that is Galois over $\mathbb{Q}$.
    \item What is $\gal(F/\mathbb{Q})$?
    \item Show there is a subgroup of $\gal(L/\mathbb{Q})$ that is isomorphic to $\gal(F/\mathbb{Q})$.
\end{enumerate}
\end{problem}


\begin{solution}$\,$
\begin{enumerate}[label=(\alph*)]
    \item Let $\xi$ be a primitive $15\thh$ root of unity. Then, the roots of $f(x)$ are exactly $\xi^i\sqrt[15]{2}$. Namely, $f$ is separbale and so $L/\mathbb{Q}$ is Galois.

    Clearly $L=\mathbb{Q}(\xi,\sqrt[15]{2})$. Now, if $\varphi(n)$ denotes the Euler totient function, then $$\varphi(15)=\varphi(3)\varphi(5)=2\cdot 4=8$$ and so there are $8$ primitive $15\thh$ roots of unity.

    Therefore, $$[L:\mathbb{Q}]=[L:\mathbb{Q}(\xi)][\mathbb{Q}(\xi):\mathbb{Q}]=[L:\mathbb{Q}(\xi)]8$$ and $$[L:\mathbb{Q}]=[L:\mathbb{Q}(\sqrt[15]{2})][\mathbb{Q}(\sqrt[15]{2}):\mathbb{Q}]=[L:\mathbb{Q}(\sqrt[15]{2})]15$$ and so $[L:\mathbb{Q}]\ge 15\cdot 8$. However, $[L:\mathbb{Q}]\le 15\cdot 8$, so we have that $$[L:\mathbb{Q}]=2^3\cdot 3\cdot 5.$$
    \item We have already found that $F=\mathbb{Q}(\xi)$ has degree $8$ over $\mathbb{Q}$. Furthermore, this extension is Galois, since $F$ is the splitting field of the seprable minimal polynomial of $\xi$, which has degree $8.$

    \item We already know that $L/\mathbb{Q}$ is Galois. Let $G=\gal(L/\mathbb{Q}).$

    By the fundamental theorem of Galois theory, subfields of $L$ $\mathbb{Q}\subset F\subset L$ correspond exactly to subgroups $H$ of $G$ satisfying $|H|=|\gal(L/F)|=[L:F]$.

    A subfield $F$ of $L$ is Galois over $\mathbb{Q}$ if and only if it corresponds to a subgroup $H$ which is normal in $G$. Then $G/H=\gal(F/\mathbb{Q})$ and $[G:H]=[F:\mathbb{Q}]=8$.

    Now, because any $\sigma\in G/H$, must permute the roots of the minimal polynomial of $\xi$, which are the primitive powers of $\xi,$ we have that $G/H$ will be abelian and namely cyclic.

    Thus, $G/H\cong\mathbb{Z}_8$.

    \item This is a direct result of the fundamental theorem of Galois theory, which states that $G/H\cong \gal(F/\mathbb{Q})$ where $H=\gal(L/F).$

    However, since $H$ is normal in $G$, $HP_2$ is a subgroup of $G$, where $P_2$ denotes a Sylow $2$-subgroup of $G.$

    Thus, because $$|HP_2|=\frac{|H||P_2|}{|H\cap P_2|}=\frac{15\cdot 8}{1}=15\cdot 8=|G|,$$ by the isomorphism theorems, $G=HP_2.$

    Thus, $G/H\cong P_2$ which is a subgroup of $G$.
\end{enumerate}
\begin{mybox}
***Note that it was not asked, but after (c), we actually have enough information to determine $G.$

Since, $H=\gal(L/F)$ is a normal subgroup of $G$ of index $8$, so $|H|=15$.

By the Sylow theorems, $n_5\equiv 1\mod 5$ and $n_5|3$, and $n_3\equiv 1\mod 3$, and $n_3|5$, so $n_5=n_3=1$ and so $H$ has only normal Sylow subgroups and so it is abelian and isomoprhic to $\mathbb{Z}_{15}.$

However, normal Sylow subgroups of normal subgroups are normal (see \textbf{Fall 2011: Problem 5 Claim 3}), and so $G$ has a normal Sylow $3$ and a normal Sylow $5$ subgroup.

Thus, $G$ is abelain and $$G\cong\mathbb{Z}_3\times\mathbb{Z}_5\times\mathbb{Z}_8$$
\end{mybox}
\end{solution}
\newpage




\begin{problem} $\,$
Let $F/\mathbb{Q}$ be a Galois extension of degree $60$, and suppose $F$ contains a primitive ninth root of unity. Show $\gal(F/\mathbb{Q})$ is solvable.
\end{problem}


\begin{solution}$\,$
Let $\xi$ be a ninth root of unity. Then if $\varphi$ is the Euler totient function, $\varphi(9)=3^2-3=6$, so $\mathbb{Q}\subset \mathbb{Q}(\xi)\subset F$, and $[\mathbb{Q}(\xi):\mathbb{Q}]=6.$

Now, $K=\mathbb{Q}(\xi)$ is clearly Galois over $\mathbb{Q}$ since it is the splitting field of a separable polynomial over $\mathbb{Q}$.

Now, by the fundamental theorem of Galois theory, subfields $\mathbb{Q}\subset K\subset F$ correspond exactly to subgroups $H\subset G=\gal(F/\mathbb{Q})$, and an extension $K/\mathbb{Q}$ is Galois if and only if $H=\gal(F/K)$ is normal in $G$.

Therefore, $H=\gal(F/K)$ is normal in $G$, and since $[G:H]=|\gal(K/\mathbb{Q})|=6$ so $|H|=10$.

Since in $H$ $n_5\equiv 1\mod 5$ and $n_5|2$, $n_5=1$ so $H$ has a normal Sylow $5$-subgroup $P_5$.

Now, since any $\sigma\in G/H=\gal(K/\mathbb{Q})$ permutes the $9\thh$ roots of unity, it will be abelian.

Therefore, we obtain a subnormal series for $G$ of $$\{e\}\trianglelefteq P_5\trianglelefteq H\trianglelefteq G$$ where $P_5\cong\mathbb{Z}_5$ is abelian, $H/P_5\cong\mathbb{Z}_2$ is abelian, and $G/H=\gal(K/\mathbb{Q})$ is abelian.

So $G$ is solvable.

\end{solution}
\newpage



\begin{problem} $\,$
Let $n$ be a positive integer. Show that $f(x,y)=x^n+y^n+1$ is irreducible in $\mathbb{C}[x,y]$.
\end{problem}


\begin{solution}$\,$
Write $x^n+1=(x-\xi)(x-\xi^2)\cdots(x-\xi^{n-1})\in\mathbb{C}[x]$ where $\xi$ is a primitive $n\thh$ root of unity.

Then, consider $f(x,y)=f(y)\in\mathbb{C}[x][y]$. Since $\mathbb{C}$ is a field, it is a UFD, so $\mathbb{C}[x]$ is a UFD and therefore, $\mathbb{C}[x][y]$ is a UFD.

Thus, we can apply Eisensten's with $p=x-\xi$. This is irreducible in $\mathbb{C}[x]$ since it is linear, and so it is prime because irreducible and prime are equivalent in a UFD.

Since $p$ divides every coefficient of $f(y)$ except the leading coefficient, and $p^2$ does not divide the constant term of $f(y)$. So by Eisenstein, $f(x,y)=f(y)$ is irreducible in $\mathbb{C}[x][y]=\mathbb{C}[x,y].$
\end{solution}



\end{document}
