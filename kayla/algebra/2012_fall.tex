\documentclass[12pt]{Qual}
\usepackage{preamble}

\name{Kayla Orlinsky}
\course{Algebra Exam}
\term{Fall 2012}
\hwnum{Fall 2012}

\begin{document}

\begin{problem} $\,$
Use Sylow's theorems directly to find, up to isomorphism, all possible structures of groups of order $5\cdot7\cdot 23$.
\end{problem}


\begin{solution}$\,$
Let $G$ be a group of order $5\cdot 7\cdot 23.$

By Sylow, $n_{23}\equiv 1\mod 23$ and $n_{23}|35$, so $n_{23}=1$.

Similarly, $n_7\equiv 1\mod 7$ and $n_7|5\cdot23$. Since $5\cdot23=115\equiv 3\mod 7$, we have that $n_7=1$.

\boxed{\text{Abelian}} If $G$ also has a normal Sylow $5$ subgroup, then $G$ is abelian and isomorphic to $\mathbb{Z}_5\times\mathbb{Z}_7\times\mathbb{Z}_{23}$.

If $G$ does not have a normal Sylow $5$-subgroup, then by the recognizing semi-direct products theorem, $G$ is isomorphic to a semi direct product of its Sylow subgroups.

\boxed{\varphi:P_5\to\aut(P_7P_{23})} Let $P_5$, $P_7, P_{23}$ be Sylow $5,7,23$-subgroups of $G$ respectively.

Then if we have a homomorphism, $\varphi:P_5\to\aut(P_7P_{23})\cong\aut(P_7)\times\aut(P_{23})\cong\mathbb{Z}_6\times\mathbb{Z}_{22}$ (since $7$ and $23$ are coprime), we have that $\varphi$ must be trivial since neither group has any elements of order $5$.

\boxed{\varphi:P_5P_7\to\aut(P_{23})} If $\varphi:P_5P_7\to\aut(P_{23})\cong\mathbb{Z}_{22}$ is a homomorphism, then $\varphi$ must be again trivial since $\mathbb{Z}_{22}$ has no elements of order $5$ or order $7.$

\boxed{\varphi:P_5P_{23}\to\aut(P_7)} if $\varphi:P_5P_{23}\to\aut(P_7)\cong\mathbb{Z}_6$ is a homomorphism, then $\varphi$ is again trivial since there are no elements of order $5$ or order $23$ in $\mathbb{Z}_6$.

Thus, there is only one group of order $5\cdot7\cdot23$, \begin{center}
    \begin{framed}
    $$\mathbb{Z}_5\times\mathbb{Z}_7\times\mathbb{Z}_{23}$$
    \end{framed}
\end{center}


\end{solution}
\newpage


\begin{problem} $\,$
Let $A,B,C$ be finitely generated $F[x]=R$ modules, for $F$ a field, with $C$ torsion free. Show that $A\otimes_RC\cong B\otimes_RC$ implies that $A\cong B$. Show by example that this conclusion can fail when $C$ is not torsion free.
\end{problem}


\begin{solution}$\,$
Because $F$ is a field, $F[x]$ is a PID, so because $A,B,C$ are finitely generated, by the structure theorem, we can write each as a direct sum of its free and torsion part.

Namely, because $C$ is torsion free, $C$ is a free module so $C\cong R^n$ for some $n$.

Thus, $$A\otimes_RC\cong A\otimes_RR^n\cong A^n\cong B\otimes_RC\cong B^n.$$

Since $A^n\cong B^n$ implies that the free parts and torsion parts of $A^n$ and $B^n$ are both isomorphic. Namely, if $A\cong R^a\oplus T(A)$ and $B\cong R^b\oplus T(B)$ with $T(A)$ and $T(B)$ the torsion parts of $A$ and $B$ respectively.

Then there exists a chain of nonzero ideals $(a_1) \subset (a_2) \subset \cdots \subset(a_k)\subset A$ and $(b_1)\subset (b_2)\subset\cdots\subset (b_l)\subset B$ with $$T(A)\cong \bigoplus_{i=1}^kR/(a_i)\qquad T(B)\cong\bigoplus_{j=1}^lR/(b_i).$$

Now, since $A^n\cong B^n$, then $$R^{an}\cong R^{bn}\implies a=b$$ and $$(T(A))^n\cong (R/(a_1))^n\oplus\cdots\oplus (R/(a_k))^n\cong (R/(b_1))^n\oplus\cdots\oplus (R/(b_l))^n.$$

Therefore, each component, $R/(a_i)$ of $T(A)$ must be represented in the decomposition for $T(B)$ so $T(A)\cong T(B)$.

Thus, $A\cong B$.

Now, assume $C$ has nontrivial torsion part. Let $A=B\oplus \ann(C)$. Then $$A\otimes_RC=(B\oplus\ann(C))\otimes_RC\cong (B\otimes_RC)\oplus(\ann(C)\otimes_RC)\cong B\otimes_RC$$ since $\ann(C)\subset R$ and so each element transfers over and kills $C.$ However, since $\ann(C)$ is nonzero, $A\not\cong B.$
\end{solution}
\newpage



\begin{problem} $\,$
Working in the polynomial ring $\mathbb{C}[x,y]$, show that some power of $(x+y)(x^2+y^4-2)$ is in $(x^3+y^3,y^3+xy)$.
\end{problem}


\begin{solution}$\,$
By Nullstellensatz, if $I=(x^3+y^3,y^3+xy)$, and $g(x,y)$ is satisfied $g(a,b)=0$ for all $(a,b)\in V(I)$, then $g(x,y)\in\sqrt{I}$ so there exists a natural number $m$ such that $g^m\in I.$

Thus, we compute $V(I).$

If $x^3+y^2=0$ and $y^3+xy=0$ simultaneously, then $x^3y+y^3-y^3-xy=0$ so $x^3y-xy=0$ so $xy(x^2-1)=0$. Thus, we have $x=0,1,-1$ or $y=0$. This gives the following points $(0,0),(1,i),(1,-i),(-1,1),(-1,-1)\in V(x^3+y^2,y^3+xy)$.

Since $(x+y)(x^2+y^4-2)$ $(0,0),(-1,1)$ immediately satisfy $(x+y)$, we need only check $(x^2+y^4-2)$.

Since $1^2+(i)^4-2=1+1-2=0$, $1^2+(-i)^4-2=0$, $(-1)^2+(-1)^4-2=2-2=0$, we have by Nullstellensatz that $(x+y)(x^2+y^4-2)$ is satisfied by every point $(a,b)\in V(x^3+y^2,y^3+xy)$, so $(x+y)(x^2+y^4-2)\in\sqrt{I}$ and there exists an integer $m$ such that $((x+y)(x^2+y^4-2))^m\in(x^3+y^2,y^3+xy)$.
\end{solution}
\newpage


\begin{problem} $\,$
For integers $n,m>1$, let $A\subset M_n(\mathbb{Z}_m)$ be a subring with the property that if $x\in A$ with $x^2=0$ then $x=0.$ Show that $A$ is commutative. Is the converse true?
\end{problem}


\begin{solution}$\,$
First, if $x^2=0\implies x=0$, then we note that if $x^n=0\implies x=0$ for all $n.$

To see this, we simply note that for any positive integer $n$, there exists natural numbers $s$ and $r<2^s$ such that $n=2^s+r$. Thus, $$x^n=0\implies x^{2^s+r}x^{2^s-r}=x^{2^{s+1}}=(x^{2^s})^2=0.$$ Therefore, $x^{2^s}=(x^{2^{s-1}})^2=0$ and so on recursively until we obtain that $x=0.$

Namely, $A$ is a finite ring with no nilpotent elements.

Let $x\in J(A)$. Then because $J(A)$ is right quasi-regular, $1-x$ is a unit in $A$.

Then, we construct a decreasing chain of ideals $$(x)\supset (x^2)\supset\cdots$$ which must terminate for some $n$. Namely, $(x^n)=(x^{n+1})$ so $x^n=rx^{n+1}$ for some $r\in A$. However, $rx\in J(A)$ and so $1-rx$ is a unit. Therefore, $$x^n=rx^{n+1}\implies x^n(1-rx)=0\implies x^n=0.$$

Namely, $x$ is nilpotent. Since $R$ has no nilpotent elements, $J(A)=0.$

Thus, by Artin Wedderburn, $$A\cong M_{n_1}(D_1)\oplus\cdots\oplus M_{n_k}(D_k)$$ where the $D_k$ are division rings.

Now, $A$ contains no nilpotent elements, however matrix rings contain nilpotent elements over any division ring, since $$\begin{bmatrix}
0 & 0 &\cdots & 0 & 1\\
0 & 0 &\cdots & 0 & 0\\
& \vdots &\ddots & \vdots &\\
0 & 0 &\cdots & 0 & 0
\end{bmatrix}$$ is nilpotent of degree $2$ over any division ring where $1\not=0$.

Namely, $n_i=1$ for all $i$.

Finally, because the $D_i$ are finite, by Wedderburn, the $D_i$ are all fields.

Thus, $A$ is a finite direct sum (isomorphic to a finite direct product) of fields and is therefore commutative.

Let $$A=\left\{\begin{bmatrix}
a & 0 & 0 & \cdots & 0 & \mathbb{Z}_m \\
0 & a  & 0 &\cdots & 0 & 0\\
0 & 0 & a  &\cdots & 0 & 0\\
& \vdots & &\ddots & \vdots & \\
0 & 0 & 0 & \cdots & a  & 0\\
0 & 0 & 0 &\cdots & 0 & a
\end{bmatrix}\,\Huge|\,a\in \mathbb{Z}_m \right\}$$

Then $A$ is indeed a subring, it is commutative since every element of $A$ is of the form $aX+bI$ where $$X=\begin{bmatrix}
0 & 0 & \cdots & 0 & 1 \\
0 & 0  &\cdots & 0 & 0\\
& \vdots &\ddots & \vdots & \\
0 & 0 &\cdots & 0 & 0
\end{bmatrix}\qquad I=I_{n\times n}.$$

However, $X^2=0$ and $X\not=0.$
\end{solution}
\newpage



\begin{problem} $\,$
Let $F$ be the splitting field of $f(x)=x^6-2$ over $\mathbb{Q}$. Show that $\gal(F/\mathbb{Q})$ is isomorphic to the dihedral group of order $12.$
\end{problem}


\begin{solution}$\,$
First, $f$ is irreducible by Eisenstein with $p=2$. $2$ divides every coefficient of $f$ except the leading coefficient and $2^2$ does not divide the constant term.

Therefore, since $f$ is irreducible over $\mathbb{Q},$ it is separable. Thus, $F$ is the splitting field of a separable polynomial over $\mathbb{Q}$ and so $F/\mathbb{Q}$ is a Galois extension.

Next, let $\xi$ be a $6\thh$ root of unity. Then $\varphi(6)=\varphi(2)\varphi(3)=1\cdot 2=2$ so there are $2$ primitive roots of unity.

Namely, $F=\mathbb{Q}(\xi,\sqrt[6]{2})$ and since $\xi\notin\mathbb{Q}(\sqrt[6]{2})$ because $\xi$ is a complex number and $\mathbb{Q}(\sqrt[6]{2})\subset\mathbb{R}$, we have that $$[\mathbb{Q}(\xi,\sqrt[6]{2}):\mathbb{Q}(\sqrt[6]{2})]=[\mathbb{Q}(\xi):\mathbb{Q}]=2.$$

Therefore, $$[F:\mathbb{Q}]=[F:\mathbb{Q}(\sqrt[6]{2})][\mathbb{Q}(\sqrt[6]{2}):\mathbb{Q}]=2\cdot 6=12$$

Let $G=\gal(F/\mathbb{Q})$, then $|G|=12$.

Now, let $\sigma\in G$ be defined by $\sigma(\sqrt[6]{2})=\sqrt[6]{2}\xi$, $\sigma(\xi)=\xi$, and $\tau\in G$ be defined by $\tau(\sqrt[6]{2}\xi)=\sqrt[6]{2}\xi^{-1}=$.

Then, $\sigma$ has order $6$ since $\xi$ is a primitive $6\thh$ root of unity, and $\tau$ has order $2$.

Now, note that $\sigma(\sqrt[6]{2}\xi^{j-1})=\sqrt[6]{2}\xi^j$ so $\sigma^{-1}(\sqrt[6]{2}\xi^j)=\sqrt[6]{2}\xi^{j-1}$

Finally, $$\sigma\tau(\sqrt[6]{2}\xi^j)=\sigma(\sqrt[6]{2}\xi^{-j})=\sqrt[6]{2}\xi^{-j+1}$$ and

$$\tau\sigma^{-1}(\sqrt[6]{2}\xi^j)=\tau(\sqrt[6]{2}\xi^{j-1})=\sqrt[6]{2}\xi^{-j+1}.$$

Therefore, $G$ is described by $$G\cong \langle \tau,\sigma\,|\,\tau^2=\sigma^6=1,\sigma\tau=\tau\sigma^{-1}\rangle\cong D_{12}.$$
\end{solution}
\newpage




\begin{problem} $\,$
Given that all groups of order $12$ are solvable show that any group of order $2^2\cdot 3\cdot 7^2$ is solvable.
\end{problem}


\begin{solution}$\,$
Let $G$ be a group of order $2^2\cdot 3\cdot 7^2$. By Sylow, $n_7\equiv 1\mod 7$ and $n_7|12$. Thus, $n_7=1$ so $G$ has a normal Sylow $7$ subgroup.

Let $P_7$ be the Sylow $7$-subgroup of $G$. Then $|P_7|=7^2$, and so $P_7$ is abelian and namely solvable. Note that groups of order $p^2$ $Q$ are abelian since they have nontrivial centers, and the quotient of their centers $Q/Z(Q)$ is cyclic so $Q$ must be abelian.

Therefore, $G$ has a normal subgroup which is solvable.

Finally, $G/P_7$ has order $12$, which we are given implies that $G/P_7$ is a solvable group.

Therefore, $G$ has a normal subgroup $P_7$ which is solvable and $G/P_7$ is solvable, so $G$ itself is solvable.
\end{solution}
\newpage


\end{document}
