\documentclass[12pt]{AlgebraQual}
\usepackage{preamble}

\name{Kayla Orlinsky}
\course{Algebra Exam}
\term{Spring 2010}
\hwnum{Spring 2010}

\begin{document}

\begin{problem} $\,$
Let $f(x)=x^6+3\in\mathbb{Q}[x].$ Show that the Galois group of $f$ is $S_3$.
\end{problem}


\begin{solution}$\,$
First, we note that $f(x)$ is separable since its roots are all distinct.

We now proceed with some computation to determine the number and order of the roots that we need to adjoint to $\mathbb{Q}$ to obtain the splitting field for $f$.

Now, if $f(x)=0$ then $x^6=-3$. Letting $z=Re^{i\theta}\in\mathbb{C}$ we get that $$(Re^{i\theta})^6=R^6e^{i6\theta}=-3=3(-1+0i)$$ so $R=\sqrt[6]{3}$ and $6\theta=(2k+1)\pi.$

This computation shows that we get $$\pm\sqrt[6]{3}i\qquad \sqrt[6]{3}\left(\pm\frac{\sqrt{3}}{2}\pm\frac{1}{2}i\right)$$ as roots.

Since $$(\sqrt[6]{3}i)^4=3^{\frac{4}{6}}=3^{\frac{1}{2}}3^{\frac{1}{6}}=\sqrt[6]{3}\sqrt{3}$$ so we finally get that all the roots of $f(x)$ can be obtained by adjoining $\sqrt[6]{3}i$ to $\mathbb{Q}$.

Namely, if $\alpha=\sqrt[6]{3}i$, then the roots of $f$ are $$\pm \alpha,\qquad \pm \alpha^4\pm \alpha$$

Namely, $\mathbb{Q}(\alpha)$ is the splitting field for $f$.

Since we already noted that $f$ is separable, we get that $|\Gal(\mathbb{Q}(\alpha)/\mathbb{Q})|=[\mathbb{Q}(\alpha):\mathbb{Q}]=6$ since the minimal polynomial of $\alpha$ is $x^6+3.$

Now, we need only prove that $\Gal(\mathbb{Q}(\alpha)/\mathbb{Q})$ is non-abelian.

However, this is straightforward since we already wrote down the roots of $f$.

If \begin{align*}
    \tau:\mathbb{Q}(\alpha)&\to\mathbb{Q}(\alpha)\\
    \alpha&\mapsto -\alpha\\
\end{align*}

and \begin{align*}
    \sigma:\mathbb{Q}(\alpha)&\to\mathbb{Q}(\alpha)\\
    \alpha&\mapsto \alpha^4+\alpha\\
\end{align*}

Note that both of these maps exist since $f$ is irreducible and separable so $\Gal(f)$ is transitive (for any two roots of $f$, there exists an automorphism sending one to the other).

Finally, $$\tau(\sigma(\alpha))=\tau(\alpha^4+\alpha)=\alpha^4-\alpha$$ and $$\sigma(\tau(\alpha))=\sigma(-\alpha)=-\alpha^4-\alpha$$

so the two maps do not commute.

Therefore, $\Gal(f)$ is non-abelian and since the only non-abelian group of order $6$ is $S_3$ we are done.


\end{solution}
\newpage

\begin{problem} $\,$
\begin{enumerate}[label=(\alph*)]
    \item Let $G$ be a group of order $pqr$, where $p<q<r$ are primes. Show that $G$ contains a normal subgroup of index $p.$
    \item Determine up to isomorphism all groups of order $3\cdot7\cdot 13$.
\end{enumerate}
\end{problem}


\begin{solution}$\,$
\begin{enumerate}[label=(\alph*)]
    \item By Lagrange's theorem, for all $p||G|$, there exists a subgroup $N$ of order $p.$ Now, we will show that if $p$ is the smallest prime dividing $|G|$ and $[G:N]=p$ then $N$ is normal.

    \begin{claim} If $p$ is the smallest prime dividing $|G|$ and $[G:N]=p$, then $N$ is normal.
    \begin{proof} Assume not. Then there exists $g\in G$ with $g\notin N$ such that $N\not=gNg^{-1}$. Let $N^g=gNg^{-1}$.

    Now, as sets, we have that $$|NN^g|=\frac{|N||N^g|}{|N\cap N^g|}.$$

    If $G=NN^g$ then $g^{-1}=n_1gn_2g^{-1}$ and so $n_1^{-1}n_2^{-1}=g\in N$, an contradiction. Namely, $|G|>|NN^g|$.

    However, we finally have that $$|NN^g|=\frac{|N||N^g|}{|N\cap N^g|}<|G|=p|N|$$ and so namely, $$\frac{|N^g|}{|N\cap N^g|}<p$$ Since $p$ is the smallest prime dividing $|G|$, there cannot be any elements of order smaller than $p$ and so namely, $|N^g|=|N\cap N^g|$ and since $N\cap N^g\subset N^g$, we get that $N\cap N^g=N^g$.

    Namely, $N=N^g$. This is a contradiction again and so no such $g$ exists.
    \end{proof}
    \end{claim}

    Therefore, from the claim, $N$ is normal and it exists by Lagrange.

    \item \boxed{\text{Abelian: }\mathbb{Z}_3\times\mathbb{Z}_7\times\mathbb{Z}_{13}} by the fundamental classification theorem of Abelian Groups.

    Now, using the Sylow Theorems, which state that $n_p\equiv 1\mod p$ and that $n_p|m$ with $|G|=p^km$.

    Thus, $n_7\equiv 1\mod 7$ and $n_7|3\cdot 13$. Since $7\nmid12$, and $7\nmid38$, $n_7=1$.

    Finally, $n_{13}=1$ trivially by the same reasoning.

    Thus, $G$ contains exactly one normal Sylow subgroup of orders $7,13$. Note that any Sylow $3$-subgroups are isomorphic to $\mathbb{Z}_3,\mathbb{Z}_7,\mathbb{Z}_{13}$ respectively.

    Now, we begin the classificiation. Starting with a normal Sylow-subgroup, we will take automorphisms of that Sylow subgroup and see how those act on the product of the remain two Sylow subgroups.

    Then we note that from (a), $P_7P_{13}$ is a normal subgroup of $G.$

    \textbf{First}, if $G$ has a normal Sylow $3$-subgroup, then $\Aut(\mathbb{Z}_3)\cong\mathbb{Z}_3^*\cong\mathbb{Z}_2$. Since there are no order two elements of $\mathbb{Z}_7\times\mathbb{Z}_{13}$ and so this yields nothing. Namely, there are no non-trivial homomorphisms $\varphi:\mathbb{Z}_7\times\mathbb{Z}_{13}\to\mathbb{Z}_2$

    \textbf{Second}, $\Aut(\mathbb{Z}_7)\cong\mathbb{Z}_6$ has two elements of order $3$, \begin{align*}
        \alpha_1:\mathbb{Z}_7&\to\mathbb{Z}_7 &&\alpha_2:\mathbb{Z}_7\to\mathbb{Z}_7\\
        b&\mapsto b^2 &&\qquad b\mapsto b^4
    \end{align*}

    Note that $\alpha_2=\alpha_1^2$


    Thus, we can let \begin{align*}
        \psi_1:\mathbb{Z}_3\times\mathbb{Z}_{13}&\to\mathbb{Z}_6\\
        (1,0)&\mapsto 2=\alpha_1\\
        (0,1)&\mapsto 0=\Id
    \end{align*} be a non-trivial homomorphism.

    Let $$\mathbb{Z}_3\times\mathbb{Z}_{13}=\langle a\rangle\times\langle c\rangle\qquad\mathbb{Z}_7=\langle b\rangle$$

    Then $\psi_1(a,0)(b)=\alpha_1(b)=b^2$ and $\psi_1(0,c)(b)=\Id(b)=b$. Finally, for $\psi_1$, this gives the relation
    $$aba^{-1}=\psi_1(a)(b)=b^2\implies ab=b^2a$$ and $$cbc^{-1}=b\implies cb=bc.$$

    Thus, we obtain the presentation $$\mathbb{Z}_7\rtimes_{\psi_1}(\mathbb{Z}_3\times\mathbb{Z}_{13})\cong\langle a,b,c\,|\, a^3=b^7=c^{13}=1, ac=ca, ab=b^2a, cb=bc\rangle\cong (\mathbb{Z}_7\rtimes_{\psi_1}\mathbb{Z}_3)\times\mathbb{Z}_{13}$$

    and similarly for $\psi_2$,  \begin{align*}
        \psi_2:\mathbb{Z}_3\times\mathbb{Z}_{13}&\to\mathbb{Z}_6\\
        (1,0)&\mapsto 4=\alpha_2\\
        (0,1)&\mapsto 0=\Id
    \end{align*}

    Now, we note that \begin{align*}
        \varphi:\mathbb{Z}_3\times\mathbb{Z}_{13}&\to\mathbb{Z}_3\times\mathbb{Z}_{13}\\
        (1,0)&\mapsto (2,0)\\
        (0,1)&\mapsto (0,1)
    \end{align*} is an automorphism of $\mathbb{Z}_3\times\mathbb{Z}_{13}$ and since $\psi_2=\psi_1\circ\varphi$, we have that $\psi_1$ and $\psi_2$ generate isomorphic semi-direct products.

    %so we get $$\mathbb{Z}_7\rtimes_{\psi_2}(\mathbb{Z}_3\times\mathbb{Z}_{13})\cong\langle a,b,c\,|\, a^3=b^7=c^{13}=1, ac=ca, ab=b^4a, cb=bc\rangle\cong(\mathbb{Z}_7\rtimes_{\psi_2}\mathbb{Z}_3)\times\mathbb{Z}_{13}$$

    \textbf{Third}, $\Aut(\mathbb{Z}_{13})\cong\mathbb{Z}_{12}$ which has $2$ elements of order $3$ and no elements of order $7$ call them $\beta_1,\beta_2$ with $\beta_1(c)=c^3$ and $\beta_2(c)=c^9$.

    Let $\psi_3:\mathbb{Z}_3\times\mathbb{Z}_7\to\mathbb{Z}_{12}$ be the map where $\psi_3(a)(c)=\beta_1(c)=c^3$ and $\psi_3(b)(c)=c$

    Similarly, $\psi_4(a)(c)=c^9$ and $\psi_4(b)(c)=c$. As from the previous case, letting $\varphi(1,0)=(2,0)$ and $\varphi(0,1)=(0,1)$, we get hat $\psi_4=\psi_3\circ\varphi$ and so again, the semi-direct products will be isomorphic.

    This gives one presentation: $$\mathbb{Z}_{13}\rtimes_{\psi_3}(\mathbb{Z}_3\times\mathbb{Z}_{7})\cong\langle a,b,c\,|\, a^3=b^7=c^{13}=1, ab=ba, ac=c^3a, bc=cb\rangle\cong (\mathbb{Z}_{13}\rtimes_{\psi_3}\mathbb{Z}_3)\times\mathbb{Z}_{7}$$.

    \textbf{Fourth} $P_7P_{13}\cong\mathbb{Z}_7\times\mathbb{Z}_{13}$ is also a normal subgroup. $\Aut(\mathbb{Z}_7\times\mathbb{Z}_{13})\cong\mathbb{Z}_6\times\mathbb{Z}_{12}$.

    Thus, we have \begin{align*}
        \psi_5:\mathbb{Z}_3&\to\mathbb{Z}_6\times\mathbb{Z}_{12}\\
        1&\mapsto (2,0)=(\alpha_1,\Id)\\
        \psi_6:\mathbb{Z}_3&\to\mathbb{Z}_6\times\mathbb{Z}_{12}\\
        1&\mapsto (4,0)=(\alpha_2,\Id)\\
        \psi_7:\mathbb{Z}_3&\to\mathbb{Z}_6\times\mathbb{Z}_{12}\\
        1&\mapsto (0,4)=(\Id,\beta_1)\\
        \psi_8:\mathbb{Z}_3&\to\mathbb{Z}_6\times\mathbb{Z}_{12}\\
        1&\mapsto (0,8)=(\Id,\beta_2)\\
        \psi_9:\mathbb{Z}_3&\to\mathbb{Z}_6\times\mathbb{Z}_{12}\\
        1&\mapsto (2,4)=(\alpha_1,\beta_1)\\
        \psi_{10}:\mathbb{Z}_3&\to\mathbb{Z}_6\times\mathbb{Z}_{12}\\
        1&\mapsto (2,8)=(\alpha_1,\beta_2)\\
        \psi_{11}:\mathbb{Z}_3&\to\mathbb{Z}_6\times\mathbb{Z}_{12}\\
        1&\mapsto (4,4)=(\alpha_2,\beta_1)\\
        \psi_{12}:\mathbb{Z}_3&\to\mathbb{Z}_6\times\mathbb{Z}_{12}\\
        1&\mapsto (4,8)=(\alpha_2,\beta_2)\\
    \end{align*}

    Where $\alpha_1:\mathbb{Z}_6\to\mathbb{Z}_6$ is defined by $\alpha_1(1)=2$, $\alpha_2(1)=4$, $\beta_1:\mathbb{Z}_{13}\to\mathbb{Z}_{13}$ is defined by $\beta_1(1)=3$, and $\beta_2(1)=9$.

    Since $\alpha_1^2=\alpha_2$, and $\beta_1^2=\beta_2$, it is clear to see that each of these homomorphisms pairs up with another one via $\varphi:\mathbb{Z}_3\to\mathbb{Z}_3$ defined by $\varphi(1)=2$. For example, $\psi_6=\psi_5\circ\varphi$.

    Now, let $\mathbb{Z}_3=\langle a\rangle$, $\mathbb{Z}_7=\langle b\rangle$ and $\mathbb{Z}_{13}=\langle c\rangle$ as before. We note that the $\psi_5$ and $\psi_6$ which generate isomorphic semi-direct products will generate the same group as $\psi_2$ from the second part. This is because, $a$ will commute with $c$ and $aba^{-1}=\psi_5(a)(b)=\alpha_1(b)=b^2$.

    Similarly, $\psi_7$ and $\psi_8$ generate the same group as $\psi_3$ from the third part.

    Finally, this will yield two sets of non-ismorphic groups. First, one defined by $\psi_9$, with relations $aba^{-1}=\psi_9(a)(b)=\alpha_1(b)=b^2$ and $aca^{-1}=\psi_9(a)(c)=c^3$, which gives $$(\mathbb{Z}_7\times \mathbb{Z}_{13})\rtimes_{\psi_9}\mathbb{Z}_3\cong\langle a,b,c\,|\, a^3=b^7=c^{13}=1, bc=cb, ab=b^2a, ac=c^3a\rangle.$$

    And the other non-isomophic group has relations defined by $aba^{-1}=\psi_{10}(a)(b)=b^2$ and $aca^{-1}=\psi_{10}(a)(c)=c^9$, which gives $$(\mathbb{Z}_7\times \mathbb{Z}_{13})\rtimes_{\psi_{10}}\mathbb{Z}_3\cong\langle a,b,c\,|\, a^3=b^7=c^{13}=1, bc=cb, ab=b^2a, ac=c^9a\rangle.$$

    \textbf{Note:} that to verify that $\psi_9$ and $\psi_{10}$ do indeed generate non-isomorphic groups we turn to a stronger theorem of Taunt in \textit{Remarks on the Isomorphism Problem in Theories of Construction of Finite Groups}.

    The theorem states that:
    \begin{framed}
    If $|N|$ and $|H|$ are comprime, then $$N\rtimes_{\psi_1} H\cong N\rtimes_{\psi_2} H$$ if and only if there exists $\alpha\in\Aut(N)$ and $\beta\in \Aut(H)$ such that $$(\psi_1\circ\beta)(h)=\alpha\circ\psi_2(h)\circ\alpha^{-1}\in\Aut(N)$$ for all $h\in H$.
    \end{framed}

    In this case, because $\Aut(N)\cong \mathbb{Z}_6\times\mathbb{Z}_{12}$ which is abelian. Namely, $\alpha\circ\psi_2(h)\circ\alpha^{-1}=\psi_2(h)$.

    Therefore, we have that two homomorphisms generate isomorphic semi-direct products, if and only if they differ by an isomorphism of $\mathbb{Z}_3$. Since there are only two isomoprhisms of $\mathbb{Z}_3$, it is easy to verify that $\psi_9$ and $\psi_{10}$ do not generate isomorphic semi-direct products.


    \textbf{Fifth} We can also define a normal subgroup $P_3P_{13}$ since both $P_3$ and $P_{13}$ normal in $G$ and intersect trivially, $P_3P_{13}\cong\mathbb{Z}_3\times\mathbb{Z}_{13}$ is normal in $G$.

    However, $\Aut(\mathbb{Z}_3\times\mathbb{Z}_{13})\cong\mathbb{Z}_2\times\mathbb{Z}_{12}$ has no elements of order $7$.

    Similarly, $\Aut(\mathbb{Z}_3\times\mathbb{Z}_7)\cong\mathbb{Z}_2\times\mathbb{Z}_6$ has no elements of order $13$.

    \textbf{As we have now ruled out all possible normal subgroups of $G$, we can conclude that we have found all of the isomorphism classes.} Listed out, the four non-abelian groups and one abelian group are

    \begin{framed}
    Groups of order $3\cdot 7\cdot 13$:


    $$\mathbb{Z}_3\times\mathbb{Z}_7\times \mathbb{Z}_{13}$$
    \vspace{-0.25cm}
    $$\mathbb{Z}_7\rtimes_{\psi_1}\mathbb{Z}_3\times\mathbb{Z}_{13}\cong \langle a,b,c\,|\, a^3=b^7=c^{13}=1, ac=ca, ab=b^2a, cb=bc\rangle$$
    \vspace{-0.25cm}
    $$\mathbb{Z}_{13}\rtimes_{\psi_3}\mathbb{Z}_3\times\mathbb{Z}_{7}\cong\langle a,b,c\,|\, a^3=b^7=c^{13}=1, ab=ba, ac=c^3a, bc=cb\rangle$$
    \vspace{-0.35cm}
    $$(\mathbb{Z}_7\times \mathbb{Z}_{13})\rtimes_{\psi_9}\mathbb{Z}_3\cong\langle a,b,c\,|\, a^3=b^7=c^{13}=1, bc=cb, ab=b^2a, ac=c^3a\rangle$$
    \vspace{-0.38cm}
    $$(\mathbb{Z}_7\times \mathbb{Z}_{13})\rtimes_{\psi_{10}}\mathbb{Z}_3\cong\langle a,b,c\,|\, a^3=b^7=c^{13}=1, bc=cb, ab=b^2a, ac=c^9a\rangle$$
    \end{framed}
\end{enumerate}
\end{solution}
\newpage

\begin{problem} $\,$
Let $R$ be a commutative Noetherian ring, and let $I,J$ and $K$ be ideals of $R$. We say $I$ is irreducible if $I=J\cap K\implies I=J$ or $I=K$.
\begin{enumerate}[label=(\alph*)]
    \item Show that every ideal of $R$ is a finite intersection of irreducible ideals.
    \item Show that every irreducible ideal is primary. (An ideal $I$ of $R$ is primary if $R/I\not=0$, and every zero-divisor in $R/I$ is nilpotent.)
\end{enumerate}
\end{problem}


\begin{solution}$\,$
\begin{enumerate}[label=(\alph*)]
    \item Assume not. Let $I$ be an ideal of $R$ which is not a finite intersection of irreducibles.

    Then, there exists ideals $J_1$ and $J_2$ such that $I=J_1\cap K_1$ with $I\subsetneq J_1$ and $I\subsetneq K_1$. Note that if $J_1$ and $K_1$ do not exist, then $I=J_1\cap K$ implies $I=J_1$ or $I=K_1$ and so $I$ is itself irreducible, a contradiction.

    Now, because $I$ is not a finite intersection of irreducibles, it must be that either $J_1$ or $K_1$ is also not a finite intersection of irreducibles. (If both were such an intersection, then $I$ would be as well).

    WLOG, take $J_1$ to be not a finite intersection of irreducibles. However, by the same argument as before, we can write $J_1=J_2\cap K_2$ with $J_1\subsetneq J_2$ and $J_1\subsetneq K_2$.

    Namely, we obtain an ascending chain $$I\subsetneq J_1\subsetneq J_2\subsetneq\cdots$$ which must terminate because $R$ is Noetherian.

    However, if the chain terminates at $J_n$, so $J_m=J_n$, then this implies that there do not exist any ideals $J$ and $K$ such that $J_n\subsetneq J\cap K$ and $J_n\subsetneq J$ and $J_n\subsetneq K$. Else, we could call $J_{n+1}=J$.

    Thus, $J_n$ is irreducible, which is a contradiction.
    \item Let $I$ be an irrediucible proper ideal of $R$. Then $R/I\not=0$.

    Let $0\not=a\in R/I$ be a zero divisor. Then there exists $0\not=b\in R/I$ such that $ab=0\in R/I$ so namely, $ab\in I$ with $a\notin I$ and $b\notin I$.

    Now, we note that this implies that $b\in\ann(a)\subset\ann(a^2)\subset\ann(a^3)\subset\cdots$ since if $ab=0$ then $a^kb=a^{k-1}0=0$.

    Now, because $R$ is Noetherian and quotients of Noetherian rings are also Noetherian, we have that $R/I$ is Noetherian. Namely, the chain $$\ann(a)\subset\ann(a^2)\subset\ann(a^3)\subset\cdots$$ must terminate.

    Say the chain terminates at $\ann(a^n)$ so $\ann(a^m)=\ann(a^n)$ for all $m\ge n$.

    Now, let $b\in\ann(a)$ and $x\in(b)\cap(a^n)$. Then $x=rb=sa^n$. However, then $0=rba=sa^{n+1}$ and so $s\in\ann(a^{n+1})=\ann(a^n)$ and so $x=sa^n=0$.

    Thus, $(b)\cap (a^n)=(0)=I$. However, $I$ is irreducible so either $I=(b)$ or $I=(a^n)$. Since $b\notin I$ by assumption, it must be that $I=(a^n)$ and so namely, $a^n=0\in R/I$.

    Thus, every zero-divisior is nilpotent.
\end{enumerate}
\end{solution}
\newpage

\begin{problem} $\,$
Let $A$ be a finite-dimensional algebra over a field $K,$ such that for every $a\in A$, $a^7=a$. Show that $A$ is a direct product (sum?) of fields. Which fields can arise?
\end{problem}


\begin{solution}$\,$
First, we note that $K\subset A$ and so the fact that $a^7=a$ for all $a\in A$ forces $k^7=k$ for all $k\in K$.

Namely, $K\cong\mathbb{F}_7$.

Now, because $A$ is a finite dimensional vector space, it is Artinian (because all ideals are finite-dimensional subspaces of $A$ so infinite chains cannot exist).

Now, let $a\in J(A)$ the Jacobson radical of $A$. Then $a^6\in J(A)$ because $J(A)$ is an ideal of $A$.

However, $J(A)$ is quasi-invertible so there exists $b\in A$ such that $$b(1-a^6)=1.$$ However, this implies that $$b(1-a^6)a=a\implies b(a-a^7)=a\implies a=0$$ so $J(A)=(0)$.

Therefore, by Artin-Wedderburn, $A$ can be written as a finite direct sum of matrix algebras over division rings. Namely, $$A\cong M_{n_1}(D_1)\oplus\cdots\oplus M_{n_l}(D_l)\qquad D_l\text{ division rings over }K.$$

Now, because $D_i$ is a division ring over $K$, it must be a field extension of $K$. However, since $A$ has the property that $a^7=a$ for all $a\in A$, each $d\in D_i$ satisfies this property as well so $D_i=K$ for all $i$.

Now, because there exist non-zero nilpotent elements in any matrix ring, it must be that $n_i=1$ for all $i$.

Namely, $$A\cong\bigoplus_{i=1}^lK.$$
\end{solution}
\newpage

\begin{problem} $\,$
Let $G$ and $H$ be finitely generated abelian groups such that $G\otimes_\mathbb{Z}H=0$. Show that $G$ and $H$ are finite and have relatively prime orders.
\end{problem}


\begin{solution}$\,$
By the fundamental theorem of finitely generated abelian groups, we can write \begin{align*}
    G&\cong\mathbb{Z}^s\oplus\mathbb{Z}_{n_1}\oplus\cdots\oplus\mathbb{Z}_{n_k}\\
    H&\cong\mathbb{Z}^t\oplus\mathbb{Z}_{m_1}\oplus\cdots\oplus\mathbb{Z}_{m_l}
\end{align*}

Now, because tensor product distributes across direct sums, we have that \begin{align*}
    G\otimes_\mathbb{Z}H&=(\mathbb{Z}^s\oplus\mathbb{Z}_{n_1}\oplus\cdots\oplus\mathbb{Z}_{n_k})\otimes_\mathbb{Z}(\mathbb{Z}^t\oplus\mathbb{Z}_{m_1}\oplus\cdots\oplus\mathbb{Z}_{m_l})\\
    &=(\mathbb{Z}^s\otimes_\mathbb{Z}\mathbb{Z}^t)\bigoplus_{i=1}^k(\mathbb{Z}_{n_i}\times_{\mathbb{Z}}\mathbb{Z}^t)\bigoplus_{j=1}^l(\mathbb{Z}^s\otimes_\mathbb{Z}\mathbb{Z}_{n_j})\bigoplus_{i,j}(\mathbb{Z}_{n_i}\otimes_\mathbb{Z}\mathbb{Z}_{m_j})\\
    &=0
\end{align*}

Since this is only possible if each individual tensor product is zero, we immediately see that $s=t=0$. Therefore, we need only show that $\mathbb{Z}_n\otimes_\mathbb{Z}\mathbb{Z}_m=0$ implies that $n$ and $m$ are coprime. In fact, we will show something far stronger:

\begin{claim} $\mathbb{Z}_n\otimes_\mathbb{Z}\mathbb{Z}_m\cong\mathbb{Z}_d$ with $d=\gcd(m,n)$.
\begin{proof} To do this, we let $f:\mathbb{Z}_n\times\mathbb{Z}_m\to\mathbb{Z}_d$ defined by $f(a,b)=(a\mod d,b\mod d)$ which is well defined because $d=\gcd(m,n)$.

Now, by the universal property of tensor products, because $\mathbb{Z}_d$ is abelian, there exists a map $\varphi:\mathbb{Z}_n\otimes_\mathbb{Z}\mathbb{Z}_m\to\mathbb{Z}_d$ such that $f=\varphi\circ i$ where $i:\mathbb{Z}_n\times\mathbb{Z}_m\to \mathbb{Z}_n\otimes_\mathbb{Z}\mathbb{Z}_m$ defined by $i(a,b)=a\otimes b$.

Now, if $f(a,b)=(0,0)$ then $d|a$ and $d|b$. Therefore, $$\frac{n}{d}(a\otimes b)=n\frac{a}{d}\otimes b=0\qquad a/d\text{ has order dividing }n$$ and similarly, $$(a\otimes b)\frac{m}{d}=a\otimes \frac{b}{d}m=0$$

Therefore, the order of $a\otimes b$ divides $n/d$ and $m/d$. However, $d=\gcd(m,n)$ so $n/d$ and $m/d$ are coprime so $a\otimes b$ has order $1$ and is trivial.

Thus, $\ker(f)\subset=\ker(\varphi\circ i)\subset \ker(i)$. However, clearly $\ker(i)\subset\ker(\varphi\circ i)$ so $\ker(f)=\ker(i)$ and therefore, $\ker(\varphi)=(0).$

Finally, $f$ is certainly surjective since $d|n$ and $d|m$ so $\varphi$ must be surjective as well.

Therefore, $\varphi$ is an isomorphism.
\end{proof}
\end{claim}
\vspace{0.1cm}

Finally, from the claim, $\mathbb{Z}_n\otimes_\mathbb{Z}\mathbb{Z}_m=0$ forces $\gcd(n,m)=1$ and so $n_i$ and $m_j$ are coprime for all $i,j$. Namely, $|G|$ and $|H|$ are coprime.

\end{solution}
\newpage

\begin{problem} $\,$
Let $S$ and $T$ be diagonalizable endomorphisms of a finite dimensional complex vector space. If $S$ and $T$ commute show that they are polynomials in each other.
\end{problem}


\begin{solution}$\,$
First, we note that it is necessary that either $S$ or $T$ has distinct eigenvalues.

For example, $I=\begin{bmatrix}
1 & 0\\
0 & 1
\end{bmatrix}$ and $A=\begin{bmatrix}
1 & 0\\
0 & 0
\end{bmatrix}$ are both diagonalizable matrices, and so represent diagonalizable endomorphisms from $\mathbb{R}^2\to\mathbb{R}^2$. Furthermore, $IA=A=AI$ so both matrices commute.

However, $A^n=A$ for all $n$ and so if $I$ is a polynomial in $A$ it is of the form $I=aA+bI$ which implies that $I=\frac{a}{1-b}A$ which is a contradiction.

Now, assume WLOG, that $S$ has distinct eigenvalues.

Then the minimal polynomial of $S$ is the characteristic polynomial of degree $n$ by Cayley.

Now, let $M$ be the space of all matrices which commute with $S$.

It is immediate that $M$ is a subspace of $M_n(\mathbb{C})$ since it is closed under addition and scalar multiplication. Namely, if $S$ commutes with $A$ and $B$, then $$S(aA+bB)=SaA+SbB=aAS+bBS=(aA+bB)S.$$

Now, we note that $S$ commutes with itself so $S^n\in M$ for all $n\in\mathbb{N}$.

\begin{claim} $M$ has dimension $n$ and $\{I,S,S^2,...,S^{n-1}\}$ is a basis for $M.$
\begin{proof} First, because $S$ has minimal polynomial of degree $n,$ this set is certainly linearly independent in $M_n(\mathbb{C})$ and so it is in $M$ as well.

Therefore, $\deg(M)\ge n$.

Now, let $T$ commute with $S$. Let $x$ be an eigenvector of $S$ with eigenvalue $\lambda$.

Then $$S(Tx)=TSx=T\lambda x=\lambda Tx$$ so $Tx$ is also an eigenvector of $S$ with eigenvalue $\lambda$.

However, the eigenvalues of $S$ are all distinct, so the eigenvectors of $S$ associated to $\lambda$ generate a $1$-dimensional subspace. Namely, there exists $\gamma$ so $Tx=\gamma x$.

Therefore, the eigenvectors of $S$ are the same as those of $T$.

Namely, $S$ and $T$ are simultaneously diagonalizable so there exists a $P$ invertible such that $PSP^{-1}=D_1$ and $PTP^{-1}=D_2$.

Thus, \begin{align*}
    M&=\{A\in M_n(\mathbb{C})\,|\, AS=SA\}\\
    &=\{P^{-1}DP\in M_n(\mathbb{C})\,|\, DD_1=D_1D\}\\
    &\cong M'\subset\{D\in M_n(\mathbb{C})\,|\, D\text{ diagonal }\}
\end{align*}
so namely, $\dim(M)\le n$.

Since $M$ has dimension $n$ and $\{I,S,S^2,...,S^{n-1}\}$ is linearly independent in $M$, then it forms a basis for $M$.
\end{proof}
\end{claim}

Finally, from the claim, $T\in M$ and so $T$ is a linear combination of basis elements and so $T$ is a polynomial in $S$.

Similarly for $S$ being a polynomial in $T$.
\end{solution}
\newpage




\begin{problem} $\,$
What are the prime ideals of $\mathbb{Z}[x]$? What are the maximal ideals? Carefully explain your answers.
\end{problem}


\begin{solution}$\,$
\boxed{\text{Prime}} Clearly, $(0), (p),(f(x)),(p,f(x))$ are all prime whenever $f(x)$ is irreducible in $\mathbb{Z}[x]$ which is a UFD.

This is because $$\mathbb{Z}[x]/(p)\cong\mathbb{Z}_p[x]\qquad\text{ PID because }\mathbb{Z}_p\text{ is a field }$$ so namely, $\mathbb{Z}[x]$ is a domain.

Similarly, if $f(x)$ is irreducible, then $\mathbb{Z}[x]/(f(x))$ is a domain. This is because if $g(x)$ is a zero divisor in $\mathbb{Z}[x]/(f(x))$, then there exists $h_1(x)\notin (f(x))$ and $h_2(x)\notin (f(x))$ such that $g(x)h_1(x)=f(x)h_2(x)$. However, $f(x)$ irreducible in $\mathbb{Z}[x]$ which is a UFD implies that $f(x)$ is prime. So this implies that $f(x)|g(x)$ or $f(x)|h_1(x)$. Since $f(x)\nmid h_1(x)$ by the assumption that $h_1(x)\notin(f(x))$, it must be that $g(x)\in(f(x))$ and so $g(x)=0\in\mathbb{Z}[x]/(f(x))$.

Finally, $(p,f(x))$ is prime for similar reasons as the first two.

Now, assume that $P$ is a non-zero prime ideal of $\mathbb{Z}[x]$. If $f(x)\in P$ is irreducible and constant, then $f=p$ for a prime $p$, else $\mathbb{Z}[x]/P$ will not be a domain. Therefore, if every $f\in P$ is constant, then $P=(p)$ for some prime $p.$

Next, let $f(x)\in P$ be non-constant and irreducible. Note that such an $f$ must exist, else $f(x)=g(x)h(x)$ and so because $P$ is prime, either $g(x)\in P$ or $h(x)\in P$. In either case, because $f$ can have only a finite number of irreducible factors, we can proceed until $P$ contains an irreducible element.

Now, we note that if $P\cap\mathbb{Z}$ is a prime ideal of $\mathbb{Z}$ since if $ab\in P\cap\mathbb{Z}$ then either $a\in P$ or $b\in P$ and certainly $a$ or $b$ is in $\mathbb{Z}$.

Therefore, $P\cap\mathbb{Z}=(0)$ or $P\cap\mathbb{Z}=(p)$ for $p$ prime.

If $P$ does not contain $p$, then $P/(p)\cong P$ and $\mathbb{Z}[x]/(p)\cong\mathbb{Z}_p[x]$ which is a PID. Therefore, $P/(p)=(h(x))\cong P$ and so because $f$ is irreducible and $f\in P$ $P=(f(x))$.

If $P$ does contain $p$, then by the exact same reasoning, $P/(p)=(f(x))$ and so $P=(f(x),p)$ since every $h\in P$ is of the form $fk_1+pk_2$.

Therefore, the above list are the only possible prime ideals of $\mathbb{Z}[x].$

\boxed{\text{Maximal}} Now, if $M$ is a maximal ideal of $\mathbb{Z}[x]$, then $M$ is prime and $\mathbb{Z}[x]/M$ is a field. Since the only prime ideal in our above list which satisfies this criteria is $(f(x),p)$, we have that the maximal ideals are of the form $(f(x),p)$ for $f$ irreducible and $p$ prime.
\end{solution}
\newpage


\end{document}
