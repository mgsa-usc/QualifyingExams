\documentclass[12pt]{Qual}
\usepackage{preamble}

\name{Kayla Orlinsky}
\course{Algebra Exam}
\term{Fall 2013}
\hwnum{Fall 2013}

\begin{document}

\begin{problem} $\,$
Let $H$ be a subgroup of the symmetric group $S_5$. Can the order of $H$ be $15,20$ or $30?$
\end{problem}


\begin{solution}$\,$
First, $S_5$ does have a subgroup of order $20$. Since by Sylow, $n_5\equiv 1\mod 5$ and $n_5|24$, $n_5=1,6$. Since $S_5$ has no normal subgroups other than $A_5$, $n_5=6$. Therefore, by Sylow, $[S_5:N_{S_5}(P_5)]=n_5=6$ where $P_5$ is a Sylow $5$-subgroup of $S_5$.

Therefore, $N_{S_5}(P_5)$ is a subgroup of $S_5$ of order $120/6=20.$

To disprove the other subgroups, we prove a claim.
\begin{claim} For $n\ge 5$, there are no subgroups of $S_n$ with $2<[S_n:H]<n$.
\begin{proof}
Note that $A_n$ is always a subgroup of $S_n$ of index $2.$

Let $H$ be a subgroup of $S_n$ such that $2<[S_n:H]=k<n$. Let $S_n$ act on $X=S_n/H$ the set of left cosets of $H$ by left-multiplication.

Then because $2<|X|<n$, this induces a homomorphism from $S_n$ to $S_k$ where $k=|X|.$

Specifically, this defines a map \begin{align*}
    \varphi:S_n&\to S_{|X|}=S_k\\
    a&\mapsto \sigma_a
\end{align*}

where $\sigma_a:X\to X$ is defined by $\sigma_a(bH)=abH$.

Now, we note that if $a$ is in the kernel of this homomorphism, then $abH=bH$ for all $b\in S_n$ and so namely, $abh=bh'$ for $h,h'\in H$ so $a=bh'h^{-1}b^{-1}\in bHb^{-1}.$

Thus, $a\in bHb^{-1}$ for all $b\in S_n$ and so $a\in eHe^{-1}=H$.

Therefore, $\ker(\varphi)\subset H$.

Finally, we note that for $n\ge 5$, the only normal subgroups of $S_n$ are the trivial subgroup, $S_n$ itself, and $A_n$. Since $[S_n:A_n]=2<[S_n:H]<n$, $\ker(\varphi)\not=S_n$ and not $A_n$.

Namely, the kernel is trivial and so we have an embedding of $S_n$ into a symmetric group of strictly smaller degree, which is of course, nonsense.

Thus, $H$ cannot exist.
\end{proof}
\end{claim}

By the claim, since $|S_5|=120,$ If $|H|=30$ then $[S_5:H]=120/30=4<5$, so there are no subgroups of order $30$.

If $H$ had a subgroup of order $15$ and $P_2$ was a sylow $2$-subgroup of $S_5$, then $$|HP_2|=\frac{|H||P_2|}{|H\cap P_2|}=\frac{15\cdot 8}{1}=120=|G|$$ it must be that $S_5=HP_2$.

Now, in $H$, by Sylow $n_5|3$ and $n_5\equiv 1\mod 5$, so $n_5=1$, and $n_3\equiv 1\mod 3$ and $n_3|5$ so $n_3=1.$ Thus $H$ has a normal Sylow $3$ and Sylow $5$-subgroup, namely $H$ is normal, since the product of two normal subgroups is normal.

However, $S_5$ has no normal non-trivial subgroups other than $A_5$ which has order $60$. Namely, this is not possible.
\end{solution}
\newpage


\begin{problem} $\,$
Let $R$ be a PID and $M$ a finitely generated torsion module of $R.$ Show that $M$ is a cyclic $R$-module if and only if for any prime $\mathfrak{p}$ of $R$ either $\mathfrak{p}M=M$ or $M/\mathfrak{p}M$ is a cyclic $R$-module.
\end{problem}


\begin{solution}$\,$

\boxed{\implies} Assume $M$ is cyclic. Then $M=(x)=xR=\{rx\,|\,r\in R\}$ for some $x\in X$. However, then $M/PM$ is certainly cyclic since any quotient of a cyclic module must also be cyclic.

    This is because we can define $\pi:M\to M/PM$ to be the quotient map, which is surjective. Then $M/PM\cong \pi((x))=(\pi(x))$ and so is cyclic.

    \begin{mybox}
    ***Note that quotiens of cyclic modules are cyclic always. $M$ need not be torsion for this to be true.
    \end{mybox}

    \boxed{\impliedby} Assume $PM=M$ or $M/PM$ is cyclic for all \textit{nonzero} prime ideals $P$.

    By the structure theorem, there is a chain of ideals $$(d_1)\subset(d_2)\subset\cdots\subset(d_n)$$ such that $$M\cong R/(d_1)\oplus\cdots\oplus R/(d_n).$$ Note that $d_i|d_{i-1}$ for all $i$.

    If $(d_n)$ is not maximal, then there is a maximal (prime) ideal $P$ such that $(d_n)\subset P$.

    Then if $PM=P/(d_1)\oplus\cdots\oplus P/(d_n)=M$ we have that $P/(d_i)\cong R/(d_i)$ for all $i$, so $P=R$ which is a contradiction.

    Thus, $M/PM$ is cyclic so $$M/PM\cong (R/(d_1))/(P/(d_1))\oplus\cdots\oplus(R/(d_n))/(P/(d_n))\cong (R/P)^n$$

    However, $M/PM$ is cyclic and $(R/P)^n\cong R/(a)$ for some $a$ forces $n=1.$ Namely, $M$ is cyclic.

    \begin{mybox}
    ***Note that torsion is not a necessary condition, only finitely generated is necessary for the backward implication.
    \end{mybox}
\end{solution}
\newpage



\begin{problem} $\,$
Let $R=\mathbb{C}[x_1,...,x_n]$ and suppose $I$ is a proper non-zero ideal of $R$. The coefficients of a matrix $A\in M_n(R)$ are polynomials in $x_1,...,x_n$ and can be evaluated at $\beta\in\mathbb{C}^n$; write $A(\beta)\in M_n(\mathbb{C})$ for the matrix so obtained. If for some $A\in M_n(R)$ and all $\alpha\in Var(I),$ $A(\alpha)=0_{n\times n}$, show that for some integer $m,$ $A^m\in M_n(I).$
\end{problem}


\begin{solution}$\,$
By Nullstellensatz, if $A(\alpha)=0$ for all $\alpha\in V(I)$, then every polynomial in every entry of $A$ is in $\sqrt{I}$. Namely, if $f_{ij}$ is the polynomial in the $(A)_{ij}$ entry, then $f_{ij}\in\sqrt{I}$ so there exists $m_{ij}$ so $f_{ij}^{m_{ij}}\in I$.

Let $m=\lcm\{m_{ij}\}$. Then the entries of $A^{n^2}$ are sum of products of $n^2$ of the $f_{ij}$. Namely, $A^{n^2m}$ will be a sum of products where at least one of the $f_{ij}$ is raised to the power $m$, and so namely, that whole product is in $I$ because $I$ is a $2$-sided (because $R$ is commutative) ideal.

Thus, $A^{n^2m}\in M_n(I).$
\end{solution}
\newpage


\begin{problem} $\,$
If $R$ is a noetherian unital ring, show that the power series ring $R[[x]]$ is also a noetherian unital ring.
\end{problem}


\begin{solution}$\,$
We will show that every ideal of $R[[x]]$ is finitely generated. Note that a formal power series $f(x)$ is invertible if and only if its constant term is a unit. Namely, $R[[x]]$ has a unit.

Now, let $I$ be an ideal of $R[[x]].$

Then, let $$I_n=\{a\in R\,|\, ax^n+\text{ higher order terms }\in I\}.$$

Then $I_n$ is an ideal of $R$ since $I$ is an ideal of $R[[x]]$

Then we have an increasing chain  $$I_0\subset I_1\subset I_2\subset\cdots$$ since if $a\in I_n$, then $ax^n+bx^{n+1}+\cdots\in I$, so $x(ax^n+bx^{n+1}+\cdots)\in I$ so $(ax^{n+1}+bx^{n+2}+\cdots)\in I$ because $I$ is a left ideal. Therefore, $a\in I_{n+1}$ so $I_n\subset I_{n+1}$.

Finally, the chain must terminate since $R$ is noetherian, and so $I_m=I_n$ for all $m\ge n$, some $n.$ Thus, if $ax^{n+1}+\cdots\in I$ then $ax^n+\cdots\in I$.

Now, because $R$ is noetherian, all ideals are finitely generated and so let $I_i=(a_1^{(i)},a_2^{(i)},...,a_{n_i}^{(i)})$ for $i=0,...,n$. Note that we can let $m=\max\{n_i\}$ and then write $$I_i=(a_1^{(i)},a_2^{(i)},...,a_m^{(i)})\qquad a_j^{(i)}=0\forall j>n_i.$$

By definition of the $I_i$, there exist the following set of polynomials in $I$ $$F=\begin{bmatrix}
a_1^{(0)}+\cdots & a_2^{(0)}+\cdots &\cdots & a_m^{(0)}+\cdots\\
a_1^{(1)}x+\cdots & a_2^{(1)}x+\cdots &\cdots & a_m^{(1)}x+\cdots\\
& \vdots & \ddots & \vdots\\
a_1^{(n)}x^n+\cdots & a_2^{(n)}x^n+\cdots &\cdots & a_m^{(n)}x^n+\cdots
\end{bmatrix}$$

Then, if $f_{i,j}=(F)_{i,j}$ we have that $f_{i,j}\in I$ for all $i,j$.

Finally, let $f\in I$. Let $\displaystyle f(x)=\sum_{i=0}^\infty\alpha_ix^i$.

Then, $\alpha_j$ is a linear combination of the $a_i^{(j)}$ because they are exactly the generators of $I_j$. Therefore, we can write the first $n$-terms of $f$ using the $f_{i,j}$, namely, $$f(x)-\sum_{i=0}^n\sum_{j=1}^mb_j^{(i)}f_{i,j}=\alpha_{n+1}'x^{n+1}+\cdots\qquad b_j^{(i)}\in R.$$

Namely, $\alpha_{n+1}'\in I_{n+1}=I_n$ because the chain terminates at $n$.

Thus, we can write the next $n+1$ terms in the sequence in terms of the $f_{n,j}$. Specifically,
$$f(x)-\sum_{i=0}^n\sum_{j=1}^mb_j^{(i)}f_{i,j}-x^{n+1}\sum_{j=1}^nb_j^{(n)}f_{n,j}=\alpha_{2n+2}''x^{2n+2}+\cdots$$

Since the next $n+1$ block can again be generated by the $f_{n,j}$ for $j=1,...,m$ we finally have by grouping, that $$f(x)=\sum_{i=0}^n\sum_{j=1}^mb_j^{(i)}f_{i,j}+\left(\sum_{k=0}^\infty c_kx^{k(n+1)}\right)f_{n,1}+\cdots+\left(\sum_{k=0}^\infty c_k'x^{k(n+1)}\right)f_{n,m}$$ and so at last, $$I=(f_{i,j})_{i=0,..,n,j=1,...,m}$$ and is finitely generated.

Thus, $R[[x]]$ is noetherian since all its ideals are finitely generated.

\end{solution}
\newpage



\begin{problem} $\,$
Let $p$ be a prime. Prove that $f(x)=x^p-x-1$ is irreducible over $\mathbb{Z}/p\mathbb{Z}$. What is the Galois group? (Hint: observe that if $\alpha$ is a root of $f(x)$, then so is $\alpha+i$ for $i\in\mathbb{Z}/p\mathbb{Z}$.)
\end{problem}


\begin{solution}$\,$
First, note that $\mathbb{Z}_p\cong\mathbb{F}_p$. Let $\alpha$ be a root of $f$ in the algebraic closure of $\mathbb{F}_p$. Then $f(\alpha)=\alpha^p-\alpha-1=0$ so $\alpha^p-\alpha=1$. Since $$f(\alpha+i)=(\alpha+i)^p-(\alpha+i)-1=\alpha^p+i^p-\alpha-i-1=\alpha^p-\alpha-1=f(\alpha)=0$$ since $i^p=i$ for all $i\in\mathbb{F}_p.$

Thus, $f$ has $p$ roots of the form, $\alpha,\alpha+1,...,\alpha+(p-1).$

Assume $f(x)=g(x)h(x)$ for $g,h\in\mathbb{F}_p[x]$ where $g$ is the minimal polynomial of $\alpha$ (so $g$ is irreducible and has $\alpha$ as a root). Then because $\alpha\notin\mathbb{F}_p$, $g$ has at least one other $\alpha+i$ as a root. Therefore, $$f(x+i)=g(x+i)h(x+i)=f(x)=g(x)h(x).$$

Thus, $g(x+i)$ is monic and also irreducible and also has $\alpha$ as a root, and so $g(x)=g(x+i)$. However, then the permutation $x\mapsto x+i$ preserves the roots of $g$, so $g$ has the same roots as $f$ and so $g=f.$

Thus, $f$ is irreducible.

Finally, let $L=\mathbb{F}_p(\alpha)$. Then $L$ is the splitting field for a separable polynomial and so $L/\mathbb{F}_p$ is Galois.

Clearly $[L:\mathbb{F}_p]=p$ and $G=\gal(L/\mathbb{F}_p)$ is generated by $\alpha\mapsto\alpha+1$. Thus, $G\cong\mathbb{Z}_p.$
\end{solution}
\newpage




\begin{problem} $\,$
Let $R$ be a finite ring with no nilpotent elements. Show that $R$ is a direct product of fields.
\end{problem}


\begin{solution}$\,$
Since $R$ is finite, it is necessarily artinian.

Let $x\in J(R)$. Then because $J(R)$ is right quasi-regular, $1-x$ is a unit in $R$.

Then, we construct a decreasing chain of ideals $$(x)\supset (x^2)\supset\cdots$$ which must terminate for some $n$. Namely, $(x^n)=(x^{n+1})$ so $x^n=rx^{n+1}$ for some $r\in R$. However, $rx\in J(R)$ and so $1-rx$ is a unit. Therefore, $$x^n=rx^{n+1}\implies x^n(1-rx)=0\implies x^n=0.$$

Namely, $x$ is nilpotent. Since $R$ has no nilpotent elements, $J(R)=0.$

Thus, by Artin Wedderburn, $$R\cong M_{n_1}(D_1)\oplus\cdots\oplus M_{n_k}(D_k)$$ where the $D_k$ are division rings.

Now, $R$ contains no nilpotent elements, however matrix rings contain nilpotent elements over any division ring, since $$\begin{bmatrix}
0 & 0 &\cdots & 0 & 1\\
0 & 0 &\cdots & 0 & 0\\
& \vdots &\ddots & \vdots &\\
0 & 0 &\cdots & 0 & 0
\end{bmatrix}$$ is nilpotent of degree $2$ over any division ring where $1\not=0$.

Namely, $n_i=1$ for all $i$.

Finally, because the $D_i$ are finite, by Wedderburn, the $D_i$ are all fields.

Thus, $R$ is a finite direct sum (isomorphic to a finite direct product) of fields.
\end{solution}
\newpage



\begin{problem} $\,$
Let $K\subset\mathbb{C}$ be the field obtained by adjoining all roots of unity in $\mathbb{C}$ to $\mathbb{Q}$. Suppose $p_1<p_2$ are primes, $a\in\mathbb{C}\backslash K$, and write $L$ for a splitting field of $$g(x)=(x^{p_1}-a)(x^{p_2}-a)$$ over $K.$ Assuming each factor of $g(x)$ is irreducible, determine the order and the structure of $\gal(L/K).$
\end{problem}


\begin{solution}$\,$
First, $g(x)$ is not a polynomial in $K[x]$, since $a\not\in K$. However, if we assume that $a\in\mathbb{Q}$ is such that each factor of $g(x)$ is irreducible, then we do have that $g\in K[x]$.

Then, since $L$ is the splitting field of a separable polynomial (since each factor of $g$ is irreducible over $\mathbb{Q}$, it is separable), we have that $L/K$ is Galois.

 Furthermore, each $\sigma\in G=\gal(L/K)$ will be uniquely determined by how it permutes the roots of each irreducible factor.

Namely, $G$ will be generated by the $\sigma_i$, where $\sigma_i$ is a permutation of the roots of $x^{p_i}-y$, fixing the other roots of $g.$

This implies that $G$ will be abelian since each $\sigma_i$ will fix all but the $p_i\thh$ roots of unity and will fix all $p_i\thh$ roots of $y.$

Therefore, $$G\cong\mathbb{Z}_{p_1p_2}.$$
\end{solution}



\end{document}
