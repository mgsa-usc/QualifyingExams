\documentclass[12pt]{AlgebraQual}
\usepackage{preamble}

\name{Kayla Orlinsky}
\course{Algebra Exam}
\term{Spring 2017}
\hwnum{Spring 2017}

\begin{document}

\begin{problem} $\,$
Let $R$ be a PID. Let $M$ be an $R$-module.
\begin{enumerate}[label=(\alph*)]
    \item Show that if $M$ is finitely generated, then $M$ is cyclic if and only if $M/PM$ is for all prime ideals $P$ of $R$.
    \item Show that the previous statement is false if $M$ is not finitely generated.
\end{enumerate}
\begin{mybox}
***Note as written this problem is wrong unless we assume $P$ is a \textit{nonzero} prime ideal.
\end{mybox}
\end{problem}


\begin{solution}$\,$
\begin{enumerate}[label=(\alph*)]
    \item This is very similar to \textbf{Fall 2013: Problem 2}.

    Let $M$ be finitely generated.

    \boxed{\implies} Assume $M$ is cyclic. Then $M=(x)=xR=\{rx\,|\,r\in R\}$ for some $x\in X$. However, then $M/PM$ is certainly cyclic since any quotient of a cyclic module must also be cyclic.

    This is because we can define $\pi:M\to M/PM$ to be the quotient map, which is surjective. Then $M/PM\cong \pi((x))=(\pi(x))$ and so is cyclic.

    \boxed{\impliedby} Assume $M/PM$ is cyclic for all \textit{nonzero} prime ideals $P$.

    By the structure theorem, there is a chain of ideals $$(d_1)\subset(d_2)\subset\cdots\subset(d_n)$$ such that $$M\cong R/(d_1)\oplus\cdots\oplus R/(d_n).$$ Note that $d_i|d_{i-1}$ for all $i$.

    If $(d_n)$ is not maximal, then there is a maximal (prime) ideal $P$ such that $(d_n)\subset P$.

    Then $PM=P/(d_1)\oplus\cdots\oplus P/(d_n)$ so
    Then $$M/PM\cong (R/(d_1))/(P/(d_1))\oplus\cdots\oplus(R/(d_n))/(P/(d_n))\cong (R/P)^n$$

    However, $M/PM$ is cyclic for all $P$, and $(R/P)^n\cong R/(a)$ for some $a$ forces $n=1.$ Namely, $M$ is cyclic.

    \item As written, the problem is true regardless of whether or not $M$ is finitely generated. Since $R$ is a PID, it is a domain, so $P=(0)$ is a prime ideal. However, then $PM=(0)M=0$ so $M/PM\cong M$ is cyclic.

    However, with the assumption that we may use only \textit{nonzero} prime ideals, let $R=\mathbb{Z}$ which is a PID and $M=\mathbb{Q}$. Then $M$ is infinitely generated.

    The \textit{nonzero} prime ideals of $R$ are exactly ideals generated by $(p)$ where $p$ is a prime and $(p)M=M$.

    Therefore, $$M/(p)M\cong M/M\cong (0)$$ which is certainly cyclic.

    However, $M$ is not cyclic.
\end{enumerate}
\end{solution}
\newpage


\begin{problem} $\,$
Prove that a power of the polynomial $(x+y)(x^2+y^4-2)$ belongs to the ideal $(x^3+y^2,x^3+xy)$ in $\mathbb{C}[x,y]$.
\end{problem}


\begin{solution}$\,$
It suffices to show that $(x+y)(x^2+y^4-2)$ is satisfied by all zeros in $V(x^3+y^2,x^3+xy)$.

By Nullstellenzatz, if $g(x,y)$ is a polynomial such that $g(a,b)=0$ for all $(a,b)\in V(I)$, then there exists an $n$ such that $g^n(x,y)\in I$.

Let $g(x,y)=(x+y)(x^2+y^4-2)$

Now, we examine $V(x^3+y^2,x^3+xy)$.

Clearly $(0,0)\in V(x^3+y^2,x^3+xy)$. If $x^3+y^2=0$ and $x^3+xy=0$ then $y^2-xy=0$, so $y(y-x)=0$.

If $y=0$ then $x=0$, and if $y=x$, then $x^2(x+1)=0$, so $x=-1$.

Thus, the only elements of $V(x^3+y^2,x^3+xy)$ are $(0,0),(-1,-1)$.

Since $g(0,0)=0$ and $g(-1,-1)=0$, we have that there exists an $n$ such that $g^n(x,y)\in(x^3+y^2,x^3+xy).$
\end{solution}
\newpage



\begin{problem} $\,$
Let $G$ be a finite group with a cyclic Sylow $2$-subgroup $S$.
\begin{enumerate}[label=(\alph*)]
    \item Show that $N_G(S)=C_G(S)$.
    \item Show that if $S\not=1$, then $G$ contains a normal subgroup of index $2$. (hint: suppose that $n=[G:S]$, consider an appropriate homomorphism from $G\to S_n$).
    \item Show that $G$ has a normal subgroup $N$ of odd order such that $G=NS$.
\end{enumerate}
\end{problem}


\begin{solution}$\,$
This problem is very similar to \textbf{Spring 2011: Problem 1.}
\begin{enumerate}[label=(\alph*)]
    \item We will prove the stronger version of this problem using \textbf{Spring 2011: Problem 1, (a)}.

    \begin{claim} If $p$ is the smallest prime dividing $|G|$ and $P$ is a cyclic Sylow $p$-subgroup, then $N_G(P)=C_G(P).$
    \begin{proof} Let $p$ be the smallest prime dividng $|G|.$ Then, since $$P\trianglelefteq C_G(P)\trianglelefteq N_G(P)$$ we have that $$[N_G(P):C_G(P)]=n\qquad \gcd(n,p)=1.$$ Furthermore, because $p$ is the smallest prime dividing $|G|,$ $n$ is only divisible by primes $q$ with $q>p.$

    Now, let \begin{align*}
        \varphi:N_G(P)&\to\aut(P)\\
        a&\mapsto\sigma_a
    \end{align*} be the map of the conjugation action of $N_G(P)$ on $P.$

    Then $C_G(P)$ is clearly the kernel of this action and so by the first isomorphism theorem, $$N_G(P)/C_G(P)\cong A\subset\aut(P).$$

    Finally, because $P=\langle x\rangle$ is cyclic, we have that the automorphisms of $P$ are exactly the maps $x\mapsto x^k$ for $\gcd(k,p)=1$. Namely, $$|\aut(P)|=p^{l-1}(p-1)\qquad\text{ by the Euler Totient Function}$$ assuming that $|P|=p^l$. Since the divisors of this are not greater than $p,$ and $|N_G(P)/C_G(P)|$ has only divisors greater than $p,$ it must be that $|N_G(P)/C_G(P)|=1.$

    Namely, $$N_G(P)=C_G(P).$$
    \end{proof}
    \end{claim}

    \item Assume $S\not=1$. Then let $n=|G|/|S|$. Note that $n$ is odd. Let \begin{align*}
        \varphi:G&\to S_{|G|}\\
        a&\mapsto \tau_a
    \end{align*} where $\tau_a(g)=ag$ is the left multiplication map.

    Then $\varphi$ is certainly injective since $\tau_a=\id$ if and only if $a=e$.

    Now, if $S=\langle a\rangle$, then $\varphi(a)=\tau_a$ is a cycle of order $|S|$ which is even. Now, let $g\in G$, then $\tau_a(g)=ag$ and $\tau_a(ag)=a^2g$ so $\varphi(a)$ has a cycle of the form $(g,ag,a^2g,...,a^{|S|-1}g)$. Since $$\tau_a(a^k)=a^{k+1}=\tau_{a^{k+1}}(e)=(\tau_a)^{k+1}(e),$$ we see that $$\varphi(a)=(a,a^2,...,a^{|S|-1})\prod_{g\in G\backslash S}(g,ag,a^2g,...,a^{|S|-1}g).$$ Namely, $\varphi(a)$ is a product of $n$ cycles of even length, so $\varphi(a)$ is an odd permutation.

    Finally, let $$\sgn:S_{|G|}\to \{1,-1\}$$ be the sign map. Then since $\sgn(\id)=1$, and $\sgn(\varphi(a))=-1$, we have that $$\sgn\circ\varphi:G\to \{1,-1\}$$ is surjective.

    Therefore, $G/\ker(\sgn\circ\varphi)\cong\mathbb{Z}_2$ so $G$ has a normal (because it is a kernel) subgroup, $H=\ker(\sgn\circ\varphi)$ of index $2$.

    \item Let $|G|=2^rn$. Then we proceed by induction on $r$.

    For $r=1$, we are done since by (b), $G$ has a normal subgroup $H$ of index $2$. Namely, $|H|=n$. Therefore, $$|SH|=|S||H|/|S\cap H|=|S||H|/1=2n=|G|$$ and since $H$ is normal, $SH$ is a subgroup of $G$ so $SH=G$.

    Now, assume the statement holds for all $1\le k\le r$. Then let $|G|=2^{r+1}n$ and have a cylic Sylow $2$-subgroup $S$.

    From (b), $G$ has a normal subgroup $H$ of order $2^rn$. Now, $S\cap H$ will also be cyclic subgroup. Now, $H$ is normal so $SH$ is a subgroup of $G$. Since $S\not\subset H$, it must be that $|SH|>H$ so $SH=G$.

    Finally $$|S\cap H|=|S||H|/|SH|=2^{r+1}2^rn/2^{r+1}n=2^r$$ so $H$ has a cyclic Sylow $2$-subgroup $S\cap H$.

    Therefore, by the inductive hypothesis, there exists an $N$ normal subgroup of $H$ of order $n$ such that $H=(S\cap H)N$. Now, $N$ is also a subgroup of $G$ so it suffices to show that $N$ is normal.

    However, clearly any element $g\in G$ normalizes $n$. Since $N$ is exactly all the elements in $G$ of odd order. Therefore, $gng^{-1}$ has odd order and so it is in $N$.

    Thus, $N$ is normal in $G$ so $$G=SN.$$
\end{enumerate}
\end{solution}
\newpage


\begin{problem} $\,$
Show that $\mathbb{Z}[\sqrt{5}]$ is not integrally closed in its quotient field.
\end{problem}


\begin{solution}$\,$
First, we note that if $a+b\sqrt{5}\in\mathbb{Q}[\sqrt{5}]$, then $a+b\sqrt{5}$ satisfies $$(x-a-b\sqrt{5})(x-a+b\sqrt{5})=x^2-2ax+a^2-5b^2\in\mathbb{Q}[x].$$ And this polynomial is minimal over $\mathbb{Q}[x]$, since $a+b\sqrt{5}\notin\mathbb{Q}$.

Now, clearly $\frac{1+\sqrt{5}}{2}$ is in the field of fractions of $\mathbb{Z}[\sqrt{5}]$. Furthermore, it has minimal polynomial $$x^2-\frac{2}{2}x+\frac{1}{4}-\frac{5}{4}=x^2-x-1\in\mathbb{Z}[x].$$

Therefore, $\frac{1+\sqrt{5}}{2}$ is integral over $\mathbb{Z}$ and so it is integral over $\mathbb{Z}[\sqrt{5}]$. However, clearly $\frac{1+\sqrt{5}}{2}\notin\mathbb{Z}[\sqrt{5}]$ so $\mathbb{Z}[\sqrt{5}]$ is not integrally closed.
\end{solution}
\newpage



\begin{problem} $\,$
Let $f(x)=x^{11}-5\in\mathbb{Q}[x]$.
\begin{enumerate}[label=(\alph*)]
    \item Show that $f$ is irreducible in $\mathbb{Q}[x].$
    \item Let $K$ be the splitting field of $f$ over $\mathbb{Q}.$ What is the Galois group of $K/\mathbb{Q}$.
    \item How many subfields $L$ of $K$ are there such that $[K:L]=11$.
\end{enumerate}
\end{problem}


\begin{solution}$\,$
\begin{enumerate}[label=(\alph*)]
    \item We will apply Eisenstein's with $p=5$. Then $p$ does not divide the leading coefficient of $f$, $p$ does divide every other coefficient, and $p^2$ does not divide the constant term.

    Therefore, by Eisenstein's Criteion, $f(x)$ is irreducible over $\mathbb{Q}[x]$.
    \item Let $K$ be the splitting field of $f$ over $\mathbb{Q}.$

    Let $z^{11}=5$ and $z=re^{i\theta}$. Then $r=\sqrt[11]{5}$ and $11\theta=2k\pi$ for $k=1,...,11$. Clearly, $\sqrt[11]{5}\xi$ where $\xi=e^{2i\pi/11}$ is a primitive root of $f(x)$.

    Therefore, $$K=\mathbb{Q}(\sqrt[11]{5},\xi).$$ Now, since $\xi^{11}$ is primitive, it satisfies $g(x)=x^{10}+x^9+\cdots+x+1$.

    Therefore, $$[K:\mathbb{Q}]=[K:\mathbb{Q}(\sqrt[11]{5})][\mathbb{Q}(\sqrt[11]{5}):\mathbb{Q}]=[K:\mathbb{Q}(\sqrt[11]{5})]11$$ and $$[K:\mathbb{Q}]=[K:\mathbb{Q}(\xi)][\mathbb{Q}(\xi):\mathbb{Q}]=[K:\mathbb{Q}(\xi)]10.$$

    Thus, $110$ divides $[K:\mathbb{Q}]$ and since $[K:\mathbb{Q}]\le10$, we have that $[K:\mathbb{Q}]=110.$ Now, since $K$ is the splitting field of a separable (no repeated roots) polynomial, $K/\mathbb{Q}$ is Galois so $G=\gal(K/\mathbb{Q})$ exists and $|G|=110=2\cdot5\cdot11$.

    Now, let $\sigma:K\to K$ be an element of $G$. Then $\sigma(\sqrt[11]{5})=\sqrt[11]{5}\xi^i$ and $\sigma(\xi)=\xi^j$ for some $i,j=1,...,11$.

    Note that if $\sigma(\sqrt[11]{5})=\sqrt[11]{5}$ and $\sigma(\xi)=\xi^i$ and $\tau(\sqrt[11]{5})=\sqrt[11]{5}\xi^j$ and $\tau(\xi)=\xi$, then $$\sigma\tau(\sqrt[11]{5}\xi)=\sigma(\sqrt[11]{5}\xi^{j+1})=\sqrt[11]{5}\xi^{(j+1)i}$$ and $$\tau\sigma(\sqrt[11]{5}\xi)=\tau(\sqrt[11]{5}\xi^i)=\sqrt[11]{5}\xi^{j+i}$$

    Namely, we obtain immediately that $G$ is non-abelian.

    Finally, if $\sigma:K\to K$ is defined by $\sigma(\sqrt[11]{5})=\sqrt[11]{5}\xi$ and $\sigma(\xi)=\xi^2$, then one can check that $\sigma$ has order $10$.

    Namely, $G$ has a subgroup $H=\langle\sigma\rangle$ of order $10$.

    Now, if $P_{11}$ is a Sylow $11$-subgroups, and since $n_{11}$ the number of Sylow $11$-subgroups must divide $|G|/11=10$ by the Sylow theorems, $n_{11}=1$. Note that $\rho:K\to K$ defined by $\rho(\sqrt[11]{5})=\sqrt[11]{5}\xi^2$ and $\rho(\xi)=\xi$ has order $11$. Thus, $P_{11}=\langle \rho\rangle.$

    So $P_{11}\cong\mathbb{Z}_{11}$ is normal in $G$. Therefore, $P_{11}H$ is a subgroup of $G$ and since $|P_{11}H|=|G|$, we have that $G$ must be a semi-direct product of $P_{11}$ and $H$.

    Now, we must identify the multiplication on $G$. Since $$G\cong \langle \rho\rangle\rtimes_\varphi \langle\sigma\rangle$$ where $\varphi:\langle \sigma\rangle\to\mathbb{Z}_{10}$ and multiplication on $G$ is defined by $\varphi(\sigma)(\rho)=\sigma\rho\sigma^{-1}=\rho^t$ for some $t$ such that $\rho\mapsto \rho^t$ is an automorphism of $P_{11}$.

    Since $\sigma(\sqrt[11]{5}\xi^5)=\sqrt[11]{5}\xi\xi^{10}=\sqrt[11]{5}$, we have that $\sigma^{-1}(\sqrt[11]{5})=\sqrt[11]{5}\xi^5$ and $\sigma(\xi^6)=\xi$ so  \begin{align*}
        \sigma\rho\sigma^{-1}(\sqrt[11]{5})&=\sigma\rho(\sqrt[11]{5}\xi^5)\\
        &=\sigma(\sqrt[11]{5}\xi^7)\\
        &=\sqrt[11]{5}\xi^{15}\\
        &=\sqrt[11]{5}\xi^4\\
        \sigma\rho\sigma^{-1}(\xi)&=\sigma\rho(\xi^6)\\
        &=\sigma(\xi^6)\\
        &=\xi
    \end{align*} so $\sigma\rho\sigma^{-1}=\rho^2.$

    Therefore, $$G\cong\langle \sigma,\rho\,|\,\sigma^{10}=\rho^{11}=1,\sigma\rho\sigma^{-1}=\rho^2\rangle.$$

    \item The subfields $L$ of $K$ such that $[K:L]=11$ correspond exactly to the subgroups $H$ of $G$ such that $|H|=11$ (namely, so $[G:H]=|G|/11=10$).

    Since if $H$ is a subgroup of $G$ of order $11$, it is a Sylow $11$-subgroup, and since $n_{11}$ the number of Sylow $11$-subgroups must divide $|G|/11=10$ by the Sylow theorems, $n_{11}=1$.

    Thus, $G$ has exactly one Sylow $11$ subgroup and it is normal.

    Thus, $K$ contains one subfield $L$ such that $[K:L]=11$ and in fact, $L/\mathbb{Q}$ is Galois.
\end{enumerate}
\end{solution}
\newpage



\begin{problem} $\,$
Suppose that $R$ is a finite ring with $1$ such that every unit of $R$ has order dividing $24$. Classify all such $R$.
\end{problem}



\begin{solution}$\,$
Since $R$ is finite, it is trivially artinian. Now, $R'=R/J(R)$ has trivial Jacobson and is also artinian. Thus, by Artin Wedderburn, $R'\cong M_{n_1}(D_1)\oplus\cdots\oplus M_{n_k}(D_k)$ where $D_i$ are division rings.

Since $R$ is finite, each $D_i$ is finite, and so each must be a finite field. Note that if $|D_i|=p_i^m$ then $|D_i^\times|=p_i^{m-1}(p_i-1)$. Furthermore, $D_i^\times$ is the group of units of $D_i$ and so since each unit of $R'$ divides $24$, we have that $p_i^{m-1}(p_i-1)|24$. Namely, we have the follwing options for pairs, $$(p,m)=(2,1),(2,2),(2,3),(2,4),(3,1),(3,2),(5,1),(7,1),(13,1).$$ Alternatively, $$|D_i|=2,4,8,16,3,9,5,7,13.$$

Now, units in $M_{n_i}(D_i)$ are exactly elements in $GL_{n_i}(D_i)$. Since $$|GL_{n_i}(D_i)|=(|D_i|^{n_i}-1)(|D_i|^{n_i}-|D_i|)\cdots(|D_i|^{n_i}-|D_i|^{n_i-1}),$$ we now have that $(|D_i|^{n_i}-1)(|D_i|^{n_i}-|D_i|)\cdots(|D_i|^{n_i}-|D_i|^{n_i-1})$ must divide $24$. If $n_i=1$, then we simply hav a copy of $D_i$ which we have already found.

Now, this gives the following possible pairs $$(n_i,|D_i|)=(2,2).$$ Everything else grows past $24$.

Therefore, $R'$ is a direct sum of copies of $D_i$, which can be any of the finite fields previously described and some number of copies of $M_2(\mathbb{Z}_2)$.

Finally, since $R/J(R)$ is finite, it is finitely generated as an $R$-module.

Write $R/J(R)=\overline{x_1}R+\cdots+\overline{x_n}R$ for some $\overline{x_i}=x_i+J(R)\in R/J(R)$.

Then, by Nakayama's Lemma, $R\cong x_1R+\cdots+x_nR$.

This fully describes $R.$
\end{solution}
\newpage


\end{document}
