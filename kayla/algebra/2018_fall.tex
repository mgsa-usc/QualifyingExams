\documentclass[12pt]{AlgebraQual}
\usepackage{preamble}

\name{Kayla Orlinsky}
\course{Algebra Exam}
\term{Fall 2018}
\hwnum{Fall 2018}

\begin{document}

\begin{problem} $\,$
Let $\mathbb{F}_p$ be a finite field with $p$ elements, and consider the group $GL_n(\mathbb{F}_p)$. Write down the order of $GL_n(\mathbb{F}_p)$ and a Sylow $p$-subgroup.
\end{problem}


\begin{solution}$\,$
If $X\in GL_n(\mathbb{F}_p)$, then $X$ must be an invertible $n\times n$ matrix with elements in $\mathbb{F}_p$. Namely, $X$ must have linearly independent columns. If $[x_i]$ are the columns of $X$, then the first column $x_1$ can by anything except the zero vector, which vies $p^n-1$ possible options.

The second column $x_2$ can be anything but a multiple of the first column. So once $x_1$ is chosen, $x_2\not=ax_1$, there are $p$ vectors that $x_2$ cannot be. Namely, there are $p^n-p$ choices for $x_2$.

Inductively, we can see that there are $p^n-p^k$ choices for $x_{k+1}$ $0\le k\le n-1$.

Thus, $$|GL_n(\mathbb{F}_p)|=(p^n-1)(p^n-p)\cdots(p^n-p^{n-1}).$$

\begin{mybox}
***Although it was not asked, we can note that the determinant function $$\det:GL_n(\mathbb{F}_p)\to\mathbb{F}_p^*$$ is a surjective homomorphism with kernel $SL_n(\mathbb{F}_p)$. Namely, $$\left|\frac{GL_n(\mathbb{F}_p)}{SL_n(\mathbb{F}_p)}\right|=|\mathbb{F}_p^*|=p-1$$ and so $$|SL_n(\mathbb{F}_p)|=\frac{|GL_n(\mathbb{F}_p)|}{p-1}.$$

We further note that if instead we were interested in $\mathbb{F}_q$ where $q=p^k$, then we could replace $p$ with $q$ in all instances and achieve the same results.
\end{mybox}

Finally, we claim that if $P$ that set of all upper triangular matrices with $1$ down the main diagonal forms a Sylow $p$-subgroup.

First, there are $$\frac{n^2-n}{2}$$ entries in matrices of this form, and $p$ possible choices for each entry, so $|P|=p^{(n-1)n/2}$.

Since Sylow $p$-subgroups have order $p^{n-1}p^{n-2}\cdots p$ or $p^{(n-1)n/2}$, we have that $P$ has the right size.

Thus, if $P$ is a subgroup it is a Sylow $p$-subgroup.

However, this is trivial since products of upper triangular matrices are upper triangular and inveres of upper triangular matrices are also upper triangular.

Since $\det(Y)=1$ if $Y\in P$, we also get that $Y^{-1}\in P$. Note that the determinant of an upper triangular matrix is the prdouct of the entries down the main diagonal.

Thus, $P$ is a subgroup and so it is a Sylow $p$-subgroup.
\end{solution}
\newpage




\begin{problem} $\,$
Prove that there are no simple groups of order $600.$
\end{problem}


\begin{solution}$\,$
Let $G$ be a group of order $600=10\cdot 10\cdot 6=2^3\cdot 3\cdot 5^2$.

Then, by Sylow Theorems, $n_5\equiv 1\mod5$ and $n_5|2^3\cdot 3$.

Namely, $n_5=1,6$.

Now, if $G$ is simple, then $n_5=6$ and so since Sylow $5$-subgroups are conjugates, $G$ can act on its Sylow $5$ subgroups by conjugation.

This action defines a homomorphism $$\varphi:G\to S_6$$ where $\varphi(g)=\sigma_g$ and $\sigma_g:\syl_5(G)\to\syl_5(G)$ with $\sigma_g(P_5)=gP_5g^{-1}$ and $P_5$ a Sylwo $5$-subgroup of $G$.

Now, since kernels of homomorphisms are normal subgroups in the domain, $\ker\varphi$ must be trivial. Namely, $\varphi$ must be an embedding.

However, $|S_6|=6!=720$, and since $|G|=600$ which does not divide $720$, there cannot be any isomorphic copies of $G$ inside $S_6$.

This is a contradiction and so $G$ cannot be simple.
\end{solution}
\newpage



\begin{problem} $\,$
Prove that $\mathbb{Z}[\sqrt{10}]$ is integrally closed in its field of fractions, but not a UFD.
\end{problem}


\begin{solution}$\,$
Let $a+b\sqrt{10}\in\mathbb{Q}[\sqrt{10}]$. Note that this is the field of fractions of $\mathbb{Z}[\sqrt{10}]$ since for $s,t,p,q\in\mathbb{Z}$, $$\frac{s+t\sqrt{10}}{p+q\sqrt{10}}=\frac{(s+t\sqrt{10})(p-q\sqrt{10})}{p^2-10q^2}\in\mathbb{Q}[\sqrt{10}].$$

Then we note that $m_{a,b}(x)=(x-a-b\sqrt{10})(x-a+b\sqrt{10})=x^2-2ax+a^2-10b^2$ is the minimal polynomial of $a+b\sqrt{10}$ over $\mathbb{Q}$. $m_{a,b}(x)$ is irreducible over $\mathbb{Q}$ and so it is also irreducible over $\mathbb{Z}$.

Now, if $a+b\sqrt{10}$ is integral over $\mathbb{Z}[\sqrt{10}]$ it satisfies $f(x)$ a monic irreducible polynomial with coefficients in $\mathbb{Z}[\sqrt{10}]$. Since $f(x)$ is irreducible, over $\mathbb{Z}[\sqrt{10}]$, by Gauss' Lemma, $f(x)$ is also irreducible over $\mathbb{Q}[\sqrt{10}]$. However, then $m_{a,b}(x)$ must divide $f(x)$ in $\mathbb{Q}[\sqrt{10}]$ and so by irreducibly, $f(x)=um_{a,b}(x)$ for $u$ a unit. Namely, $m_{a,b}(x)$ has coefficients in $\mathbb{Z}$.

Therefore, $a+b\sqrt{10}$ is integral over $\mathbb{Z}[\sqrt{10}]$ if and only if $m_{a,b}(x)$ has coefficients in $\mathbb{Z}$.

Now, if $-2a\in\mathbb{Z}$ and $a^2-10b^2\in\mathbb{Z}$ then $$4(a^2-10b^2)=(2a)^2-10(2b)^2\in\mathbb{Z}$$ however, $2a\in\mathbb{Z}$ and so $10(2b)^2\in\mathbb{Z}$.

Since $10$ is squarefree, it cannot be that $(2b)^2\notin\mathbb{Z}$ so $2b\in\mathbb{Z}$ as well.

Finally, $4(a^2-10b^2)=(2a)^2-10(2b)^2=4k\qquad k\in\mathbb{Z}$ and so $(2a)^2=2(5(2b)^2+2k),$ since $2b\in\mathbb{Z}$ we get that $(2a)^2$ is even. However, if $2|(2a)^2$ then $2|(2a)$. Namely, $a\in\mathbb{Z}$.

Immediately then, it must be that $b\in\mathbb{Z}$ since again, $10$ is squarefree and $10b^2\in\mathbb{Z}$.

Thus, $\mathbb{Z}[\sqrt{10}]$ is integrally closed.

Now, let $N(a+b\sqrt{10})=a^2-10b^2$ be the norm function on $\mathbb{Z}[\sqrt{10}]$. Note that $$N(xy)=xy\overline{xy}=xy\overline{x}\overline{y}=x\overline{x}y\overline{y}=N(x)N(y)$$ and that $$N:\mathbb{Z}[\sqrt{10}]\to\mathbb{Z}.$$

We note that if $N(a+b\sqrt{10})=3$, then \begin{align*}
    a^2-10b^2&=3\\
    a&=2n+1\qquad n\in\mathbb{Z}\\
    (2n+1)^2-10b^2&=3\\
    4n^2+4n+1-10b^2&=3\\
    2n^2+2n-1&=5b^2\\
    b&=2k+1\qquad k\in\mathbb{Z}\\
    4n^2+4n+1-10(2k+1)^2&=3\\
    4n^2+4n+1-10(4k^2+4k+1)&=3\\
    4[n^2+n-10k^2-10k]-9&=3\\
    n^2+n-10k^2-10k&=3\\
    n^2+n\equiv 1\mod 2
\end{align*} and this is not possible since if $n$ is odd, then $n^2+n$ is even, and if $n$ is even, then $n^2+n\equiv 0\mod 2$.

Thus, $N(x)\not3$ for all $x\in\mathbb{Z}[\sqrt{10}].$

Namely, since $N(3)=9$, if $3$ were reducible, then we would get $$N(3)=9=N(ab)=N(a)N(b).$$ However, since $N(a)\not3$, and $N(b)\not3$, then either $a$ or $b$ is a unit.

Therefore, $3$ is irreducible.

Now, since in $\mathbb{Z}[\sqrt{10}]$, $$9=3\cdot 3=-(1-\sqrt{10})(1+\sqrt{10}),$$ if $\mathbb{Z}[\sqrt{10}]$ were a UFD, then $3$ would be prime, since it is irreducible.

Namely, $3$ must divide $1+\sqrt{10}$.

However, if $$1+\sqrt{10}=3(a+b\sqrt{10})\implies 1=3a, 1=3b$$ which is not possible for $a,b\in\mathbb{Z}$.

Thus, $3$ is not prime and so $\mathbb{Z}[\sqrt{10}]$ is not a UFD.
\end{solution}
\newpage

\begin{problem} $\,$
If $F$ is a field and $E/F$ is an extension, then an element $a\in E$ will be called abelian if $\gal(F[a]/F)$ is an abelian group. Show taht the set of abelian elements of $E$ is a subfield of $E$ containing $F$.
\end{problem}


\begin{solution}$\,$
Let $S$ be the set of abelian elements of $E$. We need to show that $S$ is a subgfield of $E$ and that it contains $F.$

The latter is immediate since if $a\in F$ then $F[a]=F$ and so $$\gal(F[a]/F)=\gal(F/F)=\{e\}.$$ And since the trivial group is abelian, $a\in S$.

Thus, $S$ certainly contains $0$ and $1$ since $0\in F$ and $1\in F$.

Furthermore, $S$ contains inverses since $F[a]=F[a^{-1}]$ and so if $a\in S$ then $$\gal(F[a]/F)=\gal(F[a^{-1}]/F)\qquad\text{ is abelian.}$$ and so $a^{-1}\in S.$

Finally, we check that $S$ is closed under addition and multiplication.

Let $a,b\in S$. We prove a small claim.

\begin{claim} There is an injective homomorphism $$\varphi:\gal(F[a,b]/F)\to\gal(F[a]/F)\times\gal(F[b]/F).$$
\begin{proof} Let $\varphi(\sigma)=(\sigma|_{F[a]},\sigma|_{F[b]})$ which is the restriction of $\sigma$ to $F[a]$ and $F[b]$ respectively.

Note that $\varphi$ is well defined since by assumption, $F[a,b]/F$, $F[a]/F$ and $F[b]/F$ are Galois extensions, and so any automorphism $\sigma:F[a,b]\to F[a,b]$ must preserve the subfields $F[a]$ and $F[b]$.

Therefore, $\varphi$ is trivially a homomorphism since $$(\sigma\circ\tau)|_{F[a]}=\sigma(\tau|_{F[a]})=\sigma|_{F[a]}(\tau|_{F[a]})=\sigma|_{F[a]}\circ\tau|_{F[a]}$$ because $\tau|_{F[a]}:F[a]\to F[a]$.

Similarly for $(\sigma\circ\tau)|_{F[b]}$.

Finally, if $\varphi(\sigma)=(\id,\id)$ then $\sigma$ acts as the identity on $F[a]$ and on $F[b]$ so it must be the identity on $F[a,b]$.

Thus, $\ker\varphi=0$.

Therefore, there is an isomorphic copy of $\gal(F[a,b]/F)$ in $\gal(F[a]/F)\times\gal(F[b]/F).$
\end{proof}
\end{claim}

Since $a,b\in S$, $\gal(F[a]/F)\times\gal(F[b]/F)$ is a product of two abelian groups and is therefore abelian.

Finally, $\gal(F[a,b]/F)$ is therefore abelain  and so all subgroups are normal. Therefore, since $F[a-b]\subset F[a,b]$ and $F[ab]\subset F[a-b]$ are both subfields, by the fundamental theorem of Galois theory $F[a-b]/F$ is a Galois extension with abelian Galois group since it is a subgroup of $\gal(F[a,b]/F)$.

Therefore $a-b\in S$ and similarly $ab^{-1}\in S$.

Thus, $S$ is a subfield of $E$ containing $F$.
\end{solution}
\newpage



\begin{problem} $\,$
Let $K$ be the splitting field of $x^4-2\in\mathbb{Q}[x]$. Prove that $\gal(K/\mathbb{Q})$ is $D_8$ the dihedral group of order $8$, (i.e., the group of isometries of the square). Find all subfields of $K$ that have degree $2$ over $\mathbb{Q}$.
\end{problem}


\begin{solution}$\,$
Let $K$ be the splitting field of $$x^4-2=(x^2-\sqrt{2})(x^2+\sqrt{2})=(x-2^{1/4})(x+2^{1/4})(x-2^{1/4}i)(x+2^{1/4}i)$$

Then $K=\mathbb{Q}(2^{1/4},i)$ clearly.

Now, we note that $2^{1/4}$ has minimal polynomial $x^4-2$ over $\mathbb{Q}$ and $i$ has minimal polynomial $x^2+1$. Since $i\notin\mathbb{Q}(2^{1/4})$ because it is not in $\mathbb{R}$ and $\mathbb{Q}(2^{1/4})\subset\mathbb{R}$, then this must the minimal polynomial of $i$ over $\mathbb{Q}(2^{1/4})$.

Namely, $$[K:\mathbb{Q}]=[K:\mathbb{Q}(2^{1/4})][\mathbb{Q}(2^{1/4}):\mathbb{Q}]=2\cdot 4=8.$$

Now, because $x^4-2$ is separable and $K$ is its splitting field, $K/\mathbb{Q}$ is Galois.

Therefore, $G=\gal(K/\mathbb{Q})$ is of order $8$.

Let $\sigma\in G$ be defined by $\sigma(2^{1/4})=2^{1/4}i$ and $\sigma(i)=i$.

Then $\sigma$ clearly has order $4$ since $$\sigma^4(2^{1/4})=\sigma^3(2^{1/4}i)=\sigma^2(-2^{1/4})=\sigma(-2^{1/4}i)=2^{1/4}.$$

Let $\tau\in G$ be defined by $\tau(2^{1/4})=2^{1/4}$ and $\tau(i)=-i$. Then $\tau$ has order $2$.

Finally, \begin{align*}
    \sigma(\tau(2^{1/4}i))&=\sigma(-2^{1/4}i)=2^{1/4}\\
    \tau(\sigma(2^{1/4}i))&=\tau(-2^{1/4})=-2^{1/4}
\end{align*} and so $\sigma$ and $\tau$ do not commute.

Therefore, $G$ is non-abelian and so clearly $$G\cong D_8=\langle\sigma,\tau\,|\,\sigma^4=\tau^2=1,\sigma\tau=\tau\sigma^{-1}\rangle.$$

Now, the subfields $F$ of $K$ which have degree $2$ over $\mathbb{Q}$ correspond by the Galois Correspondence Theorem, to the subgroups of $G$ which have index $2$. Namely, to the subgroups of $G$ of order $4$.

Note that $\langle\sigma\rangle$, $\langle\tau,\sigma^2\rangle$, $\langle \tau\sigma,\sigma^2\rangle$ all have order $4$.

One can check that these are the only subgroups of order $4$. Namely, if $H$ were another subgroup of order $4$ containing $\sigma$ or $\sigma^3$, then $H$ would be the first subgroup.

Similarly, if $H$ contains $\sigma^2$, then it must contain at least one of the $\tau,\tau\sigma,\tau\sigma^2,\tau\sigma^3$ and these would all result in either the second or third subgroup listed.

Thus, no such fourth $H$ exists.

Finally, $\sigma$ fixes $i$ and so $$\langle\sigma\rangle=\gal(K/\mathbb{Q}(i))\leadsto \mathbb{Q}(i)$$

$\sigma^2$ fixes $i$ and $\sqrt{2}$ since $$\sigma^2((2^{1/4})^2)=\sigma((2^{1/4}i)^2)=\sigma(-(2^{1/4})^2)=-(2^{1/4}i)^2=(2^{1/4})^2$$, and $\tau$ fixes $\sqrt{2}$, so $$\langle\sigma\rangle=\gal(K/\mathbb{Q}(\sqrt{2}))\leadsto \mathbb{Q}(\sqrt{2})$$

and $\tau\sigma$ and $\sigma^2$ both fix $\sqrt{2}i$ since $$\tau\sigma((2^{1/4})^2i)=\tau(-(2^{1/4})^2i)=(2^{1/4})^2i$$ and $\sigma^2$ fixes $\sqrt{2}$ and $i$, so $$\langle\sigma\rangle=\gal(K/\mathbb{Q}(i\sqrt{2}))\leadsto \mathbb{Q}(i\sqrt{2}).$$
\end{solution}
\newpage



\begin{problem} $\,$
Let $F$ be a field, and suppose $A$ is a finite-dimensional $F$-algebra. Write $[A,A]$ for the $F$-subspace of $A$ spanned by elements of the form $ab-ba$ with $a,b\in A$. Show that $[A,A]\not=A$ in the following two cases:
\begin{enumerate}[label=(\alph*)]
    \item When $A$ is a matrix algebra over $F;$
    \item When $A$ is a central division algebra over $F.$
\end{enumerate} (Recall that a division algebra over $F$ is called central if its center is isomorphic with $F$).
\end{problem}


\begin{solution}$\,$
\begin{enumerate}[label=(\alph*)]
    \item Let $A=M_n(F)$ the algebra of $n\times n$ matrices with coefficients in $F$.

    Now, we note that $$\tr(XY-YX)=\tr(XY)-\tr(YX)=\tr(XY)-\tr(XY)=0$$ where $\tr(X)$ is the trace of $X.$

    Therefore, $[A,A]\subsetneq A$ since there are clearly matrices in $A$ with nonzero trace.
    \item Let $A$ be a central division algebra over $F$.

    We prove several small claims.

    \begin{claim} The center with $A,B$ both $F$-algebras ($F$ a field), $Z(A\bigotimes_F B)=Z(A)\bigotimes_F Z(B)$.
    \begin{proof} \boxed{\subset} Let $x\in Z(A\bigotimes_F B)$, then $x$ commutes with all elementary tensors. WLOG we can write $x=\sum_{j=1}^n\alpha_j(a_j\otimes b_j)$ where the $a_j$ and $b_j$ are linearly independent, then \begin{align*}
        x(a\otimes 1)&=\sum_{j=1}^n\alpha_j(a_j\otimes b_j)(a\otimes 1)\\
        &=\sum_{j=1}^n\alpha_j(a_ja\otimes b_j)\\
        &=(a\otimes b)x\\
        &=\sum_{j=1}^n\alpha_j(aa_j\otimes b_j)
    \end{align*} and so $$\sum_{j=1}^n\alpha_j((a_ja-aa_j)\otimes b_j)=0$$ and since the $b_j$ are linearly independent, this forces $a_ja-aa_j=0$ for all $j$.

    Thus, $a_j\in Z(A)$ for all $j.$

    Similarly, checking $x(1\otimes b)$ we get that $b_j\in Z(B)$ for all $j.$

    Thus, $x\in Z(A)\bigotimes_F Z(B)$

    \boxed{\supset} This is immediate since if $x\in Z(A)\bigotimes_F Z(B)$ then $x=\sum_{j=1}^n\alpha_j(a_j\otimes b_j)$ with $a_j\in Z(A)$ and $b_j\in Z(B)$ so $x$ will commute with all elementary tensors since $$x(a\otimes b)=\sum_{j=1}^n\alpha_j(a_j\otimes b_j)(a\otimes b)=\sum_{j=1}^n\alpha_j(a_ja\otimes b_jb)=\sum_{j=1}^n\alpha_j(aa_j\otimes bb_j)=(a\otimes b)x.$$

    Therefore, $x\in Z(A\bigotimes_F B)$
    \end{proof}
    \end{claim}

    \begin{claim} If $A$ is a central simple algebra, and $B$ is simple then $A\bigotimes_F B$ is simple where $F$ is a field and $A,B$ are $F$-algebras.
    \begin{proof} Let $I$ be an ideal of $A\bigotimes_F B$.

    Then there exists an $x\in I$ with $x=\sum_{j=1}^n\alpha_j(a_j\otimes b_j)$ where $n$ is minimal and the $b_j$ are linearly independent.

    Then $a_j\not=0$ for all $j$ and so the two sided ideal $I_1=(a_1)$ is a nonzero ideal of $A$, so namely, $I_j=A$ since $A$ is simple.

    Therefore, $1=t_1a_1s_1$ for some $t_1,s_1\in A$.

    Thus, $$x'=(t_1\otimes 1)x(s_1\otimes 1)=\alpha_1(1\otimes b_1)+\sum_{j=2}^n\alpha_j(t_1a_js_1\otimes b_j)\in I$$ since $I$ is a two sided ideal.

    Now, let $a\in A$ be arbitrary, then $$x_0=(a\otimes 1)x'-x'(a\otimes 1)=\sum_{j=2}^n\alpha_j(at_1a_js_1-t_1a_js_1a\otimes b_j)$$

    which is in $I$ and is of length strictly smaller than $x$. Thus, $x_0=0$ and so because the $b_j$ are linearly independent, this forces $at_1a_js_1-t_1a_js_1a=0$ for all $j.$ Therefore, since $a$ was arbitrary, $t_1a_js_1\in Z(A)=F$ because $A$ is central.

    However, then $x'=1\otimes b$ for some $b\in B$.

    However, then $1\otimes (b)\subset I$ where $(b)$ is a two sided ideal of $B.$ However, since $B$ is also simple, $(b)=B$ and so $1\otimes B\subset I$.

    Therefore, $(A\otimes 1)(1\otimes B)=A\otimes B\subset I$

    So $A\bigotimes_F B$ is simple.
    \end{proof}
    \end{claim}

    Finally, let $\overline{F}$ be the algebraic closure of $F.$ Let $C=A\otimes_F\overline{F}$. From \textbf{Claim 2} and \textbf{Claim 3}, $C$ is simple and has center $F\otimes_F\overline{F}=\overline{F}.$

    Therefore, by Artin-Wedderburn, $C= M_n(D_i)$ for some $D_i$ division ring over $\overline{F}$. However, since $Z(D_i)=\overline{F}$, and $\overline{F}$ is algebraically closed, $D_i=\overline{F}$. Note $D_i$ is finite dimensional over $\overline{F}$ by Artin-Wedderburn, and so it must be an algebraic extension, however $\overline{F}$ is algebraically closed so $D_i=\overline{F}.$

    Thus, $C=M_n(\overline{F})$ and so by (a), $[C,C]\not=C$.

    Since \begin{align*}
        [C,C]&=[A\otimes_F\overline{F},A\otimes_F\overline{F}]\\
        &=\{\text{linear combinations of }(a\otimes f)(b\otimes g)-(b\otimes g)(a\otimes f)\}\\
        &=\{\text{linear combinations of }ab\otimes fg-ba\otimes gf\}\\
        &=\{\text{linear combinations of }ab\otimes fg-ba\otimes fg\}\\
        &=\{\text{linear combinations of }(ab-ba)\otimes fg)\}\\
        &=[A,A]\otimes_F\overline{F}\not=A\otimes_F\overline{F}
    \end{align*} it must be that $[A,A]\not=A$.
\end{enumerate}
\end{solution}
\newpage



\begin{problem} $\,$
If $\varphi:A\to B$ is a surjective homomorphism of rings, show that the image of the Jacobson radical of $A$ under $\varphi$ is contained in the Jacobson radical of $B$.
\end{problem}


\begin{solution}$\,$
First, let $M$ be a maximal ideal of $A$. Let $\varphi(M)\subset N$ where $N\subset B$ is maximal.

Then we can define $$\Tilde{\varphi}:A\to B/N$$ defined by $\Tilde{\varphi}(a)=\varphi(a)+N.$

Clearly $\Tilde{\varphi}(M)\subset 0$ and so $M\subset\ker(\Tilde{\varphi})$. Therefore, either $M=\ker\Tilde{\varphi}$ or $\ker\Tilde{\varphi}=A$.

In the first case, we get that $\varphi(M)=N$ since if $x\in N$, then $\varphi$ is surjective so there exists $a\in A$ with $\varphi(a)=x$, namely, $\Tilde{\varphi}(a)=0$ and so $a\in M$.

In the second case, we get that $N=B$, else we could take $1\in B\backslash N$ and again, there would exist some $a\in A$ such that $\Tilde{\varphi}(a)=1+N\not=0$.

Namely, $\varphi$ sends maximal ideals to maximal ideals.

Therefore, \begin{align*}
    \varphi(J(A))&=\varphi\left(\bigcap_{M\text{ max }\subset A}M\right)\\
    &=\bigcap_{M\text{ max }\subset A}\varphi(M)\\
    &\subset \bigcap_{N\text{ max }\subset B}N\\
    &=J(B)
\end{align*}
\end{solution}
\newpage



\end{document}
