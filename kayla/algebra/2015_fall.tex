\documentclass[12pt]{Qual}
\usepackage{preamble}

\name{Kayla Orlinsky}
\course{Algebra Exam}
\term{Fall 2015}
\hwnum{Fall 2015}

\begin{document}

\begin{problem} $\,$
If $M$ is a maximal ideal in $\mathbb{Q}[x_1,...,x_n]$ show that there are only finitely many maximal ideals in $\mathbb{C}[x_1,...,x_n]$ that contain $M.$
\end{problem}


\begin{solution}$\,$
This question is actually a specific case of \textbf{Spring 2015: Problem 3}.

First, we note that by Nullstellensatz, since $M$ is a proper ideal of $\mathbb{Q}[x_1,...,x_n]$, $V(M)\not=\varnothing$ as a subset of $\mathbb{C}^n$. Namely, there exists $(a_1,...,a_n)\in\mathbb{C}^n$ such that every polynomial in $M$ is satisfied by $(a_1,...,a_n)$.

Thus, by Nullstellensatz, for every $f\in M$ considered as a polynomial in $\mathbb{C}[x_1,...,x_n]$, there exists $r$ such that $f^r\in (x_1-a_1,...,x_n-a_n)$. However, by Nullstellensatz, $(x_1-a_1,...,x_n-a_n)$
 is a maximal ideal of $\mathbb{C}[x_1,...,x_n]$ and so it is prime. Thus, recursively, $f\in (x_1-a_1,...,x_n-a_n)$, for all $f\in M$.

Thus, $M\subset (x_1-a_1,...,x_n-a_n).$

Therefore, $M$ is contained in at least one maximal ideal of $\mathbb{C}[x_1,...,x_n].$

Now, for each maximal ideal $N\subset\mathbb{C}[x_1,...,x_n]$ such that $M\subset N$, there is clearly an induced injection of fields $$\mathbb{Q}[x_1,...,x_n]/M\hookrightarrow\mathbb{C}[x_1,...,x_n]/N$$ where $L=\mathbb{Q}[x_1,...,x_n]$ is a field extension of $\mathbb{Q}$ and $\mathbb{C}[x_1,...,x_n]/N\cong \mathbb{C}$ since $\mathbb{C}$ is algebraically closed and so the only field extension of it is itself.

Clearly, the correspondence is $\oto$. Namely, for every distinct maximal ideal $N$ containing $M$ there corresponds one injection of fields from $L$ into $\mathbb{C}.$

Now we prove two claims. First, that $L/\mathbb{Q}$ is finite, and second that there are only finitely many injections from a finite field extension of $\mathbb{Q}$ into its algebraic closure.

Both of these were proved in \textbf{Spring 2015: Problem 3 Claim 1, Claim 2.}

Both claims are proved by induction.

First, we argue that $L$ is an algebraic extension of $\mathbb{Q}$ (and hence finitely generated), then we argue that any injection of $L$ into $\mathbb{C}$ is uniquely determined by how it permutes the roots of the minimal polynomial of each generator of $L$, of which there are only finitely many options.

Finally, we obtain that there are only finitely many maximal ideals $N$ of $\mathbb{C}[x_1,...,x_n]$ containing $M.$
\end{solution}
\newpage


\begin{problem} $\,$
Let $R$ be a right Noetherian ring with $1.$ Prove that $R$ has a \textit{unique} maximal nilpotent ideal $P(R)$. Argue that $R[x]$ also has a unique maximal nilpotent ideal $P(R[x])$. Show that $P(R[x])=P(R)[x].$
\end{problem}


\begin{solution}$\,$
Let $\mathscr{S}$ be the set of nilpotent right-ideals of $R$.

Since $R$ is right-Noetherian, every set of ideals contains a unique maximal element. Thus, $\mathscr{S}$ contains a maximal nilpotent right ideal $N$ of order $n.$

Let $J$ be a second such maximal ideal of order $j$. Then $J+N=\{a+b\,|\,a\in J, b\in N\}$ will also be a nilpotent ideal since $(a+b)^{jn}=0$.

Since $N\subset J+N$ and $N$ is maximal, $J+N=N$, however $J\subset J+N$ as well and so $J+N=J.$ Therefore, $N=J.$ Thus, $N$ is unique.

Now, let $P(R)$ be the two-sided ideal generated by $N.$ We would like to show that $P(R)$ is nilpotent.

Let $x_1,...,x_n,x_{n+1}\in N$, and $r_1,...,r_n,r_{n+1}\in R$. It suffices to show that any product of $k$ things of the form $nr$ where $n\in N$ and $r\in R$ is $0$ for some $k.$

Then $$(x_1r_1)(x_2r_2)\cdots(x_nr_n)(x_{n+1}r_{n+1})=x_1(r_1x_2)(r_2x_3)\cdots(r_nx_{n+1})r_{n+1}=x_10r_{n+1}=0$$ since $r_ix_i\in N$ and $N$ is nilpotent of order $n.$

Therefore, $P(R)$ is nilpotent of order at most $n+1.$ Since $P(R)$ is generated by the unique maximal nilpotent right-ideal of $R$, it is the unique maximal nilpotent $2$-sided ideal of $R.$

By the Hilbert Basis theorem, $R[x]$ is also right-Noetherian, and so it too will contain a unique maximal $2$-sided nilpotent ideal, $P(R[x])$

Let $f(x)=a_mx^m+\cdots+a_1x+a_0\in P(R[x]).$ We induct on the degree of $f.$

If $f(x)=a_0$, then $f^n=a_0^n=0$ so trivially $f(x)\in P(R)[x].$

Assume $f\in P(R[x])\implies f\in P(R)[x]$ for $f$ having degree $k\le m-1$.

Now, assume $f$ has degree $m$.

Because $f^n=0$, we have that $a_m^n=0$, so $a_mx^m\in P(R)[x]$. Therefore, $f-a_mx^m\in P(R)[x]$ by the inductive hypothesis since $f-a_mx^m$ has degree strictly less than $m$ and is a sum of nilpotent elements (which is also nilpotent).

Therefore, since $f-a_mx^m\in P(R)[x]$ and $a_mx^m\in P(R[x])$, we have that $f\in P(R)[x].$

If $f(x)\in P(R)[x]$, then every coefficent of $f$ is nilpotent of degree less than or equal to $n$, so $f^{n^2}(x)=0$ since each coefficient will be raised to at least the $n\thh$ power. Therefore, $f\in P(R[x])$ since this is the unique largest nilpotent ideal of $R[x].$

\end{solution}
\newpage



\begin{problem} $\,$
Up to isomorphism, describe the possible structures of any group of order $182$ as a direct sum of cyclic groups, dihedral groups, other semi-direct products, symmetric groups, or matrix groups. (Note: 91 is not prime!)
\end{problem}


\begin{solution}$\,$
Let $G$ be a group of order $182=2\cdot 7\cdot 13$. By Sylow, $n_7\equiv 1\mod 7$ and $n_7|2\cdot 13$. Therefore, $n_7=1$ so $G$ has a normal Sylow $7$-subgroup.

\boxed{\text{Abelian}} By the fundamental theorem of abelian groups, $G\cong\mathbb{Z}_{182}.$

Let $P_2,P_7,P_{13}$ be Sylow $2,7,13$-subgroups of $G$ respectively.

Therefore, $P_7P_{13}$ is a subgroup of $G$ and it is normal in $G$ since it has index $2$ which is the smallest prime dividing the order of $G$. (see \textbf{Spring 2010: Problem 2 Claim 1}).

Now, in $P_7P_{13}$, because $n_{13}|7$ and $n_{13}\equiv 1\mod 13$, $n_{13}=1$ so $P_7P_{13}$ has a normal Sylow $13$-subgroup.

Therefore, because $P_{13}$ is normal in $P_7P_{13}$ which is normal in $G$, $P_{13}$ is also normal in $G$ so $G$ has one normal Sylow $13$-subgroup (see \textbf{Fall 2011: Problem 5 Claim 3}).

Thus, we need to check only three homomorphisms.

\boxed{\varphi:P_2P_7\to\aut(P_{13})} Since $P_7$ is normal, $P_2P_7$ is a subgroup of $G$. Let $\varphi:P_2P_7\to\aut(P_{13})\cong\mathbb{Z}_{13}^\times\cong\mathbb{Z}_{12}$.

Let $P_2\cong\langle a\rangle$, $P_7\cong\langle b\rangle$, $P_{13}\cong\langle c\rangle$.

Then since there are no elements of order $7$ in $\mathbb{Z}_{12}$, $\varphi(b)=\id$.

There is one possible elements of order $2$ to send $a$, namely, $\varphi(a)=\alpha$ where $\alpha(c)=c^{12}$. This defines multiplication on $G$ by $aca^{-1}=\varphi_1(a)(c)=c^{12}=c^{-1}$ $$G\cong\langle a,b,c\,|\, a^2=b^7=c^{13}=1,ab=ba,bc=cb,ac=c^{-1}a\rangle$$ We note that $$G\cong \mathbb{Z}_{13}\rtimes_{\varphi_1}\mathbb{Z}_2\times\mathbb{Z}_7\cong D_{26}\times\mathbb{Z}_7$$ where $D_{26}$ is the dihedreal group of $26$ elements.


\boxed{\varphi:P_2P_{13}\to\aut(P_7)} $\varphi:P_2P_{13}\to\mathbb{Z}_6$. We have one choice: $\varphi_2(c)=\id$, and $\varphi_2(a)=\beta$ where $\beta(b)=b^5=b^{-1}$, since this is the only element of $\aut(P_7)$ of order $2$.

This gives $$G\cong\langle a,b,c\,|\,a^2=b^7=c^{13}=1,ac=ca,bc=cb,ab=b^{-1}a\rangle\cong D_{14}\times\mathbb{Z}_{13}.$$

\boxed{\varphi:P_2\to\aut(P_7P_{13})} Let $\varphi:P_2\to\aut(P_7P_{13})\cong\aut(P_{13})\times\aut(P_7)$ since $7$ and $13$ are coprime and $P_7$ and $P_{13}$ are cyclic.

Now, defining $\varphi(a)=(\alpha,\id)$ or $\varphi(\id,\beta)$ will yield the same multiplication as we found previously.

Thus, the only new structure can be given by $\varphi(a)=(\alpha,\beta)$.

Therefore, we get one more possible structure for $G.$

$$G\cong\langle a,b,c\,|\,a^2=b^7=c^{13}=1,ac=c^{-1}a,bc=cb,ab=b^{-1}a\rangle\cong D_{182}.$$

Finally, we have $4$ possible group structures for $G.$

\begin{center}
    \begin{framed}
    $$\mathbb{Z}_{182}$$

    $$D_{14}\times\mathbb{Z}_{13}$$

    $$D_{26}\times\mathbb{Z}_7$$

    $$D_{182}$$
    \end{framed}
\end{center}
\end{solution}
\newpage


\begin{problem} $\,$
Let $K=\mathbb{C}(y)$ for an indeterminate $y$ and let $p_1<p_2<\cdots<p_n$ be primes (in $\mathbb{Z}$). Let $f(x)=(x^{p_1}-y)\cdots(x^{p_n}-y)\in K$ with splitting field $L$ over $K$.
\begin{enumerate}[label=(\alph*)]
    \item Show each $x^{p_j}-y$ is irreducible over $K$
    \item Describe the structure of $\gal(L/K)$.
    \item How many intermediate fields are between $K$ and $L$.
\end{enumerate}
\end{problem}


\begin{solution}$\,$
This problem is a more extensive version of \textbf{Fall 2013: Problem 7}
\begin{enumerate}[label=(\alph*)]
    \item We use generalized Eisenstein's criterion with $p=y$. Clearly $(y)$ is a prime (in fact maximal) ideal since $\mathbb{C}(y)/(y)\cong\mathbb{C}$ which is a domain.

    Now, $0\in(y)$ and $y\in (y)$ and $1\notin (y)$ which is the criterion on the coefficeints of $x^{p_j}-y$. Finally, $y\notin(y)^2$ and so $x^{p_j}-y$ is irreducible over $K$.

    \item Each irreducible $x^{p_j}-y$ factor of $f$ has as its roots, the $p_j$ roots of $y$ multiplying the $p_j\thh$ roots of unity. Namely, $f$ is separable and so $L/K$ is indeed Galois.

    Furthermore, each $\sigma\in G=\gal(L/K)$ will be uniquely determined by how it permutes the roots of each irreducible factor.

    Namely, $G$ will be generated by the $\sigma_i$, where $\sigma_i$ is a permutation of the roots of $x^{p_j}-y$, fixing the other roots of $f.$

    This implies that $G$ will be abelian since each $\sigma_i$ will fix all but the $p_i\thh$ roots of unity and will fix all $p_i\thh$ roots of $y.$

    Therefore, $$G\cong\mathbb{Z}_{p_1p_2\cdots p_n}.$$

    \item Since $G$ is abelian, any product of subgroups of $G$ will also be a subgroup and so there are $$\sum_{i=1}^{n-1}\binom{n}{i}=\sum_{i=0}^n\binom{n}{i}-2=2^n-2$$ possible subgroups.
\end{enumerate}
\end{solution}
\newpage



\begin{problem} $\,$
In any finite ring $R$ with $1$ show that some element in $R$ is not a sum of nilpotent elements. Note that in all $M_n(\mathbb{Z}/n\mathbb{Z})$ the identity matrix is a sum of nilpotent elements. (Hint: What is the trace of a nilpotent element in a matrix ring over a field?)
\end{problem}


\begin{solution}$\,$
Since $R$ is finite, it is clearly artinian. Thus, by Artin Wedderburn, $R/J(R)$ is semi-simple and so isomorphic to $M_{n_1}(D_1)\oplus\cdots\oplus M_{n_k}(D_k)$ where $D_i$ are division rings over $R$.

Since the only finite division rings are finite fields, we have that $$R/J(R)\cong M_{n_1}(\mathbb{F}_{q_1})\oplus\cdots\oplus  M_{n_k}(\mathbb{F}_{q_k})$$ where $\mathbb{F}_{q_i}$ is the field of $q_i$ elements.

Now, we note that the trace $\tr(A)=0$ if $A\in M_n(\mathbb{F}_q)$ is nilpotent.

Therefore, if $A$ is a sum of nilpotent matrices $A=N_1+\cdots+N_l$ then $$\tr(A)=\tr(N_1+\cdots+N_l)=\tr(N_1)+\cdots+\tr(N_l)=0+\cdots+0=0.$$

Therefore, if $\tr(A)\not=0$ then $A$ cannot be a sum of nilpotent elements.

Let $$A=\begin{bmatrix}
1 & 0 & \cdots & 0 & 0\\
0 & 0 &\cdots & 0 & 0\\
& \vdots & \ddots & \vdots & \\
0 & 0 & \cdots & 0 & 0\\
0 & 0 & \cdots & 0 & 0
\end{bmatrix}$$

Then $A$ is not nilpotent, and it is not a sum of nilpotents since $\tr(A)=1\not=0$ over any finite field.

Since $A+J(R)$ is not a sum of nilpotent elements of $R/J(R)$, $A$ cannot be a sum of nilpotent elements of $R.$
\end{solution}
\newpage




\begin{problem} $\,$
Let $R$ be a commutative principal ideal domain.
\begin{enumerate}[label=(\alph*)]
    \item If $I$ and $J$ are ideals of $R$ show that $R/I\otimes_R R/J\cong R/(I+J)$.
    \item If $V$ and $W$ are finitely generated $R$ modules so that $V\otimes_RW=0$, show that $V$ and $W$ are torsion modules whose annihilators in $R$ are relatively prime.
\end{enumerate}
\end{problem}


\begin{solution}$\,$
\begin{enumerate}[label=(\alph*)]
    \item Let $I$ and $J$ be ideals of $R$. Define \begin{align*}
        f:R/I\times R/J&\to R/(I+J)\\
        (a+I,b+J)&\mapsto ab+I+J
    \end{align*}

    Then $f$ is well defined since if $(a+I,b+J)=(a'+I,b'+J)$ then $a=a'+i$ for $i\in I$ and $b=b'+j$ for $j\in J.$

    Thus, \begin{align*}
        f(a+I,b+J)&=ab+I+J\\
        &=(a'+i)(b'+j)+I+J\\
        &=a'b'+a'j+b'i+ij+I+J\\
        &=a'b'+I+J\\
        &=f(a'+I,b'+J)
    \end{align*}

    Furthermore, $$f(ra+I,b+J)=rab+I+J=rf(a+I,b+J)$$ and $$f(a+I,br+J)=abr+I+J=f(a+I,b+J)r$$ since $R$ is commutative.

    Finally, $$f(a+a'+I,b+J)=(a+a')b+I+J=f(a+I,b+J)+f(a'+I,b+J)$$ and similarly for linearity on the right.

    Thus, $f$ is bilinear.

    Thus, because $R$ is commutative $R/(I+J)$ is an abelian group under multiplication. Thus, by the universal property of tensor product, $f$ induces a map $\overline{f}:R/I\otimes_R R/J\to R/(I+J)$ defined by $\overline{f}((a+I)\otimes(b+J))=ab+I+J.$

    Let $r\in R/I\otimes_R R/J$ then \begin{align*}
        r&=\sum_{i=1}^n(a_i+I)\otimes(b_i+J)\\
        &=\sum_{i=1}^n(a_i+I)\otimes b_i(1+J)\\
        &=\sum_{i=1}^nb_i(a_i+I)\otimes (1+J)\\
        &=\sum_{i=1}^n(b_ia_i+I)\otimes (1+J)\\
        &=\left(\sum_{i=1}^nb_ia_i+I\right)\otimes (1+J)
    \end{align*}

    Thus, $r=(s+I)\otimes(1+J)$ for some $s\in R$.

    Now, $g:R/(I+J)\to R/I\otimes_R R/J$ defined by $g(a+I+J)=(a+I)\otimes 1.$

    Then $g$ is clearly well defined since if $a+I+J=b+I+J$ then there exists $i,j\in I,J$ respectively so $a+I+J=b+i+j+I+J$. Thus, \begin{align*}
        g(a+I+J)&=(a+I)\otimes(1+J)\\
        &=(b+i+j+I)\otimes(1+J)\\
        &=(b+I)\otimes(1+J)+(j+I)\otimes(1+J)\\
        &=(b+I)\otimes(1+J)+j(1+I)\otimes(1+J)\\
        &=(b+I)\otimes(1+J)+(1+I)\otimes(j+J)\\
        &=(b+I)\otimes(1+J)+(1+I)\otimes0\\
        &=(b+I)\otimes(1+J)\\
        &=g(b+I+J)
    \end{align*}

    Furthermore, $$\overline{f}(g(a+I+J))=\overline{f}((a+I)\otimes(1+J))=a+I+J$$

    $$g(\overline{f}((a+I)\otimes(b+J))=g(ab+I+J)=(ab+I)\otimes(1+J)=b(a+I)\otimes(1+J)=(a+I)\otimes(b+J).$$

    Thus, $g$ is the inverse of $f$ so $f$ defines an isomorphism.
    \item Let $V$ and $W$ be finitely generated $R$ modules so that $V\otimes_RW=0$.

    By the structure theorem, $V\cong R^m\oplus T(V)$ and $W\cong R^n\oplus T(W)$ where $T(V)$ and $T(W)$ are the torsion parts of $V$ and $W$ respectively.

    Then \begin{align*}
        0&=V\otimes_RW\\
        &\cong(R^m\oplus T(V))\otimes_R(R^n\oplus T(W))\\
        &\cong(R\oplus R\oplus\cdots\oplus R\oplus T(V))\otimes_R(R^n\oplus T(W))\\
        &\cong R\otimes_R(R^n\oplus T(W))\oplus \cdots\oplus R\otimes_R(R^n\oplus T(W))\oplus T(V)\otimes_R(R^n\oplus T(W))\\
        &\cong (R^n\oplus T(W))\oplus (R^n\oplus T(W))\oplus\cdots\oplus (R^n\oplus T(W))\oplus T(V)\otimes_R(R^n\oplus T(W))\\
        &\cong (R^{nm}\oplus T(W))\oplus T(V)\otimes_R(R^n\oplus T(W))\\
        &\cong (R^{nm}\oplus T(W))\oplus (T(V)\otimes_R R^n)\oplus (T(V)\otimes_R T(W))\\
        &\cong (R^{nm}\oplus T(W))\oplus (T(V)\otimes_R R^n)\oplus (T(V)\otimes_R T(W))\\
        &\cong (R^{nm}\oplus T(W))\oplus [T(V)]^n \oplus (T(V)\otimes_R T(W))
    \end{align*}

    This can only be zero if each component in the direct sum is zero.

    Namely $n=m=0$, else if $T(W)=T(V)=0$, then $0\cong R^{nm}$ which is a contradiction.

    Finally, $V\cong T(V)$ and $W\cong T(W)$ and $T(V)\otimes_R T(W)\cong 0$.

    Since $V\cong T(V)$ is finitely generated, by the structure theorem, $T(V)\cong R/(r_1)\oplus\cdots\oplus R/(r_n)$ for some ideals $(a_i)\subset R.$

    Now, there is a clear homomorphism $f:R\to T(V)$ defined by $f(a)=(a+(r_1),a+(r_2),\cdots,a+(r_n)).$ $f$ will certainly be surjective.

    Now, if $f(a)=0$ then $a\in(r_1)\cap\cdots\cap(r_n)$. Namely, $a\in\ann(V)$. Similarly, if $a\in\ann(V)$ then $av=0$ for all $v\in V$ and so $aR/(r_1)\oplus aR/(r_2)\oplus\cdots\oplus aR/(r_n)=0$ so $a\in(r_1)\cap\cdots\cap(r_n)$ so $a\in\ker(f).$

    Thus, $$T(V)\cong R/\ann(V).$$

    Now, finally, by (a), $$R/(\ann(V)+\ann(W))\cong R/\ann(V)\otimes_RR/\ann(W)\cong T(V)\otimes_RT(W)\cong 0.$$

    Therefore $\ann(V)+\ann(W)=R$ so $1\in \ann(V)+\ann(W)$ and so $\ann(V)$ and $\ann(W)$ are relatively prime.
\end{enumerate}
\end{solution}
\newpage




\begin{problem} $\,$
Let $g(x)=x^{12}+5x^6-2x^3+17\in\mathbb{Q}[x]$ and $F$ a splitting field of $g(x)$ over $\mathbb{Q}$. Determine if $\gal(F/\mathbb{Q})$ is solvable.
\end{problem}


\begin{solution}$\,$
Let $h(x)=x^4+5x^2-2x+17$. Then if $\alpha$ is a root, $h(x)$, the third roots of $\alpha$ are all roots of $g(x).$ Namely, if $K$ is the splitting field of $h$, then $\gal(F/K)$ is clearly solvable since every root of $g$ to the third power is in $K$. Thus, it suffices to show that $\gal(K/\mathbb{Q})$ is solvable.

Now, $h'(x)=4x^3+10x-2$ which is negative for all $x\le 1/10$ and positive for all $x\ge1/5$. Thus, $h$ has a minimum value somewhere between $1/10$ and $1/5$.

However, for all $\alpha\in(0,1)$, $h(\alpha)\ge -2+17>0.$ Thus, $h$ has no real roots.

Since $h$ has only complex roots, it has a conjugate pair of roots, $\alpha,\overline{\alpha},\beta,\overline{\beta}.$

Note that since $F$ is the splitting field of a separable polynomial $F/\mathbb{Q}$ is indeed Galois.

Now, $K=\mathbb{Q}(a,\overline{a},b,\overline{b})$ is Galois over $\mathbb{Q}$. Thus, $H=\gal(F/K)$ is normal in $G=\gal(F/\mathbb{Q})$ and $\gal(K/\mathbb{Q})=G/H$.

Now, each third rood in $F$ clearly has minimal polynomial $x^3-\alpha$, $x^3-\beta$, $x^3-\overline{\alpha}$, $x^3-\overline{\beta}$ over $K$. These are irreducible since factoring would force a linear term to appear over $K$, and $K$ does not contain any third roots of $\alpha,\beta,\overline{\alpha},\overline{\beta}$.

So $[F:K]\le 3^{12}$. Specifically, since each of these is irreducible over $F$, $[F:K]=3^r$ for some $r\le12$.

However, then clearly $H$ has order $3^r$ and so it must be solvable. This is because $p$-groups have non-trivial centers, and so recursively, we could obtain a subnormal chain by examining $H/Z(H)$, $H/Z(H)/Z(H/Z(H))$, etc.

Finally, $\alpha,\beta,\overline{\alpha}, \overline{\beta}$ all have minimal polynomial of degree $4$ over $\mathbb{Q}$, so $[G:H]\le 4^4$ and namely $[G:H]=4^s=2^{2s}$ for $s\le 4$, so $G/H$ is solvable.

Therefore, since $H$ is normal in $G$, and $H$ is solvable and $G/H$ is solvable, then $G$ is solvable.

\end{solution}



\end{document}
