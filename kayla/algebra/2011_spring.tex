\documentclass[12pt]{Qual}
\usepackage{preamble}

\name{Kayla Orlinsky}
\course{Algebra Exam}
\term{Spring 2011}
\hwnum{Spring 2011}

\begin{document}

\begin{problem} $\,$
Let $G$ be a finite group with a cyclic Sylow $2$-subgroup $S.$
\begin{enumerate}[label=(\alph*)]
    \item Show that any element of odd order in $N_G(S)$ centralizes $S.$
    \item Show that $N_G(S)=C_G(S).$
    \item Give an example to show that (a) can fail if $S$ is abelian.
\end{enumerate}
\end{problem}


\begin{solution}$\,$
\begin{enumerate}[label=(\alph*)]
    \item Since $S\subset C_G(S)\subset N_G(S)$, (a) and (b) are equivalent. Namely, $[N_G(S):C_G(S)]=2n+1$ for some $n\in\mathbb{N}$.

    Therefore, we will prove (b) directly. In fact, we will prove something stronger.

    \begin{claim} If $p$ is the smallest prime dividing $|G|$ and $P$ is a cyclic Sylow $p$-subgroup, then $N_G(P)=C_G(P).$
    \begin{proof} Let $p$ be the smallest prime dividng $|G|.$ Then, since $$P\trianglelefteq C_G(P)\trianglelefteq N_G(P)$$ we have that $$[N_G(P):C_G(P)]=n\qquad \gcd(n,p)=1.$$ Furthermore, because $p$ is the smallest prime dividing $|G|,$ $n$ is only divisible by primes $q$ with $q>p.$

    Now, let \begin{align*}
        \varphi:N_G(P)&\to\Aut(P)\\
        a&\mapsto\sigma_a
    \end{align*} be the map of the conjugation action of $N_G(P)$ on $P.$

    Then $C_G(P)$ is clearly the kernel of this action and so by the first isomorphism theorem, $$N_G(P)/C_G(P)\cong A\subset\Aut(P).$$

    Finally, because $P=\langle x\rangle$ is cyclic, we have that the automorphisms of $P$ are exactly the maps $x\mapsto x^k$ for $\gcd(k,p)=1$. Namely, $$|\Aut(P)|=p^{l-1}(p-1)\qquad\text{ by the Euler Totient Function}$$ assuming that $|P|=p^l$. Since the divisors of this are not greater than $p,$ and $|N_G(P)/C_G(P)|$ has only divisors greater than $p,$ it must be that $|N_G(P)/C_G(P)|=1.$

    Namely, $$N_G(P)=C_G(P).$$
    \end{proof}
    \end{claim}
    \item Since $2$ is clearly the smallest prime dividing $|G|$, the claim in (a) applies and we are done.
    \item There is a small example where $S$ is not abelian to show how (b) can fail.

    Assume $S$ is a $2$-Sylow subgroup and $$S\cong D_8=\langle r,s\,|\,s^4=r^2=1,sr=r^{-1}s\rangle.$$ which is non-abelian.

    Let $G=S_4.$ Since $S$ is non-abelian, $C_G(S)$ does not contain $S$ but $S\subset N_G(S)$ so the two are certainly not equal.

    However, in this case, $N_G(S)=S$ and so it contains no elements of odd order.

    To contradict (a), we can consider $G=A_4$ which has a normal $2$-Sylow subgroup $S$ isomorphic to $\mathbb{Z}_2\times\mathbb{Z}_2.$

    Thus, $N_G(S)=G$ and $G$ certainly contains elements of odd order. However, one can check that $(1\,\, 2\,\, 3)\in G$ has odd order and $(1\,\, 2\, \, 3)\notin C_G(S).$ In fact, it is true that $C_G(S)=S.$
    \end{enumerate}
\end{solution}
\newpage


\begin{problem} $\,$
Let $G$ be a finite group with a cyclic Sylow $2$-subgroup $S\not=1.$
\begin{enumerate}[label=(\alph*)]
    \item Let $\rho:G\to S_n$ be the regular representation with $n=|G|.$ Show that $\rho(G)$ is not contained in $A_n.$
    \item Show that $G$ has a normal subgroup of index $2.$
    \item Show that the set of elements of odd order in $G$ form a normal subgroup $N$ and $G=NS.$
\end{enumerate}
\end{problem}


\begin{solution}$\,$
\begin{enumerate}[label=(\alph*)]
    \item The regular representation $\rho$ is the map which sends $g\mapsto\lambda_g$ which is the left multiplication map, namely, $\lambda_g(h)=gh$ for all $h\in G$.

    Therefore, by construction, $\lambda_g$ has no fixed points and, because $\rho$ is a homomorphism, $\lambda_g$ has order $o(g).$

    \begin{claim} $\lambda_g$ can be represented in $S_n$ as a product of $\frac{|G|}{o(g)}$ cycles each of length $o(g).$
    \begin{proof} Let $\lambda_g=\sigma_1\cdots\sigma_l$ with $\sigma_i$ disjoint cycles.

    Now, because $\lambda_g$ has no fixed points, the product of the $\sigma_i$ also have no fixed points.

    Next, we note that $\lambda_{g^t}=(\lambda_g)^t$ is non-trival for all $t<o(g)$ and $\lambda_{g^t}$ also has no fixed points.

    Therefore, $$(\sigma_1\cdots\sigma_l)^t=\sigma_1^t\cdots\sigma_l^t$$ has no fixed points for all $t<o(g)$ and so, letting $k_i$ be the length of $\sigma_i$ for all $i,$ we get that $k_i\ge o(g)$ for all $i$.

    However, since $o(\lambda_g)=o(g)=\lcm($distinct cycle lengths$),$ we get that $k_i\le o(g)$ for all $i.$

    Therefore, $k_i=o(g)$ for all $i.$

    Finally, the only way for there to be no fixed points is if all $n$ integers are expressed in some $\sigma_i.$ Therefore, $$n=\sum_{i=1}^lk_i=lo(g)\implies l=\frac{n}{o(g)}.$$

    Thus, $\lambda_g$ can be expressed as $\frac{n}{o(g)}$ cycles each of length $o(g).$
    \end{proof}
    \end{claim}

    Let $S=\langle x\rangle$ since it is cyclic, $o(x)=2^k$ for $|S|=2^k.$

    From the claim, $\rho(x)=\lambda_x$ can be written as a product of $\frac{n}{2^k}$ cycles, each of length $2^k.$

    Since $S$ is a $2$-Sylow subgroup, $\frac{n}{2^k}$ is odd, and so $\lambda_x$ is a product of an odd number of even length cycles. Since cycles of even length are expressed as an odd number of transpositions, $\lambda_x$ is a product of an odd number of transpositions, an odd number of times.

    Therefore, $\rho(x)=\lambda_x\notin A_n$ and so $\rho(G)\not\subset A_n.$
    \item Since $\rho(G)\not\subset A_n$ by (a), and since $A_n$ is normal in $S_n,$ we have that $$A_n\subsetneq\rho(G)A_n\subset S_n.$$

    However, since $[S_n:A_n]=2,$ we have that $A_n$ is maximal and so $\rho(G)A_n=S_n$.

    Now, because $$\frac{|S_n|}{|A_n|}=2$$ and by the first isomorphism theorem, $$\rho(G)A_n/A_n\cong\rho(G)/(\rho(G)\cap A_n)$$ so we get that $$2=\frac{|S_n|}{|A_n|}=\frac{|\rho(G)A_n|}{|A_n|}=\frac{|\rho(G)|}{|\rho(G)\cap A_n|}.$$

    Thus, because $\rho(G)\cap A_n\subset \rho(G)$ is a subgroup, we have that $\rho(G)$ has a subgroup of index $2.$

    And since $\rho(G)\cong G$, $G$ has a subgroup of index $2$ which is normal because $2$ is the smallest prime dividing $|G|.$ (For a proof see \textbf{Spring 2010, Claim 1})
    \item Let $N$ be the set of elements of odd order in $G$.

    Now, let $|G|=n=2^km$. Then, because $G$ by (b), we can let $K_1$ be a normal subgroup of index $2$. Then $|K_1|=2^{k-1}m.$

    If we can show that $K_1$ has a cyclic Sylow $2$-subgroup, then (b) will apply again and $K_1$ will have a normal subgroup of index $2.$

    Let $S=\langle x\rangle.$ Then $x$ has order $2^k$ by assumption. Therefore, $x^2$ has order $2^{k-1}$ since $$(x^2)^{2^{k-1}}=x^{2\cdot 2^{k-1}}=x^{2^k}=e$$ so $o(x^2)|2^{k-1}$ and also clearly $o(x^2)\ge 2^{k-1}.$

    So, we claim that $\langle x^2\rangle$ is a copy of a Sylow $2$-subgroup of $K_1.$

    However, this follows since $\rho(K_1)\cong\rho(G)\subset A_n$ and since $\rho(x^2)=\lambda_{x^2}\in A_n.$

    This is because $\lambda_{x^2}$ can be represented as a product of $\frac{|G|}{o(x^2)}=\frac{2^km}{2^{k-1}}=2k$ cycles of length $2^{k-1}.$ Since even length cycles are odd and the product of two odd cycles is even, we get that $\lambda_{x^2}$ is even.

    Therefore, $x^2\in K_1$ and so $K_1$ has a cyclic Sylow $2$-subgroup.

    Thus, (b) applies and so we repeat to obtain a chain $$K_k\trianglelefteq K_{k-1}\trianglelefteq \cdots\trianglelefteq K_1\trianglelefteq G$$ with $|K_j|=2^{k-j}m.$

    Therefore, $|K_k|=m$ and is a subgroup of $G$ containing only odd order elements. Let $K_k=N.$

    Finally, $G\cong K_1\langle x\rangle$ since $x\notin K_1$ and $K_1$ is of minimal index and so is of maximal order.

    Similarly, $K_1\cong K_2\langle x^2\rangle$. Thus,

    Therefore, $$G\cong Ne\langle x^{2^{k-1}}\rangle\cdots\langle x^2\rangle\langle x\rangle=NS.$$ Note that this follows from order arguments and uses no assumptions that $N$ is normal in $G.$ Namely, if $|HK|=|G|$, then $HK=G$ regardless of whether or not $H$ or $K$ is normal.

    Now, we simply note that if $n\in N$ with order $t$ for $t$ odd, then $$(x^lnx^{-l})^t=x^ln^tx^{-l}=x^lex^{-l}=e$$ and so $x^lnx^{-l}$ has order dividing $t$, and so namely, it has odd order.

    Therefore, $x^lnx^{-l}\in N$ for all $l$, so $N\subset N_G(S).$

    Now, let $g\in G.$ Then since $G=NS,$ $g=n_0x^l$ for some $n_0\in N$ and some $l.$

    Therefore, $$gng^{-1}=n_0x^lnx^{-l}n_0^{-1}=n_0n'n_0^{-1}\in N$$ since $x^l\in S$ and $N\subset N_G(S).$

    Therefore, $N$ is normal in $G.$
\end{enumerate}
\end{solution}
\newpage

\begin{problem} $\,$
For a group $G$ and $p$ a prime let $G(p)=\{g\in G\,|\,g^p=1\}.$
\begin{enumerate}[label=(\alph*)]
    \item Show that if $G$ is abelian, then $G(p)$ is a subgroup of $G.$ Give an example to show that $G(p)$ need not be a subgroup in general.
    \item Let $G,H$ be finitely generated abelian groups with $G/G(p)\cong H/H(p)$ and $G/G(q)\cong H/H(q)$ for different primes $p,q$. Show that $G\cong H.$
\end{enumerate}
\end{problem}


\begin{solution}$\,$
\begin{enumerate}[label=(\alph*)]
    \item Assume $G$ is abelian. Then let $a,b\in G(p)$. Then $(ab^{-1})^p=a^pb^{-p}=1$ since $G$ is abelian and so $ab^{-1}\in G(p).$

    Let $G=S_3$. Then $$G(2)=\{1,(1\,\, 2),(1\,\, 3),(2\,\,3)\}$$ which is clearly not a subgroup since $$(1\,\, 2)(2\,\,3)=(1\,\,2\,\,3)\notin G(2).$$
    \item Let $G,H$ be finitely generated abelian groups with $G/G(p)\cong H/H(p)$ and $G/G(q)\cong H/H(q)$ for different primes $p,q$.

    By the fundamental theorem of abelian groups, we can write \begin{align*}
        G&\cong\mathbb{Z}^m\oplus\mathbb{Z}_{p_1}^{\alpha_1}\oplus\cdots\oplus\mathbb{Z}_{p_k}^{\alpha_k}\\
        H&\cong\mathbb{Z}^n\oplus\mathbb{Z}_{q_1}^{\beta_1}\oplus\cdots\oplus\mathbb{Z}_{q_l}^{\beta_l}
    \end{align*}

    Then, if $a\in G(p)$, then $o(a)|p$ and so namely, either $a=1$ or $a\in\mathbb{Z}_p$.

    If $G(p)=1$ and $H(p)=1$, then we are done.

    Assume $G(p)\not=1.$ Then $G(p)\cong\mathbb{Z}_p^t$ for some $t>0$

    \boxed{H(p)=1} Then $p_i=p$ for some $i$. WLOG, say $p_1=p.$ Then \begin{align*}
        G/G(p)&\cong H\\
        \mathbb{Z}^m\oplus\mathbb{Z}_{p}^{\alpha_1-t}\oplus\cdots\oplus\mathbb{Z}_{p_k}^{\alpha_k}&\cong \mathbb{Z}^n\oplus\mathbb{Z}_{q_1}^{\beta_1}\oplus\cdots\oplus\mathbb{Z}_{q_l}^{\beta_l}
    \end{align*}

    Therefore, with possible reindexing, $m=n,$ $k=l$, and $\alpha_i=\beta_i$ for all $i\not=1$, and $\alpha_1-t=\beta_1.$ Note that this can be proved using projection maps, or by counting arguments.

    Now, regardless of what $G(q)$ and $H(q)$ are, we will get a contradiction.

    If $G(q)$ and $H(q)$ are both trivial, then $H\cong G$ so $H\not\cong G/G(p).$ If $G(q)\not=1,$ then $G/G(q)\cong H/H(q),$ however, this will imply, after possibly reindexing, that $p_1=q_1$ and $\alpha_1=\beta_1.$

    However, this contradicts the above, that $\alpha_1-t=\beta_1.$ \contradiction

    \boxed{H(p)\not=1} Then $H(p)\cong\mathbb{Z}_p^s$ for $s>0$ so, after possibly reindexing, we can take $q_1=p.$ \begin{align*}
        G/G(p)&\cong H/H(p)\\
        \mathbb{Z}^m\oplus\mathbb{Z}_{p}^{\alpha_1-t}\oplus\cdots\oplus\mathbb{Z}_{p_k}^{\alpha_k}&\cong \mathbb{Z}^n\oplus\mathbb{Z}_{p}^{\beta_1-s}\oplus\cdots\oplus\mathbb{Z}_{q_l}^{\beta_l}
    \end{align*} Thus, $m=n$, $k=l$, $\alpha_i=\beta_i$ for all $i\not=1$, and $\alpha_1-t=\beta_1-s$.

    Now, we repeat with $G(q)$ and $H(q)$, which both cannot be trivial by the same argument as earlier, to get that $\alpha_1=\beta_1$ and we are done.
\end{enumerate}
\end{solution}
\newpage

\begin{problem} $\,$
Let $R$ be a prime ring with only finitely many right ideals.
\begin{enumerate}[label=(\alph*)]
    \item Show that $R$ is a simple ring.
    \item Prove that either $R$ is finite or $R$ is a division ring.
\end{enumerate}
\end{problem}


\begin{solution}$\,$
\begin{enumerate}[label=(\alph*)]
    \item A prime ring is a ring satisfying: if $a,b\in R$, and $arb=0$ for all $r\in R$ implies $a=0$ or $b=0.$ Alternatively, if $I,J$ are both ideals of $R$ and $IJ=0$, then $I=0$ or $J=0.$

    Now, since $R$ has only finitely many right ideals, it is right artinian and so $J(R)$ is nilpotent.

    However, if $(J(R))^n=0$, then either $J(R)=0$ or $(J(R))^{n-1}=0$ because $R$ is prime. Recursively, we get that $J(R)=0.$

    Thus, by Artin-Wedderburn, $R$ is semisimple and so $$R\cong M_{n_1}(D_1)\oplus\cdots\oplus M_{n_k}(D_k)$$ for $D_i$ division rings.

    Now, recall that the matrix rings $M_{n_i}(D_i)$ represent simple submodules of $R$ and further note that the submodules of $R$ (considered as an $R$-module) are exactly the ideals of $R$ as a ring.

    Finally, because the $M_{n_i}(D_i)$ are simple submodules, they correspond exactly to minimal ideals $I_i$ of $R$. Namely, $M_{n_i}(D_i)\cong R/M_i$ for some maximal ideal $M_i\subset R$.

    Therefore, because $I_iI_j$ is an ideal for all $i,j$ and $I_iI_j\subsetneq I_i$ which is minimal, we get that $I_iI_j=0$ for all $i\not=j.$

    However, because $R$ is prime, this forces $I_i=0$ or $I_j=0$.

    Recursively, we lose all but one of the matrix rings in the decomposition and so $$R\cong M_n(D)\qquad\text{ which is simple.}$$

    \item If $R$ is finite we are done.

    Assume $R$ is not finite. From (a), $$R\cong M_n(D)$$ for some division ring $D$. Note that because $R$ is assumed infinite, $D$ is infinite.

    However, the right ideals of $M_n(D)$ correspond exactly to the right $D$-submodules of the free $D$-module $D^n.$ %This comes from $M_n(D)\cong\End(D^n).$

    If $n>1$, then $D^n$ has infinitely many submodules. For example, $D^n(1,a,0,...,0)\cong D\oplus Da$ is a non-trivial proper submodule for all $a\in D$ (of which there are infinitely many because $D$ must be infinite).

    This implies that $R$ has infinitely many right ideals, which is a contradiction.

    Thus, $n=1$ and so $R\cong D.$
\end{enumerate}
\end{solution}
\newpage



\begin{problem} $\,$
Let $R=\mathbb{C}[x_1,...,x_n]$ and let $J$ be a nonzero proper ideal of $R.$ Let $A=A(X),$ $B=B(X)\in M_r(R)$ and assume that det$(A)$ is a product of distinct monic irreducible polynomials in $R.$ Assume that for each $\alpha=(a_1,...,a_n)\in\mathbb{C},$ $B(\alpha)\in M_r(\mathbb{C})$ invertible implies that $A(\alpha)$ is invertible. Show that det$(A)$ divides det$(B)$ in $R.$
\end{problem}


\begin{solution}$\,$
if whenever $B(\alpha)$ is invertible $A(\alpha)$ is also invertible, then whenever $\det(B)\not=0$, $\det(A)\not=0.$

Thus, if $\det(A)=0$, $\det(B)=0$. Therefore, if $I=(\det(A))$, every $\alpha\in (V(I))$ also satisfies $\det(B)(\alpha)=\det(B(\alpha))=0$.

Therefore, by Nullstellenzat's Part II, there exists an $n>0$ such that $\det(B)^n\in I.$

Therefore, there exists $f(X)\in R$ such that $\det(B)^n=f(X)\det(A).$ Since $\det(A)$ consists of a product of distinct monic irreducible polynomials, say $\det(A)=g_1(X)\cdots g_k(X)$, for each $g_i(X)|\det(A),$ $g_i(X)|\det(B)^n$. Inductively, by the irreducibility of $g_i$, we get that $g_i(X)|\det(B)$ for all $i.$

Therefore, $\det(A)|\det(B)$ in $R.$
\end{solution}
\newpage



\begin{problem} $\,$
Let $L$ be the splitting field over $\mathbb{Q}$ for $p(x)=x^{10}+3x^5+1$. Let $G=\Gal(L/\mathbb{Q}).$
\begin{enumerate}[label=(\alph*)]
    \item Show that $G$ has a normal subgroup of index $2$.
    \item Show that $4$ divides $|G|.$
    \item Show that $G$ is solvable.
\end{enumerate}
\end{problem}


\begin{solution}$\,$
\begin{enumerate}[label=(\alph*)]
    \item Let $u=x^5$. Then $p(x)=x^{10}+3x^5+1=u^2+3u+1.$ Thus, using the quadratic formula, $$u=\frac{-3\pm\sqrt{9-4}}{2}=\frac{-3\pm\sqrt{5}}{2}\notin\mathbb{Q}.$$

    Therefore, $u^2+3u+1$ is irreducible over $\mathbb{Q}$ and so $p(x)$ is irreducible over $\mathbb{Q}$. We can note also that $p(x)$ is separable since $x^5=\frac{-3\pm\sqrt{5}}{2}$, yields no repeated roots. Namely, $L$ is indeed a Galois extension over $\mathbb{Q}$.

    Now, if $\alpha=\sqrt[5]{\frac{-3+\sqrt{5}}{2}}$ and $\beta=\sqrt[5]{\frac{-3-\sqrt{5}}{2}}$, then the roots of $p(x)$ are exactly, $\alpha\xi^i$ and $\beta\xi^j$ where $\xi$ is a $5^{th}$ root of unity and for $i,j\in\{1,...,5\}$.

    Therefore, $$G=\mathbb{Q}(\sqrt[5]{\alpha},\sqrt[5]{\beta},\xi).$$

    Now, by the Galois Correspondence Theorem, the normal subgroups of $G$ correspond exactly to the Galois extensions of $\mathbb{Q}$ contained in $L$, and furthermore, there is an $N\trianglelefteq G$ with $[G:N]=2$ if and only if there is a $K\subset L$ such that $[L:K]=\frac{|G|}{2}$, or alternatively, if and only if there is a $K\subset L$ with $[K:\mathbb{Q}]=2.$ Since $$\alpha^5-\beta^5=\frac{-3+\sqrt{5}}{2}-\frac{-3-\sqrt{5}}{2}=\frac{2\sqrt{5}}{2}=\sqrt{5}\in L$$ we have that $\mathbb{Q}(\sqrt{5})\subset L$.

    Since $[\mathbb{Q}(\sqrt{5}):\mathbb{Q}]=2$, there exists a subgroup $N\subset G$ with $[G:N]=2$ which is normal since $2$ is the smallest prime dividing $|G|.$ \textit{(To see a proof of this see \textbf{Spring 2010, Problem 2, Claim 1})}.

    Show that $G$ has a normal subgroup of index $2$.
    \item Since $\xi$ satisfies $x^4+x^3+x^2+x+1$, $[\mathbb{Q}(\xi):\mathbb{Q}]=4$ and so $G$ has a subgroup of index $4.$

    Namely, $4||G|.$

    \item $G$ is solvable if and only if $L$ is contained in a radical extension of $\mathbb{Q}$.

    However, $L$ is a radical extension of $\mathbb{Q}$.

    Recall that a radical extension is one in which $\mathbb{Q}=K_1\subset K_2\subset\cdots\subset K_n=L$ with $K_i=K_{i-1}(\alpha_i)$ for all $i$ for $\alpha_i$ satisfying that there exists $t$ with $\alpha_i^t\in K_{i-1}$.

    Therefore, since $$\mathbb{Q}\subset\mathbb{Q}(\xi)\subset\mathbb{Q}(\sqrt{5},\xi)\subset \mathbb{Q}(\sqrt{5},\alpha,\xi)\subset\mathbb{Q}(\sqrt{5},\alpha,\beta,\xi)=L$$ and \begin{align*}
        (\beta)^5&=3-\alpha^5\in\mathbb{Q}(\sqrt{5},\alpha,\xi)\\
        (\alpha)^5&=\frac{-3+\sqrt{5}}{2}\in\mathbb{Q}(\sqrt{5},\xi)\\
        (\sqrt{5})^2&=5\in\mathbb{Q}(\xi)\\
        (\xi)^5&=1\in\mathbb{Q}
    \end{align*} we have that $L$ is a radical extension.

    Therefore, $G$ is solvable.
\end{enumerate}
\end{solution}
\newpage


\end{document}
