\documentclass[12pt]{AlgebraQual}
\usepackage{preamble}

\name{Kayla Orlinsky}
\course{Algebra Exam}
\term{Fall 2016}
\hwnum{Fall 2016}

\begin{document}

\begin{problem} $\,$
If $R:=\mathbb{C}[x,y]/(y^2-x^3-1)$, then describe all the maximal ideals in $R.$
\end{problem}


\begin{solution}$\,$
By the correspondence theorem, there is a $\oto$ correspondence between maximal ideals of $R$ and maximal ideals of $\mathbb{C}[x,y]$ containing $(y^2-x^3-1).$

By Nullstellensatz, maximal ideals of $\mathbb{C}[x,y]$ are of the form $(x-a,y-b)$ for some $a,b\in\mathbb{C}$.

Now, again by Nullstellensatz, $a,b\in\mathbb{C}$ must be such that $(a,b)\in V(y^2-x^3-1)$. Namely, the maximal ideals of $R$ correspond exactly to $(a,\pm\sqrt{a^3+1})$ where $a\in\mathbb{C}.$
\end{solution}
\newpage


\begin{problem} $\,$
Suppose $F$ is a field, and $\mathfrak{b}_n(F)$ is the $F$-algebra of upper-triangular matrices, i.e., the subalgebra of $M_n(F)$ consisting of matrices $X$ such that $X_{ij}=0$ when $i>j$. Describe the Jacobson radical of $\mathfrak{b}_n(F),$ the simple modules, and the maximal semi-simple quotient.
\end{problem}


\begin{solution}$\,$
Let $A=\mathfrak{b}_n(F)$.

\boxed{J(A)} Note that $A$ is finite dimensional over $F$ and so $J(A)$ is nilpotent.

Now, if $X\in J(A)$ then $X$ is noninvertible, however, because $J(A)$ is quasi-regular, $I-X$ has a left inverse in $A$.

Namely, $X$ has a $0$ eigenvalue while $I-X$ does not. Since the eigenvalues of upper triangular matrices are exactly the values down the main diagonal, we get that $1$ is not an eigenvalue of $X$, else $I-X$ has a $0$ eigenvalue.

However, $aX\in J(A)$ for $a\in F$, and so if $X$ has any non-zero eigenvalue $\lambda$ then $\lambda^{-1}X\in J(A)$ has $1$ as an eigenvalue. This contradicts that $I-\lambda^{-1}X$ is invertible since this matrix will have a $0$ down the main diagonal.

Namely, $X$ cannot have any non-zero eigenvalues.

Therefore, every matrix in $J(A)$ has zeros down the main diagonal.

Now, if $Y$ is a matrix that has zeros down the main diagonal, then $Y$ has only $0$s as an Eigenvalue so $Y^n=0$ by Cayley Hamilton. Thus, $Y$ is nilpotent.

Since all nilpotent ideals are contained in $J(A),$ we have that $Y\in J(A)$.

Namely, $J(A)$ is exactly the set of strictly upper triangular matrices, or upper triangular matrices with zeros down the main diagonal.

\boxed{\text{Simple modules}} A simple module of $A$ is a simple left $A$-module, namely a quotient of $A$ by a maximal left ideal.

Since maximal left ideals are exactly $$I_i=\{X\in A\,|\, (X)_{ij}=0, j=1,...,n\}$$ namely, the matrices in $A$ with the $i\thh$ column zeros, we have that $A/I_i\cong F^i$ where $i=1,...,n$.

\boxed{\text{Maximal Semi-simple Quotient}} I believe that we are being asked to find is an ideal $I\subset A$ such that the quotient $A/I$ is semi-simple and $A/I$ is the largest of these quotients.

This will clearly be $A/J(A).$

Since $A$ is artinian $A/I$ is artinian for all ideals $I$ (quotients of artinian rings are artinian).

Now, $A/I$ is semi-simple if and only if $A/I$ artinian and $J(A/I)=0$ by Artin Wedderburn.

Since there is a $\oto$ correspondence between maximal ideals of $A$ containing $I$ and maximal ideals of $A/I$, we see that $J(A/I)=0$ implies that the intersection of every maximal ideal of $A/I$ is contained in $I$. Therefore, the intersection of all maximal ideals containing $I$ is contained in $I$ and so $J(A)\subset I$. Finally, $A/I$ is isomorphic to a subset of  $A/J(A)$ and so $A/J(A)$ is maximal.

\begin{comment}
which are exactly equivalent to $A/M$ for some maximal ideal $M.$

Since $J(A)\subset M$ for all $M$ (because the Jacobson is the intersection of all maximal ideals of $A$), we have that there is a $\oto$ correspondence between maximal ideals of $A$ and maximal ideals of $A/J(A)$.

However, clearly $A/J(A)\cong F^n$ since by the previous part, $J(A)$ contains all strictly upper triangular matrices so $A/J(A)$ is the set of diagonal matrices.

Therefore, the maximal ideals of $A/J(A)$ are exactly isomorphic to $F^{n-1}$.
These correspond to matrices in $A$ where exactly one component down the main diagonal is $0$.

Therefore, if $M_i$ is the maximal ideal corresponding to matrices $X\in A$ where $X_{ii}=0$, then $A/M_i$ is exactly the set of matrices $X$ where the $i\thh$ column is nonzero, and the $j\thh$ column is $0$ for all $j\not=i.$

Note that simple right $A$-modules would correspond exactly to matrices where a single row is nonzero.

Note further that all these simple $A$-modules will be isomorphic trivially since they are all isomorphic to $F$.


Finally, we get that the simple left $A$-modules are exactly $A/M$
\end{comment}

\end{solution}
\newpage



\begin{problem} $\,$
Let $\mathbb{F}_5$ be the finite field with $5$ elements, and consider the group $G=PGL_2(\mathbb{F}_5)$ (i.e., the quotient of the group of invertible $2\times 2$ matrices over $\mathbb{F}_5$ by the subgroup of scalar multiple of the identity.
\begin{enumerate}[label=(\alph*)]
    \item What is the order of $G$?
    \item Describe $N_G(P)$ where $P$ is a Sylow $5$-subgroup of $G$.
    \item If $H\subset G$ is a subgroup, can $H$ have order $15,20, 30?$
\end{enumerate}
\end{problem}


\begin{solution}$\,$
\begin{enumerate}[label=(\alph*)]
    \item $$|GL_2(\mathbb{F}_5)|=(5^2-1)(5^2-5)=(25-1)\cdot(25-5)=24\cdot 20=8\cdot 3\cdot 5\cdot 4=2^5\cdot3\cdot 5.$$

    Now, scalar multiples of the identity is of course a subgroup of size $5-1=4$ so $$|PGL_2(\mathbb{F}_5)|=2^5\cdot3\cdot 5/2^2=2^3\cdot3\cdot 5=120=5!$$

    \item Let $P$ be a Sylow $5$-subgroup of $G$.

    By Sylow, the number of Sylow $5$-subgroups of $G$, $n_5$ satisfies that $n_5|2^3\cdot3$ and $n_5\equiv 1\mod 5$. Therefore, $n_5=1,6.$

    Now, in $G$, scalar multiples of the identity are the same. Namely,  $$\begin{bmatrix}
    a & b\\
    c & d
    \end{bmatrix}=\begin{bmatrix}
    xa & xb\\
    xc & xd
    \end{bmatrix}\in G\qquad x\in \mathbb{F}_5.$$

    Thus, \begin{align*}
    \begin{bmatrix}
    2 & 4\\
    0 & 2
    \end{bmatrix}^2&=\begin{bmatrix}
    4 & 1\\
    0 & 4
    \end{bmatrix}\\
    \begin{bmatrix}
    2 & 4\\
    0 & 2
    \end{bmatrix}^5&=\begin{bmatrix}
    4 & 1\\
    0 & 4
    \end{bmatrix}\begin{bmatrix}
    4 & 1\\
    0 & 4
    \end{bmatrix}\begin{bmatrix}
    2 & 4\\
    0 & 2
    \end{bmatrix}\\
    &=\begin{bmatrix}
    1 & 3\\
    0 & 1
    \end{bmatrix}\begin{bmatrix}
    2 & 4\\
    0 & 2
    \end{bmatrix}\\
    &=\begin{bmatrix}
    2 & 0\\
    0 & 2
    \end{bmatrix}\\
    &=I
    \end{align*}

    Therefore, $$P=\left\langle\begin{bmatrix}
    2 & 4\\
    0 & 2
    \end{bmatrix}\right\rangle$$ is a Sylow $5$-subgroup.

    Now, $$\begin{bmatrix}
    0 & 2\\
    3 & 0
    \end{bmatrix}\begin{bmatrix}
    0 & 2\\
    3 & 0
    \end{bmatrix}=\begin{bmatrix}
    1 & 0\\
    0 & 1
    \end{bmatrix}$$ so $\begin{bmatrix}
    0 & 2\\
    3 & 0
    \end{bmatrix}$ is its own inverse. However, \begin{align*}
        \begin{bmatrix}
    0 & 2\\
    3 & 0
    \end{bmatrix}\begin{bmatrix}
    2 & 4\\
    0 & 2
    \end{bmatrix}\begin{bmatrix}
    0 & 2\\
    3 & 0
    \end{bmatrix}&=\begin{bmatrix}
    0 & 4\\
    1 & 2
    \end{bmatrix}\begin{bmatrix}
    0 & 2\\
    3 & 0
    \end{bmatrix}\\
    &=\begin{bmatrix}
    2 & 0\\
    1 & 2
    \end{bmatrix}
    \end{align*} which is not an element of $P$. Therefore, $N_G(P)\not=G$ so $n_5=6$ and $$|N_G(P)|=120/6=20.$$

    Since the normalizers will be isomorphic by the conjugation isomoprhism (because Sylow $p$-subgroups are all conjugates), it suffices to examine $N_G(P)$ where $P$ is the Sylow $5$-subgroup given above.

    Note \begin{align*}
        \begin{bmatrix}
    1 & 0\\
    0 & 4
    \end{bmatrix}\begin{bmatrix}
    2 & 4\\
    0 & 2
    \end{bmatrix}\begin{bmatrix}
    1 & 0\\
    0 & 4
    \end{bmatrix}^{-1}&=\begin{bmatrix}
    1 & 0\\
    0 & 4
    \end{bmatrix}\begin{bmatrix}
    2 & 4\\
    0 & 2
    \end{bmatrix}\begin{bmatrix}
    1 & 0\\
    0 & 4
    \end{bmatrix}\\
    &=\begin{bmatrix}
    2 & 4\\
    0 & 3
    \end{bmatrix}\begin{bmatrix}
    1 & 0\\
    0 & 4
    \end{bmatrix}\\
    &=\begin{bmatrix}
    2 & 1\
    0 & 2
    \end{bmatrix}\\
    &=\begin{bmatrix}
    1 & 3\
    0 & 1
    \end{bmatrix}\in P
    \end{align*} so $\begin{bmatrix}
    1 & 0\\
    0 & 4
    \end{bmatrix}\in N_G(P).$

    Now, $$\begin{bmatrix}
    1 & 0\\
    0 & 4
    \end{bmatrix}\begin{bmatrix}
    2 & 4\\
    0 & 2
    \end{bmatrix}=\begin{bmatrix}
    2 & 4\\
    0 & 3
    \end{bmatrix}\not=\begin{bmatrix}
    2 & 4\\
    0 & 2
    \end{bmatrix}\begin{bmatrix}
    1 & 0\\
    0 & 4
    \end{bmatrix}=\begin{bmatrix}
    2 & 1\\
    0 & 3
    \end{bmatrix}$$

    so $N_G(P)$ is non-abelian.

    Now, it is quickly verified that $$\begin{bmatrix}
    1 & 0\\
    0 & 3
    \end{bmatrix}\begin{bmatrix}
    2 & 4\\
    0 & 2
    \end{bmatrix}\begin{bmatrix}
    1 & 0\\
    0 & 2
    \end{bmatrix}=\begin{bmatrix}
    2 & 4\\
    0 & 1
    \end{bmatrix}\begin{bmatrix}
    1 & 0\\
    0 & 2
    \end{bmatrix}=\begin{bmatrix}
    2 & 3\\
    0 & 2
    \end{bmatrix}=\begin{bmatrix}
    4 & 1\\
    0 & 4
    \end{bmatrix}\in P$$ so $\begin{bmatrix}
    1 & 0\\
    0 & 3
    \end{bmatrix}\in N_G(P)$ and has order $4$. Therefore, $N_G(P)$ is a non-abelian group of order $20$ which Sylow $2$-subgroups isomorphic to $\mathbb{Z}_4$.

    Therefore, $N_G(P)$ is a semi-direct product.

    Let $\varphi:\mathbb{Z}_4\to\aut(\mathbb{Z}_5)$. Then if $\mathbb{Z}_4\cong\langle a\rangle$ and $\mathbb{Z}_5\cong\langle b\rangle$, $\varphi_i(a)=\sigma_i$ $i=1,2,3$ where $\sigma_1(b)=b^2$, $\sigma_2(b)=b^3$ and $\sigma_3(b)=b^4$.

    Clearly $\sigma_1^3=\sigma_2$ so $\varphi_1(a^3)=\varphi_2(a)$. Since $a\mapsto a^3$ is an isomorphism of $\mathbb{Z}_4$, the following diagram commutes and so $\varphi_1$ and $\varphi_2$ generate isomorphic semi-direct products.
    \begin{center}
    \begin{tikzcd}
    \langle a\rangle \arrow[d,"a\mapsto a^3",swap] \arrow[r,"\varphi_2"]
        & \aut(\mathbb{Z}_5) & \\
    \langle a\rangle  \arrow[ru,"\varphi_1",swap] &
    \end{tikzcd}
    \end{center}

    Now, this gives two possible multiplications for $N_G(P),$ either through $\varphi_1$ or $\varphi_3$. Namely, $$N_G(P)\cong\langle a,b\,|\,a^4=b^5=1,aba^{-1}=b^2\rangle$$ $$N_G(P)\cong\langle a,b\,|\,a^4=b^5=1,aba^{-1}=b^4\rangle$$

    Thus, we need only check if an element $a$ of order $4$ and a generator $b$ of $P$ satisfy $ab=b^2a$ or $ab=b^{-1}a$.

    Since $$ab=\begin{bmatrix}
    1 & 0\\
    0 & 3
    \end{bmatrix}\begin{bmatrix}
    2 & 4\\
    0 & 2
    \end{bmatrix}=\begin{bmatrix}
    2 & 4\\
    0 & 1
    \end{bmatrix}=\begin{bmatrix}
    4 & 3\\
    0 & 2
    \end{bmatrix}=\begin{bmatrix}
    4 & 1\\
    0 & 4
    \end{bmatrix}\begin{bmatrix}
    1 & 0\\
    0 & 3
    \end{bmatrix}=b^2a$$ we have at last that $$N_G(P)\cong\langle a,b\,|\,a^4=b^5=1,aba^{-1}=b^2\rangle$$

    \begin{mybox}
    ***Note that $PGL_2(\mathbb{F}_5)\cong S_5$, so perhaps showing such an isomoprhism would allow us to reach the conclusion of (b) faster.
\end{mybox}

    \item

    \boxed{20} Let $H\subset G$ be a subgroup. First, $|N_G(P)|=20$ so $|H|=20$ is fine.

    \boxed{15} Now, assume that $H$ has order $15.$ Then $H$ necessarily contains a Sylow $5$-subgroup $P.$ However, $|H|=15$ so $n_5|3$ and $n_5\equiv 1\mod 5$ implies that $n_5=1$ where $n_5$ here is the number of Sylow $5$-subgroups of $H.$ Namely, $P$ is normal in $H.$

    However, if $g\in G$ normalizes $P,$ then $g\in N_G(P)$ by definition, thus $H\subset N_G(P).$ However, $|N_G(P)|=20$ and so it does not have any elements of order $3$, namely $H$ cannot be a subset of $N_G(P).$

    Thus, $H$ does not exist.

    \boxed{30} Now, let $H$ have order $30.$ By the same argument as before, $H$ cannot have only one normal Sylow $5$ subgroup, and so it must contain all $6$ Sylow $5$ subgroups since by Sylow, $n_5|6=|H|/5$ and $n_5\equiv 1\mod 5.$

    Now, we note that in $H$, $n_3\equiv 1\mod 3$ and $n_3|10$. Thus, $n_3=1,10.$ Since $H$ contains all the Sylow $5$ subgroups of $G$, it cannot contain $10$ Sylow $3$-subgroups. Since every Sylow $5$-subgroup has order $5$ and every Sylow $3$-subgroup has order $3$, and since by the Sylow theorems, Sylow $p$-subgroups are all conjugates of each other, for each $p,$ this would force $H$ to have $4\cdot 6$ non-trivial elements of order $5$ and $2\cdot 10$ non-trivial elements of order $3$. Since this is $4\cdot 6+2\cdot 10=24+20=44$ distinct non-trivial elements and $H$ has order $30$, we reach a contradiction.

    Thus, $H$ has one normal Sylow $3$-subgroup $Q$. However, then for any Sylow $5$-subgroup $P$ of $H,$ $PQ$ will be a subgroup of $H$ of order $15$.

    Namely, then $G$ will have a subgroup of order $15.$ Since this is not possible we are done.

\begin{comment}
    Therefore, $$|H\cap N_G(P)|=|N_H(P)|=|H|/n_5=30/6=5$$ and so $$|HN_G(P)|=\frac{|H||N_G(P)|}{|H\cap N_G(P)|}=\frac{30\cdot 20}{5}=120.$$

    Namely, $G=HN_G(P)=N_G(P)H$



    However, because $3$ does not divide $N_G(P)$ this implies that $G$ has one normal Sylow $3$-subgroup.

    To see this, we examine $N_G(Q)$ where $Q$ is a Sylow $3$-subgroup of $H$ (and so of $G$). $H\subset N_G(Q)$. If $N_G(Q)=H$ then $N_G(Q)\cap N_G(P)=H\cap N_G(P)=P$ so $P$ normalizes $Q.$

    However, then $PQ=QP$ since for all $p\in P$ and $q\in Q$, we have that $pqp^{-1}=q_0$ some $q_0\in Q$ so $pq=q_0p\in QP$ and similarly for the other inclusion.

    Therefore, $PQ$ is a subgroup and it has order $3\cdot 5=15$. However, we already showed no such subgroups existed.

    If $N_G(Q)$ is strictly greater than $H$, then it has order $60$ or order $120$. If it has order $60$, then it is normal in $G$ because it has index $2.$ However, by Sylow, $$G=N_G(N_G(Q))=N_G(Q)\qquad Q\text{ Sylow subgroups }$$ so this contradicts that $N_G(Q)$ has order $60.$

    Finally, $N_G(Q)=G$ and so $G$ has a single Sylow $3$ subgroup.

    Now,
    \begin{center}
\begin{minipage}{0.4\textwidth}
    \begin{align*}
        \begin{bmatrix}
    4 & 2\\
    2 & 0
    \end{bmatrix}^3&= \begin{bmatrix}
    0 & 3\\
    3 & 4
    \end{bmatrix}\begin{bmatrix}
    4 & 2\\
    2 & 0
    \end{bmatrix}\\
    &= \begin{bmatrix}
    1 & 0\\
    0 & 1
    \end{bmatrix}\\
    &=I
    \end{align*}
\end{minipage}
\begin{minipage}{0.4\textwidth}
    \begin{align*}
        \begin{bmatrix}
    1 & 3\\
    3 & 0
    \end{bmatrix}^3&= \begin{bmatrix}
    0 & 3\\
    3 & 4
    \end{bmatrix}\begin{bmatrix}
    1 & 3\\
    3 & 0
    \end{bmatrix}\\
    &= \begin{bmatrix}
    4 & 0\\
    0 & 4
    \end{bmatrix}\\
    &=I
    \end{align*}
\end{minipage}
   \end{center}

These elements clearly generate distinct Sylow $3$-subgroups and so we contradict that $H$ has a normal Sylow $3$-subgroup. Namely, no such $H$ can exist.
\end{comment}

\end{enumerate}
\end{solution}
\newpage


\begin{problem} $\,$
Let $A$ be an $n\times n$ matrix over $\mathbb{Z}$. Let $V$ be the $\mathbb{Z}$-module of column vectors of size $n$ over $\mathbb{Z}$.
\begin{enumerate}[label=(\alph*)]
    \item Prove that the size of $V/AV$ is equal to the absolute value of $\det(A)$ if $\det(A)\not=0$.
    \item Prove that $V/AV$ is infinite if $\det(A)=0$.
\end{enumerate}
(hint: use the theory of finitely generated modules $\mathbb{Z}$-modules)
\end{problem}


\begin{solution}$\,$
\begin{enumerate}[label=(\alph*)]
    \item We use that $A$ has a smith normal form (since $\mathbb{Z}$ is a PID). Namely, there exists invertible matrices $P,Q$ so $A=PDQ$ and $D$ is diagonal. Since $P,Q$ are inverible over $\mathbb{Z}$, $\det(P)=\pm1$ and $\det(Q)=\pm1$.

    Namely, $\det(A)=\pm \det(D)$.

    Now, $V=\mathbb{Z}e_1\oplus\cdot\oplus\mathbb{Z}e_n$ where $e_i$ is the standard basis vector with $1$ in the $i\thh$ position.

    Now, because $P,Q$ are invertible, $QV=V$ and $PV=V$ so $$AV=PDQV=PDV=DV.$$

    If $$D=\begin{bmatrix}
    d_1 & 0 & \cdots & 0 & 0\\
    0 & d_2 & \cdots & 0 & 0\\
    & \vdots & \cdots & \vdots &\\
    0 & 0 &\cdots & d_{n-1} & 0\\
    0 & 0 & \cdots & 0 & d_n
    \end{bmatrix}$$ then $$V/DV=\mathbb{Z}/(d_1)e_1\oplus\cdots\oplus \mathbb{Z}/(d_n)e_n\cong\mathbb{Z}_{d_1}e_1\oplus\cdots\oplus\mathbb{Z}_{d_n}e_n.$$

    Namely, $|V/DV|=|d_1\cdot d_2\cdot\cdots\cdot d_n|=|\det(D)|=|\det(A)|.$

    \item  Note that $\mathbb{Z}$ is a PID. Thus, by the structure theorem of finitely generated modules over a PID, $$V=\mathbb{Z}^n\oplus T(V)$$ where $T(V)$ is the torsion part of $V.$

    Note that the rank of the free part of $V$ has size $n$ since there are $n$ linearly independent vectors of length $n$ over $\mathbb{Z}$, namely the standard basis vectors.

    If $\det(A)=0$, then the columns of $A$ cannot span $\mathbb{Z}^n$.

    Therefore, $$AV=\mathbb{Z}^m\oplus T(AV)$$ where $m<n$ and $T(AV)$ is the torsion part of $AV.$ Namely, $V/AV$ will have at least one copy of $\mathbb{Z}$ in its decomposition. Namely, it will be infinite.

    Again, this follows since rank$(V/AV)=$rank$(V)-$rank$(AV)>0$.
\end{enumerate}
\end{solution}
\newpage



\begin{problem} $\,$
Let $V$ be a finite dimensional right module over a division ring $D.$ Let $W$ be a $D$-submodule of $V$.
\begin{enumerate}[label=(\alph*)]
    \item Let $I(W)=\{f\in\End_D(V)\,|\,f(W)=0\}.$ Prove that $I(W)$ is a left ideal of $\End(V)$.
    \item Prove that any left ideal of $\End_D(V)$ is $I(W)$ for some submodule $W$.
\end{enumerate}
\end{problem}


\begin{solution}$\,$
\begin{enumerate}[label=(\alph*)]
    \item  $I(W)$ is nonempty, it contains the $0$ map. Let $f,g\in I(W)$ then by linearity, $(f-g)(W)=f(W)-g(W)=0+0=0$ so $f-g\in I(W)$. Thus, $I(W)$ is closed as an additive abelian group.

    Now, let $h\in\End(V)$. Then ``multiplication'' is actually composition in $\End(V)$ so if $f\in I(W)$ then $$(hf)(W)=(h\circ f)(W)=h(f(W))=h(0)=0$$ because $h$ is an endomorphism and so preserves the origin.

    Thus, $hf\in I(W)$ so $I(W)$ is a left ideal.
    \item Let $J$ be any left ideal of $\End_D(V)$. Note that $V$ is finite dimensional so there exists $v_i\in V$ so $$V=v_1D+\cdots v_nD.$$

    Let $$W=\bigcap_{f\in J}\ker(f).$$ Note that $0\in W$ so $W$ is nonempty. Then, $W\subset V.$ If $x,y\in W$ and $f\in J$ then $$f(x-y)=f(x)-f(y)=0-0=0$$ so $x-y\in W.$

    If $a\in D$ then $$f(ax)=af(x)=0$$ so $ax\in W.$

    Now, clearly $J\subset I(W)$ since every $f\in J$ satisfies that $f(W)=0.$

    Let $g\in I(W)$.

    Let $f_i\in J$ such that $f_i(v_i)\not=0.$ Note that if $f(v_i)=0$ for all $f\in J$, then $v_i\in W$ so $g(v_i)=0.$

    Now, let $h_i\in\End(V)$ such that $h_i(f(v_i))=g(v_i)$ and $h_i(f(v_j))=0$ for all $j\not=i$. If $g(v_i)=0$ then take $h_i\equiv 0.$

    Let $x\in V$, then $$x=\sum_{i=1}^na_iv_i\qquad a_i\in D.$$ Thus, \begin{align*}
        \sum_{j=1}^n(h_j\circ f_j)(x)&=\sum_{j=1}^n\left(h_j\circ f_j\right)\left(\sum_{i=1}^na_iv_i\right)\\
        &=\sum_{j=1}^n\sum_{i=1}^n(h_j\circ f_j)(a_iv_i)\\
        &=\sum_{j=1}^n\sum_{i=1}^na_ih_j(f_j(v_i))\\
        &=\sum_{j=1}^na_jh_j(f_j(v_j))\\
        &=\sum_{j=1}^na_jg(v_j)\\
        &=g\left(\sum_{j=1}^na_jv_j\right)\\
        &=g(x)
    \end{align*}

    Therefore, $$g= \sum_{j=1}^n(h_j\circ f_j)\in J$$ so $I(W)\subset J.$

\end{enumerate}
\end{solution}
\newpage



\begin{problem} $\,$
Let $p$ and $q$ be distinct primes. Let $F$ be the subfield of $\mathbb{C}$ generated by the $pq$-roots of unity. Let $a,b$ be squarefree integers all greater than $1.$ Let $c,d\in\mathbb{C}$ with $c^p=a$ and $d^q=b.$ Let $K=F(c,d).$
\begin{enumerate}[label=(\alph*)]
    \item Show that $K/\mathbb{Q}$ is a Galois extension.
    \item Describe the Galois group $K/F$
    \item Show that any intermediate field $F\subset L\subset K$ satisfies $L=F(S)$ where $S$ is some subset of $\{c,d\}.$
\end{enumerate}
\end{problem}


\begin{solution}$\,$
\begin{enumerate}[label=(\alph*)]
    \item Let $\xi$ be a primitive $pq\thh$-root of unity in $F$. Then $$\xi^{pq}=(\xi^p)^q=1$$ so $F$ contains a primitive $p\thh$-root of unity as well. Similarly, it contains a primitive $q\thh$ root of unity.

    We claim that $K$ is the splitting field of $f(x)=(x^p-a)(x^q-b)$. Clearly $c,d$ satisfy these polynomials. Now, if $\alpha$ is a root of $f(x),$ then $\alpha^p=a$, or $\alpha^q=b$. Thus, $\alpha=c(\xi^q)^t$ or $d(\xi^p)^s$ for some $t$ or some $s$ so $\alpha\in K.$

    Thus, $f(x)$ splits completely over $K$ so $K$ is the splitting field of a separable polynomial over $\mathbb{Q}$ so $K/\mathbb{Q}$ is Galois.

    \item
    Since $[K:\mathbb{Q}]\le pq$, and $$[K:\mathbb{Q}]=[K:F(c)][F(c):\mathbb{Q}]=[K:F(c)]p$$ and $$[K:\mathbb{Q}]=[K:F(d)][F(d):\mathbb{Q}]=[K:F(d)]q$$ we have that $[K:\mathbb{Q}]=pq.$

    Thus, $G=\gal(K/\mathbb{Q})$ has order $pq.$ WLOG, take $p<q$.

    Now, let $\sigma,\tau\in G$ be defined by $\sigma(c)=c\xi^q$ and $\sigma(d)=d$, and $\tau(c)=c$ and $\tau(d)=d\xi^p$. Then $\sigma$ clearly has order $p$ and $\tau$ has order $q$.

    Furthermore, $\sigma$ and $\tau$ commute so any permutation of the roots of $f(x)$ will be given by some power of $\sigma$ and $\tau.$

    Specifically, the map $c\mapsto c(\xi^q)^i$ and $d\mapsto d(\xi^p)^j$ is given by $\sigma^i\tau^j.$

    Therefore, $G$ is abelian and so it is isomoprhic to $\mathbb{Z}_{pq}.$


    \begin{comment}
    Namely, $\langle\sigma\rangle$ generates a Sylow $p$-subgroup and $\langle\tau\rangle$ generates a Sylow $q$-subgroup.

    Finally, by Sylow, $n_q=1$ since $n_q|p$ and $n_q\equiv 1\mod q$ but $p<q.$

    Thus, $\langle\tau\rangle$ is the unique Sylow $q$-subgroup, and since it commutes with $\langle\sigma\rangle$,

    Since $\sigma$ and $\tau$ commute, we actually have that any


    Then If $p\nmid|(q-1)$ then by Sylow, $n_p\equiv 1\mod p$ and $n_p|q$ so $n_p=1$.

    Also

    Thus, in this case, $n_p=n_q=1$ so $$G\cong \mathbb{Z}_{pq}.$$


    If $p|(q-1)$, and $n_p=q-1$, then because $n_q=1$, by the recognizing semi-direct products theorem, we have that $G\cong \mathbb{Z}_q\rtimes_\varphi\mathbb{Z}_p$ for some $\varphi.$

    Let $\varphi:\mathbb{Z}_p\to\aut(\mathbb{Z}_q)\cong\mathbb{Z}_{q-1}.$

    Then if $\mathbb{Z}_p=\langle x\rangle$, and $\mathbb{Z}_q=\langle y\rangle$, then let $y\mapsto y^t$ be a non-trivial element of $\aut(\mathbb{Z}_q)$ of order $p$ (which exists since $p|(q-1)$. Then $\varphi$ defines multiplication by $xyx^{-1}=\varphi(x)(y)=y^t$. This gives the semi-direct product structure of $G$ as $$G\cong\langle x,y\,|\,x^p=y^q=1,xyx^{-1}=y^t\rangle.$$

    Furthermore, since for any other map $\sigma\in\aut(\mathbb{Z}_q)$ with $\sigma(y)=y^s$ having order $p$ there exists some map $\tau\in\aut(\mathbb{Z}_q)$ such that $\tau\sigma\tau^{-1}(y)=y^t$. Namely, all other possible structures for $G$ as a semi-direct prduct will be isomorphic.
\end{comment}

    \item By the Galois correspondence theorem, each intermediate field $F\subset L\subset K$ corresponds to a subgroup $H$ of $G$ where $|H|=[K:L].$ Since if $H\not=G,\{e\}$, we have that $|H|=p,q$, we have that $[K:L]=p,q.$

    Since $G$ is abelian, there are exactly two nontrivial proper subgroups $H$ of order $p$ and $q.$ Therefore, there are two field extensions of $F$ contained strictly in $K$. Since $F(c)$ and $F(d)$ are two such extensions, these must be the only two.
\end{enumerate}
\end{solution}
\newpage


\end{document}
