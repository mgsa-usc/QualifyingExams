\documentclass{article}

\usepackage[margin=1in]{geometry}
\usepackage{amsmath,amsthm,amssymb}
\usepackage{bbm, enumerate}

\newenvironment{problem}[2][Problem]{\begin{trivlist}
\item[\hskip \labelsep {\bfseries #1}\hskip \labelsep {\bfseries #2.}]}{\end{trivlist}}
\newenvironment{note}[1][Note.]{\begin{trivlist}
\item[\hskip \labelsep {\bfseries #1}]}{\end{trivlist}}

\begin{document}

\title{Spring 2013: Real Analysis Graduate Exam}
\author{Peter Kagey}

\maketitle

% -----------------------------------------------------
% First problem
% -----------------------------------------------------
\begin{problem}{1}
  Suppose that $\{ f_n \}$ is a sequence of real valued continuously
  differentiable functions on $[0, 1]$ such that \[
    \lim_{n \rightarrow \infty} \int_0^1 |f_n(x)| dx = 0
    \text{ and }
    \lim_{n \rightarrow \infty} \int_0^1 |f'_n(x)| dx = 0.
  \]
  Show that $\{ f_n \}$ converges to 0 uniformly on $[0, 1]$.
\end{problem}

\begin{proof}
\end{proof}

% -----------------------------------------------------
% Second problem
% -----------------------------------------------------
\pagebreak

\begin{problem}{2}
  Investigate the convergence of $\sum_{n=0}^\infty a_n$, where \[
    a_n = \int_0^1 \frac{x^n}{1 - x}\sin(\pi x) dx
  \]
\end{problem}

\begin{proof}
  Because the integrand $sin(\pi x)x^n/(1-x)$ is positive, Tonelli's theorem
  gives \begin{align}
    \sum_{n=0}^\infty \int_0^1 \frac{x^n}{1 - x}\sin(\pi x) dx &=
    \int_0^1 \sum_{n=0}^\infty \frac{x^n}{1 - x}\sin(\pi x) dx \\ &=
    \int_0^1 \frac{\sin(\pi x)}{1 - x} \sum_{n=0}^\infty x^n dx \\ &=
    \int_0^1 \frac{\sin(\pi x)}{1 - x} \cdot \frac{1}{1 - x} dx \\ &=
    \int_0^1 \frac{\sin(\pi x)}{(1 - x)^2} dx.
  \end{align}
  Notice that (2) implies (3) because the bounds of the integral ensure that $x$
  is $m$-almost everwhere within the radius of convergece of the sum. \\

  Because the integrand is non-negative, by elementary calculus \[
    \int_0^1 \frac{\sin(\pi x)}{(1 - x)^2} dx \geq
    \int_{1/2}^1 \frac{\sin(\pi x)}{(1 - x)^2} dx \geq
    \int_{1/2}^1 \frac{1 - x}{(1 - x)^2} dx =
    \int_{1/2}^1 (1 - x)^{-1} dx =
    \infty.
  \] Therefore $\sum_{n=0}^\infty a_n = \infty$.
\end{proof}

% -----------------------------------------------------
% Third problem
% -----------------------------------------------------
\pagebreak

\begin{problem}{3}
  Let $(X, \mathcal{M}, \mu)$ be a measure space, $f_n,\ f \in L^1(\mu)$.
  Show that $\int_X |f_n - f| d\mu \rightarrow 0$ as $n \rightarrow \infty$
  if and only if \[
    \sup_{A\in\mathcal{M}} \left|\int_A f_n d\mu - \int_A f d\mu \right| \rightarrow 0
  \] as $n \rightarrow \infty$
\end{problem}

\begin{proof}
  ($\Longrightarrow$)
  Assume that $\int_X |f_n - f| d\mu \rightarrow 0$
  as $n \rightarrow \infty$; in particular choose sufficiently large $n$ so that
  $\int_X |f_n - f| d\mu < \varepsilon$. Then for any set $A \subset X$ \[
    \varepsilon =
    \int_X |f_n - f| d\mu =
    \int_A |f_n - f| d\mu +
    \int_{A^{\mathsf{c}}} |f_n - f| d\mu \geq
    \int_A |f_n - f| d\mu \geq
    \left| \int_A f_n - f\ d\mu \right|.
  \]
  Since this holds for all $A\in\mathcal{M}$, it also holds for the supremum.
\end{proof}

% -----------------------------------------------------
% Fourth problem
% -----------------------------------------------------
\pagebreak

\begin{problem}{4}
  Let $\mu$ and $\nu$ be $\sigma$-finite positive meastures, $\mu \geq \nu$
  and assume that $\nu \ll \mu - \nu$ ($\nu$ is absolutely continuous with
  respect to $\mu - \nu$). \\
  Prove that \[
    \mu\left(\{\frac{d\nu}{d\mu} = 1\}\right) = 0.
  \]
\end{problem}

\begin{proof}
\end{proof}

\end{document}
