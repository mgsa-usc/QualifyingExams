\documentclass{article}

\usepackage[margin=1in]{geometry}
\usepackage{amsmath,amsthm,amssymb}
\usepackage{bbm, enumerate, mathtools}

\newenvironment{problem}[2][Problem]{\begin{trivlist}
\item[\hskip \labelsep {\bfseries #1}\hskip \labelsep {\bfseries #2.}]}{\end{trivlist}}
\newenvironment{note}[1][Note.]{\begin{trivlist}
\item[\hskip \labelsep {\bfseries #1}]}{\end{trivlist}}

\begin{document}

\title{Spring 2012: Geometry/Topology Graduate Exam}
\author{Peter Kagey}

\maketitle

% -----------------------------------------------------
% First problem
% -----------------------------------------------------
\begin{problem}{1} \textit{(Topology)}
\end{problem}

\begin{proof}
\end{proof}

% -----------------------------------------------------
% Second problem
% -----------------------------------------------------
\pagebreak

\begin{problem}{2} \textit{(Topology)}
\end{problem}

\begin{proof}
\end{proof}

% -----------------------------------------------------
% Third problem
% -----------------------------------------------------
\pagebreak

\begin{problem}{3} \textit{(Topology)}
\end{problem}

\begin{proof}
\end{proof}

% -----------------------------------------------------
% Fourth problem
% -----------------------------------------------------
\pagebreak

\begin{problem}{4}
  Does there exist a smooth embedding of the projective plane $\mathbb{R}P^2$
  into $\mathbb{R}^2$? Justify your answer.
\end{problem}

\begin{proof}
\end{proof}

% -----------------------------------------------------
% Fifth problem
% -----------------------------------------------------
\pagebreak

\begin{problem}{5}
  Let $M$ be a manifold and let $C^\infty(M)$ be the algebra of $C^\infty$
  functions $M \rightarrow \mathbb{R}$. Explain the relationship between vector
  fields on $M$ and $C^\infty(M)$.
  If we consider vector fields as maps
  $C^\infty(M) \rightarrow C^\infty(M)$ is the composition map $XY$ also a
  vector field? What about $[X, Y] = XY - YX$?
\end{problem}

\begin{proof}
  
\end{proof}

% -----------------------------------------------------
% Sixth problem
% -----------------------------------------------------
\pagebreak

\begin{problem}{6}
  Let $S$ be the unit sphere defined by $x^2 + y^2 + z^2 + w^2 = 1$ in
  $\mathbb{R}^4$. Compute $\int_S\omega$ where \[
    \omega = (w + w^2)\, dx \wedge dy \wedge dz
  \]
\end{problem}

\begin{proof}
  By Stokes' Theorem, knowing that $\partial B^4 = S^3$, we can rewrite the
  integral as \[
    \int_S\omega = \int_{\partial B^4}\omega = \int_{B^4} d\omega.
  \]
  Then \begin{align*}
    d\omega
    &= d((w + w^2)\, dx \wedge dy \wedge dz) \\
    &= (1 + 2w)\,dw \wedge dx \wedge dy \wedge dz \\
    &= -(1 + 2w)\, dx \wedge dy \wedge dz \wedge dw,
  \end{align*} where $dx \wedge dy \wedge dz \wedge dw$ is the usual volume
  form.
  \\~\\
  By linearity, \begin{align*}
    \int_{B^4} d\omega
    &= -\underbrace{\int_{B^4}\, dx \wedge dy \wedge dz \wedge dw}_{\operatorname{volume}(B^4)} -
      \underbrace{\int_{B^4}2w\, dx \wedge dy \wedge dz \wedge dw}_{=0 \text{ by symmetry}} \\
    &= -\operatorname{volume}(B^4).
  \end{align*}
\end{proof}

% -----------------------------------------------------
% Seventh problem
% -----------------------------------------------------
\pagebreak

\begin{problem}{7}
  Does the equation $x^2 = y^3$ define a smooth submanifold in $\mathbb{R}^3$?
\end{problem}

\begin{proof}
  Consider the map $f\colon \mathbb R^3 \rightarrow \mathbb R$ which sends
  $(x,y,z) \xmapsto f x^2  - y^3$. The function $f$ is not submersive at $f^{-1}(0)$ because
  \[
    df_{(x,y,z)} =
    \begin{bmatrix}
      2x & -3y^2 & 0
    \end{bmatrix}
  \] has rank $0$ when $x=y=0$. Thus $0 \in \mathbb R$ is a critical value, and
  $f^{-1}(0)$ cannot be given the structure of a smooth manifold.
\end{proof}
\end{document}
