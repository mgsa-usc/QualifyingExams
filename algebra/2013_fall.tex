\documentclass{article}
\usepackage[margin=1in]{geometry}
\usepackage{amsmath,amsthm,amssymb,mathtools}
\usepackage{bbm, enumerate, tikz}
\usepackage{multicol, hyperref}

\newenvironment{problem}[2][Problem]{\begin{trivlist}
\item[\hskip \labelsep {\bfseries #1}\hskip \labelsep {\bfseries #2.}]}{\end{trivlist}}
\newenvironment{note}[1][Note.]{\begin{trivlist}
\item[\hskip \labelsep {\bfseries #1}]}{\end{trivlist}}
\newenvironment{hint}[1][Hint.]{\begin{trivlist}
\item[\hskip \labelsep {\bfseries #1}]}{\end{trivlist}}

\renewcommand{\thesection}{}
\renewcommand{\thesubsection}{}

\newcommand{\C}{\mathbb C}
\newcommand{\N}{\mathbb N}
\newcommand{\Q}{\mathbb Q}
\newcommand{\Z}{\mathbb Z}
\newcommand{\set}[1]{\{#1\}}
\newcommand{\normalsubgroup}{\trianglelefteq}
\newcommand{\chr}{\operatorname{char}}
\newcommand{\rank}{\operatorname{rank}}
\newcommand{\Ann}{\operatorname{Ann}}
\newcommand{\Aut}{\operatorname{Aut}}
\newcommand{\Gal}{\operatorname{Gal}}
\newcommand{\Var}{\operatorname{Var}}
\newcommand{\fn}[3]{{#1 \colon #2 \rightarrow #3}}


\begin{document}

\section{Fall 2013: Algebra Graduate Exam}
% -----------------------------------------------------
% First problem
% -----------------------------------------------------
\begin{subsection}{Problem 1.}
  Let $H$ be a subgroup of the symmetric group $S_5$. Can the order of $H$ be
  $15$, $20$ or $30$?
\end{subsection}

\begin{proof}
  First note that the only normal subgroup of $S_5$ is $A_5$, which has order
  $60$.
  \\~\\
  \textbf{Case 1.} Assume $|H| = 15$.
  Then $H$ must have both a Sylow $5$-subgroup and a Sylow
  $3$-subgroup, and thus $H$ must contain a $5$-cycle and a $3$-cycle.
  Since neither of the subgroups generated by these elements is normal in $S_5$,
  their product is not a subgroup.
  Therefore any subgroup of $S_5$ containing a $5$-cycle and a $3$-cycle has
  more than $15$ elements.
  \\~\\
  \textbf{Case 2.} Assume $|H| = 20$.
  Of course, $H$ must have both a Sylow $5$-subgroup and a Sylow $2$-subgroup,
  so
  \\
  \indent\textbf{Case 2a.} Assume the Sylow $2$-subgroup contains a
  transposition. Then a $5$-cycle and a transposition generates $S_5$,
  so $|H| = 120$, a contradiction.
  \\
  \indent\textbf{Case 2b.} If the Sylow $2$-subgroup does not contain a
  transposition, it must contain element of the form $(s_1s_2)(s_3s_4)$. (...?)
  \\~\\
  \textbf{Case 3.}
  Assume $|H| = 30$. Thus $H$ has a Sylow $5$-subgroup, a Sylow $3$-subgroup, and a Sylow $2$-subgroup.
  Based on Case 2a, if $H$ has a Sylow $2$-subgroup, it must be of the form $(s_1s_2)(s_3s_4)$ (...?)
  % Firstly, $H$ must have Sylow $5$-subgroup, which means that $H$ must contain
  % an element of order $5$, and thus a five-cycle. WLOG, say that $(12345) \in H$.
  % Then $H_{15}$ must contain a Sylow $3$-subgroup, and thus an element of order
  % $3$. Note that the only normal subgroup in $S_5$ is $A_5$, so the product of
  % the Sylow $3$-subgroup and the Sylow $5$-subgroup is not itself a subgroup.


\end{proof}
\pagebreak

% -----------------------------------------------------
% Second problem
% -----------------------------------------------------
\begin{subsection}{Problem 2.}
  Let $R$ be a PID and $M$ a finitely generated torsion module of $R$. Show that
  $M$ is a cyclic $R$-module if and only if for any prime $\mathfrak p$ of $R$,
  either $\mathfrak pM = M$ or $M/\mathfrak pM$ is a cyclic $R$-module.
\end{subsection}

\begin{proof}
\end{proof}
\pagebreak

% -----------------------------------------------------
% Third problem
% -----------------------------------------------------
\begin{subsection}{Problem 3.}
  Let $R = \C[x_1, \hdots, x_n]$ and suppose $I$ is a proper non-zero ideal of
  $R$. The coefficients of a matrix $A \in M_n(R)$ are polynomials in
  $x_1, \hdots, x_n$ and can be evaluated at $\beta \in \C^n$; write
  $A(\beta) \in M_n(\C)$ for the matrix so obtained. If for some $A \in M_n(R)$
  and all $\alpha \in \Var(I)$, $A(\alpha) = 0_{n\times n}$, show that for some
  integer $m$, $A^m \in M_n(I)$.
\end{subsection}

\begin{proof}
\end{proof}
\pagebreak

% -----------------------------------------------------
% Fourth problem
% -----------------------------------------------------
\begin{subsection}{Problem 4.}
  If $R$ is a noetherian unital ring, show that the power series ring $R[[x]]$
  is also a noetherian unital ring.
\end{subsection}

\begin{proof}
\end{proof}
\pagebreak

% -----------------------------------------------------
% Fifth problem
% -----------------------------------------------------
\subsection{Problem 5.}
  Let $p$ be a prime. Prove that $f(x) = x^p - x - 1$ is irreducible over
  $\Z/p\Z$. What is the Galois group?

\begin{hint}
  Observe that if $\alpha$ is a root of $f(x)$, then so is $\alpha + i$) for $i \in \Z/p\Z$.
\end{hint}
\begin{proof}
\end{proof}
\pagebreak

% -----------------------------------------------------
% Sixth problem
% -----------------------------------------------------
\begin{subsection}{Problem 6.}
  Let $K \subset \C$ be the field obtained by adjoining all roots of unity in
  $\C$ to $\Q$. Suppose $p_1 < p_2$ are primes, $a \in \C \setminus K$, and
  write $L$ for a splitting field of \[
    g(x) = (x^{p_1} - a)(x^{p_2} - a)
  \] over $K$. Assuming each factor of $g(x)$ is irreducible, determine the
  order and the structure of $\Gal(L/K)$.
\end{subsection}

\begin{proof}
\end{proof}
\end{document}
