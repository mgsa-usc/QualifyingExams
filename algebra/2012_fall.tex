\documentclass{article}

\usepackage[margin=1in]{geometry}
\usepackage{amsmath,amsthm,amssymb}
\usepackage{bbm, enumerate, tikz}
\usepackage{multicol}

\newenvironment{problem}[2][Problem]{\begin{trivlist}
\item[\hskip \labelsep {\bfseries #1}\hskip \labelsep {\bfseries #2.}]}{\end{trivlist}}
\newenvironment{note}[1][Note.]{\begin{trivlist}
\item[\hskip \labelsep {\bfseries #1}]}{\end{trivlist}}

\newcommand{\Z}{\mathbb Z}
\newcommand{\normalsubgroup}{\trianglelefteq}

\begin{document}

\title{Fall 2012: Algebra Graduate Exam}
\author{Peter Kagey}

\maketitle

% -----------------------------------------------------
% First problem
% -----------------------------------------------------
\begin{problem}{1}
  Use Sylow's theorems directly to find, up to isomorphism, all possible structures of groups of order $5 \cdot 7 \cdot 23$.
\end{problem}

\begin{proof}
  Sylow's theorems tell us that any group $G$ must have \begin{align*}
    r_5 &\text{ Sylow } 5\text{-subgroups,} \\
    r_7 &\text{ Sylow } 7\text{-subgroups, and} \\
    r_{23} &\text{ Sylow } 23\text{-subgroups}
  \end{align*} where $r_5, r_7$, and $r_{23}$ divide $5 \cdot 7 \cdot 23$, and $r_p \equiv 1 \bmod p$. \[
    r_p = 1, 5, 7, 5 \cdot 7, 23, 5 \cdot 23, 7 \cdot 23, \text{ or } 5 \cdot 7 \cdot 23
  \] considering the restriction on modulus,
    $r_5 \in \{1, 7 \cdot 23\}$,
    $r_7 = 1$, and
    $r_{23} = 1$.
\end{proof} Let $P$ and $Q$ be the unique Sylow $23$-subgroup and Sylow
$7$-subgroup respectively.
Since $P \cap Q = 1$, $PQ \cong P \times Q$. Let $R$ be a Sylow $5$-subgroup.
\\
Since $R \normalsubgroup G$ (why?), and $R$ has a complement $P \times Q$, $G$ is a
semidirect product of $R$ by $P \times Q$, that is $G = R \ltimes (P \times Q)$.
\\
% This can be done by counting homomorphisms from $\Z_5 \rightarrow \operatorname{Aut}(\Z_{23} \times \Z_7)$
By Rotman Lemma 7.21, there is a homomorphism \[
  \theta\colon \underbrace{R \rightarrow \operatorname{Aut}(P \times Q)}_{\Z_5 \rightarrow \Z_{22} \times \Z_6}.
\]
But since $\gcd(5, 22) = \gcd(5, 6) = 1$, the only homomorphism is trivial.
Therefore there is only one group of order $5 \cdot 7 \cdot 23$, the abelian
group \[
  G \cong \Z_5 \oplus \Z_7 \oplus \Z_{23}.
\]
% -----------------------------------------------------
% Second problem
% -----------------------------------------------------
\pagebreak

\begin{problem}{2}
  Let $A$, $B$, and $C$ be finitely generated $F[x] = R$ modules for $F$ a field with $C$ torsion free. Show that $A \otimes_R C \cong B \otimes_R C$ implies that $A \cong B$.
  Show by example that this conclusion can fail when $C$ is not torsion free.
\end{problem}

\begin{proof}
\end{proof}

\end{document}
