\documentclass{article}

\usepackage[margin=1in]{geometry}
\usepackage{amsmath,amsthm,amssymb}
\usepackage{bbm, enumerate, tikz}
\usepackage{multicol}

\newenvironment{problem}[2][Problem]{\begin{trivlist}
\item[\hskip \labelsep {\bfseries #1}\hskip \labelsep {\bfseries #2.}]}{\end{trivlist}}
\newenvironment{note}[1][Note.]{\begin{trivlist}
\item[\hskip \labelsep {\bfseries #1}]}{\end{trivlist}}

\newcommand{\C}{\mathbb C}
\newcommand{\N}{\mathbb N}
\newcommand{\Q}{\mathbb Q}
\newcommand{\Z}{\mathbb Z}
\newcommand{\set}[1]{\{#1\}}
\newcommand{\normalsubgroup}{\trianglelefteq}
\newcommand{\chr}{\operatorname{char}}
\newcommand{\rank}{\operatorname{rank}}
\newcommand{\Ann}{\operatorname{Ann}}
\newcommand{\Aut}{\operatorname{Aut}}
\newcommand{\Gal}{\operatorname{Gal}}
\newcommand{\Var}{\operatorname{Var}}
\newcommand{\fn}[3]{{#1 \colon #2 \rightarrow #3}}

\begin{document}

\title{Fall 2012: Algebra Graduate Exam}
\author{Peter Kagey}

\maketitle

% -----------------------------------------------------
% First problem
% -----------------------------------------------------
\begin{problem}{1}
  Use Sylow's theorems directly to find, up to isomorphism, all possible
  structures of groups of order $5 \cdot 7 \cdot 23$.
\end{problem}

\begin{proof}
  Sylow's theorems tell us that any group $G$ must have \begin{align*}
    r_5 &\text{ Sylow } 5\text{-subgroups,} \\
    r_7 &\text{ Sylow } 7\text{-subgroups, and} \\
    r_{23} &\text{ Sylow } 23\text{-subgroups}
  \end{align*} where $r_5, r_7$, and $r_{23}$ divide $5 \cdot 7 \cdot 23$, and $r_p \equiv 1 \bmod p$. \[
    r_p = 1, 5, 7, 5 \cdot 7, 23, 5 \cdot 23, 7 \cdot 23, \text{ or } 5 \cdot 7 \cdot 23
  \] considering the restriction on modulus,
  $r_5 \in \{1, 7 \cdot 23\}$, $r_7 = 1$, and $r_{23} = 1$.
  %
  Let $P$ and $Q$ be the unique Sylow $23$-subgroup and Sylow
  $7$-subgroup respectively.
  Since $P \cap Q = 1$, $PQ \cong P \times Q$. Let $R$ be a Sylow $5$-subgroup.
  \\
  Since $R \normalsubgroup G$ (why?), and $R$ has a complement $P \times Q$, $G$ is a
  semidirect product of $R$ by $P \times Q$, that is $G = R \ltimes (P \times Q)$.
  \\
  % This can be done by counting homomorphisms from $\Z_5 \rightarrow \operatorname{Aut}(\Z_{23} \times \Z_7)$
  By Rotman Lemma 7.21, there is a homomorphism \[
    \theta\colon \underbrace{
      R \rightarrow \operatorname{Aut}(P \times Q)
    }_{\Z_5 \rightarrow \Z_{22} \times \Z_6}.
  \]
  But since $\gcd(5, 22) = \gcd(5, 6) = 1$, the only homomorphism is trivial.
  Therefore there is only one group of order $5 \cdot 7 \cdot 23$, the abelian
  group \[
    G \cong \Z_5 \oplus \Z_7 \oplus \Z_{23}.
  \]
\end{proof}
\pagebreak

% -----------------------------------------------------
% Second problem
% -----------------------------------------------------
\begin{problem}{2}
  Let $A$, $B$, and $C$ be finitely generated $F[x] = R$ modules for $F$ a field
  with $C$ torsion free. Show that $A \otimes_R C \cong B \otimes_R C$ implies
  that $A \cong B$.
  Show by example that this conclusion can fail when $C$ is not torsion free.
\end{problem}
\begin{proof}(From Nicolle)\\
  $R$ is a PID since $F$ is a field, so by the
  \textit{structure theorem for finitely generated modules over a PID},
  \begin{align*}
    A &\cong T(A) \oplus R^n \\
    B &\cong T(B) \oplus R^m \\
    C &\cong R^t,
  \end{align*} were $T(M)$ denotes the torsion submodule of $M$.
  Since $A \otimes_R C \cong B \otimes_R C$, it follows that \begin{align*}
    (T(A) \oplus R^n) \otimes_R R^t &\cong (T(B) \oplus R^m) \otimes_R R^t \\
    (T(A) \otimes_R R^t) \oplus (R^n \otimes_R R^t) &\cong (T(B) \otimes_R R^t) \oplus (R^m \otimes_R R^t)
  \end{align*}
  Thus the free part of $A \otimes_R C$ is isomorphic to the free part of
  $B \otimes_R C$: \[
    R^n \otimes_R R^t \cong R^m \otimes_R R^t,
  \] so $n = m$. Similarly, the torsion submodules of $A \otimes_R C$ and
  $B \otimes_R C$ are isomorphic: \[
    (T(A) \otimes_R R^t) \cong (T(B) \otimes_R R^t),
  \] so $T(A) = T(B)$. Therefore, \[
    A \cong T(A) \oplus R^n \cong T(B) \oplus R^m \cong B,
  \] as desired.
  \\

  As a counterexample, consider $A = B \oplus \Ann(C)$. Then \[
    A \otimes_R C
    \cong B \otimes_R C \oplus \underbrace{\Ann(C) \otimes_R C}_0
    \cong B \otimes_R C,
  \] but $A \not\cong B$.
\end{proof}
\pagebreak

% -----------------------------------------------------
% Third problem
% -----------------------------------------------------
\begin{problem}{3}
  Working in the polynomial ring $\C[x, y]$, show that some power of
  $f(x,y) = (x + y)(x^2 + y^4 -2)$ is in $I = (x^3 + y^2, y^3 + xy)$.
\end{problem}
\begin{note}
  This is identical to the Problem 5 in the 2014 fall exam.
\end{note}
\begin{proof}
  It is sufficient to show that $f(x,y)$ vanishes on $\Var(I)$; by
  Hilbert's Nullstellensatz,
  this implies that $f(x,y)^m \in I$ for some $m \in \N$.
  \\~\\
  First note that $y^3 + xy = y(y^2 + x)$ vanishes when $y = 0$ or $x = -y^2$.
  \\
  \textbf{Case 1.} Assume $y = 0$. Then $x^3 + y^2$ vanishes at $(0, 0)$.
  \\
  % \textbf{Case 2.} Assume $x = -y^2$. Substituting this yields
  % $(-y^2)^3 + y^2 = -y^6 + $
  % % $x^3 - x = x(x^2 - 1)$ vanishes at $(0,0)$, $(-1,1)$, $(-1,-1)$, $(1,i)$, and $(1,-i)$.
  % Checking these against
  \textbf{Case 2.} Assume $x = -y^2$. Substituting this yields
  $(-y^2)^3 + y^2 = y^2(-y^4 + 1)$, so the polynomial vanishes at
  $(0,0),(-1,1), (-1,-1),(1,i),(1,-i)$
  Checking these: \begin{alignat*}{2}
    0^3 + 0^2 &= 0^3 + 0 \cdot 0 &&= 0 \\
    (-1)^3 + 1^2 &= 1^3 + (-1) \cdot 1 &&= 0 \\
    (-1)^3 + (-1)^2 &= (-1)^3 + (-1)(-1) &&= 0 \\
    1^3 + i^2 &= i^3 + 1 \cdot i &&= 0 \\
    1^3 + (-i)^2 &= (-i)^3 + 1(-i) &&= 0.
  \end{alignat*}
  Now it is enough to check that $f(x,y)$ vanishes on
  $\Var(I) = \set{(0,0),(-1,1), (-1,-1),(1,i),(1,-i)}$: \begin{align*}
    f(0, 0) &= \underbrace{(0 + 0)}_0(0^2 + 0^4 - 2) = 0\\
    f(-1, 1) &= \underbrace{(-1 + 1)}_0((-1)^2 + 1^4 - 2) \\
    f(-1, -1) &= (-1 + (-1))\underbrace{((-1)^2 + (-1)^4 - 2)}_0 \\
    f(1, i) &= (1 + i)\underbrace{(1^2 + i^4 - 2)}_0 \\
    f(1, -i) &= (1 + (-i))\underbrace{(1^2 + (-i)^4 - 2)}_0.
  \end{align*}
  Thus by Hilbert's Nullstellensatz, since $f$ vanishes on $\Var(I)$, a power of
  $f$ is in $I$.
\end{proof}
\pagebreak

% -----------------------------------------------------
% Fourth problem
% -----------------------------------------------------
\begin{problem}{4}
  For integers $n ,m > 1$, let $A \subseteq M_n(\mathbb Z_m)$ be a subring with the property that if $x \in A$ with $x^2 = 0$ then $x = 0$. Show that $A$ is commutative. Is the converse true?
\end{problem}

\begin{proof}
  % First, $A$ is a subring because if $A^2 \neq 0$ and $B^2 \neq 0$, then $(AB)^2 = ABAB \neq 0$.
  The idea here it to show that $A$ is semisimple, and so by Artin-Wedderburn
  can be written as \[
    A \cong M_{n_1}(\Delta_1) \times \hdots \times M_{n_m}(\Delta_m)
  \] where $\Delta_i$ is a field because it is finite and $n_i = 1$.
  \\~\\
  The converse is false. Let $A$ be the ring generated by a single element with $n = m = 2$: \[
    A = \left\langle\begin{bmatrix}0 & 0\\0 & 1\end{bmatrix}\right\rangle.
  \]  Then $A$ is commutative, but
  $\begin{bmatrix}0 & 0\\0 & 1\end{bmatrix}^2 = \begin{bmatrix}0 & 0\\0 & 0\end{bmatrix}$ while $\begin{bmatrix}0 & 0\\0 & 1\end{bmatrix} \neq \begin{bmatrix}0 & 0\\0 & 0\end{bmatrix}$.
\end{proof}
\pagebreak

% -----------------------------------------------------
% Fifth problem
% -----------------------------------------------------
\begin{problem}{5}
  Let $F$ be the splitting field of $f(x) = x^6 - 2$ over $\Q$.
  Show that $\Gal(F/\Q)$ is isomorphic to the dihedral group of order $12$.
\end{problem}

\begin{proof}
  Firstly, $F = \Q[\sqrt[3]2, \omega]$ where $\omega$ is a sixth root of unity.
  Then \begin{align*}
    [\Q[\sqrt[3]2]:\Q] &= 6, \text{ and} \\
    [F:\Q[\sqrt[3]2]] &= \varphi(6) = 2,
  \end{align*}
  so $[F:\Q] = [F:\Q[\sqrt[3]2]]\cdot[\Q[\sqrt[3]2]:\Q] = 12$ and
  $\Gal(F/\Q) = 12$.
  Now consider the automorphisms \[
    \tau: \begin{cases}
      \omega \mapsto \overline\omega \\
      \sqrt[3]2 \mapsto \sqrt[3]2
    \end{cases} \text{ and ~~~}
    \sigma: \begin{cases}
        \omega \mapsto \omega \\
        \sqrt[3]2 \mapsto \omega\sqrt[3]2
      \end{cases}.
  \]
  Now $\tau$ is of order $2$ and $\sigma$ is of order $6$, and the dihedral
  relation is satisfied: \begin{alignat*}{4}
    \sigma\tau\sigma\tau(\omega) &= \sigma\tau\sigma(\overline\omega) &&= \sigma\tau(\overline\omega) &&= \sigma(\omega) &&= \omega \\
    \sigma\tau\sigma\tau(\sqrt[3]2) &= \sigma\tau\sigma(\sqrt[3]2) &&= \sigma\tau(\omega\sqrt[3]2) &&= \sigma(\overline\omega\sqrt[3]2) &&= \underbrace{\overline\omega\omega}_1\sqrt[3]2 = \sqrt[3]2.
  \end{alignat*}
\end{proof}
\pagebreak

% -----------------------------------------------------
% Sixth problem
% -----------------------------------------------------
\begin{problem}{6}
  Given that all groups of order $12$ are solvable, show that any group of order $2^2 \cdot 3 \cdot 7^2$ is solvable.
\end{problem}

\begin{proof}
  Let $r_p$ denote the number of Sylow $p$-subgroups of $G$. Sylows theorems
  state that $r_p$ divides $2^2 \cdot 3 \cdot 7^2$, so \begin{align*}
    r_2 &\in \set{1, 3, 7, 3 \cdot 7, 7^2, 3 \cdot 7^2} \\
    r_3 &\in \set{1, 2, 2^2, 7, 2 \cdot 7, 2^2 \cdot 7, 7^2, 2 \cdot 7^2, 2^2 \cdot 7^2} \\
    r_7 &\in \set{1, 2, 2^2, 3, 2 \cdot 3, 2^2 \cdot 3}
  \end{align*} also $r_p \equiv 1 \bmod p$, so
  \begin{align*}
    r_2 &\in \set{1, 3, 7, 3 \cdot 7, 7^2, 3 \cdot 7^2} \\
    r_3 &\in \set{1, 2^2, 7, 2^2 \cdot 7, 7^2, 2^2 \cdot 7^2} \\
    r_7 &= 1
  \end{align*}
  This means that there is a unique---and thus normal---Sylow $7$-subgroup,
  call it $N \cong \Z_7$.
  Therefore $G \cong N \rtimes K$ where $K$ is a subgroup of order $12$.
  \\
  Now a group is solvable if it has a normal series whose factor groups are
  cyclic of prime order. Since $K$ is solvable, it has a normal series \[
    K = K_0 \leq K_1 \leq K_2 \leq \hdots \leq K_n = 1.
  \] where $K_{i}/K_{i+1}$ is a cyclic group of prime order.
  Moreover, since $N$ is normal, $NK_{i+1}$ is a subgroup of $NK_{i}$. Thus \[
    G = NK_0 \leq NK_1 \leq NK_2 \leq \hdots \leq \underbrace{NK_n}_N \leq 1
  \] is a normal series of $G$ where $NK_{i}/NK_{i+1} \cong K_{i}/K_{i+1}$ is a
  cyclic group of prime order for $i \in \set{0,1,..., n-1}$, and
  $N/1 \cong N \cong \Z_7$ is a cyclic group of prime order.
  Therefore $G$ is solvable.
\end{proof}
\end{document}
