\documentclass{article}

\usepackage[margin=1in]{geometry}
\usepackage{amsmath,amsthm,amssymb}
\usepackage{bbm, enumerate, tikz}
\usepackage{multicol}

\newenvironment{problem}[2][Problem]{\begin{trivlist}
\item[\hskip \labelsep {\bfseries #1}\hskip \labelsep {\bfseries #2.}]}{\end{trivlist}}
\newenvironment{note}[1][Note.]{\begin{trivlist}
\item[\hskip \labelsep {\bfseries #1}]}{\end{trivlist}}

\newcommand{\Z}{\mathbb Z}
\newcommand{\Q}{\mathbb Q}
\newcommand{\C}{\mathbb C}
\newcommand{\set}[1]{\{#1\}}
\newcommand{\normalsubgroup}{\trianglelefteq}
\newcommand{\Gal}{\operatorname{Gal}}

\begin{document}

\title{Fall 2012: Algebra Graduate Exam}
\author{Peter Kagey}

\maketitle

% -----------------------------------------------------
% First problem
% -----------------------------------------------------
\begin{problem}{1}
  Use Sylow's theorems directly to find, up to isomorphism, all possible structures of groups of order $5 \cdot 7 \cdot 23$.
\end{problem}

\begin{proof}
  Sylow's theorems tell us that any group $G$ must have \begin{align*}
    r_5 &\text{ Sylow } 5\text{-subgroups,} \\
    r_7 &\text{ Sylow } 7\text{-subgroups, and} \\
    r_{23} &\text{ Sylow } 23\text{-subgroups}
  \end{align*} where $r_5, r_7$, and $r_{23}$ divide $5 \cdot 7 \cdot 23$, and $r_p \equiv 1 \bmod p$. \[
    r_p = 1, 5, 7, 5 \cdot 7, 23, 5 \cdot 23, 7 \cdot 23, \text{ or } 5 \cdot 7 \cdot 23
  \] considering the restriction on modulus,
    $r_5 \in \{1, 7 \cdot 23\}$,
    $r_7 = 1$, and
    $r_{23} = 1$.
\end{proof} Let $P$ and $Q$ be the unique Sylow $23$-subgroup and Sylow
$7$-subgroup respectively.
Since $P \cap Q = 1$, $PQ \cong P \times Q$. Let $R$ be a Sylow $5$-subgroup.
\\
Since $R \normalsubgroup G$ (why?), and $R$ has a complement $P \times Q$, $G$ is a
semidirect product of $R$ by $P \times Q$, that is $G = R \ltimes (P \times Q)$.
\\
% This can be done by counting homomorphisms from $\Z_5 \rightarrow \operatorname{Aut}(\Z_{23} \times \Z_7)$
By Rotman Lemma 7.21, there is a homomorphism \[
  \theta\colon \underbrace{
    R \rightarrow \operatorname{Aut}(P \times Q)
  }_{\Z_5 \rightarrow \Z_{22} \times \Z_6}.
\]
But since $\gcd(5, 22) = \gcd(5, 6) = 1$, the only homomorphism is trivial.
Therefore there is only one group of order $5 \cdot 7 \cdot 23$, the abelian
group \[
  G \cong \Z_5 \oplus \Z_7 \oplus \Z_{23}.
\]
\pagebreak

% -----------------------------------------------------
% Second problem
% -----------------------------------------------------
\begin{problem}{2}
  Let $A$, $B$, and $C$ be finitely generated $F[x] = R$ modules for $F$ a field
  with $C$ torsion free. Show that $A \otimes_R C \cong B \otimes_R C$ implies
  that $A \cong B$.
  Show by example that this conclusion can fail when $C$ is not torsion free.
\end{problem}

\begin{proof}
\end{proof}
\pagebreak

% -----------------------------------------------------
% Third problem
% -----------------------------------------------------
\begin{problem}{3}
  Let $F$ be a finite field and $G$ a finite group with
  $\gcd\set{\operatorname{char} F, |G|} = 1.$ The group algebra $F[G]$ is an
  algebra over $F$ with $G$ as an $F$-basis, elements $\alpha = \sum_G a_gg$ for
  $g \in F$, and multiplication that extends $ag \cdot bh = ab \cdot gh$.
  Show that any $x \in F[G]$ that is not a zero left divisor must be invertible
  in $F[G]$.
  \\~\\
  \textbf{Note:} Since $x$ is not a zero left divisor, if $xy = 0$ for
  $y \in F[G]$ then $y=0$.
\end{problem}

\begin{proof}
\end{proof}
\pagebreak

% -----------------------------------------------------
% Fourth problem
% -----------------------------------------------------
\begin{problem}{4}
  If $p(x) = x^8 + 2x^6 + 3x^4 + 2x^2 + 1 \in \Q[x]$ and if
  $\Q \subseteq M \subseteq \C$ is a splitting field for $p(x)$ over $\Q$,
  argue that $\Gal(M/\Q)$ is solvable.
\end{problem}

\begin{proof}
\end{proof}
\pagebreak

% -----------------------------------------------------
% Fifth problem
% -----------------------------------------------------
\begin{problem}{5}
  Let $R$ be a commutative ring with $1$ and let $x_1, \hdots, x_n \in R$
  so that $x_1y_1 + \hdots + x_ny_n = 1$ for some $y_j \in R$. Let
  $A = \set{(r_1, r_2, \hdots, r_n) \in R^n \mid x_1r_1 + \hdots + x_nr_n = 0}.$
  Show that \begin{enumerate}[(i)]
    \item $R^n \cong_R A \oplus R$,
    \item $A$ has $n$ generators, and
    \item when $R = F[x]$ for $F$ a field, then $A_R$ is free of rank $n-1$.
  \end{enumerate}
\end{problem}

\begin{proof}
\end{proof}
\pagebreak

% -----------------------------------------------------
% Sixth problem
% -----------------------------------------------------
\begin{problem}{6}
  For $p$ a prime, let $F_p$ be the field of $p$ elements and $K$ and extension
  field of $F_p$ of dimension $72$. \begin{enumerate}[(i)]
    \item Describe the possible structures of $\Gal(K/F_p)$.
    \item If $g(x) \in F_p[x]$ is irreducible of degree $72$, argue that $K$ is
      a splitting field of $g(x)$ over $F_p$
    \item Which integers $d > 0$ have irreducibles in $F_p[x]$ of degree $d$
      that split in $K$?
  \end{enumerate}
\end{problem}

\begin{proof}
\end{proof}
\end{document}
