\documentclass{article}
\usepackage[margin=1in]{geometry}
\usepackage{amsmath,amsthm,amssymb,mathtools}
\usepackage{bbm, enumerate, tikz}
\usepackage{multicol, hyperref}

\newenvironment{problem}[2][Problem]{\begin{trivlist}
\item[\hskip \labelsep {\bfseries #1}\hskip \labelsep {\bfseries #2.}]}{\end{trivlist}}
\newenvironment{note}[1][Note.]{\begin{trivlist}
\item[\hskip \labelsep {\bfseries #1}]}{\end{trivlist}}
\newenvironment{hint}[1][Hint.]{\begin{trivlist}
\item[\hskip \labelsep {\bfseries #1}]}{\end{trivlist}}

\renewcommand{\thesection}{}
\renewcommand{\thesubsection}{}

\newcommand{\C}{\mathbb C}
\newcommand{\N}{\mathbb N}
\newcommand{\Q}{\mathbb Q}
\newcommand{\Z}{\mathbb Z}
\newcommand{\set}[1]{\{#1\}}
\newcommand{\normalsubgroup}{\trianglelefteq}
\newcommand{\chr}{\operatorname{char}}
\newcommand{\rank}{\operatorname{rank}}
\newcommand{\Ann}{\operatorname{Ann}}
\newcommand{\Aut}{\operatorname{Aut}}
\newcommand{\Gal}{\operatorname{Gal}}
\newcommand{\Var}{\operatorname{Var}}
\newcommand{\fn}[3]{{#1 \colon #2 \rightarrow #3}}


\begin{document}

\section{Spring 2012: Algebra Graduate Exam}
\label{sec:spring2012}

% -----------------------------------------------------
% First problem
% -----------------------------------------------------
\begin{subsection}{Problem 1.}
  Let $I$ be an ideal of $R = \C[x_1, \hdots, x_n]$. Show that $\dim_\C(R/I)$ is finite if and only if $I$ is contained in only finitely many maximal ideals of $R$.
\end{subsection}

\begin{proof}
\end{proof}
\pagebreak

% -----------------------------------------------------
% Second problem
% -----------------------------------------------------
\begin{subsection}{Problem 2.}
  If $G$ is a group with $|G| = 7^2 \cdot 11^2 \cdot 19$, show that $G$ must be
  abelian and describe the possible structures of $G$.
\end{subsection}

\begin{proof}
  We'll start by using Sylow's theorems.
  Firstly, let $r_p$ denote the number of Sylow $p$-subgroups.
  Since $p$ divides $|G|$, \begin{align*}
    r_{19} &\in \set{1, 7, 7^2, 11, 11\cdot7, 11\cdot7^2, 11^2, 11^2\cdot7,  11^2\cdot7^2}, \\
    r_{11} &\in \set{1, 7, 7^2, 19, 19\cdot7, 19\cdot7^2}, \\
    r_7  &\in \set{1, 11, 11^2, 19, 19\cdot11, 19\cdot11^2}.
  \end{align*}
  Since $r_p \cong 1 \bmod p$, we can further refine this to \begin{align*}
    r_{19} &= 1, \\
    r_{11} &\in \set{1, 19\cdot7}, \\
    r_7  &= 1.
  \end{align*}
  This means that we have unique subgroups $H_{19}$ and $H_7$ of orders
  $19$ and $7$ respectively. Since $H_7$ and $H_{19}$ are unique and thus normal,
  the product of $H_7$ and $H_{19}$ forms a normal subgroup, call it $N$.
  Since $H_7 \cap H_{19} = \set{e}$, $H_7H_{19} \cong H_7 \times H_{19}$,
  where $H_{19}$ is abelian because it is cyclic, and $H_7$ is abelian because all
  groups of order $p^2$ are abelian. Thus
  $N \cong \Z_7 \times \Z_7 \times \Z_{19}$ or $N \cong \Z_{49} \times \Z_{19}$.

  Since $N$ and $H_{11}$ are complementary,
  that is $N \cap H_{11} = \set{e}$ and $|N||H_{11}| = |G|$,
  $G$ can be realized as the semidirect product of $N$ and $H_{11}$ \[
    G = N \rtimes H_{11}.
  \]
  Thus it is enough to consider the possible structures of the semidirect
  product.
  \\~\\
  \textbf{Case 1.} Assume $N \cong \Z_7 \times \Z_7 \times \Z_{19}$.
  Consider homomorphisms $\fn \varphi {H_{11}} {\Aut(N)}$, noting that \[
    \Aut(N)
    \cong \Aut(\Z_7 \times \Z_7 \times \Z_{19})
    \cong \Aut(\Z_7 \times \Z_7) \times \Aut(\Z_{19})
    \cong \underbrace{\Aut(\Z_7 \times \Z_7)}_{\text{order } 48 \cdot 42} \times \Z_{18}.
  \]
  Since $\gcd(11, 48\cdot42\cdot18) = 1$, the only homomorphism is trivial.
  So the semidirect product is direct \[
    G \cong \Z_{7} \times \Z_7 \times \Z_{19} \times H_{11}
  \]
  \\~\\
  \textbf{Case 2.} Assume $N \cong \Z_{49} \times \Z_{19}$.
  Consider homomorphisms $\fn \varphi {H_{11}} {\Aut(N)}$, noting that \[
    \Aut(N)
    \cong \Aut(\Z_{49} \times \Z_{19})
    \cong \Aut(\Z_{49}) \times \Aut(\Z_{19})
    \cong \underbrace{\Aut(\Z_7 \times \Z_7)}_{\text{order } 7 \cdot 6} \times \Z_{18}.
  \]
  Since $\gcd(11, 7\cdot6\cdot18) = 1$, the only homomorphism is trivial.
  So the semidirect product is direct \[
    G \cong \Z_{49} \times \Z_{19} \times H_{11}
  \]
  Since $|H_{11}| = 11^2$, it is abelian, so by the fundamental theorem
  of abelian groups, $G$ is isomorphic to \begin{alignat*}{5}
    \Z_{7} &\times \Z_{7} &&\times \Z_{11} &&\times \Z_{11} &&\times \Z_{19},&& \\
    \Z_{7} &\times \Z_{7} &&\times \Z_{121} &&\times \Z_{19}, && &&\\
    \Z_{49} &\times \Z_{11} &&\times \Z_{11} &&\times \Z_{19},&& && \text{ or}\\
    \Z_{49} &\times \Z_{121} &&\times \Z_{19}.  && && &&
  \end{alignat*}
\end{proof}
\pagebreak

% -----------------------------------------------------
% Third problem
% -----------------------------------------------------
\begin{subsection}{Problem 3.}
  Let $F$ be a finite field and $G$ a finite group with
  $\gcd\set{\chr F, |G|} = 1.$ The group algebra $F[G]$ is an
  algebra over $F$ with $G$ as an $F$-basis, elements $\alpha = \sum_G a_gg$ for
  $g \in F$, and multiplication that extends $ag \cdot bh = ab \cdot gh$.
  Show that any $x \in F[G]$ that is not a zero left divisor must be invertible
  in $F[G]$.
  \\~\\
  \textbf{Note:} Since $x$ is not a zero left divisor, if $xy = 0$ for
  $y \in F[G]$ then $y=0$.
\end{subsection}

\begin{proof}
  Since $\chr F$ does not divide $|G|$, by Mashke's Theorem, $F[G]$ is semisimple, so by the Artin-Wedderburn theorem, \[
    F[G] \cong M_{n_1}(D_1) \times M_{n_2}(D_2) \times \hdots \times M_{n_k}(D_k)
  \] where $M_{n_i}(D_i)$ is an $n_i$-by-$n_i$ matrix ring over a division ring
  $D_i$.
  \\~\\
  Thus any $\alpha = \sum_{g \in G} a_gg \in F[G]$ maps under the isomorphism to \[
    \varphi(\alpha) = (a_1, a_2, \hdots, a_k) \in M_{n_1}(D_1) \times \hdots \times M_{n_k}(D_k).
  \]
  Now suppose for the sake of contradiction that some $a_i$ is not invertible for
  some $i$; without loss of generality, say that $i=1$.
  Then there exists some $b \neq 0 \in M_{n_1}(D_1)$ such that $a_1b = 0$ (why?),
  and \[
    (a_1, a_2, \hdots, a_k)\cdot(b, 0, 0, \hdots, 0) = (\underbrace{a_1b}_0, 0, 0, \hdots, 0).
  \] Therefore $\varphi^{-1}(a_1, a_2, \hdots, a_k) = x$ is a left divisor.

  Thus in order for $x$ not to be a left divisor, all $a_i$ must be invertible.
  Thus $x^{-1} = \varphi^{-1}(a_1^{-1}, a_2^{-1}, \hdots, a_k^{-1})$.
\end{proof}
\pagebreak

% -----------------------------------------------------
% Fourth problem
% -----------------------------------------------------
\begin{subsection}{Problem 4.}
  If $p(x) = x^8 + 2x^6 + 3x^4 + 2x^2 + 1 \in \Q[x]$ and if
  $\Q \subseteq M \subseteq \C$ is a splitting field for $p(x)$ over $\Q$,
  argue that $\Gal(M/\Q)$ is solvable.
\end{subsection}

\begin{proof}
  Let $q(y) = y^4 + 2y^3 + 3y^2 + 2y + 1$ so that $q(x^2) = p(x)$. Since
  $\deg(q) = 4$, $q$ is solvable by radicals with roots
  $\set{a_1, a_2, a_3, a_4}$ expressible as radicals. Thus $p$ is also solvable
  by radicals with roots
  $\set{\pm\sqrt{a_1}, \pm\sqrt{a_2}, \pm\sqrt{a_3}, \pm\sqrt{a_4}}$.
\end{proof}
\pagebreak

% -----------------------------------------------------
% Fifth problem
% -----------------------------------------------------
\begin{subsection}{Problem 5.}
  Let $R$ be a commutative ring with $1$ and let $x_1, \hdots, x_n \in R$
  so that $x_1y_1 + \hdots + x_ny_n = 1$ for some $y_j \in R$. Let
  $A = \set{(r_1, r_2, \hdots, r_n) \in R^n \mid x_1r_1 + \hdots + x_nr_n = 0}.$
  Show that \begin{enumerate}[(i)]
    \item $R^n \cong_R A \oplus R$,
    \item $A$ has $n$ generators, and
    \item when $R = F[x]$ for $F$ a field, then $A_R$ is free of rank $n-1$.
  \end{enumerate}
\end{subsection}

\begin{proof}
  First consider the map $\fn \varphi {R^n} R$ that sends
  $(r_1, \hdots, r_n) \mapsto x_1r_1 + \hdots + x_nr_n$ so that
  $\varphi(y_1, \hdots, y_n) = 1$ and thus is surjective.
  Notice also that $\ker(\varphi) = A$. So the short exact sequence splits:
  \[
    0 \rightarrow A \hookrightarrow R^n \twoheadrightarrow R \rightarrow 0
  \]
  \begin{enumerate}[(i)]
    \item Since $R$, as a module over itself, is free and thus projective,
      so $R^n \cong_R A \oplus R$.
    \item (?)
    \item If $R = F[x]$, then $R$ is a PID. Thus by the
    structure theorem for finitely generated modules over a PID, \[
      A \cong T(A) \oplus R^k
    \] and since $R^n \cong A \oplus R = T(A) \oplus R^{k+1}$, $T(A) \cong 0$
    and $k = n - 1$, so $\rank(A) = \rank(R^{n-1}) = n - 1$.
  \end{enumerate}

\end{proof}
\pagebreak

% -----------------------------------------------------
% Sixth problem
% -----------------------------------------------------
\begin{subsection}{Problem 6.}
  For $p$ a prime, let $F_p$ be the field of $p$ elements and $K$ and extension
  field of $F_p$ of dimension $72$. \begin{enumerate}[(i)]
    \item Describe the possible structures of $\Gal(K/F_p)$.
    \item If $g(x) \in F_p[x]$ is irreducible of degree $72$, argue that $K$ is
      a splitting field of $g(x)$ over $F_p$
    \item Which integers $d > 0$ have irreducibles in $F_p[x]$ of degree $d$
      that split in $K$?
  \end{enumerate}
\end{subsection}

\begin{proof}
\end{proof}
\end{document}
