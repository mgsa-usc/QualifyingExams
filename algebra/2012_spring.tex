\documentclass{article}

\usepackage[margin=1in]{geometry}
\usepackage{amsmath,amsthm,amssymb}
\usepackage{bbm, enumerate, tikz}
\usepackage{multicol}

\newenvironment{problem}[2][Problem]{\begin{trivlist}
\item[\hskip \labelsep {\bfseries #1}\hskip \labelsep {\bfseries #2.}]}{\end{trivlist}}
\newenvironment{note}[1][Note.]{\begin{trivlist}
\item[\hskip \labelsep {\bfseries #1}]}{\end{trivlist}}

\newcommand{\Z}{\mathbb Z}
\newcommand{\Q}{\mathbb Q}
\newcommand{\C}{\mathbb C}
\newcommand{\set}[1]{\{#1\}}
\newcommand{\normalsubgroup}{\trianglelefteq}
\newcommand{\Gal}{\operatorname{Gal}}

\begin{document}

\title{Spring 2012: Algebra Graduate Exam}
\author{Peter Kagey}

\maketitle

% -----------------------------------------------------
% First problem
% -----------------------------------------------------
\begin{problem}{1}
  Let $I$ be an ideal of $R = \C[x_1, \hdots, x_n]$. Show that $\dim_\C(R/I)$ is finite if and only if $I$ is contained in only finitely many maximal ideals of $R$.
\end{problem}

\begin{proof}
\end{proof}
\pagebreak

% -----------------------------------------------------
% Second problem
% -----------------------------------------------------
\begin{problem}{2}
  If $G$ is a group with $|G| = 7^2 \cdot 11^2 \cdot 19$, show that $G$ must be abelian and describe the possible structures of $G$.
\end{problem}

\begin{proof}
\end{proof}
\pagebreak

% -----------------------------------------------------
% Third problem
% -----------------------------------------------------
\begin{problem}{3}
  Let $F$ be a finite field and $G$ a finite group with
  $\gcd\set{\operatorname{char} F, |G|} = 1.$ The group algebra $F[G]$ is an
  algebra over $F$ with $G$ as an $F$-basis, elements $\alpha = \sum_G a_gg$ for
  $g \in F$, and multiplication that extends $ag \cdot bh = ab \cdot gh$.
  Show that any $x \in F[G]$ that is not a zero left divisor must be invertible
  in $F[G]$.
  \\~\\
  \textbf{Note:} Since $x$ is not a zero left divisor, if $xy = 0$ for
  $y \in F[G]$ then $y=0$.
\end{problem}

\begin{proof}
\end{proof}
\pagebreak

% -----------------------------------------------------
% Fourth problem
% -----------------------------------------------------
\begin{problem}{4}
  If $p(x) = x^8 + 2x^6 + 3x^4 + 2x^2 + 1 \in \Q[x]$ and if
  $\Q \subseteq M \subseteq \C$ is a splitting field for $p(x)$ over $\Q$,
  argue that $\Gal(M/\Q)$ is solvable.
\end{problem}

\begin{proof}
\end{proof}
\pagebreak

% -----------------------------------------------------
% Fifth problem
% -----------------------------------------------------
\begin{problem}{5}
  Let $R$ be a commutative ring with $1$ and let $x_1, \hdots, x_n \in R$
  so that $x_1y_1 + \hdots + x_ny_n = 1$ for some $y_j \in R$. Let
  $A = \set{(r_1, r_2, \hdots, r_n) \in R^n \mid x_1r_1 + \hdots + x_nr_n = 0}.$
  Show that \begin{enumerate}[(i)]
    \item $R^n \cong_R A \oplus R$,
    \item $A$ has $n$ generators, and
    \item when $R = F[x]$ for $F$ a field, then $A_R$ is free of rank $n-1$.
  \end{enumerate}
\end{problem}

\begin{proof}
\end{proof}
\pagebreak

% -----------------------------------------------------
% Sixth problem
% -----------------------------------------------------
\begin{problem}{6}
  For $p$ a prime, let $F_p$ be the field of $p$ elements and $K$ and extension
  field of $F_p$ of dimension $72$. \begin{enumerate}[(i)]
    \item Describe the possible structures of $\Gal(K/F_p)$.
    \item If $g(x) \in F_p[x]$ is irreducible of degree $72$, argue that $K$ is
      a splitting field of $g(x)$ over $F_p$
    \item Which integers $d > 0$ have irreducibles in $F_p[x]$ of degree $d$
      that split in $K$?
  \end{enumerate}
\end{problem}

\begin{proof}
\end{proof}
\end{document}
