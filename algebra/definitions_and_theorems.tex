\documentclass{article}
\usepackage[utf8]{inputenc}
\usepackage{amsmath, amssymb, amsthm, enumerate}

\title{Algebra Definitions}
\author{Peter Kagey}
\date{May 2019}

\theoremstyle{definition}
\newtheorem{theorem}{Theorem}
\numberwithin{theorem}{subsection} % important bit

\theoremstyle{definition}
\newtheorem{definition}[paragraph]{Definition}
\setcounter{secnumdepth}{4}

\newcommand{\Gal}{\operatorname{Gal}}
\newcommand{\Q}{\mathbb Q}
\newcommand{\set}[1]{\{#1\}}
\newcommand{\fn}[3]{{#1 \colon #2 \rightarrow #3}}

\begin{document}

\maketitle

\begin{section}{Groups}
  \begin{subsection}{Notation and definitions}
    \begin{subsubsection}{Basic definitions}
      \begin{definition}[Normal subgroup]
        Let $G$ be a group and $K$ be a subgroup of $G$. If $gkg^{-1} \in K$
        for all $k \in K$ and $g \in G$, then $K$ is called a normal subgroup of
        $G$ and is denoted $K \trianglelefteq G$.
      \end{definition}
      \begin{definition}[Simple group]
        A group $G$ is called a simple group is a group whose only normal
        subgroups are $\set e$ and $G$.
      \end{definition}
      \begin{definition}[Semidirect product]
        Let $K \trianglelefteq G$ and $Q \leq G$. A group $G$ is a semidirect
        product of $K$ by $Q$ (denoted $G = K \ltimes Q$) if there exists
        $Q_1 \cong Q$ such that $Q_1$ is a
        complement of $K$ in $G$, that is $K \cap Q_1 = 1$ and $KQ_1 = G$.
      \end{definition}
    \end{subsubsection}
    \begin{subsubsection}{Galois Theory}
      \begin{definition}[Normal series]
        A normal series of a group $G$ is a sequence of subgroups \[
          G = G_0 \geq G_1 \geq \hdots \geq G_n = 1
        \] in which $G_{i+1} \trianglelefteq G_i$ for all $i$.
      \end{definition}
      \begin{definition}[Factor groups]
        The factor groups of a normal series are the groups $G_i/G_{i+1}$ for
        $i = 0, 1, \hdots, n-1$.
      \end{definition}
      \begin{definition}[Length]
        The length of a a normal series is the number of nontrivial factor
        groups.
      \end{definition}
      \begin{definition}[Solvable group]
        A finite group is solvable if it has a normal series whose factor groups
        are cyclic of prime order.
      \end{definition}
    \end{subsubsection}
    \begin{subsubsection}{Centralizer/Normalizer}
      \begin{definition}[Center] % Rotman 44
        The center of a group $G$, denoted by $Z(G)$, is the set of all
        $a \in G$ that commute with every element of $G$.
      \end{definition}
      \begin{definition}[Centralizer]
        The centralizer of a subset $S$ of a group $G$ is defined to be \[
          C_G(S) = \set{g \in G \mid gs = sg \text{ for all } s \in S}.
        \]
      \end{definition}
      \begin{definition}[Normalizer]
        The centralizer of a subset $S$ in the group $G$ is defined to be \[
          N_G(S) = \set{g \in G \mid gS = Sg}.
        \]
      \end{definition}
      \begin{definition}[Commutator]
        If $a, b \in G$, the commutator of $a$ and $b$, denoted $[a, b]$, is \[
          [a, b] = a b a^{-1} b^{-1},
        \] and the commutator subgroup of $G$, denoted $G'$, is the subgroup of
        $G$ generated by all of the commutators.
      \end{definition}
      \begin{definition}[Class equation]
        Partition $G$ into its conjugacy classes, with $x_i$ the representative
        of the $i$th conjugacy class.
        The class equation of the finite group $G$ is \[
          |G| = |Z(G)| + \sum_{i} [G : C_G(x_i)].
        \]
      \end{definition}
    \end{subsubsection}
    \begin{subsubsection}{Group Actions}
      \begin{definition}[Group action]
        Let $G$ be a group and $X$ be a set. Then a group action on $X$ is a
        function $\fn{\varphi}{G \times X}{X}$ denoted
        $\varphi(g, x) = g \cdot x$ and satisfying \begin{enumerate}[(i)]
          \item Identity: group action by the identity is trivial for all
          $x \in X$: $1 \cdot x = x$.
          \item Compatibility: $(gh) \cdot x = g \cdot (h \cdot x)$.
        \end{enumerate}
        And $X$ is called a $G$-set.
      \end{definition}
      \begin{definition}[Orbit]
        The orbit of an element $x \in X$ is denoted by \[
          G \cdot x = \set{g \cdot x \mid g \in G}
        \]
      \end{definition}
      \begin{definition}[Stabilizer subgroup]
        The stabilizer subgroup of $G$ with respect to $x \in X$ is denoted \[
          G_x = \set{g \in G \mid g \cdot x = x}
        \]
      \end{definition}
      \begin{definition}[Transitive]
        A group action is called transitive is for each $x, y \in X$ there
        exists some $g \in G$ such that $g \cdot x = y$.
      \end{definition}
    \end{subsubsection}
  \end{subsection}
  \begin{subsection}{Theorems}
    \begin{theorem}[First isomorphism theorem]
      If $\varphi\colon G \rightarrow H$ is a group homomorphism then
      $\ker(\varphi) \trianglelefteq G$ and $G/\ker(\varphi) \cong \varphi(G)$.
    \end{theorem}
    \begin{theorem}[Second isomorphism theorem]
      Let $G$ be a group with $S \leq G$ and $N \trianglelefteq G$. Then
      \begin{enumerate}
        \item $SN \leq G$
        \item $S \cap N \trianglelefteq S$, and
        \item $(SN)/N \cong S/(S \cap N)$.
      \end{enumerate}
    \end{theorem}
    Strictly speaking, $N$ does not have to be a normal subgroup as long as $S$
    is a subgroup of the normalizer of $N$, $S \leq N_G(N)$.
    \begin{theorem}[Third isomorphism theorem]
      Let $G$ be a group with normal subgroup $N \trianglelefteq G$. Then
      \begin{enumerate}
        \item If $K \leq G$ (resp. $K \trianglelefteq G$) such that
        $N \subseteq K \subseteq G$, then $K/N \leq G/N$
        (resp. $K/N \trianglelefteq G/N$).
        \item Every subgroup (resp. normal subgroup) of $G/N$ is of the form
        $K/N$, for some subgroup (resp. normal subgroup) $K \subset G$ such that
        $N \subseteq K \subseteq G$.
        \item If $K \trianglelefteq G$ such that $N \subseteq K \subseteq G$,
        then $(G/N)/(K/N) \cong G/K$.
      \end{enumerate}
    \end{theorem}
    \begin{theorem}[Simplicity of the $A_n$] % Rotman, page 50–51
      $A_n$ is simple for all $n \geq 5$.
    \end{theorem}
    \begin{theorem}[Sylow's theorem] ~ % Rotman, page 78–81
      \begin{enumerate}[(i)]
        \item If $P$ is a Sylow $p$-subgroup of a finite group $G$, then all
        Sylow $p$-subgroups of $G$ are conjugate to $P$.
        \item If there are $r$ Sylow $p$-subgroups, then $r$ divides $|G|$ and
        $r \equiv 1 \bmod p$.
      \end{enumerate}
    \end{theorem}
    \begin{theorem}[Fundamental Theorem of Abelian Groups] ~ % Rotman, page 132
      If $G$ and $H$ are finite abelian groups, then $G \cong H$ if and only if,
      for all primes $p$, they have the same elementary divisors.
    \end{theorem}
    \begin{theorem} % https://groupprops.subwiki.org/wiki/Subgroup_of_index_equal_to_least_prime_divisor_of_group_order_is_normal
      Let $G$ be a finite group and $p$ be the least prime divisor of $|G|$.
      Then if $H$ is a subgroup of $G$ such that $[G: H] = p$,
      then $H \trianglelefteq G$.
    \end{theorem}
  \end{subsection}
\end{section}
\pagebreak
%
% Fields
%
\begin{section}{Fields}
  \begin{subsection}{Notation and definitions}
    \begin{subsubsection}{Basic definitions}
      \begin{definition}[Degree of a field extension]
        Suppose that $E/k$ is a field extension. Then $E$ may be considered as a
        vector space over $k$. The dimension of this vector space is called
        the degree of the field extension and is denoted by $[E:k]$.
      \end{definition}
      \begin{definition}[Field automorphism]
        A field automorphism of a field $K$ is an isomorphism $\fn \phi K K$.
        In particular, \begin{align*}
          \phi(a + b) &= \phi(a) + \phi(b) \text{ and}\\
          \phi(ab) &= \phi(a)\phi(b).
        \end{align*}
      \end{definition}
      \begin{definition}[Splitting field]
        A splitting field of a polynomial $p$ over a field $K$ is a field
        extension $L \supseteq K$ over which $p$ factors into linear factors.
      \end{definition}
      \begin{definition}[Separable polynomial]
        A polynomial $p$ is called separable if if factors into distinct linear
        factors in its splitting field.
      \end{definition}
      \begin{definition}[Separable extension]
        A separable extension is an field extension $E \supseteq F$ such that
        for every $\alpha \in E$, the minimal polynomial of $\alpha$ over $F$ is
        a separable polynomial.
      \end{definition}
      \begin{definition}[Normal extension]
        A normal extension $K \supseteq L$ is one for which every polynomial that
        is irreducible over $K$ either has no root in $L$ or splits into linear
        factors in $L$.
      \end{definition}
      \begin{definition}[Galois extension]
        A Galois extension is an algebraic field extension $E/F$ that is normal
        and separable.
      \end{definition}
      \begin{definition}[Galois group]
        Let $E \supseteq F$ be a field extension. The Galois group $\Gal(E/F)$
        is the set of automorphisms of $E$ that fix $F$ under function
        composition.
      \end{definition}
      \begin{definition}[Galois correspondence]
        Let $E \supseteq F$ be a finite, Galois extension.
        The Galois correspondence is the bijection between intermediate fields
        $F \supseteq K \supset E$ and subgroups of the Galois group $E/F$.
      \end{definition}
      \begin{definition}[Trace]
        ???
      \end{definition}
      \begin{definition}[Norm]
        ???
      \end{definition}
      \begin{definition}[Radical extension]
        A radical extension of a field $K$ is an extension that is obtained by
        adjoining a sequence of $n$th roots of elements of $K$.
      \end{definition}
      \begin{definition}[Finite field]
        A finite field is a field with a finite number of elements.
        Note: any finite field has $p^k$ elements for some prime $p$ and
        $k \in \mathbb N$.
      \end{definition}
      \begin{definition}[Cyclotomic extension]
        A cyclotomic extension $\Q(\xi_n)$ of $\Q$ is an extension formed by
        adjoining a primitive $n$th root of unity.
      \end{definition}
      \begin{definition}[Algebraic closure]
        An algebraic closure of a field $K$ is an algebraic extension $F/K$
        such that $F$ contains a root for every non-constant polynomial in
        $F[x]$.
      \end{definition}
    \end{subsubsection}
  \end{subsection}
  \begin{subsection}{Theorems}
    \begin{theorem}[Isomorphism extension theorem]
      Let $F$ be a field and $\fn \phi F {F'}$ an isomorphism. Then if $E$ is an
      extension field of $F$, $\phi$ can be extended into an isomorphism
      $\fn \tau E E'$.
    \end{theorem}
    \begin{theorem}[Fundamental theorem of Galois theory] % Rotman, p. 201
      Let $E/k$ be a finite Galois extension with Galois group $G = \Gal(E/k)$.
      The function \[
        \fn \gamma {\operatorname{Sub}(\Gal(E/k))} {\operatorname{Int}(E/k)},
      \]
      defined by $H \mapsto E^H$, is an order reversing bijection whose
      inverse maps ${B \mapsto \Gal(E/B)}$.
    \end{theorem}
    \begin{theorem}[Primitive element theorem]
      Finite separable extensions are simple.
    \end{theorem}
  \end{subsection}
\end{section}
\end{document}
